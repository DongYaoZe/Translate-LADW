
\chapter{第三章~~行列式}

\section{1. 引言}

读者可能已经遇到过行列式,至少是在微积分或代数学中遇到的 $2 \times 2$ 和 $3 \times 3$ 矩阵的行列式。对于 $2 \times 2$ 矩阵 
$$\begin{pmatrix} a & b \\ c & d \end{pmatrix},$$
行列式就是 $ad - bc$;$3 \times 3$ 矩阵的行列式可以通过“大卫之星”(Star of David)法则
\footnote{
译者注:大卫之心,即六芒星,可以将两个三角形中心重合地以反方向覆盖而得到。其名称与犹太历史上的大卫王有关。
}
找到。

在本章中,我们想要引入 $n \times n$ 矩阵的行列式,而不是仅仅给出一个形式定义。首先我想给出一些动机,然后推导出行列式应具备的一些性质。然后,如果我们想要让这些性质同时成立,我们就别无选择,只能得到行列式的几个等价定义。

从矩阵的行列式开始引入,而不是从向量组的行列式开始,因为前者更为方便:这里没有真正的区别,因为我们可以始终将向量拼接在一起(比如作为列)来构成一个矩阵。

设我们有 $\RR^n$ 中的 $n$ 个向量 $\vv_1, \vv_2, \dots, \vv_n$(注意向量的数量与维数一致),我们想找到由这些向量确定的平行六面体的\textbf{$n$维体积}。

由向量 $\vv_1, \vv_2, \dots, \vv_n$ 确定的平行六面体可以定义为所有向量 $\vv \in \RR^n$ 的集合,这些向量可以表示为 
$$\vv = t_1 \vv_1 + t_2 \vv_2 + \dots + t_n \vv_n, \quad 0 \le t_k \le 1 \quad \forall k = 1, 2, \dots, n.$$
当 $n=2$(平行四边形)和 $n=3$(平行六面体)时,这很容易可视化。那么 $n$ 维体积是什么?

在维数为 2 时,它表示面积;在维数为 3 时,它表示体积;在维数为 1 时,它则是长度。

最后,让我们引入一些符号。对于(列)向量组$\vv_1, \vv_2, \dots, \vv_n$,我们把它的行列式(我们将要构造的)表示为 $D(\vv_1, \vv_2, \dots, \vv_n)$.~如果我们把这些向量拼接成矩阵 $A$($A$ 的第 $k$ 列是 $\vv_k$),那么我们将使用符号 $\det A$, 
$$\det A = D(\vv_1, \vv_2, \dots, \vv_n).$$

对于矩阵 
$$A = \begin{pmatrix} a_{1,1} & a_{1,2} & \dots & a_{1,n} \\ a_{2,1} & a_{2,2} & \dots & a_{2,n} \\ \vdots & \vdots & \ddots & \vdots \\ a_{n,1} & a_{n,2} & \dots & a_{n,n} \end{pmatrix},$$
它的行列式也常常表示为
$$
\begin{vmatrix}
a_{1,1} & a_{1,2} & \dots & a_{1,n} \\
a_{2,1} & a_{2,2} & \dots & a_{2,n} \\
\vdots & \vdots & \ddots & \vdots \\
a_{n,1} & a_{n,2} & \dots & a_{n,n}
\end{vmatrix}.
$$

\section{2. 行列式应具备的性质}

我们知道,对于维度 2 和 3,平行六面体的“体积”由“底乘以高”法则确定:如果我们选择一个向量,那么高是该向量到由其余向量张成的子空间的距离,底是其余向量确定的平行六面体的($n-1$ 维)体积。

现在让我们将这个想法推广到更高维度。我们暂时不关心如何精确地确定高和底。我们将表明,如果我们假设高和底满足某些自然性质,那么我们就别无选择,行列式就被唯一确定了。

\subsection{2.1. 关于列向量的线性性质}

首先,如果我们把向量 $\vv_1$ 乘以一个正数 $a$,那么,高(即到线性张成 $\LL(\vv_2, \dots, \vv_n)$ 的距离)就会乘以 $a$.~如果我们允许负高度(和负体积),那么这个性质对所有标量 $a$ 都成立,因此向量组 $\vv_1, \vv_2, \dots, \vv_n$ 的行列式 $D(\vv_1, \vv_2, \dots, \vv_n)$ 应该满足
$$
D(\alpha \vv_1, \vv_2, \dots, \vv_n) = \alpha D(\vv_1, \vv_2, \dots, \vv_n)
$$
当然,向量 $\vv_1$ 没有什么特别之处,所以对于任何下标 $k$
$$
(2.1) \quad D(\vv_1, \dots, \alpha \underset{k}{\vv_k}, \dots, \vv_n) = \alpha D(\vv_1, \dots, \underset{k}{\vv_k}, \dots, \vv_n) 
$$
为了得到下一个性质,让我们注意到,如果我们将两个向量相加,那么结果的“高度”应该等于被加数“高度”的总和,即
$$
(2.2) \quad D(\vv_1, \dots, \underset{k}{\underbrace{\uu_k + \vv_k}}, \dots, \vv_n) =\\
D(\vv_1, \dots, \underset{k}{\uu_k}, \dots, \vv_n) + D(\vv_1, \dots, \underset{k}{\vv_k}, \dots, \vv_n) 
$$
换句话说,上述两个性质表明,行列式是\textbf{每个参数(向量)的线性},这意味着如果我们固定 $n-1$ 个向量,并将剩余向量看作一个变量(参数),我们就会得到一个线性函数。

{\heiti 注记}~~ 我们已经知道\textbf{线性性质}(linearity)是一个非常好的性质,在许多情况下对我们都有帮助。因此,允许负高度(以及由此而来的负体积)是获得线性的一个非常小的代价,因为我们之后总是可以取绝对值。

实际上,通过允许负高度,我们并没有牺牲任何东西!相反,我们甚至获得了一些东西,因为行列式的符号包含了有关向量系统(方向)的一些信息。

\subsection{2.2. “列替换”下保持不变性}

下一个性质也看起来很自然。也就是说,如果我们取一个向量,比如 $\vv_j$,并将其加上另一个向量 $\vv_k$ 的倍数,“高度”不会改变,所以
$$
(2.3) \quad D(\vv_1, \dots, \underset{j}{\underbrace{\vv_j + \alpha \vv_k}}, \dots, \underset{k}{\vv_k}, \dots, \vv_n) = D(\vv_1, \dots, \underset{j}{\vv_j}, \dots, \underset{k}{\vv_k}, \dots, \vv_n)
$$
换句话说,如果我们应用第三种类型的\textbf{列运算},行列式不会改变。

{\heiti 注记}~~ 虽然在此并非必需,但让我们注意到第二个部分的线性性质(性质 (2.2))不是独立的:它可以从性质 (2.1) 和 (2.3) 推导出来。

我们将证明留作读者的练习。

\subsection{2.3. 反对称性}

行列式应该具备的另外一个性质,是如果我们交换了两个向量,行列式的符号改变一次。
\footnote{具有多个变量的一些函数的性质,是在交换任意两个参数时会变号,这类函数称为反对称函数。}

也就是说,如果我们交换两个向量,行列式会变号:
$$
 (2.4) \quad D(\vv_1, \dots, \underset{j}{\vv_k}, \dots, \underset{k}{\vv_j}, \dots, \vv_n) = - D(\vv_1, \dots, \underset{j}{\vv_j}, \dots, \underset{k}{\vv_k}, \dots, \vv_n)
$$
第一次看到时,这个性质看起来并不自然,但它可以从前面的性质推导出来。也就是说,三次应用性质 (2.3),然后使用 (2.1),我们得到
\begin{align*} 
&D(\vv_1, \dots, \underset{j}{\vv_j}, \dots, \underset{k}{\vv_k}, \dots, \vv_n) \\
&= D(\vv_1, \dots, \underset{j}{\vv_j}, \dots, \underset{k}{\underbrace{\vv_k - \vv_j}}, \dots, \vv_n) \\ 
&= D(\vv_1, \dots, \underset{j}{\underbrace{\vv_j + (\vv_k - \vv_j)}}, \dots, \underset{k}{\underbrace{\vv_k - \vv_j}}, \dots, \vv_n) \\ 
&= D(\vv_1, \dots, \underset{j}{\vv_k}, \dots, \underset{k}{\underbrace{\vv_k - \vv_j}}, \dots, \vv_n) \\ 
&= D(\vv_1, \dots, \underset{j}{\vv_k}, \dots, \underset{k}{\underbrace{(\vv_k - \vv_j) - \vv_k}}, \dots, \vv_n) \\ 
&= D(\vv_1, \dots, \underset{j}{\vv_k}, \dots, -\underset{k}{\vv_j}, \dots, \vv_n) \\ 
&= - D(\vv_1, \dots, \underset{j}{\vv_k}, \dots, \underset{k}{\vv_j}, \dots, \vv_n). \end{align*}

\subsection{2.4. 归一化}

最后一个性质是最简单的。对于 $\RR^n$ 中的标准基 $\ee_1, \ee_2, \dots, \ee_n$,对应的平行六面体是 $n$ 维单位立方体,所以
$$
(2.5)\quad D(\ee_1, \ee_2, \dots, \ee_n) = 1.
$$
在矩阵表示中,这可以写成 
$$\det (I) = 1.$$


\section{3. 行列式的构造}

我们现在的计划是:利用我们从第 2 节认为的行列式应该具有的性质,推导出行列式的其他性质,其中一些性质非常不平凡。我们将展示如何使用这些性质通过我们熟悉的朋友——行约简来计算行列式。

稍后,在第 4 节,我们将展示行列式,即具有所需性质的函数,是存在且唯一的。毕竟,我们必须确信我们正在计算和研究的对象是存在的。

虽然我们引入行列式及其性质的初始几何动机来自于考虑 $\RR^n$ 中的向量,因此它们只与实数项的矩阵相关,但以下所有构造只使用代数运算(加法、乘法、除法)并且适用于具有复数项的矩阵,甚至适用于具有任意域项的矩阵。

因此,在以下内容中,我们不仅为实数矩阵构造行列式,也为复数矩阵(以及具有任意域项的矩阵)构造行列式。虽然我们最初的几何动机仅适用于实数情况,但在我们确定了行列式的性质(见本节的性质 1—3)之后,所有内容都适用于一般情况。

\subsection{3.1. 基本性质}

在这一节中,我们将使用以下行列式性质:

1. 行列式在每一列中都是线性的,即,在向量表示中,对于每个下标 $k$,
$$
D(\vv_1, \dots, \underset{k}{\underbrace{\alpha \uu_k + \beta \vv_k}}, \dots, \vv_n) = \alpha D(\vv_1, \dots, \underset{k}{\uu_k}, \dots, \vv_n) + \beta D(\vv_1, \dots, \underset{k}{\vv_k}, \dots, \vv_n)
$$
对所有标量 $\alpha, \beta$ 成立。

2. 行列式是\textbf{反对称}(antisymmetric)的,即,如果我们交换两列,行列式前正负符号改变一次。

3. 归一化性质:$\det I = 1$.~

所有这些性质在第 2 节中都已讨论过。第一个性质只是 (2.1) 和 (2.2) 的组合。第二个是 (2.4),最后一个是归一化性质 (2.5)。注意,我们没有使用性质 (2.3):它可以从上述三个性质中推导出来。因此,这三个性质就完全定义了行列式!

\subsection{3.2.从行列式基本性质中推导出的性质}

{\heiti 命题 3.1}~~ 对于方阵 $A$,以下陈述成立:

1. 如果 $A$ 有一个零列,那么 $\det A = 0$.~

2. 如果 $A$ 有两列相等,那么 $\det A = 0$;

3. 如果 $A$ 的一列是另一列的倍数,那么 $\det A = 0$;

4. 如果 $A$ 的列是线性相关的,即如果矩阵不可逆,那么 $\det A = 0$.~

\begin{proof} 陈述 1 由线性性质直接得出。如果我们用零乘以零列,我们不会改变矩阵及其行列式。但根据上面的性质 1,我们应该得到 0。

行列式的反对称性蕴含了陈述 2。
实际上,如果我们交换两列相等的列,我们什么也没改变,所以行列式保持不变。另一方面,交换两列改变了行列式的符号,所以 $$\det A = -\det A,$$
这只有在 $\det A = 0$ 时才可能。

陈述 3 是陈述 2 和线性性质的直接推论。

要证明最后一个陈述,让我们首先假设第一个向量 $\vv_1$ 是其他向量的线性组合,$$\vv_1 = \alpha_2 \vv_2 + \alpha_3 \vv_3 + \dots + \alpha_n \vv_n = \sum_{k=2}^n \alpha_k \vv_k.$$
那么根据线性性质,我们有(在向量表示中)
$$
D(\vv_1, \vv_2, \dots, \vv_n) = 
D\begin{pmatrix}
(\sum_{k=2}^n \alpha_k \vv_k), \vv_2, \dots, \vv_n \end{pmatrix}= \sum_{k=2}^n \alpha_k D(\vv_k, \vv_2, \dots, \vv_n)
$$
并且和中的每个行列式都为零,因为存在两个相等的列。

现在考虑一般情况,即假设系统 $\vv_1, \vv_2, \dots, \vv_n$ 是线性相关的。那么其中一个向量,比如 $\vv_k$,可以表示为其他向量的线性组合。将此向量与 $\vv_1$ 交换,我们得到我们刚刚处理过的情况,所以 $$D(\vv_1, \dots, \underset{k}{\vv_k}, \dots, \vv_n) = -D(\vv_k, \dots, \underset{k}{\vv_1}, \dots, \vv_n) = -0 = 0,$$
所以这种情况下的行列式也为零。\end{proof}

下一个命题推广了性质 (2.3)。正如我们上面已经说过的,这个性质可以从我们本节中使用的三个“基本”性质中推导出来。

{\heiti 命题 3.2}~~ 当我们向一列添加其他列的线性组合时,行列式不会改变(保持其他列不变)。特别地,行列式在“列替换”(第三类列运算)下保持不变。
\footnote{
把某列加上自己本身的某个倍数在这里是被禁止的。我们只能在其余非自身列上相加。
}

\begin{proof} 固定一个向量 $\vv_k$,令 $\uu$ 为其他向量的线性组合,$$\uu = \sum_{j \neq k} \alpha_j \vv_j.$$
那么根据线性性质
$$
D(\vv_1, \dots, \underset{k}{\underbrace{\vv_k + \uu}}, \dots, \vv_n) = D(\vv_1, \dots, \underset{k}{\vv_k}, \dots, \vv_n) + D(\vv_1, \dots, \underset{k}{\uu}, \dots, \vv_n),
$$
并且根据命题 3.1,最后一项为零。\end{proof}

\subsection{3.3. 对角和三角矩阵的行列式}

现在我们准备计算一些重要的特殊矩阵类别的行列式。第一类是所谓的\textbf{对角}(diagonal)矩阵。让我们回顾一下,一个方阵 $A = \{a_{j,k}\}^{n}_{j,k=1}$ 称为\textbf{对角}矩阵,如果\textbf{主对角线之外}的所有项都为零,即如果 $a_{j,k} = 0$ $\forall j \neq k$.~我们将经常使用 $\diag\{a_1, a_2, \dots, a_n\}$ 来表示对角矩阵:
$$
\begin{pmatrix}
a_1 & 0 & \dots & 0 \\
0 & a_2 & \dots & 0 \\
\vdots & \vdots & \ddots & \vdots \\
0 & 0 & \dots & a_n
\end{pmatrix}.
$$

由于对角矩阵 $\diag\{a_1, a_2, \dots, a_n\}$ 可以通过将第 $k$ 列乘以 $a_k$ 从单位矩阵 $I$ 得到,所以

\noindent\fbox{对角矩阵的行列式等于所有对角项的乘积,
$\det(\diag\{a_1, a_2, \dots, a_n\}) = a_1 a_2 \dots a_n.$}

下一个重要类别是所谓的\textbf{三角}(triangular)矩阵。一个方阵 $A = \{a_{j,k}\}^{n}_{j,k=1}$ 称为\textbf{上三角}\index{juzhen@矩阵!shangsanjiao@上三角}矩阵,如果主对角线以下的所有项都为零,即如果 $a_{j,k} = 0$ $\forall k < j$.~一个方阵称为\textbf{下三角}\index{juzhen@矩阵!xiansanjiao@下三角}矩阵,如果主对角线以上的所有项都为零,即如果 $a_{j,k} = 0$ $\forall j < k$.~我们称矩阵为\textbf{三角}\index{juzhen@矩阵!sanjiao@三角}\index{sanjiaojuzhen@三角矩阵}矩阵,如果它是下三角或上三角矩阵。

很容易看出,

\noindent\fbox{
三角矩阵的行列式等于所有对角项的乘积,
$\det A = a_{1,1} a_{2,2} \dots a_{n,n}.$
}

实际上,如果一个三角矩阵的主对角线上有零出现,那么它就是不可逆的(这可以通过列运算很容易地核验出来),因此两边都等于零。如果所有对角项都非零,那么使用列替换(第三类列运算)可以将矩阵转化为具有相同对角项的对角矩阵:
对于上三角矩阵,首先应该从第 $2,3,...,n$ 列减去第 1 列的适当倍数,“消去”第 1 行中的所有项,然后从第 $3,...,n$ 列减去第 2 列的适当倍数,依此类推。

为了处理下三角矩阵的情况,需要从左到右进行“列约简”,即首先从最后一列开始,将适当倍数的最后一列减去第 $n-1, \dots, 2, 1$ 列,依此类推。

\subsection{3.4. 计算行列式}

现在我们知道如何通过使用行列式的性质来计算行列式:只需进行列约简(即对 $A^T$ 进行行约简),并注意可能会改变行列式符号的列运算。幸运的是,最常使用的运算——行替换,即第三类行运算,也不会改变行列式(见下一小节)。所以我们只需要留心列的交换和用标量乘以列。

如果 $A^T$ 的阶梯形不在每一列(和每一行)都有主元,那么 $A$ 是不可逆的,因此 $\det A = 0$.~如果 $A$ 是可逆的,我们得到一个三角矩阵,而 $\det A$ 是对角项的乘积,乘以来自列交换和乘法的校正因子(correction)。

上述算法暗示 $\det A$ 仅在矩阵 $A$ 不可逆时才可能为零。结合命题 3.1 的最后一个陈述,我们得到:

{\heiti 命题 3.3}~~ $\det A = 0$ 当且仅当 $A$ 不可逆,或者等价地说:$\det A \neq 0$ 当且仅当 $A$ 可逆。

注意,虽然我们现在知道如何计算行列式,但行列式仍然没有被定义。有人可能会问:为什么我们不将其定义为通过上述算法得到的结果?问题在于,从形式上看,这个结果并非被良好定义:我们没有证明不同的列运算顺序会得到相同的结果。

\subsection{3.5. 矩阵转置和乘积的行列式~~初等矩阵的行列式}

在本节中,我们将证明两个重要定理。

\begin{theorem}[\normalfont\heiti 定理 3.4\nopunct] (矩阵转置的行列式)对于方阵 $A$,$$\det A = \det(A^T).$$\end{theorem}

这个定理意味着,我们之前讨论过的所有关于列的陈述,相应关于行的陈述也都是正确的。特别地,行列式在\textbf{行运算}下的行为与在\textbf{列运算}下的行为相同。因此,我们可以使用行运算来计算行列式。

\begin{theorem}[\normalfont\heiti 定理 3.5\nopunct] (矩阵乘积的行列式) 对于 $n \times n$ 矩阵 $A$ 和 $B$:
$$
\det(AB) = (\det A)(\det B).$$
\end{theorem}
换句话说,

\fbox{矩阵乘积的行列式等于矩阵行列式的乘积。}

为了证明这两个定理,我们需要先证以下引理。

\begin{lemma}[\normalfont\heiti 引理 3.6\nopunct] 对于方阵 $A$ 和初等矩阵 $E$(要求大小相同,即同型):
$$
\det(AE) = (\det A)(\det E)
$$\end{lemma}

\begin{proof} 证明可以通过直接检验等式两边各自的计算结果来完成:特殊矩阵的行列式很容易计算;右乘初等矩阵对应进行列运算,而列运算对行列式的作用是众所周知的。

这可能看起来像一个幸运的巧合,即初等矩阵的行列式与其相应的列运算一致,但这并非巧合。

我们知道,每个列运算对应的初等矩阵,可以由对单位矩阵 $I$ 的该列加以相应列运算得到。所以,它的行列式是 $1$($I$ 的行列式)乘以这一列运算的作用。

要论证的一切就是这样子了!读者一开始可能很难意识到,但上述段落就是对引理\textbf{完整且严谨}的证明!\end{proof}

应用引理 3.6 $N$ 次,我们得到以下推论。

{\heiti 推论 3.7}~~ 对于任何矩阵 $A$ 和任何初等矩阵序列 $E_1, E_2, \dots, E_N$(所有矩阵均为 $n \times n$):
$$
\det(A E_1 E_2 \dots E_N) = (\det A)(\det E_1)(\det E_2) \dots (\det E_N)
$$

\begin{lemma}[\normalfont\heiti 引理 3.8\nopunct] 任何可逆矩阵都可以表示为初等矩阵的乘积。\end{lemma}

\begin{proof} 我们知道任何可逆矩阵都可以通过行运算化为单位矩阵(其简化阶梯形和单位矩阵行等价)。所以 
$$I = E_N E_{N-1} \dots E_2 E_1 A,$$
因此任何可逆矩阵都可以表示为初等矩阵的乘积,
$$A = (E_N E_{N-1} \dots E_2 E_1)^{-1} I = E_1^{-1} E_2^{-1} \dots E_{N-1}^{-1} E_N^{-1}$$
(初等矩阵的逆也是初等矩阵)。\end{proof}

\begin{proof}[\normalfont\heiti 定理 3.4 的证明~\nopunct] 首先,容易验证,对于初等矩阵 $E$, $\det E = \det(E^T)$.~请注意,只需为可逆矩阵 $A$ 证明该定理即可,因为如果 $A$ 不可逆,那么 $A^T$ 也不可逆,并且它们两个的行列式都为零。
根据引理 3.8,矩阵 $A$ 可以表示为初等矩阵的乘积,
$$A = E_1 E_2 \dots E_N,$$
并且根据推论 3.7, $A$ 的行列式是初等矩阵行列式的乘积。由于取转置只是转置每个初等矩阵并反转它们的顺序,因此推论 3.7 蕴含了 $\det A = \det A^T$.~\end{proof}

\begin{proof}[\normalfont\heiti 定理 3.5 的证明~\nopunct]首先让我们假设矩阵 $B$ 是可逆的。那么引理 3.8 蕴含了 $B$ 可以表示为初等矩阵的乘积 
$$B = E_1 E_2 \dots E_N,$$
因此根据推论 3.7
$$\det(AB) = (\det A)\left[(\det E_1)(\det E_2) \dots (\det E_N)\right] = (\det A)(\det B).$$

如果 $B$ 不可逆,那么乘积 $AB$ 也不可逆,而定理仅仅说明 $0 = 0$.~

要验证上面的乘积 $AB =: C$ 是不可逆的,就让我们假设 $C$ 是可逆的。那么将恒等式 $AB = C$ 从左边乘以 $C^{-1}$,我们得到 $C^{-1} AB = I$,所以 $C^{-1} A$ 是 $B$ 的左逆。因此 $B$ 是左可逆的,并且由于它是方阵,所以 $B$ 是可逆的。我们得到了一个矛盾。\end{proof}

\subsection{3.6. 行列式的性质总结}

首先,让我们再说一遍,\textbf{行列式仅对方阵有定义!} 由于我们现在知道 $\det A = \det(A^T)$,我们之前关于列的所有陈述也对行成立。

1. 行列式在每个行(列)中是线性的,当其他行(列)固定不变时。

2. 如果我们交换矩阵 $A$ 的两行(列),行列式改变符号(正负号)。

3. 对于三角矩阵(特别地,对角矩阵),其行列式是对角项的乘积。特别地,$\det I = 1$.~

4. 如果矩阵 $A$ 有一个零行(或零列),则 $\det A = 0$.~

5. 如果矩阵 $A$ 有两行(列)相等,则 $\det A = 0$.~

6. 如果 $A$ 的某一行(列)是其他行(列)的线性组合,即如果矩阵不可逆,则 $\det A = 0$;更一般地,

7. $\det A = 0$ 当且仅当 $A$ 不可逆,或者等价地说

8. $\det A \neq 0$ 当且仅当 $A$ 可逆。

9. 如果我们将行(列)的线性组合加到某个行(列)上,行列式不改变。特别地,行列式在行(列)替换,即第三类行(列)运算下保持不变。

10. $\det A^T = \det A$.~

11. $\det(AB) = (\det A)(\det B)$.最后,

12. 如果 $A$ 是一个 $n \times n$ 矩阵,那么 $\det( \alpha A ) = \alpha^n \det A$.

最后一个性质是从行列式的线性性质质得出的,如果我们回忆起,若要将矩阵 $A$ 乘以 $\alpha$,我们必须将每一行乘以 $\alpha$,并且每次乘法都会将行列式乘以 $\alpha$.~

\begin{exer} {\heiti 练习}~~

3.1. 如果 $A$ 是一个 $n \times n$ 矩阵,$\det(3A)$ 与 $\det A$ 有何关系?\\
{\heiti 注记}:$\det(3A) = 3 \det A$ 仅在 $1 \times 1$ 矩阵的平凡情况下成立。

3.2. 下面$A$ 和 $B$ 各自的行列式之间有什么关系?
\\
a) $A = \begin{pmatrix} a_1 & a_2 & a_3 \\ b_1 & b_2 & b_3 \\ c_1 & c_2 & c_3 \end{pmatrix}$, $\quad B = \begin{pmatrix} 2a_1 & 3a_2 & 5a_3 \\ 2b_1 & 3b_2 & 5b_3 \\ 2c_1 & 3c_2 & 5c_3 \end{pmatrix}$;
\\
b) $A = \begin{pmatrix} a_1 & a_2 & a_3 \\ b_1 & b_2 & b_3 \\ c_1 & c_2 & c_3 \end{pmatrix}$, $\quad B = \begin{pmatrix} 3a_1 & 4a_2 + 5a_1 & 5a_3 \\ 3b_1 & 4b_2 + 5b_1 & 5b_3 \\ 3c_1 & 4c_2 + 5c_1 & 5c_3 \end{pmatrix}$.

3.3. 使用列或行运算计算行列式:
$$\begin{vmatrix} 0 & 1 & 2 \\ -1 & 0 & -3 \\ 2 & 3 & 0 \end{vmatrix},\quad \begin{vmatrix} 1 & 2 & 3 \\ 4 & 5 & 6 \\ 7 & 8 & 9 \end{vmatrix},\quad \begin{vmatrix} 1 & 0 & -2 & 3 \\ -3 & 1 & 1 & 2 \\ 0 & 4 & -1 & 1 \\ 2 & 3 & 0 & 1 \end{vmatrix},\quad \begin{vmatrix} 1 & x \\ 1 & y \end{vmatrix}.$$

3.4. 一个方阵($n \times n$)称为\textbf{反对称}(skew-symmetric)(或\textbf{反交换})矩阵,如果 $A^T = -A$.~证明如果 $A$ 是反对称的且 $n$ 是奇数,则 $\det A = 0$.~这对偶数 $n$ 是否成立?

3.5. 一个方阵称为\textbf{幂零}(nilpotent)矩阵,如果$\exists k \in \mathbb{N}_+$,使得  $A^k = \oo$ 成立。证明如果 $A$ 是幂零的,则 $\det A = 0$.~

3.6. 证明如果矩阵 $A$ 和 $B$ 相似,则 $\det A = \det B$.~

3.7. 一个实方阵 $Q$ 称为\textbf{正交}的,如果 $Q^T Q = I$.~证明如果 $Q$ 是正交矩阵,那么 $\det Q = \pm 1$.~

3.8. 证明 
$$\begin{vmatrix} 1 & x & x^2 \\ 1 & y & y^2 \\ 1 & z & z^2 \end{vmatrix} = (z-x)(z-y)(y-x).$$
这是所谓的范德蒙德 (Vandermonde) 行列式的特例。

3.9. 设平面 $\RR^2$ 中的点 $A, B, C$ 的坐标分别为 $(x_1, y_1), (x_2, y_2), (x_3, y_3)$.~证明三角形 $ABC$ 的面积是 
$$\frac{1}{2}  \begin{vmatrix} 1 & x_1 & y_1 \\ 1 & x_2 & y_2 \\ 1 & x_3 & y_3 \end{vmatrix} $$
的绝对值。{\heiti 提示}:使用行运算和 $2 \times 2$ 行列式的几何解释(面积)。

3.10. 设 $A$ 和 $C$ 是方阵,证明分块三角矩阵 
$$\begin{pmatrix} I & * \\ \oo & A \end{pmatrix},\quad \begin{pmatrix} A & * \\ \oo & I \end{pmatrix},\quad \begin{pmatrix} I & \oo \\ * & A \end{pmatrix},\quad \begin{pmatrix} A & \oo \\ * & I \end{pmatrix}$$ 
\\的行列式都等于 $\det A$.~这里 $*$ 可以是任何东西。以下问题说明了分块矩阵表示的力量。

3.11. 使用上一个问题证明,如果 $A$ 和 $C$ 是方阵,那么 
$$\det \begin{pmatrix} A & B \\ \oo & C \end{pmatrix} = (\det A)(\det C).$$
{\heiti 提示}:$\begin{pmatrix} A & B \\ \oo & C \end{pmatrix} = \begin{pmatrix} I & B \\ \oo & C \end{pmatrix} \begin{pmatrix} A & \oo \\ \oo & I \end{pmatrix}.$

3.12. 设 $A$ 是 $m \times n$ 矩阵,$B$ 是 $n \times m$ 矩阵。证明 
$$\det \begin{pmatrix} \oo & A \\ -B & I \end{pmatrix} = \det(AB).$$
{\heiti 提示}:虽然可以通过对矩阵进行行运算得到行列式易于计算的形式,但最简单的方法是右乘矩阵 $\begin{pmatrix} I & \oo \\ B & I \end{pmatrix}$.\end{exer}

\section{4. 行列式的正式定义~~存在性与唯一性}

在本节中,我们将得到行列式的正式定义。我们表明了,确实有函数满足第 3 节中的基本性质 1, 2, 3 的存在性,而且,这样的函数是唯一的,也就是说,在构造行列式时我们别无选择。

考虑一个 $n \times n$ 矩阵 $A = \{a_{j,k}\}^{n}_{j,k=1}$,并设 $\vv_1, \vv_2, \dots, \vv_n$ 是它的列,即
$$
\vv_k = \begin{pmatrix} a_{1,k} \\ a_{2,k} \\ \vdots \\ a_{n,k} \end{pmatrix} = a_{1,k} \ee_1 + a_{2,k} \ee_2 + \dots + a_{n,k} \ee_n = \sum_{j=1}^n a_{j,k} \ee_j
$$
使用行列式的线性性质,我们在第 1 列展开:
$$
 (4.1)\quad D(\vv_1, \vv_2, \dots, \vv_n) = D(\sum_{j=1}^n a_{j,1} \ee_j, \vv_2, \dots, \vv_n) = \sum_{j=1}^n a_{j,1} D(\ee_j, \vv_2, \dots, \vv_n) 
$$
然后我们在第 2 列展开,然后是第 3 列,依此类推。我们得到
$$
D(\vv_1, \vv_2, \dots, \vv_n) = \sum_{j_1=1}^n \sum_{j_2=1}^n \dots \sum_{j_n=1}^n a_{j_1,1} a_{j_2,2} \dots a_{j_n,n} D(\ee_{j_1}, \ee_{j_2}, \dots, \ee_{j_n})
$$
注意,我们必须为每一列使用不同的求和下标(哑指标):我们称它们为 $j_1, j_2, \dots, j_n$;这里 $j_1$ 的下标与 (4.1) 中的下标 $j$ 相同。

这是一个巨大的求和,包含 $n^n$ 项。幸运的是,其中一些项为零。也就是说,如果 $j_1, j_2, \dots, j_n$ 中有任何两个下标相同,则行列式 $D(\ee_{j_1}, \ee_{j_2}, \dots, \ee_{j_n})$ 为零,因为这里有两个相等的列。

因此,让我们重写求和,省略所有零项。最方便的方式是使用\textbf{排列}(permutation)的概念。
非正式地说,一个有序集 $\{1, 2, \dots, n\}$ 的\textbf{排列}是其元素的重新排列。一种方便表示这种重新排列的形式是通过使用一个函数
$$\sigma: \{1, 2, \dots, n\} \to \{1, 2, \dots, n\},$$
其中 $\sigma(1), \sigma(2), \dots, \sigma(n)$ 给出了集合 $1, 2, \dots, n$ 的新顺序。换句话说,排列 $\sigma$ 将有序集 $1, 2, \dots, n$ 重排为 $\sigma(1), \sigma(2), \dots, \sigma(n)$.~

这样的函数 $\sigma$ 必须是\textbf{单射}(对不同的自变量取不同的值)和\textbf{满射}(取到目标空间的所有可能值)。既是单射又是满射的函数称为\textbf{双射}(bijection),它们在定义域和目标空间之间建立了一一对应关系。
\footnote{
还有一种常用的方式来表示交换,即用一个双射 \(\sigma\),在这个表示中,\(\sigma(k)\) 给出元素编号 \(k\) 在排列中的新位置。在这个表示中,\(\sigma\) 会将 \(\sigma(1), \sigma(2), \ldots, \sigma(n)\) 排成 1,2,…,n 的顺序。

虽然在第一种表示中写出函数比较容易(如果你知道排列的重排),但第二种更适合排列的组成:它与函数的组成是相符的。具体来说,如果我们先执行对应于函数 \(\sigma\) 的重排,然后再执行对应于 \(\tau\) 的重排,得到的排列就等于 \(\tau \circ \sigma\).
}

尽管这在此处不直接相关,但让我们注意到,在组合学中,众所周知,集合 $\{1, 2, \dots, n\}$ 的不同排列的数量恰好是 $n!$.~所有 $n$ 的排列的集合将被记为 $\text{Perm}(n)$.~


使用排列的概念,我们可以将行列式重写为:
$$
D(\vv_1, \vv_2, \dots, \vv_n) = \sum_{\sigma \in \text{Perm}(n)} a_{\sigma(1),1} a_{\sigma(2),2} \dots a_{\sigma(n),n} D(\ee_{\sigma(1)}, \ee_{\sigma(2)}, \dots, \ee_{\sigma(n)})
$$
% 其中求和是遍历 $\{1, 2, \dots, n\}$ 的所有排列。
矩阵 $\ee_{\sigma(1)}, \ee_{\sigma(2)}, \dots, \ee_{\sigma(n)}$ 的列可以从单位矩阵通过有限次数的列交换得到,所以行列式 
$$D(\ee_{\sigma(1)}, \ee_{\sigma(2)}, \dots, \ee_{\sigma(n)})$$
是 $1$ 或 $-1$,取决于列交换的次数。

为了将这一点形式化,我们(非正式地)定义排列 $\sigma$ 的\textbf{符号}(记作 $\text{sign } \sigma$)为,如果将 $n$ 元组 $1, 2, \dots, n$ 重排为 $\sigma(1), \sigma(2), \dots, \sigma(n)$ 所需的交换次数是偶数,则$\text{sign}(\sigma) = 1$,如果交换次数是奇数,则 $\text{sign}(\sigma) = -1$.~

这是组合学中的一个事实,符号是良好定义的,即虽然有无数种方法可以从 $1, 2, \dots, n$ 得到 $n$ 元组 $\sigma(1), \sigma(2), \dots, \sigma(n)$,但交换次数要么总是奇数,要么总是偶数。

一种证明这一点的方法是引入另一种定义。设 $K(\sigma)$ 为 $\sigma$ 的\textbf{逆序对}(disorder)的数量,即满足 $\sigma(j) > \sigma(k)$ 的整数对 $(j, k)$ 的数量,其中 $j, k \in \{1, 2, \dots, n\}$, $j < k$,然后检查该数量是偶数还是奇数。我们将排列 $\sigma$ 称为\textbf{奇排列}如果 $K$ 是奇数,称为\textbf{偶排列}如果 $K$ 是偶数。然后定义 $\text{sign } \sigma := (-1)^{K(\sigma)}$;注意通过这种方式定义的 $\text{sign } \sigma$ 是明确的。

我们现在想要证明 $\text{sign } \sigma = (-1)^{K(\sigma)}$ 可以通过将 $n$ 元组 $1, 2, $ $\dots, n$ 重排为 $\sigma(1), \sigma(2), \dots, \sigma(n)$ 并计算交换次数来得到,如上所述。

如果 $\sigma(k) = k \ \forall k$,那么\textbf{逆序对}的数量 $K(\sigma) = 0$ ,所以这种\textbf{恒等}排列的符号是 1。还请注意,任何两个相邻元素的\textbf{置换}(仅交换两个相邻元素)会改变排列的符号,因为它会改变逆序对的数量(增加或减少 1)。因此,要从一个排列得到另一个排列,当排列具有相同的符号时,总是需要偶数次初等置换,而当符号不同时,则需要奇数次。

最后,任何两个元素的交换都可以通过奇数次初等置换来实现。这意味着当两个元素被交换时,符号会改变。因此,要从 $1, 2, \dots, n$ 得到偶排列(正的符号)总是需要偶数次交换,而得到奇排列(负的符号)需要奇数次交换。

因此,如果我们希望行列式满足第 3 节中的基本性质 1—3,我们必须将其定义为:
$$
(4.2) \quad \det A = \sum_{\sigma \in \text{Perm}(n)} a_{\sigma(1),1} a_{\sigma(2),2} \dots a_{\sigma(n),n} \text{sign}(\sigma)
$$
其中求和遍历集合 $\{1, 2, \dots, n\}$ 的所有排列。

如果我们这样定义行列式,可以很容易地验证它满足第 3 节中的基本性质 1—3。实际上,因为每个乘积项在每一列中恰好有一个因子,
并且对于任何两个相邻的列交换,我们得到的符号会改变,所以满足线性性质和反对称性。

而且,对于单位矩阵 $I$,右侧实质上只有一项(对应于恒等排列 $\sigma(k)=k \ \forall k$),(4.2)中右侧的符号是 1,所以 $D(I)=1$.~

\begin{exer} {\heiti 练习}~~

4.1. 假设排列 $\sigma$ 将 $(1, 2, 3, 4, 5)$ 映射到 $(5, 4, 1, 2, 3)$.~
\\
a) 确定 $\sigma$ 的符号;
\\
b) $\sigma^2 := \sigma \circ \sigma$ 会对 $(1, 2, 3, 4, 5)$ 做什么?
\\
c) 逆排列 $\sigma^{-1}$ 会对 $(1, 2, 3, 4, 5)$ 做什么?
\\
d) $\sigma^{-1}$ 的符号是什么?

4.2. 设 $P$ 是一个\textbf{排列矩阵}(permutation matrix),即一个仅由0和1组成的 $n \times n$ 矩阵,并且每行每列恰好有一个 1。
\\
a) 你能描述相应的线性变换吗?这将会解释它的名称的由来。
\\
b) 证明 $P$ 是可逆的。你能描述 $P^{-1}$ 吗?
\\
c) 证明存在 $N > 0$, 
$$P^N := \underset{N ~\rm times}{\underbrace{P P \dots P}}= I.$$
利用排列只有有限个的事实。

4.3. 为什么 $(1, 2, \dots, 9)$ 的排列有偶数个,并且其中恰好一半是奇排列?
\\
{\heiti 提示}:这个问题用排列来解决可能很难,但有一个非常简单的行列式解。

4.4. 如果 $\sigma$ 是一个奇排列,解释为什么 $\sigma^2$ 是偶数但 $\sigma^{-1}$ 是奇数。

4.5. 使用 (4.2) 的行列式形式计算一个 $n \times n$ 矩阵的行列式需要多少次乘法和加法?无需计数计算 $\text{sign } \sigma$ 所需的操作。\end{exer}


\section{5. 代数余子式展开}

对于 $n \times n$ 矩阵 $A = \{a_{j,k}\}^{n}_{j,k=1}$,设 $A_{j,k}$ 表示通过划掉第 $j$ 行和第 $k$ 列得到的 $(n-1) \times (n-1)$ 矩阵,称为余子式。

\begin{theorem}[\normalfont\heiti 定理 5.\nopunct] 1(行列式的代数余子式展开)(cofactor expansion)~ 设 $A$ 是一个 $n \times n$ 矩阵。对于每个 $j$, $1 \le j \le n$,行列式 $A$ 可以按第 $j$ 行展开为:
\begin{equation} \notag\begin{split}
\det A =&\ a_{j,1} (-1)^{j+1} \det A_{j,1} + a_{j,2} (-1)^{j+2} \det A_{j,2} + \dots + a_{j,n} (-1)^{j+n} \det A_{j,n} \\
=&\ \sum_{k=1}^n a_{j,k} (-1)^{j+k} \det A_{j,k}.\end{split}\end{equation}
类似地,对于每个 $k$, $1 \le k \le n$,行列式可以按第 $k$ 列展开为:
$$
\det A = \sum_{j=1}^n a_{j,k} (-1)^{j+k} \det A_{j,k}.
$$\end{theorem}

\begin{proof} 我们首先证明第 1 行的展开公式。第 2 行的展开公式可以通过交换第 1 行和第 2 行从它得到。然后交换第 2 行和第 3 行,得到第 3 行的展开公式,依此类推。

由于 $\det A = \det A^T$,列展开将自动跟上。

让我们首先考虑一个特殊情况,即第 1 行只有一个非零项 $a_{1,1}$.~通过对第 $2,3,...,n$ 列进行列运算,我们可以将 $A$ 转化为下三角形式。那么 $A$ 的行列式可以计算为:

\fbox{三角矩阵的所有对角项的乘积} $\times$ \fbox{来源于列运算的修正因子}.

但是,除了 $a_{1,1}$ 之外的所有对角项的乘积(即不包括 $a_{1,1}$)乘上修正因子恰好是 $\det A_{1,1}$,所以在这种特定情况下 $\det A = a_{1,1} \det A_{1,1}$.~

现在考虑第 1 行中除了 $a_{1,2}$ 外的其余项都为零的情况。这种情况可以通过交换第 1 列和第 2 列来化简到前面的情况,因此在这种情况下 $\det A = (-1)^{1+2} a_{1,2} \det A_{1,2}$.~

当 $a_{1,3}$ 是第 1 行唯一非零项的情况,可以通过交换第 2 行和第 3 行来简化到前面情况,所以在这种情况下 $\det A = a_{1,3} \det A_{1,3}$.~

重复这个过程,我们得到,当 $a_{1,k}$ 是第 1 行中的唯一非零项时,$\det A = (-1)^{1+k} $ $a_{1,k} \det A_{1,k}$.~
\footnote{
在 $a_{1,k}$ 是第 1 行中唯一非零项的情况下,这可能会诱使我们交换第 1 列和第 $k$ 列,将问题简化为 $a_{1,1} \neq 0$ 的情况。然而,当我们交换第 1 列和第 $k$ 列时,我们会改变其他列的顺序:如果我们划掉第 $k$ 列,那么第 1 列将是剩余列中的第 1 列。但是,如果我们交换第 1 列和第 $k$ 列,然后划掉第 $k$ 列(它现在是第 1 列),那么第 1 列现在将是第 $k-1$ 列。为了避免跟踪复杂的列交换,我们可以像上面那样,交换第 $k$ 列和第 $k-1$ 列,将一切简化为我们在上一步处理的情况。这样的操作不会改变其余列的顺序。
}

在一般情况下,行列式在每一行上的线性意味着
$$\det A = \det A^{(1)} + \det A^{(2)} + \dots + \det A^{(n)} = \sum_{k=1}^n \det A^{(k)},$$
其中矩阵 $A^{(k)}$ 是通过将 $A$ 的第 1 行中除 $a_{1,k}$ 之外的所有项替换为 0 而得到的。正如我们上面所讨论的,
$$\det A^{(k)} = (-1)^{1+k} a_{1,k} \det A_{1,k},$$
所以 
$$\det A = \sum_{k=1}^n (-1)^{1+k} a_{1,k} \det A_{1,k}.$$

为了得到第 2 行的展开式,我们可以交换第 1 行和第 2 行,然后应用上面的公式。行交换改变了符号,所以我们得到 $$\det A = -\sum_{k=1}^n (-1)^{1+k} a_{2,k} \det A_{2,k} = \sum_{k=1}^n (-1)^{2+k} a_{2,k} \det A_{2,k}.$$
通过交换第 3 行和第 2 行并按第 2 行展开,我们得到公式 
$$\det A = \sum_{k=1}^n (-1)^{3+k} a_{3,k} \det A_{3,k},$$
依此类推。

要将行列式 $\det A$ 按列展开,只需对 $A^T$ 应用行展开公式即可。\end{proof}

{\heiti 定义}~~ 这些数
$$C_{j,k} = (-1)^{j+k} \det A_{j,k}$$
称为 $A$ 的\textbf{代数余子式}(cofactor)。

使用这个符号,在第 $j$ 行展开行列式的公式可以重写为 
$$\det A = a_{j,1} C_{j,1} + a_{j,2} C_{j,2} + \dots + a_{j,n} C_{j,n} = \sum_{k=1}^n a_{j,k} C_{j,k}.$$
类似地,在第 $k$ 列展开可以写成 
$$\det A = a_{1,k} C_{1,k} + a_{2,k} C_{2,k} + \dots + a_{n,k} C_{n,k} = \sum_{j=1}^n a_{j,k} C_{j,k}.$$

{\heiti 注记}~~ 代数余子式展开公式经常被用作行列式的定义。不难证明由该公式给出的数满足行列式的基本性质:归一化性质是显然的,反对称性的证明也很容易。然而,线性性质的证明虽然不难,但有点繁琐。

{\heiti 注记}~~ 虽然它看起来非常不错,但代数余子式展开公式不适用于直接计算大于 $3 \times 3$ 的一般矩阵的行列式。

可以计算,它需要超过 $n!$ 次乘法(准确地说,需要 $\sum_{k=2}^n n!/k!$ 次乘法),而 $n!$ 的增长非常快。例如,计算一个 $20 \times 20$ 矩阵的代数余子式展开需要超过 $20! \approx 2.4 \times 10^{18}$ 次乘法。一台每秒执行十亿次乘法的计算机需要 77 年才能执行 $20!$ 次乘法;事实上,计算一个 $20 \times 20$ 矩阵的代数余子式展开所需总运算将需要 132 多年的时间来完成。
\footnote{
读者可以自行验证这一数字,比如使用WolframAlpha软件。
}

另一方面,使用行约简计算 $n \times n$ 矩阵的行列式需要 $(n^3 + 2n - 3) / 3$ 次乘法(以及大约相同数量的加法)。对于一台每秒执行一百万次运算(按当前标准非常慢)的计算机来说,计算 $100 \times 100$ 矩阵的行列式只需要一秒钟时间里的一小部分。

只有当某一行(或列)包含很多零项时,代数余子式展开公式才会显得实用。

然而,代数余子式展开公式具有重要的理论价值,正如下一节所示。

\subsection{5.1. 逆矩阵的代数余子式公式}

由代数余子式 $C_{j,k} = (-1)^{j+k} \det A_{j,k}$ 组成的矩阵 $C = \{C_{j,k}\}^{n}_{j,k=1}$ 称为 $A$ 的\textbf{代数余子式矩阵}(cofactor matrix)。

\begin{theorem}[\normalfont\heiti 定理 5.2\nopunct]  设 $A$ 是一个可逆矩阵,并设 $C$ 是它的代数余子式矩阵。
\footnote{
译者注:在国内,一般记$A^*$为书中的$C^T$,读作“$A$的伴随”。但本书的$A^*$符号已经名花有主:它在后面的章节(始见于第五章第5节)中用于表示埃尔米特伴随(Hermitian adjoint).
}
那么
$$
A^{-1} = \frac{1}{\det A} C^T
$$\end{theorem}

\begin{proof} 让我们计算乘积 $AC^T$.~第 $j$ 个对角项是通过将 $A$ 的第 $j$ 行与 $C$ 的第 $j$ 列(即 $C^T$ 的第 $j$ 行)相乘得到的,所以 根据代数余子式展开公式,
$$(AC^T)_{j,j} = a_{j,1} C_{j,1} + a_{j,2} C_{j,2} + \dots + a_{j,n} C_{j,n} = \det A,$$

为了得到非对角项,我们需要将 $A$ 的第 $k$ 行与 $C^T$ 的第 $j$ 列相乘,$j \neq k$,得到 
$$a_{k,1} C_{j,1} + a_{k,2} C_{j,2} + \dots + a_{k,n} C_{j,n}.$$
根据代数余子式展开公式(在第 $j$ 行展开),这是将 $A$ 中第 $j$ 行替换为第 $k$ 行(而所有其他行保持不变)得到的矩阵的行列式。但是,这个矩阵的第 $j$ 行和第 $k$ 行是相同的,所以行列式为 0。因此,$AC^T$ 的所有非对角项都为零(而所有对角项都等于 $\det A$),所以 
$$AC^T = (\det A) I.$$
这意味着矩阵 $\frac{1}{\det A} C^T$ 是 $A$ 的右逆,由于 $A$ 是方阵,所以它是逆。\end{proof}

回忆一下,对于可逆矩阵 $A$,方程 $A \xx = \bb$ 的解是 
$$\xx = A^{-1} \bb = \frac{1}{\det A} C^T \bb,$$
我们得到以下定理的推论。

{\heiti 推论 5.3(Cramer 法则)}~~ 对于可逆矩阵 $A$,方程 $A \xx = \bb$ 的解的第 $k$ 个项由以下公式给出:
$$
x_k = \frac{\det B_k}{\det A},
$$
其中矩阵 $B_k$ 是通过将 $A$ 的第 $k$ 列替换为向量 $\bb$ 而得到的。

\subsection{5.2. 逆矩阵的代数余子式公式的应用}

{\heiti 例子(求 $2 \times 2$ 矩阵的逆)}~~ 代数余子式公式在求 $2 \times 2$ 矩阵 
$$A = \begin{pmatrix} a & b \\ c & d \end{pmatrix}$$
的逆时,确实非常有用。代数余子式仅仅是$A$中的项($1 \times 1$ 矩阵),代数余子式矩阵是 $$\begin{pmatrix} d & -c \\ -b & a \end{pmatrix},$$
所以逆矩阵 $A^{-1}$ 由公式给出:
$$
A^{-1} = \frac{1}{\det A} \begin{pmatrix} d & -b \\ -c & a \end{pmatrix}.
$$

虽然对于维数大于 3 的情况,逆矩阵的代数余子式公式看起来并不实用,但它的确具有重要的理论价值,正如下面的例子所示。

{\heiti 例子(整数逆矩阵)}~~ 假设我们想构造一个具有整数项的矩阵 $A$,使得其逆也具有整数项(对这样的矩阵求逆可以出成一个很好的家庭作业:你无需处理分数)。如果 $\det A = 1$ 且其项是整数,那么逆矩阵的代数余子式公式表明了 $A^{-1}$ 也具有整数项。

注意,构造一个 $\det A = 1$ 的整数矩阵很容易:可以从主对角线上为 1 的三角矩阵开始,然后应用几次行或列替换(第三类运算)来使矩阵看起来是一般的。

{\heiti 例子(多项式矩阵的逆)}~~ 另一个例子是考虑一个\textbf{多项式矩阵}\index{duoxiangshijuzhen@多项式矩阵}(polynomial matrix) $A(x)$,即其项不是数字而是变量 $x$ 的多项式 $a_{j,k}(x)$.~如果 $\det A(x) \equiv 1$,那么逆矩阵 $A^{-1}(x)$ 也是一个多项式矩阵。

如果 $\det A(x) = p(x) \neq 0$,则从代数余子式展开可知,$p(x)$ 是一个多项式,因此 $A^{-1}(x)$ 具有有理数项:更重要的是,$p(x)$ 是每个分母的倍数。

\begin{exer} {\heiti 练习}~~

5.1. 你可以使用任何方法计算行列式:
$$\begin{vmatrix} 0 & 1 & 1 \\ 1 & 2 & -5 \\ 6 & 4 & -3 \end{vmatrix},\quad \begin{vmatrix} 1 & -2 & 3 &-12 \\ -5 & 12 & -14 & 19 \\ -9 & 22 & -20 & 31 \\ -4 & 9 & -14 & 15 \end{vmatrix}.$$

5.2. 使用行(列)展开计算以下行列式。注意,你没有必要从第 1 行(列)开始展开:选择具有更多零的行(列)将简化你的计算。
$$\begin{vmatrix} 1 & 2 & 0 \\ 1 & 1 & 5 \\ 1 & -3 & 0 \end{vmatrix},\quad\begin{vmatrix} 4 & -6 & -4 & 4 \\ 2 & 1 & 0 & 0 \\ 0 & -3 & 1 & 3 \\ -2 & 2 & -3 & -5 \end{vmatrix}.$$

5.3. 对于$n \times n$矩阵 
$$A = \begin{pmatrix} 0 & 0 & 0 & \dots & 0 & a_0 \\ -1 & 0 & 0 & \dots & 0 & a_1 \\ 0 & -1 & 0 & \dots & 0 & a_2 \\ \vdots & \vdots & \vdots & \ddots & \vdots & \vdots \\ 0 & 0 & 0 & \dots & 0 & a_{n-2} \\ 0 & 0 & 0 & \dots & -1 & a_{n-1} \end{pmatrix},$$
计算 $\det(A + tI)$,其中 $I$ 是 $n \times n$ 单位矩阵。行展开和归纳可能是最好的方法。这时你将得到一个涉及 $a_0, a_1, \dots, a_{n-1}$ 和 $t$ 的漂亮表达式。

5.4. 使用代数余子式公式计算下列矩阵的逆。$$\begin{pmatrix} 1 & 2 \\ 3 & 4 \end{pmatrix}, \quad \begin{pmatrix} 19 & -17 \\ 3 & -2 \end{pmatrix}, \quad \begin{pmatrix} 1 & 0 \\ 3 & 5 \end{pmatrix}, \quad \begin{pmatrix} 1 & 1 & 0 \\ 2 & 1 & 2 \\ 0 & 1 & 1 \end{pmatrix}$$

5.5. 设 $D_n$ 是 $n \times n$ 三对角矩阵
$$
\begin{pmatrix}
1 & -1 &  & \dots &  &  \\
1 & 1 & -1 & \dots &  &  \\
 & 1 & 1 & \dots &  &  \\
\vdots & \vdots & \ddots & \ddots & \vdots & \vdots \\
 &  &  & \dots & 1 & -1 \\
 &  &  & \dots & 1 & 1
\end{pmatrix}
$$
的行列式。使用代数余子式展开证明 $D_n = D_{n-1} + D_{n-2}$.~这表明数列 $D_n$ 是斐波那契数列 $1, 2, 3, 5, 8, 13, 21, \dots$.~

5.6. 重新回顾范德蒙德行列式。我们的目标是证明 $(n+1) \times (n+1)$ 范德蒙德行列式的公式:
$$
\begin{vmatrix}
1 & c_0 & c_0^2 & \dots & c_0^n \\
1 & c_1 & c_1^2 & \dots & c_1^n \\
\vdots & \vdots & \vdots & \ddots & \vdots \\
1 & c_n & c_n^2 & \dots & c_n^n
\end{vmatrix} = \prod_{0 \le j < k \le n} (c_k - c_j).
$$
我们将使用归纳法。为此:
\\
a) 验证公式对 $n=1, n=2$ 成立。
\\
b) 将最后一行中的变量 $c_n$ 看作 $x$,并证明行列式是一个 $n$ 次多项式 $A_0 + A_1 x + A_2 x^2 + \dots + A_n x^n$,其中系数 $A_k$ 由 $c_0, c_1, \dots, c_{n-1}$决定。
\\
c) 证明该多项式在 $x = c_0, c_1, \dots, c_{n-1}$ 处均有零点,因此可以表示为 $A_n \cdot (x - c_0)(x - c_1) \dots (x - c_{n-1})$,其中 $A_n$ 如b)中所述。

d) 假设范德蒙德行列式的公式对 $n-1$ 成立,计算 $A_n$ 并证明对 $n$ 的公式。

5.7. 使用代数余子式展开来计算 $n \times n$ 矩阵的行列式需要多少次乘法?证明这个公式。\end{exer}


\section{6. 子式与秩}

对于矩阵 $A$,让我们考虑它的 $k \times k$ \textbf{子矩阵}\index{zijuzhen@子矩阵}(submatrix),它通过选取原矩阵中的 $k$ 行和 $k$ 列得到。该矩阵的行列式称为 $k$ 阶\textbf{子式}\index{zishi@子式}(minor)。注意,一个 $m \times n$ 矩阵有 $\binom{m}{k} \cdot \binom{n}{k}$ 个不同的 $k \times k$ 子矩阵,因此它就有这么多个 $k$ 阶子式。

\begin{theorem}[\normalfont\heiti 定理 6.1\nopunct] 对于一个非零矩阵 $A$,它的秩等于能使 $k$ 阶非零子式存在的最大整数 $k$.~\end{theorem}

\begin{proof} 首先,让我们证明,如果 $k > \rank A$,则所有 $k$ 阶子式都为零。实际上,由于 $A$ 的列空间 $\Ran A$ 的维数是 $\rank A < k$,因此 $A$ 中任何含个数为 $k$ 的列的系统都是线性相关的。因此,对于 $A$ 的任何 $k \times k$ 子矩阵,它的列都是线性相关的,所以所有 $k$ 阶子式都为零。

为了完成证明,我们需要证明存在一个非零的 $k$ 阶子式,其中 $k = \rank A$.~可能存在许多这样的子式,但也许最简单的方法是取主元行和主元列(即原始矩阵中包含主元的行和列)。这个 $k \times k$ 子矩阵具有与原始矩阵相同的主元,因此它是可逆的(每一列和每一行都有主元),并且其行列式非零。\end{proof}

这个定理看起来不是很有用,因为进行行约简比计算所有子式要容易得多。然而,它同样具有重要的理论价值,正如以下推论所示。

{\heiti 推论 6.2}~~ 设 $A=A(x)$ 是一个 $m \times n$ 多项式矩阵(即其项是变量 $x$ 的多项式)。那么 $\rank A(x)$ 在除了有限个可能的点之外的地方是恒定的,而在这些点上秩会变小。

\begin{proof} 设 $r$ 是满足至少存在一个 $x$ 使得 $\rank A(x) = r$ 成立的最大整数。为了证明这样的 $r$ 存在,我们首先尝试 $r = \min\{m, n\}$.~如果确实存在一个 $x$ 使得 $\rank A(x) = r$,我们就找到了 $r$.~如果不是,我们则将 $r$ 替换为 $r-1$ 并重试。经过有限步操作,我们要么停止,要么得到 $0$.~因此,$r$ 是存在的。

设 $x_0$ 是使得 $\rank A(x_0) = r$成立的一个点,并且设 $M$ 是一个 $k$ 阶子式,使得 $M(x_0) \neq 0$.~由于 $M(x)$ 是一个 $k \times k$ 多项式矩阵的行列式,所以可以将 $M(x)$ 看作是一个多项式。由于 $M(x_0) \neq 0$,它不是恒零的,因此它只能在有限个点处为零。所以,除了可能有限个点之外,$\rank A(x) \geq r$.~但是根据 $r$ 的定义,对于所有的 $x$,$\rank A(x) \leq r$.~\end{proof}


\section{7. 第三章复习题}

\begin{exer}
7.1. 判断正误:
\\
a) 行列式只对方阵有定义。
\\
b) 如果 $A$ 的两行或两列相同,则 $\det A = 0$.~
\\
c) 如果 $B$ 是通过交换 $A$ 的两行(或两列)得到的矩阵,则 $\det B = \det A$.~
\\
d) 如果 $B$ 是通过将 $A$ 的某一行(列)乘以一个标量 $\alpha$ 得到的矩阵,则 $\det B = \det A$.~
\\
e) 如果 $B$ 是通过将 $A$ 的某一行乘以一个数加到另一行得到的矩阵,则 $\det B = \det A$.~
\\
f) 三角矩阵的行列式是其对角线元素的乘积。
\\
g) $\det(A^T) = -\det(A)$.~
\\
h) $\det(AB) = \det(A)\det(B)$.~
\\
i) 矩阵 $A$ 可逆当且仅当 $\det A \neq 0$.~
\\
j) 如果 $A$ 是可逆矩阵,则 $\det(A^{-1}) = 1/\det(A)$.~

7.2. 设 $A$ 是一个 $n \times n$ 矩阵。$\det(3A)$, $\det(-A)$ 和 $\det(A^2)$ 与 $\det A$ 的关系是什么?

7.3. 如果 $A$ 和 $A^{-1}$ 的所有元素都是整数,那么 $\det A = 3$ 是否可能?
\\
{\heiti 提示:} $\det(A)\det(A^{-1})$ 是什么?

7.4. 设 $\vv_1, \vv_2$ 是 $\RR^2$ 中的向量,然后设 $A$ 是以 $\vv_1, \vv_2$ 为列的 $2 \times 2$ 矩阵。证明 $|\det A|$ 是由向量 $\vv_1, \vv_2$ 作为两邻边确定的平行四边形的面积。
\\
首先考虑 $\vv_1 = (x_1, 0)^T$ 的情况。对于一般情况 $\vv_1 = (x_1, y_1)^T$,左乘一个旋转矩阵,将向量 $\vv_1$ 变换为 $(\widetilde{x}_1, 0)^T$ 来处理。\\
{\heiti 提示:} 旋转矩阵的行列式是什么?
\\
以下问题说明了行列式的符号与向量组的“定向”(orientation)之间的关系。

7.5. 设 $\vv_1, \vv_2$ 是 $\RR^2$ 中的向量。证明 $D(\vv_1, \vv_2) > 0$ 当且仅当存在一个旋转矩阵 $T_\alpha$ 使得向量 $T_\alpha \vv_1$ 与 $\ee_1$ 平行(并且方向相同),且 $T_\alpha \vv_2$ 位于上半平面 ( $\ee_2$ 所在的半平面),即分量$x_2 > 0$.~\\
{\heiti 提示:} 同样地,旋转矩阵的行列式是什么?

\end{exer}





