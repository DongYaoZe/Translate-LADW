

\chapter{第八章~~对偶空间与张量}

本章中的所有向量空间都是有限维的。

\section{1. 对偶空间}

\subsection{1.1. 线性泛函与对偶空间~~对偶空间中的坐标变换}

\textbf{定义 1.1}~~
向量空间 $V$(域为 $\FF$)上的\textbf{线性泛函}(linear functional)是一个线性变换 $L : V \to \FF$.~

这类特殊的线性变换足够重要,值得一个单独的名称。

如果我们把向量看作是某种物理对象,比如力和速度,那么线性泛函可以被看作是一个(线性的)测量,它给你一个标量作为结果:可以想象一下给定方向上的力和速度。

\textbf{定义 1.2}~~

有限维\footnote{
我们这里只考虑有限维空间,因为对于无限维空间,对偶空间并不完全由所谓的\textbf{有界}(bounded)线性泛函组成。在不给出精确定义的情况下,我们只提一句:在有限维情况下(域和目标空间都是有限维的),所有线性变换都是有界的,因此我们不需要提及“有界”这个词。
}向量空间 $V$ 上所有线性泛函的集合被称为 $V$ 的\textbf{对偶空间}(dual space),通常记作 $V'$ 或 $V^*$.~

正如我们在第一章第四节中讨论过的,从 $V$ 到 $W$ 的所有线性变换的集合 $\LL(V, W)$(具有自然定义的加法和标量乘法)是一个向量空间。 因此,对偶空间 $V' = \LL(V, F)$ 是一个向量空间。

让我们来看一个例子。设空间 $V$ 是 $\RR^n$,那么它的对偶是什么?我们知道,线性变换 $T : \RR^n \to \RR^m$ 由一个 $m \times n$ 矩阵表示,所以 $\RR^n$ 上的线性泛函(即线性变换 $L : \RR^n \to \RR$)由一个 $1 \times n$ 矩阵(行向量)给出,我们记之为 $[L]$.~所有这些行向量的集合与 $\RR^n$ 同构(同构是通过取转置 $[L] \to [L]^T$ 给出的)。

因此,$\RR^n$ 的对偶就是 $\RR^n$ 本身。对于复数空间 $\CC^n$ 也是如此,当然,对于任意域$\FF$上的 $\FF^n$ 也是如此。由于域 $\FF$(这里我们主要关心 $\FF = \RR$ 或 $\FF = \CC$ 的情况)上 $n$ 维空间 $V$ 与 $\FF^n$ 同构,而 $\FF^n$ 的对偶与 $\FF^n$ 同构,我们可以得出对偶空间 $V'$ 与 $V$ 同构。

因此,对偶空间的定义开始显得有些“愚蠢”,因为它似乎没有提供任何新的东西。

然而,事实并非如此!如果我们仔细观察,就会发现对偶空间确实是一个新对象。为了说明这一点,让我们分析一下当我们在 $V$ 中改变基时,矩阵 $[L]$ 的项(我们称之为 $L$ 的坐标)是如何变化的。

\subsubsection{1.1.1. 坐标变换公式}
设 
$$\A = \{\aaa_1, \aaa_2, \dots, \aaa_n\},\quad \B = \{\bb_1, \bb_2, \dots, \bb_n\}$$ 
是 $V$ 中的两个基,设 $[L]_\A = [L]_{\SSS,\A}$ 和 $[L]_\B = [L]_{\SSS,\B}$ 分别是 $L$ 在基 $\A$ 和 $\B$ 下的矩阵(我们假设标量目标空间中的基总是标准基,所以我们可以在记号中省略下标 $\SSS$)。然后,回忆第二章 8.4 节中的坐标变换规则,我们得到 
$$[L]_\B = [L]_\A [I]_{\A,\B}.$$
回忆一下,对于向量 $\vv \in V$,其在不同基下的坐标由公式 
$$[\vv]_\B = [I]_{\B,\A} [\vv]_\A$$ 
相关联,并且 
$$[I]_{\A,\B} = [I]_{\B,\A}^{-1}.$$

如果我们令 $S := [I]_{\B,\A}$,那么 $[\vv]_\B = S [\vv]_\A$.~那么 $[L]^T_\B$ 和 $[L]^T_\A$ 的项由公式  
$$(1.1)\quad [L]^T_\B = (S^{-1})^T [L]^T_\A $$
相关联。
(由于我们通常将向量表示为其坐标的列向量,所以我们使用 $[L]^T_\A$ 和 $[L]^T_\B$ 而不是 $[L]_\A$ 和 $[L]_\B$)。

用文字来说,

\fbox{\begin{minipage}{0.9\textwidth}
如果 $S$ 是 $X$ 中的坐标变换矩阵(从旧坐标到新坐标),那么对偶空间 $X'$ 中的坐标变换矩阵是 $(S^{-1})^T$.
\end{minipage}}

因此,对偶空间 $V'$ 虽然与 $V$ 同构,但实际上是一个不同的对象:区别在于当改变 $V$ 中的基时,$V$ 和 $V'$ 中的坐标如何变化。

\textbf{注记}~~有人可能会问:为什么我们不能在 $X$ 中选择一个基,而在对偶空间 $X'$ 中选择一个完全不相关的基呢?当然,我们可以这样做,但试想一下,如果我们知道 $\xx$ 在某个基下的坐标,以及 $L$ 在某个完全不相关的基下的坐标,该如何计算 $L(\xx)$?

因此,如果我们想(知道向量 $\xx$ 在某个基下的坐标)用矩阵代数的标准规则来计算线性泛函 $L$ 的作用,即用行(泛函)乘以列(向量),那么我们别无选择:线性泛函 $L$ 的“坐标”应该是它在(相同基下的)矩阵的项。正如我们稍后会在下面第 1.3 节看到的,线性泛函的项(“坐标”)确实是某些基(所谓的\textbf{对偶基})下的坐标。

\subsubsection{1.1.2. 唯一性定理}

\textbf{引理 1.3} 设 $\vv \in V$.~如果对所有 $L \in V'$ 都有 $L(\vv) = 0$,那么 $\vv = 0$.~
其推论是,如果对所有 $L \in V'$ 都有 $L(\vv_1) = L(\vv_2)$,那么 $\vv_1 = \vv_2$.

\textbf{证明}~~
固定 $V$ 中的一个基 $\B$.~则 $$L(\vv) = [L]_\B [\vv]_\B.$$
通过选取不同的矩阵(即不同的 $L$),我们可以轻易看出 $[\vv]_\B = \oo$.~
的确,如果 
$$L_k = [0, \dots, 0, \underset{k}{1}, 0, \dots, 0],$$
那么等式 $$L_k [\vv]_\B = 0$$ 暗示 $[\vv]_\B$ 的第 $k$ 个坐标为 0.

对所有 $k$ 使用这个等式,我们得出 $[\vv]_\B = \oo$,所以 $\vv = \oo$.~

\subsection{1.2. 二次对偶空间}
正如我们上面讨论的,对偶空间 $V'$ 是一个向量空间,因此我们可以考虑它的对偶 $V'' = (V')'$.~看起来我们可以考虑 $V''$ 的对偶 $V'''$ ……以此类推。然而,有趣的讨论在 $V''$ 处停止,因为

\fbox{\begin{minipage}{0.9\textwidth}
二次对偶空间 $V''$ \textbf{在概念上}(即以一种自然的方式)同构于 $V$.~
\end{minipage}}

让我们解读一下这个陈述。任何向量 $\vv \in V$ 在概念上定义了 $V'$ 上的一个线性泛函 $L_\vv$(即二次对偶空间 $V''$ 的一个元素),其规则是 $$L_\vv(f) = f(\vv)\quad \forall f \in V'.$$
可以很容易地验证映射 $T: V \to V''$, $T\vv = L_\vv$ 是一个线性变换。

注意,$\text{Ker } T = \{\oo\}$.~的确,如果 $T\vv = \oo$,那么 $$f(\vv) = 0 \quad \forall f \in V',$$
根据上面的引理 1.3,我们得到 $\vv = \oo$.~

由于 $\dim V'' = \dim V' = \dim V$,条件 $\text{Ker } T = \{\oo\}$ 暗示 $T$ 是一个可逆变换(同构)。

这个同构 $T$ 是非常自然的(至少对数学家而言)。特别是,它没有使用基来定义,因此它不依赖于基的选择。
所以,非正式地说,我们说 $V''$ 在概念上同构于 $V$:更严谨的陈述是,上面描述的映射 $T$(我们认为它是自然和概念上的)是从 $V$ 到 $V''$ 的一个同构。


\subsection{1.3. 对偶基,又称双正交基}

在前面的章节中,我们多次提到线性泛函的矩阵项为“坐标”。但这里的“坐标”通常是指某个基下的坐标。线性泛函的“坐标”真的是某个基下的坐标吗?事实证明答案是“是”,所以术语保持一致。让我们找出与 $L \in V'$ 的坐标对应的基。

设 $\{\bb_1, \bb_2, \dots, \bb_n\}$ 是 $V$ 中的一个基。对于 $L \in V'$,设 $[L]_\B = [L_1, L_2, \dots, L_n]$ 是其在基 $\B$ 下的矩阵(行向量)。考虑线性泛函 $\bb'_1, \bb'_2, \dots, \bb'_n \in V'$,它们由
$$(1.2)\quad \bb'_k(\bb_j) = \delta_{k,j} $$
定义。其中 $\delta_{k,j}$ 是克罗内克符号:
$$\delta_{k,j} = \begin{cases} 1, & j=k \\ 0, & j \neq k \end{cases}.$$
回忆一下,一个线性变换由其在基上的作用定义,因此泛函 $\bb'_k$ 是明确定义的。

正如人们可以很容易地看到的那样,泛函 $L$ 可以表示为 
$$L = \sum_{k=1}^n L_k \bb'_k.$$
确实,取任意 $\vv = \sum_{k=1}^n \alpha_k \bb_k \in V$,其在基 $\B$ 下的坐标为 $[\vv]_\B = [\alpha_1, \alpha_2, \dots, \alpha_n]^T$.~根据线性和 $\bb'_k$ 的定义:
$$\bb'_k(\vv) = \bb'_k \left( \sum_{j=1}^n \alpha_j \bb_j \right) = \sum_{j=1}^n \alpha_j \bb'_k(\bb_j) = \alpha_k.$$
因此,
$$L(\vv) = [L]_\B [\vv]_\B = \sum_{k=1}^n L_k \alpha_k = \sum_{k=1}^n L_k \bb'_k(\vv).$$
由于这个恒等式对所有 $\vv \in V$ 都成立,我们得出 $L = \sum_{k=1}^n L_k \bb'_k$.~

因为我们没有对 $L \in V'$ 做任何假设,我们刚才已经证明了任何线性泛函 $L$ 都可以表示为 $\bb'_1, \bb'_2, \dots, \bb'_n$ 的线性组合,所以系统 $\{\bb'_k\}_{k=1}^n$ 是生成集。

现在我们证明这个系统是线性无关的(因此它是一个基)。设 $\oo = \sum_{k=1}^n L_k \bb'_k$.~那么对于任意 $j = 1, 2, \dots, n$,
$$0 = \oo  \bb_j = \left( \sum_{k=1}^n L_k \bb'_k \right) (\bb_j) = \sum_{k=1}^n L_k \bb'_k(\bb_j) = L_j,$$
所以 $L_j = 0$.~因此,所有的 $L_k$ 都为 0,并且该系统是线性无关的。

所以,系统 $\{\bb'_1, \bb'_2, \dots, \bb'_n\}$ 确实是 $V'$ 中的一个基,并且 $[L]_\B$ 的项是 $L$ 相对于基 $\B$ 的坐标。

\textbf{定义 1.4}~~

设 $\{\bb_1, \bb_2, \dots, \bb_n\}$ 是 $V$ 中的一个基。由方程 (1.2) 唯一确定的向量组 
$$\{\bb'_1, \bb'_2, \dots, \bb'_n\} \subset V'$$
被称为与 $\{\bb_1, \bb_2, \dots, \bb_n\}$ \textbf{对偶的(或双正交的)基}。

注意,我们已经证明了基的对偶系统也是一个基。还请注意,如果 $\{\bb'_1, \bb'_2, \dots, \bb'_n\}$ 是基 $\{\bb_1, \bb_2, \dots, \bb_n\}$ 的对偶系统,那么 $\{\bb_1, \bb_2, \dots, \bb_n\}$ 也是基 $\{\bb'_1, \bb'_2, \dots, \bb'_n\}$ 的对偶系统。

\subsubsection{1.3.1. 抽象非正交傅里叶展开}

对偶系统可用于计算基 $\{\bb_1, \bb_2, \dots, \bb_n\}$ 下向量的坐标。

设 $\{\bb'_1, \bb'_2, \dots, \bb'_n\}$ 是 $\{\bb_1, \bb_2, \dots, \bb_n\}$ 的双正交系统,设 $\vv = \sum_{k=1}^n \alpha_k \bb_k$.~那么,正如之前所示:
$$ \bb'_j(\vv) = \bb_j \left( \sum_{k=1}^n \alpha_k \bb_k \right) = \sum_{k=1}^n \alpha_k \bb_j(\bb_k) = \alpha_j \bb'_j(\bb_j) = \alpha_j, $$
所以 $\alpha_k = \bb'_k(\vv)$.~那么我们可以写成:
$$(1.3)\quad \vv = \sum_{k=1}^n \bb'_k(\vv) \bb_k .$$

换句话说,


% ³ 我们可以简单地认为,对于 $\vv \in V$,其坐标是 $\vv$ 在某个基下的坐标。当引入对偶基时,我们对偶地引入了对偶空间的坐标。
% \noindent
\fbox{\begin{minipage}{0.9\textwidth}
第 $k$ 个坐标(在基 $\B = \{\bb_1, \bb_2, \dots, \bb_n\}$ 下)是 $\bb'_k(\vv)$,其中 $\B' = \{\bb'_1, \bb'_2, \dots, \bb'_n\}$ 是对偶基。
\end{minipage}}

这个公式被称为 $\vv$ 的(一个简化的)\textbf{抽象非正交傅里叶展开}(abstract non-orthogonal Fourier decomposition)(在基 $\bb_1, \bb_2, \dots, \bb_n$ 下)。之所以这样命名,稍后在 2.3 节中会清楚。

\textbf{注记 1.5}~~
设 $\A = \{\aaa_1, \aaa_2, \dots, \aaa_n\}$ 和 $\B = \{\bb_1, \bb_2, \dots, \bb_m\}$ 分别是 $X$ 和 $Y$ 中的基,设 $\B' = \{\bb'_1, \bb'_2, \dots, \bb'_m\}$ 是 $\B$ 的对偶基。那么变换 $T$ 在基 $\A, \B$ 下的矩阵 $[T]_{\B,\A} =: A = \{a_{k,j}\}_{k=1}^{m} ~_{ j=1}^{n}$ 由下式给出:
$$ a_{k,j} = \bb'_k(T \aaa_j), \quad j = 1, 2, \dots, n, \quad k = 1, 2, \dots, m. $$


\subsection{1.4. 对偶系统的例子}
我们考虑的第一个例子是平凡的。设 $V$ 为 $\RR^n$(或 $\CC^n$),设 $\ee_1, \ee_2, \dots, \ee_n$ 是那里的标准基。对偶空间将是 $n$ 维行向量的空间,它与 $\RR^n$(或复数情况下的 $\CC^n$)同构,那里的标准基是对偶于 $\ee_1, \ee_2, \dots, \ee_n$ 的。$( \RR^n )'$(或 $(\CC^n)'$)中的标准基是从 $\ee_1, \ee_2, \dots, \ee_n$ 通过转置得到的 $\ee^T_1, \ee^T_2, \dots, \ee^T_n$.~

\subsubsection{1.4.1. 泰勒公式}
下一个例子更有趣。让我们考虑次数最多为 $n$ 的多项式空间 $\PP_n$.~我们知道,幂 $\{\ee_k\}_{k=0}^n, \ee(t)=t^n$ 构成了该空间中的标准基。这个基的对偶是什么?

这个答案可能很难猜测,但一旦你知道了,验证起来就非常容易。
也就是说,考虑线性泛函 $\ee'_k \in (\PP_n)',\quad k = 0, 1, \dots, n$,它们对多项式 $p$ 的作用如下:
$$ \ee'_k(p) := \frac{1}{k!} \frac{\dif ^k}{\dif t^k} p(t) \Big|_{t=0} = \frac{1}{k!} p^{(k)}(0); $$
这里我们使用常规约定 $0! = 1$ 和 $\dif ^0 f / \dif t^0 = f$.~

由于
$$ \frac{\dif ^k}{\dif t^k} t^j = \begin{cases} j(j-1)\dots(j-k+1)t^{j-k}, & k \le j \\ 0, & k > j \end{cases} $$
我们可以很容易地看出系统 $\{\ee'_k\}_{k=0}^n$ 是幂集 $\{\ee_k\}_{k=0}^n$ 的对偶。

将 (1.3) 应用于上述系统 $\{\ee_k\}_{k=0}^n$ 及其对偶,我们得到次数最多为 $n$ 的任何多项式 $p$ 都可以表示为:
$$(1.4) \quad p(t) = \sum_{k=0}^n \frac{p^{(k)}(0)}{k!} t^k   $$
这个公式在微积分中作为多项式的泰勒公式是众所周知的。更确切地说,这是泰勒公式的一个特殊情况,即所谓的麦克劳林公式。一般的泰勒公式 
$$p(t) = \sum_{k=0}^n \frac{p^{(k)}(a)}{k!} (t-a)^k$$
可以通过对多项式 $p(\tau-a)$ 应用 (1.4) 然后令 $t := \tau-a$ 来得到。它也可以通过考虑幂 $(t-a)^k, k=0, 1, \dots, n$ 并以与我们为 $t^k$ 相同的方式找到对偶系统来得到。\footnote{一般的泰勒公式比这里得到的用于多项式的公式包含了更多信息:它说明任何 $n$ 次可微的函数都可以用其泰勒多项式在点 $a$ 附近进行近似。更重要的是,如果函数是 $n+1$ 次可微的,它允许我们估计误差。上面多项式的公式作为一般情况的动机和起点。}

\subsubsection{1.4.2. 拉格朗日插值}

我们的下一个例子涉及所谓的拉格朗日插值公式。
设 $a_1, a_2, \dots, a_{n+1}$ 是互不相同的点(在 $\RR$ 或 $\CC$ 中),设 $\PP_n$ 是次数最多为 $n$ 的多项式空间。定义泛函 $\ff_k \in \PP'_n$ 为:
$$ \ff_k(p) = p(a_k) \quad \forall p \in \PP_n .$$

这个泛函系统的对偶是什么?注意,虽然证明泛函 $\ff_k$ 是线性无关的(因此,因为 $\dim(\PP_n)' = \dim \PP_n = n+1$,它们构成 $(\PP_n)'$ 中的一个基)并不难,但我们不需要这样做。我们将直接构造对偶系统,然后就能看出系统 $\ff_1, \ff_2, \dots, \ff_{n+1}$ 确实是一个基。

也就是说,让我们定义多项式 $p_k,\quad k = 1, 2, \dots, n+1$ 为:
$$ p_k(t) = \prod_{j: j \neq k} (t - a_j) / \prod_{j: j \neq k} (a_k - a_j) $$
其中乘积中的 $j$ 从 1 遍历到 $n+1$.~
显然,$p_k(a_k) = 1$,且如果 $j \neq k$,则 $p_k(a_j) = 0$.~因此,系统 $\{p_1, p_2, \dots, p_{n+1}\}$ 确实是对 $\{\ff_1, \ff_2, \dots, \ff_{n+1}\}$ 的对偶。

这里有一个小细节,因为对偶系统的概念只针对基定义的,而我们没有证明这两个系统中的任何一个是一个基。但人们可以立即看出系统 $\{p_1, p_2, \dots, p_{n+1}\}$ 是线性无关的(你能解释为什么吗?),并且由于它包含 $n+1 = \dim \PP_n$ 个向量,它是一个基。因此,泛函系统 $\{\ff_1, \ff_2, \dots, \ff_{n+1}\}$ 也是 $(\PP_n)'$ 对偶空间中的一个基。

\textbf{注记}~~
注意,我们在这并不走运,这是一个普遍现象。也就是说,正如练习 1.1 所断言的,任何拥有“对偶”系统的向量系统都必须是线性无关的。因此,构造一个对偶系统是证明线性无关性的一种方法(如果你能像上面的例子那样轻易做到,那么这种方法就很简单)。

应用公式 (1.3) 到上面的例子,我们可以看出满足$$(1.5)\quad p(a_k)=y_k,\quad k=1, 2, \dots, n+1$$
$\deg p \le n$ 的(唯一)多项式 $p$,可以由公式  
$$(1.6)\quad p(t) = \sum_{k=1}^{n+1} y_k p_k(t). $$
重构。
这个公式在数学中作为“拉格朗日插值公式”是众所周知的。

\begin{exer} \textbf{练习}~~

1.1. 设 $\vv_1, \vv_2, \dots, \vv_r$ 是 $X$ 中的一个向量系统,使得存在一个线性泛函系统 $\vv'_1, \vv'_2, \dots, \vv'_r$ 满足 $$\vv'_k(\vv_j) = \begin{cases} 1, & j=k \\ 0, & j \neq k .\end{cases}$$

a) 证明系统 $\vv_1, \vv_2, \dots, \vv_r$ 是线性无关的。

b) 证明如果系统 $\vv_1, \vv_2, \dots, \vv_r$ 不是生成集,那么“双正交”系统 $\vv'_1, \vv'_2, \dots, \vv'_r$ 不是唯一的。

\textbf{提示}:可能最简单的证明方法是将其扩展为基,见第二章命题 5.4.

1.2. 证明对于给定的互不相同的点 $a_1, a_2, \dots, a_{n+1}$ 和值 $y_1, y_2, \dots, y_{n+1}$(不一定互不相同),满足 (1.5) 的多项式 $p$,$\deg p \le n$,是唯一的。尝试使用线性代数的思想来证明,而不是你所知道的多项式知识。\end{exer}

\section{2. 内积空间的对偶}

让我们回顾一下,任意域上的内积空间并不存在,我们所有的内积空间都是实数或复数。

\subsection{2.1. 里斯表示定理}

\textbf{定理 2.1} (里斯(Riesz)表示定理)
设 $H$ 是一个内积空间。给定 $H$ 上的一个线性泛函 $L$,存在一个唯一的向量 $\yy \in H$ 使得
$$(2.1)\quad L(\vv) = (\vv, \yy) \quad \forall \vv \in H .  $$

\textbf{证明}~~
在 $H$ 中固定一个标准正交基 $\ee_1, \ee_2, \dots, \ee_n$,设 $$[L] = [L_1, L_2, \dots, L_n]$$ 是 $L$ 在这个基下的矩阵。定义向量 $\yy$ 为:
$$(2.2) \quad \yy := \sum_{k=1}^n L_k \ee_k  $$
其中 $\overline{L}_k$ 表示 $L_k$ 的复共轭。在实数空间的情况下,共轭运算不起作用,可以简单地忽略。

我们声称 $\yy$ 满足 (2.1)。

事实上,取任意向量 $\vv = \sum_{k=1}^n \alpha_k \ee_k$.~那么 
$$[\vv] = [\alpha_1, \alpha_2, \dots, \alpha_n]^T,$$
并且 
$$L(\vv) = [L][\vv] = \sum_{k=1}^n L_k \alpha_k.$$
另一方面,
\footnote{
回忆一下,如果我们知道两个向量在标准正交基下的坐标,我们就可以通过取这些坐标并计算 $\CC^n$(或 $\RR^n$)中的标准内积来计算内积。}
$$(\vv, \yy) = \sum_{k=1}^n \alpha_k \overline{\overline{L}}_k = \sum_{k=1}^n \alpha_k L_k.$$
所以 (2.1) 成立。

为了证明向量 $\yy$ 是唯一的,我们假设 $\yy$ 满足 (2.1)。那么对于 $k = 1, 2, \dots, n$,
$$ (\ee_k, \yy) = L(\ee_k) = L_k ,$$
所以 $(\yy, \ee_k) = \overline{L}_k$.~然后,使用在标准正交基下的分解公式,见第五章 2.1 节,我们得到:
$$ \yy = \sum_{k=1}^n (\yy, \ee_k) \ee_k = \sum_{k=1}^n \overline{L}_k \ee_k $$
这意味着任何满足 (2.1) 的向量必须由 (2.2) 表示。

\textbf{注记}~~
虽然定理的陈述不要求基,但这里提出的证明利用了 $H$ 中的一个标准正交基,尽管得到的向量 $\yy$ 不依赖于基的选择。
\footnote{
另一种需要基的证明也是可能的。这个替代的证明(在无限维情况下有效)利用了单位球在内积空间中的强凸性,以及来自分析学的完备性思想。
}
这个证明的一个优点是它给出了表示向量 $\yy$ 的计算公式。

\subsection{2.2. 内积空间自身是其对偶吗?}
对于内积空间 $H$ 中的一个向量 $\yy$,可以通过 
$$L_\yy(\vv) := (\vv, \yy)$$
定义一个线性泛函。很容易看出映射 $\yy \mapsto L_\yy$ 是一个从 $H$ 到其对偶 $H^*$ 的单射映射。上面的定理 2.1 断言这个映射是一个满射,所以人们倾向于说内积空间 $H$ 的对偶(规范同构地)是空间 $H$ 本身,其中规范同构由 $\yy \mapsto L_\yy$ 给出。

这对于\textbf{实}内积空间$H$确实是如此,并且很容易证明映射 $\yy \mapsto L_\yy$ 是一个\textbf{线性}变换。我们已经讨论过这个映射是单射和满射,所以它是一个可逆线性变换,即\textbf{同构}。

然而,如果 $H$ 是一个\textbf{复}空间,则需要更加谨慎。即,映射 $\yy \mapsto L_\yy$,它将向量 $\yy \in H$ 映射到线性泛函 $L_\yy$,如定理 2.1 所示($L_\yy(\vv) = (\vv, \yy)$),不是线性的。更准确地说,虽然很容易证明:
$$ (2.3) \quad L_{\yy_1+\yy_2} = L_{\yy_1} + L_{\yy_2}, $$
然而,从 $L_\yy$ 的定义和内积的性质可得:
$$ (2.4) \quad  L_{\alpha \yy}(\vv) = (\vv, \alpha \yy) = \overline{\alpha} (\vv, \yy) = \overline{\alpha} L_\yy(\vv) ,$$
所以 $L_{\alpha \yy} = \overline{\alpha} L_\yy$.~

换句话说,我们可以说,一个复数内积空间的对偶是该空间本身,但具有\textbf{不同的线性结构}:两个向量的加法等同于相应线性泛函的加法,但一个向量乘以 $\alpha$ 等同于相应泛函乘以 $\overline{\alpha}$.~

\fbox{\begin{minipage}{0.9\textwidth}
一个满足 $T(\alpha \xx + \beta \yy) = \overline{\alpha} T \xx + \overline{\beta} T \yy$ 的变换有时被称为\textbf{共轭线性}变换。
\end{minipage}}


因此,对于复数内积空间 $H$,其对偶可以通过一个共轭线性同构(即可逆共轭线性变换)与 $H$ 规范对应(canonically identified)。

当然,对于实内积空间,复共轭可以简单地忽略(因为 $\alpha$ 是实数),所以映射 $\yy \mapsto L_\yy$ 是线性的。在这种情况下,我们确实可以说内积空间 $H$ 的对偶就是其本身。

在实数和复数情况下的两种情况下,我们仍然可以认为内积空间 $H$ 的对偶可以规范对应为空间 $H$ 本身。

\subsection{2.3. 双正交系统与标准正交基}

\textbf{定义 2.2}~~
设 $\{\bb_1, \bb_2, \dots, \bb_n\}$ 是内积空间 $H$ 中的一个基。在 $H$ 中由 
$$(\bb_j, \bb'_k) = \delta_{j,k},$$
定义的唯一系统 $\{\bb'_1, \bb'_2, \dots, \bb'_n\}$,其中 $\delta_{j,k}$ 是克罗内克符号,被称为与基 $\{\bb_1, \bb_2, \dots, \bb_n\}$ \textbf{双正交}或\textbf{对偶}。

这个定义显然与定义 1.4 一致,如果我们像上面讨论的那样将对偶 $H'$ 与 $H$ 对应。那么从 1.3 节的讨论中可以立即得出,基 $\{\bb_1, \bb_2, \dots, \bb_n\}$ 的对偶系统 $\{\bb'_1, \bb'_2, \dots, \bb'_n\}$ 是唯一确定的,并且构成一个基,并且 $\{\bb'_1, \bb'_2, \dots, \bb'_n\}$ 的对偶是 $\{\bb_1, \bb_2, \dots, \bb_n\}$.~

抽象的非正交傅里叶展开公式 (1.3) 可以重写为:
$$ \vv = \sum_{k=1}^n (\vv, \bb'_k) \bb_k .$$

注意,一个标准正交基是它自身的对偶。所以,如果 $\{\ee_1, \ee_2, \dots, \ee_n\}$ 是一个标准正交基,那么上面的公式重写为:
$$ \vv = \sum_{k=1}^n (\vv, \ee_k) \ee_k ,$$
这就是经典的(正交的)抽象傅里叶展开,见第五章 2.1 节公式 (2.2)。

\section{3. 伴随(对偶)变换与转置,基本子空间再回顾(又一次)}

类比内积空间的情况,见定理 2.1,通常将 $L(\vv)$ 写成类似于内积的形式,其中 $L$ 是一个线性泛函(即 $L \in V',\quad \vv \in V$):
$$ L(\vv) = \langle \vv, L \rangle $$
注意,表达式 $\langle \vv, L \rangle$ 在两个参数上都是线性的,这与内积不同,后者在复数情况下是第一个参数为线性的,第二个参数为共轭线性的。
所以,为了区分它与内积,我们使用尖括号。\footnote{
这个记号虽然被广泛使用,但远非标准。有时也使用 $(\vv, L)$,有时尖括号用于内积。因此,在文本中遇到这样的表达式时,必须非常小心地从线性泛函的作用中区分出内积。
}

还请注意,虽然在内积中两个向量属于同一空间,但上面的 $\vv$ 和 $L$ 属于不同的空间:特别地,我们不能将它们相加。

\subsection{3.1. 对偶(伴随)变换}

\textbf{定义 3.1}~~
设 $A : X \to Y$ 是一个线性变换。变换 $A' : Y' \to X'$(其中 $X'$ 和 $Y'$ 分别是 $X$ 和 $Y$ 的对偶空间),使得
$$ \langle A\xx, \yy' \rangle = \langle \xx, A'\yy' \rangle \quad \forall \xx \in X, \yy' \in Y' $$
被称为 $A$ 的\textbf{伴随(对偶)变换}。


当然,事先并不清楚为什么变换 $A'$ 存在。下面我们将表明,确实存在这样的变换,而且它是唯一的。

\subsubsection{3.1.1. $A : \FF^n \to \FF^m$ 情况下的对偶变换}

让我们首先考虑 $X = \FF^n$, $Y = \FF^m$ 的情况(这里的$\FF$通常是 $\RR$ 或 $\CC$,但一切都适用于任意域)。

像往常一样,我们将 $\FF^n$ 中的向量 $\vv$ 与其坐标列向量进行标识,并将线性变换与其矩阵(在标准基下)进行标识。

如上所述,$\FF^n$ 的对偶是大小为 $n$ 的行向量空间,所以我们可以将其与 $\FF^n$ 进行对应。同样,我们将 $(\FF^n)'$ 中的元素视为其坐标的列向量。

在这些约定下,我们对于 $\xx \in \FF^n$ 和 $\xx' \in (\FF^n)'$ 有:
$$ \xx'(\xx) = \langle \xx, \xx' \rangle = (\xx')^T \xx $$
其中右侧是矩阵乘法(或行向量乘以列向量)。那么,对于任意 $\xx \in X = \FF^n$ 和 $\yy' \in Y' = (\FF^m)'$,
$$ \langle A\xx, \yy' \rangle = (\yy')^T A\xx = (A^T \yy')^T \xx = \langle \xx, A^T \yy' \rangle. $$
(中间的表达式是矩阵乘法)。

所以我们已经证明了伴随变换存在。
让我们证明它是唯一的。假设存在某个变换 $B$ 使得
$$ \langle A\xx, \yy' \rangle = \langle \xx, B\yy' \rangle \quad \forall \xx \in \FF^n, \forall \yy' \in (\FF^m)' $$
这意味着对于任意 $\xx$ 和 $\yy'$,
$$ \langle \xx, (A^T - B)\yy' \rangle = 0 \quad \forall \xx \in \FF^n, \forall \yy' \in (\FF^m)'$$
通过选择 $\xx$ 和 $\yy'$ 分别为 $\FF^n$ 和 $(\FF^m)' \cong \FF^m$ 中的标准基向量,我们得到矩阵 $B$ 和 $A^T$ 是相同的。

所以,对于 $X = \FF^n$, $Y = \FF^m$,

\fbox{\begin{minipage}{0.9\textwidth}
伴随变换 $A'$ 存在且唯一。而且,其矩阵(在标准基下)等于 $A^T$($A$ 矩阵的转置)。
\end{minipage}}


\subsubsection{3.1.2. 抽象设置下的对偶变换}

现在,让我们考虑一般情况。事实上,我们不需要做太多,因为一切都可以归约到 $\FF^n$ 的情况。

也就是说,我们固定 $X$ 中的基 $\A = \{\aaa_1, \aaa_2, \dots, \aaa_n\}$ 和 $Y$ 中的基 $\B = \{\bb_1, \bb_2, \dots, \bb_m\}$,以及它们对应的对偶基 $\A' = \{\aaa'_1, \aaa'_2, \dots, \aaa'_n\}$ 和 $\B' = \{\bb'_1, \bb'_2, \dots, \bb'_m\}$(分别在 $X'$ 和 $Y'$ 中)。对于一个向量 $\vv$ (来自空间或其对偶),我们像往常一样用 $[\vv]_\B$ 表示其在基 $\B$ 下的坐标。那么
$$ \langle \xx, \xx' \rangle = ([\xx']_{\A'})^T [\xx]_\A, \quad \forall \xx \in X, \forall \xx' \in X' ,$$
也就是说,与 $\xx \in X$ 和 $\xx' \in X'$ 上的作用相比,我们可以使用它们坐标的列向量,以绝对相同的方式操作,就像在 $\FF^n$ 的情况下一样。当然,对于 $Y$ 也是如此,所以通过使用坐标列向量然后将一切翻译回抽象设置,我们得到在这种情况下对偶变换也存在且唯一。而且,利用(我们刚刚证明的)对于 $A : \FF^n \to \FF^m$, $A'$ 的矩阵是 $A^T$ 的事实,我们得到:
$$(3.1)\quad [A']_{\A', \B'} = ([A]_{\B, \A})^T, $$
或者用通俗的语言说:

\fbox{\begin{minipage}{0.9\textwidth}
对偶变换在对偶基下的矩阵是变换在原始基下的矩阵的转置。
\end{minipage}}


\textbf{注记 3.2}~~
请注意,虽然我们使用基来构造对偶变换,但得到的变换不依赖于基的选择。

\subsubsection{3.1.3. 定义对偶变换的无坐标方法}

现在,让我们提出另一种更“高深”的方法来定义线性变换的对偶。也就是说,对于 $\xx \in X$, $\yy' \in Y'$,让我们暂时固定 $\yy'$,并将表达式 $\langle A\xx, \yy' \rangle = \yy'(A\xx)$ 看作是 $\xx$ 的函数。很容易看出这是一个(哪些?)两个线性变换的复合,因此它是 $\xx$ 的线性函数,即 $X$ 上的一个线性泛函,即 $X'$ 中的一个元素。

让我们称这个线性泛函为 $B(\yy')$,以强调它依赖于 $\yy'$.~由于我们可以对每个 $\yy' \in Y'$ 执行此操作,我们可以定义一个变换 $B : Y' \to X'$ 使得
$$ \langle A\xx, \yy' \rangle = \langle \xx, B(\yy') \rangle $$
我们的下一步是证明 $B$ 是一个线性变换。请注意,由于变换 $B$ 是以一种相当间接的方式定义的,我们无法立即从定义中看出它是线性的。为了证明 $B$ 的线性,让我们取 $\yy'_1, \yy'_2 \in Y'$.~对于 $\xx \in X$:
\begin{equation} \notag
\begin{split}
 \langle \xx, B(\alpha \yy'_1 + \beta \yy'_2) \rangle =&\ \langle A\xx, \alpha \yy'_1 + \beta \yy'_2 \rangle \quad (\text{根据 } B \text{ 的定义}) \\
 =&\ \alpha \langle A\xx, \yy'_1 \rangle + \beta \langle A\xx, \yy'_2 \rangle \quad (\text{根据线性性质}) \\
 =&\ \alpha \langle \xx, B(\yy'_1) \rangle + \beta \langle \xx, B(\yy'_2) \rangle \quad (\text{根据 } B \text{ 的定义}) \\
 =&\ \langle \xx, \alpha B(\yy'_1) + \beta B(\yy'_2) \rangle \quad (\text{根据线性性质}) 
 \end{split}\end{equation}
由于这个恒等式对所有 $\xx$ 都成立,我们得出 $B(\alpha \yy'_1 + \beta \yy'_2) = \alpha B(\yy'_1) + \beta B(\yy'_2)$,即 $B$ 是线性的。

这种方法的主要优点是它不需要基,因此可以(并且被)用于无限维情况。然而,我们在 3.1.1 和 3.1.2 节中给出的证明提供了一种构造性计算对偶变换的方法,所以我们使用了那个证明而不是更通用的无坐标证明。

\textbf{注记 3.3}~~
注意,上面的无坐标方法可以用来定义内积空间中算子的埃尔米特伴随。与上面呈现的推理相比,唯一需要添加的是使用里斯表示定理(定理 2.1)。我们把细节留给读者作为练习,见下面的问题 3.2。

\subsection{3.2. 零化子与基本子空间之间的关系}

\textbf{定义 3.4}~~
设 $X$ 是一个向量空间,设 $E \subset X$.~$E$ 的\textbf{零化子}(annihilator),记作 $E^\perp$,是所有 $\xx' \in X'$ 的集合,使得 $\langle \xx, \xx' \rangle = 0$ 对所有 $\xx \in E$.~

利用 $X''$ 与 $X$ 规范同构的这一事实(见 1.2 节),我们说对于 $E \subset X'$,其\textbf{零化子} $E^\perp$ 由所有满足 $\langle \xx, \xx' \rangle = 0$ 对所有 $\xx' \in E$ 的向量 $\xx \in X$ 组成。

\textbf{注记 3.5}~~
严格来说,对于 $E \subset X'$,集合 $E^\perp$ 应该定义为所有 $\xx'' \in X''$ 的集合,使得 $\langle \xx', \xx'' \rangle = 0$ 对所有 $\xx' \in E$.~符号 $E^\perp$ 通常用于定义 3.4 的第二部分中的零化子。然而,由于 $X''$ 和 $X$ 之间的自然同构,这两种情况之间没有真正的区别,所以我们总是使用 $E^\perp$.~

区分 $E \subset X$ 和 $E \subset X'$ 的情况在无限维情况下非常有意义,其中 $X''$ 并不总是与 $X$ 规范同构。

满足 $X''$ 与 $X$ 规范同构的空间被称为\textbf{自反空间}。

\textbf{命题 3.6}~~
设 $E$ 是 $X$ 的一个子空间。那么 $(E^\perp)^\perp = E$.~
这个命题看起来完全像第五章命题 3.6。然而,它的证明有点复杂,因为第五章命题 3.6 的建议证明严重依赖于内积空间结构:它使用了 $X = E \oplus E^\perp$ 的分解,这在我们的情况下是不成立的,因为例如,$E$ 和 $E^\perp$ 属于不同的空间。

\textbf{证明}~~
设 $\{\vv_1, \vv_2, \dots, \vv_r\}$ 是 $E$ 的一个基(回忆一下本章中的所有空间都是有限维的),所以 $E = \text{span}\{\vv_1, \vv_2, \dots, \vv_r\}$.~

根据第二章命题 5.4,该系统可以扩展为 $X$ 的一个基,也就是说,我们可以找到向量 $\vv_{r+1}, \dots, \vv_n$($n = \dim X$),使得 $\{\vv_1, \vv_2, \dots, \vv_n\}$ 是 $X$ 的一个基。

设 $\{\vv'_1, \vv'_2, \dots, \vv'_n\}$ 是与 $\{\vv_1, \vv_2, \dots, \vv_n\}$ 对偶的基。根据问题 3.3,$E^\perp = \text{span}\{\vv'_{r+1}, \dots, \vv'_n\}$.~再次将这个问题应用于 $E^\perp$,我们得到 $$(E^\perp)^\perp = \text{span}\{\vv_1, \vv_2, \dots, \vv_n\} = E.$$

\textbf{定理 3.7}~~
设 $A : X \to Y$ 是一个从一个向量空间到另一个向量空间的算子。那么:

a) Ker $A' = (\text{Ran } A)^\perp$;

b) Ker $A = (\text{Ran } A')^\perp$;

c) Ran $A = (\text{Ker } A')^\perp$;

d) Ran $A' = (\text{Ker } A)^\perp$.~

\textbf{证明}~~
首先,让我们注意到,由于对于子空间 $E$,我们有 $(E^\perp)^\perp = E$,所以命题 1 和 3 是等价的。类似地,出于同样的原因,命题 2 和 4 是等价的。最后,命题 2 正是应用于算子 $A'$ 的命题 1(我们使用 $(A')' = A$ 的平凡事实,这例如是因为转置的相应事实)。

因此,为了证明定理,我们只需要证明命题 1。

回忆一下,$A' : Y' \to X'$.~包含 $\yy' \in (\text{Ran } A)^\perp$ 意味着 $\yy'$ 零化了所有形式为 $A\xx$ 的向量,即 $$\langle A\xx, \yy' \rangle = 0\quad \forall \xx \in X.$$
由于 $\langle A\xx, \yy' \rangle = \langle \xx, A'\yy' \rangle$,最后一个恒等式等价于 $$\langle \xx, A'\yy' \rangle = 0\quad \forall \xx \in X.$$
但这表示 $A'\yy' = \oo$($A'\yy'$ 是零泛函)。

所以我们证明了 $\yy' \in (\text{Ran } A)^\perp$ 当且仅当 $A'\yy' = \oo$,或者等价地,当且仅当 $\yy' \in \text{Ker } A'$.~

\begin{exer} \textbf{练习}~~

3.1. 证明如果对于线性变换 $T, T_1 : X \to Y$,
$$\langle T\xx, \yy' \rangle = \langle T_1\xx, \yy' \rangle$$
对所有 $\xx \in X$ 和所有 $\yy' \in Y'$ 成立,那么 $T = T_1$.~

也许最简单的证明方法是使用引理 1.3。

3.2. 结合里斯表示定理(定理 2.1)和上面 3.1.3 节的推理,给出一个内积空间中算子的埃尔米特伴随的无坐标定义。


下一个问题给出了证明命题 3.6 的一种方法。

3.3. 设 $\vv_1, \vv_2, \dots, \vv_n$ 是 $X$ 中的一个基,设 $\vv'_1, \vv'_2, \dots, \vv'_n$ 是它的对偶基。设 $E := \text{span}\{\vv_1, \vv_2, \dots, \vv_r\},\quad r < n$.~证明 $E^\perp = \text{span}\{\vv'_{r+1}, \dots, \vv'_n\}$.~

3.4. 使用前一个问题来证明对于子空间 $E \subset X,$
$$ \dim E + \dim E^\perp = \dim X.$$\end{exer}

\section{4. 空间与其对偶之间的区别}

我们知道对偶空间 $X'$ 与 $X$ 具有相同的维度,所以空间与其对偶是同构的。因此,人们可能会认为空间与其对偶之间实际上没有区别。然而,正如我们在 1.1 节中讨论过的,当我们在空间 $X$ 中改变基时,$X$ 中的坐标和 $X'$ 中的坐标根据不同的规则变化,见上面公式 (1.1)。

另一方面,利用 $X$ 和 $X''$ 的自然同构,我们可以说 $X$ 是 $X'$ 的对偶。从这个角度来看,$X$ 和 $X'$ 之间没有区别:我们可以从 $X$ 开始,说 $X'$ 是它的对偶,或者我们可以反过来,从 $X'$ 开始。

我们已经在上面使用了这种观点,例如在定理 3.7 的证明中。

还请注意,坐标变换公式 (1.1)(也见它下方的方框语句)与这种观点一致:如果 $\tilde{S} := (S^{-1})^T$,那么 $(\tilde{S}^{-1})^T = S$,所以我们通过相同的规则从 $X'$ 中的坐标变换公式得到了 $X$ 中的坐标变换公式!

\subsection{4.1. $X$ 与 $X'$ 之间的同构}

定义 $X$ 和 $X'$ 之间的同构存在无数种可能性。

如果 $X = \FF^n$,那么最自然地对应 $X$ 和 $X'$ 的方法是将 $\FF^n$ 中的标准基与 $(\FF^n)'$ 中的标准基进行对应。在这种情况下,线性泛函的作用将由“内积类型”的表达式 $$\langle \vv, \vv' \rangle = (\vv')^T \vv$$ 给出。
为了将其推广到一般情况,必须固定 $X$ 中的一个基 $\B = \{\bb_1, \bb_2, \dots, \bb_n\}$ 并考虑其对偶基 $\B' = \{\bb'_1, \bb'_2, \dots, \bb'_n\}$,并定义一个同构 $T : X \to X'$ 为 $T \bb_k = \bb'_k,\quad k = 1, 2, \dots, n$.~

这个同构在某种意义上是自然的,但它依赖于基的选择,所以在一般情况下没有自然的方式来对应 $X$ 和 $X'$.~

唯一的例外是当 $X$ 是一个实内积空间时:里斯表示定理(定理 2.1)提供了一种自然的方式将线性泛函与 $X$ 中的向量进行对应。请注意,这种方法仅适用于\textbf{实}内积空间。对于复数情况,里斯表示定理给出了 $X$ 和 $X'$ 的自然对应,但这种对应不是线性的,而是\textbf{共轭线性}的。

\subsection{4.2. 例子:速度(微分算子),微分形式作为向量,线性泛函}

为了说明向量和线性泛函之间的关系,让我们考虑一个来自多变量微积分的例子,它引出了微分几何中的重要思想,如切线丛和余切线丛。

让我们回忆一下微积分中第二类路径积分的概念。回忆一下,$\RR^n$ 中的一条路径 $\gamma$ 由其参数化定义,即由一个从区间 $[a, b]$ 到 $\RR^n$ 的函数 
$$t \mapsto \xx(t) = (x_1(t), x_2(t), \dots, x_n(t))^T.$$
如果 $\omega$ 是所谓的\textbf{微分形式}(一阶微分形式),
$$ \omega = f_1(\xx) \dif x_1 + f_2(\xx) \dif x_2 + \dots + f_n(\xx) \dif x_n ,$$
那么\textbf{路径积分} $$\int_\gamma \omega = \int_\gamma f_1 \dif x_1 + f_2 \dif x_2 + \dots + f_n \dif x_n$$ 
是通过将 $\xx(t) = (x_1(t), x_2(t), \dots, x_n(t))^T$ 代入表达式来计算的,即 $\int_\gamma \omega$ 计算为:
$$ \int_a^b \left( f_1(\xx(t)) \frac{\dif x_1(t)}{\dif t} + f_2(\xx(t)) \frac{\dif x_2(t)}{\dif t} + \dots + f_n(\xx(t)) \frac{\dif x_n(t)}{\dif t} \right) \dif t .$$

换句话说,在每个时刻 $t$,我们必须计算速度 
$$\vv = \frac{\dif \xx(t)}{\dif t} = \left( \frac{\dif x_1(t)}{\dif t}, \frac{\dif x_2(t)}{\dif t}, \dots, \frac{\dif x_n(t)}{\dif t} \right)^T,$$
将其应用于线性泛函 $\ff = (f_1, f_2, \dots, f_n),\quad \ff(\vv) = \sum_{k=1}^n f_k v_k$(这里 $f_k = f_k(\xx(t))$ 但对于固定的 $t$, 每个 $f_k$ 只是一个数字,所以我们只写 $f_k$),然后对结果(它依赖于 $t$)关于 $t$ 进行积分。

\subsubsection{4.2.1. 速度作为向量}

让我们固定 $t$ 并分析 $\ff(\vv)$.~我们将根据微积分的规则表明 $\vv$ 的坐标如何变化,以及 $\ff$ 的坐标如何变化。

假设正如微积分中的惯例, $x_k$ 是 $\RR^n$ 标准基下的坐标,设 $\B = \{\bb_1, \bb_2, \dots, \bb_n\}$ 是 $\RR^n$ 中的另一个基。我们将使用记号 $\tilde{x}_k$ 来表示向量 $\xx = (x_1, x_2, \dots, x_n)^T$ 的坐标,即 $[\xx]_\B = (\tilde{x}_1, \tilde{x}_2, \dots, \tilde{x}_n)^T$.~

设 $A = \{a_{k,j}\}_{k,j=1}^n$ 是坐标变换矩阵,$A = [I]_{\B,\SSS}$,所以新的坐标 $\tilde{x}_k$ 用旧的坐标 $x_j$ 表示为:
$$ \tilde{x}_k = \sum_{j=1}^n a_{k,j} x_j, \quad k = 1, 2, \dots, n .$$
所以向量 $\vv$ 的新坐标 $\tilde{v}_k$ 是从其旧坐标 $v_k$ 得到的:
$$ \tilde{v}_k = \sum_{j=1}^n a_{k,j} v_j, \quad k = 1, 2, \dots, n .$$

\subsubsection{4.2.2. 微分形式作为线性泛函(余向量)}

现在让我们用新的坐标 $\tilde{x}_k$ 来计算微分形式  
$$(4.1)\quad\omega = \sum_{k=1}^n f_k \dif x_k.$$
从旧坐标到新坐标的坐标变换矩阵是 $A^{-1}$.~设 $A^{-1} = \{\tilde{a}_{k,j}\}_{k,j=1}^n$,所以 
$$x_k = \sum_{j=1}^n \tilde{a}_{k,j} \tilde{x}_j, \text{~~并且~~} \dif x_k = \sum_{j=1}^n \tilde{a}_{k,j} \dif \tilde{x}_j,\quad k = 1, 2, \dots, n.$$
将此代入 (4.1) 中,我们得到:
\begin{equation} \notag
\begin{split}
\omega =&\ \sum_{k=1}^n f_k  \sum_{j=1}^n \tilde{a}_{k,j} \dif \tilde{x}_j  \\
=&\ \sum_{j=1}^n \left( \sum_{k=1}^n \tilde{a}_{k,j} f_k \right) \dif \tilde{x}_j\\
=&\ \sum_{j=1}^n \tilde{f}_j \dif \tilde{x}_j,
\end{split}\end{equation}
其中 $$\tilde{f}_j = \sum_{k=1}^n \tilde{a}_{k,j} f_k.$$
但这正是对偶空间中坐标的变换规则!所以

\fbox{\begin{minipage}{0.9\textwidth}
根据微积分的规则,一阶微分形式的系数按照与对偶空间中的坐标相同的规则进行变换。
\end{minipage}}


因此,根据被接受了的微积分的规则,速度 $\vv$ 的坐标像向量的坐标一样变化,而一阶微分形式的系数(坐标)像线性泛函的系数一样变化。在微分几何中,所有速度的集合被称为\textbf{切空间}(tangent space),而所有一阶微分形式的集合是其对偶,被称为\textbf{余切空间}(cotangent space)。


\subsubsection{4.2.3. 微分算子作为向量}

正如我们上面讨论的,在微分几何中,向量用速度表示,即用导数 $\dif \xx(t)/\dif t$ 表示。这是一个简单且直观清晰的观点,但有时被认为有点天真。

更“高深”的观点,在微分几何中(尽管是在更高级的文本中)也使用,是向量由\textbf{微分算子}表示:
$$(4.2) \quad  D = \sum_{k=1} v_k \frac{\partial}{\partial x_k}. $$
这样非正式做的原因是,假设我们想沿着由函数 $t \mapsto \xx(t)$ 给出的路径计算函数 $\Phi$ 的导数,即导数 $$\frac{\dif \Phi(\xx(t))}{\dif t}.$$根据链式法则,在给定时间 $t$:
$$ \frac{\dif \Phi(\xx(t))}{\dif t} = \sum_{k=1}^n \left( \frac{\partial \Phi}{\partial x_k} \Big|_{\xx=\xx(t)} \right) \xx'_k(t) = D\Phi \Big|_{\xx=\xx(t)} ,$$
其中微分算子 $D$ 由 (4.2) 给出, $v_k = x'_k(t)$.~

当然,我们需要根据向量坐标的坐标变换规则来表明微分形式的系数 $\vv_k$ 如何变化。
这是直观清晰的,并且可以通过使用多变量链式法则轻松证明。我们将其留给读者作为练习,见下面的问题 4.1.

\subsection{4.3. 实内积空间的情况}

正如我们上面已经讨论过的,根据里斯表示定理(定理 2.1),实内积空间 $X$ 及其对偶 $X'$ 是规范同构的。因此,我们可以说向量和泛函存在于同一个空间中,这使得事情既更简单也更混乱。

\textbf{注记}~~
首先,让我们注意到,如果坐标变换矩阵 $S$ 是正交的($S^{-1} = S^T$),那么 $(S^{-1})^T = S$.~因此,对于正交坐标变换矩阵,向量和线性泛函的坐标根据相同的规则变化,所以人们无法真正区分向量和泛函。

坐标变换矩阵是正交的,例如,当我们从一个标准正交基改变到另一个标准正交基时。

\subsubsection{4.3.1. 爱因斯坦记法,度量张量}

设 $\B = \{\bb_1, \bb_2, \dots, \bb_n\}$ 是实内积空间 $X$ 中的一个基,并设 $\B' = \{\bb'_1, \bb'_2, \dots, \bb'_n\}$ 是它的对偶基(我们通过里斯表示定理将对偶空间 $X'$ 与 $X$ 对应,所以 $\bb'_k$ 可以在 $X$ 中)。

在这里,我们介绍了在这些基下工作时处理坐标的标准记法(所谓的爱因斯坦记法(Einstein notation)\footnote{
译者注:又称为爱因斯坦求和约定(Einstein summation convention)
}
)。由于我们只处理坐标,我们可以假设我们在 $\RR^n$ 空间中工作,其非标准内积为 $( \cdot, \cdot )_G$,由正定矩阵 $G = \{g_{j,k}\}^{n}_{j,k=1}$ 定义,其中 $g_{j,k} = (\bb_k, \bb_j)_X$,这通常被称为\textbf{度量张量}(metric tensor)。
$$(4.3)\quad (\xx, \yy) = (\xx, \yy)_G = \sum_{j=1}^n \sum_{k=1}^n g_{j,k} x_j y_k, \quad \xx, \yy \in \RR^n $$
(见第七章第 5 节)。



为了区分向量和线性泛函(余向量),约定将向量的坐标写成上标,线性泛函的坐标写成下标:因此 $x^j, j = 1, 2, \dots, n$ 表示向量 $\xx$ 的坐标,而 $f_k, k = 1, 2, \dots, n$ 表示线性泛函 $\ff$ 的坐标。

\textbf{注记}~~

将下标写成上标可能会令人困惑,因为需要将其与幂区分开。然而,这是一个标准且广泛使用的记法,所以我们需要熟悉它。虽然我个人,以及许多数学家,更喜欢使用无坐标记法,但所有最终的计算都是在坐标中进行的,所以坐标记法必须被使用。而且就坐标记法而言,你会发现这种记法在处理时相当方便。

爱因斯坦记法的另一个约定是,每当在乘积中出现相同的下标和上标时,意味着需要对该下标进行求和。因此,$x^j f_j$ 表示 $\sum_j x^j f_j$,所以我们可以写成 $\ff(\xx) = x^j f_j$.~同样的约定适用于我们有多个求和下标的情况,所以 (4.3) 可以重写为:
$$(4.4)  \quad (\xx, \yy) = g_{j,k} \xx^k \yy^j \quad \xx, \yy \in \RR^n .$$
(数学家很懒,总是试图避免编写额外的符号,只要他们能做到)。

最后,爱因斯坦记法的最后一个约定是\textbf{位置的保持}:如果我们不对某个下标求和,它将保持与之前相同的 位置(下标或上标)。因此,我们可以写 $y^j = a^j_k x^k$,但不能写 $f_j = a^j_k x^k$,因为下标 $j$ 必须保持为下标。

注意,为了计算两个向量的内积,仅知道它们的坐标是不够的。你还需要知道矩阵 $G$(通常称为\textbf{度量张量})。这与爱因斯坦记法一致:如果我们试图将 $(\xx, \yy)$ 写成标准内积,那么 $x^k y_k$ 的表达式意味着仅仅是坐标的乘积,因为为了求和,我们需要同时作为下标和上标的相同下标。另一方面,表达式 (4.4) 完全符合这个约定。

\subsubsection{4.3.2. 协变和逆变坐标~~上/下指标的升降}

让我们回忆一下,在实内积空间中,我们有一个基 $\{\bb_1, \bb_2, \dots, \bb_n\}$,以及它的对偶基 $\{\bb'_1, \bb'_2, \dots, \bb'_n\}$,$\bb'_k \in X$(我们通过里斯表示定理将对偶空间 $X'$ 与 $X$ 对应,所以 $\bb'_k$ 可以在 $X$ 中)。给定向量 $\xx \in X$,它可以表示为:
$$ (4.5)  \quad \xx = \sum_{k=1}^n (\xx, \bb'_k) \bb_k =: \sum_{k=1}^n x^k \bb_k, $$
以及:
$$(4.6)  \quad \xx = \sum_{k=1}^n (\xx, \bb_k) \bb'_k =: \sum_{k=1}^n x_k \bb'_k .$$
坐标 $x_k$ 被称为向量 $\xx$ 的\textbf{协变}(covariant)坐标,而坐标 $x^k$ 被称为\textbf{逆变}(contravariant)坐标。

现在问自己一个问题:如何从逆变坐标 $x^k$ 得到向量的协变坐标 $x_k$?

根据爱因斯坦记法,我们使用逆变坐标来处理向量,而协变坐标用于线性泛函(即当我们把向量 $\xx \in X$ 解释为线性泛函时)。我们知道 $ x_k = (\xx, \bb_k)$(见 (4.6)),因此:
$$ x_k = (\xx, \bb_k) = \left( \sum_j x^j \bb_j, \bb_k \right) = \sum_j x^j (\bb_j, \bb_k) = \sum_j g_{k,j} x^j $$
或者用爱因斯坦记法:
$$ x_k = g_{k,j} x^j $$
换句话说,

\fbox{\begin{minipage}{0.9\textwidth}
度量张量 $G$ 是从逆变坐标 $x^j$ 到协变坐标 $x_k$ 的变换矩阵。
\end{minipage}}

从逆变坐标获得协变坐标的操作被称为\textbf{下标的降}(lowering of the indices)。

注意公式 (4.4) 对于内积的解释:正如我们所知,对于向量 $\xx$,我们得到其协变坐标为 $x_j = g_{j,k} x^k$.~因此,$(\xx, \yy) = x_j y^j$.~类似地,由于 $G$ 是对称的,我们可以说 $y_k = g_{j,k} y^k$ 并且 $(\xx, \yy) = x^k y_k$.~换句话说,

\fbox{\begin{minipage}{0.9\textwidth}
为了计算两个向量的内积,首先需要使用度量张量 $G$ 来降低一个向量的下标,然后将其视为一个泛函,计算它在另一个向量上的值.
\end{minipage}}



当然,我们也可以从协变坐标 $x_j$ 变到逆变坐标 $x^j$(\textbf{上标的升})。由于 
$$(x^1, x^2, \dots, x^n)^T = G^{-1}(x_1, x_2, \dots, x_n)^T,$$
我们得到 
$$(x^1, x^2, \dots, x^n)^T = G^{-1}(x_1, x_2, \dots, x_n)^T,$$
所以这种情况下的坐标变换矩阵是 $G^{-1}$.~

我们知道,由于坐标变换矩阵就是度量张量,我们可以立即得出 $G^{-1}$ 是协变度量张量,即如果 $G^{-1} = \{g^{k,j}\}^{n}_{j,k=1}$,那么
 $$(\xx, \yy) = g^{k,j} x_j y_k.$$

\textbf{注记}~~
注意,如果从宏观角度来看,协变和逆变坐标是完全可互换的。这仅仅取决于我们选择哪一个基对作为“主要”基,哪一个作为“对偶”基。

选择什么作为“主要”对象,什么作为“对偶”对象,主要取决于公认的约定。

\textbf{注记 4.1}~~
爱因斯坦记法通常用于微分几何,特别是黎曼几何,其中向量被对应为速度,而余向量(线性泛函)被对应为一阶微分形式,见上面 4.2 节。这里的向量和余向量是明显不同的对象,构成了所谓的\textbf{切空间}和\textbf{余切空间}。

在黎曼几何中,我们接下来可以在切空间上引入内积(即度量张量,如果从坐标的角度来看),这允许我们对应向量和余向量(线性泛函)。在坐标表示中,这种对应是通过升降指标来完成的,如上所述。

\subsection{4.4. 结论}

让我们总结一下上面关于空间与其对偶是否不同的讨论。

简而言之,答案是“是的”,它们是不同的对象。虽然在本手册所讨论的有限维情况下,它们是同构的,但将空间与其对偶进行对应通常没有什么好处。

即使在 $\FF^n$ 最简单的情况下,认为 $\FF^n$ 的元素是列向量,而其对偶的元素是行向量(尽管在处理对偶空间元素时,我们经常将行向量垂直放置)也是有用的。
更显着的例子是 1.4.1 和 1.4.2 节中关于泰勒公式和拉格朗日插值的内容。在那里,你可以清楚地看到线性泛函确实与多项式是完全不同的对象,并且通过对应泛函与多项式几乎没有什么好处。

对于内积空间,情况有所不同,因为这样的空间可以\textbf{规范地}与其对偶进行对应。这种对应对于实内积空间是线性的,所以一个实内积空间与其对偶是规范同构的。对于复数空间,这种对应只是\textbf{共轭线性的},但它仍然非常有助于将线性泛函与向量进行对应,并利用内积空间结构和正交性、自伴性、正交投影等思想。

然而,有时即使在实内积空间的情况下,考虑空间及其对偶作为不同的对象也更自然。例如,在黎曼几何中,见上面注记 4.1,向量和余向量来自不同的对象,分别是速度和一阶微分形式。尽管引入度量张量允许我们对应向量和余向量,但有时更方便记住它们的起源,将它们视为不同的对象。

\begin{exer} \textbf{练习}~~

4.1. 设 
$$D = \sum_{k=1}^n v_k \frac{\partial}{\partial x_k}$$
是一个微分算子。利用链式法则,证明当我们改变基并用新坐标写出 $D$ 时,其系数 $v_k$ 按照向量的坐标变换规则变化。\end{exer}

\section{5. 多线性函数~~张量}

\subsection{5.1. 多线性函数}

\textbf{定义 5.1}~~
设 $V_1, V_2, \dots, V_p, V$ 是向量空间(在同一个域$\FF$上)。一个\textbf{多线性}(multilinear)($p$-线性)函数 $F$,具有 $p$ 个向量变量 $\vv_1, \vv_2, \dots, \vv_p, \vv_k \in V_k$,其目标空间为 $V$,在每个变量 $\vv_k$ 上都是线性的。换句话说,这意味着如果我们固定除 $\vv_k$ 之外的所有变量,我们得到一个线性映射,并且这对所有 $k = 1, 2, \dots, p$ 都成立。我们将使用符号 $L(V_1, V_2, \dots, V_p; V)$ 表示所有这些多线性函数的集合。

如果目标空间 $V$ 是标量域 $\FF$,我们称 $F$ 为\textbf{多线性泛函},或\textbf{张量}(tensor)。数字 $p$ 被称为多线性泛函(张量)的\textbf{次性}(valency)。因此,次性为 1 的张量是线性泛函,次性为 2 的张量称为\textbf{双线性型}。

\textbf{例子}~~
设 $\ff_k \in (V_k)'$.~定义一个多线性泛函 $F = \ff_1 \otimes \ff_2 \otimes \dots \otimes \ff_p$ 通过乘以泛函 $\ff_k$:
$$(5.1) \quad   \ff_1 \otimes \ff_2 \otimes \dots \otimes \ff_p (\vv_1, \vv_2, \dots, \vv_p) = \ff_1(\vv_1) \ff_2(\vv_2) \dots \ff_p(\vv_p),$$
其中 $\vv_k \in V_k,\quad k = 1, 2, \dots, p$.~多线性泛函 $\ff_1 \otimes \ff_2 \otimes \dots \otimes \ff_p$ 被称为泛函 $\ff_k$ 的\textbf{张量积}(tensor product)。

\subsubsection{5.1.1. 多线性函数构成向量空间}

注意到,在空间 $L(V_1, V_2, \dots, V_p; V)$ 中可以引入加法和标量乘法的自然运算:
$$ (F_1 + F_2)(\vv_1, \vv_2, \dots, \vv_p) := F_1(\vv_1, \vv_2, \dots, \vv_p) + F_2(\vv_1, \vv_2, \dots, \vv_p) $$
$$ (\alpha F_1)(\vv_1, \vv_2, \dots, \vv_p) := \alpha F_1(\vv_1, \vv_2, \dots, \vv_p) $$
其中 $F_1, F_2 \in L(V_1, V_2, \dots, V_p; V), \alpha \in \FF$.~

装备了这些运算后,空间 $L(V_1, V_2, \dots, V_p; V)$ 就是一个向量空间。

为了看出这一点,我们首先需要证明 $F_1 + F_2$ 和 $\alpha F_1$ 是多线性函数。由于“多线性”意味着它在每个参数上都是线性的(固定其他变量),这来源于线性变换的相应事实;即,线性变换的和以及线性变换的标量倍数是线性变换,参见第一章第四节。

然后很容易证明 $L(V_1, V_2, \dots, V_p; V)$ 满足所有向量空间的公理;我们只需要使用 $V$ 满足这些公理的事实。我们将细节留给读者作为练习。他/她可以参考第一章第四节,其中证明了线性变换集合满足公理 7。所有其他公理也得到满足的证明非常相似。

\subsubsection{5.1.2. $L(V_1, V_2, \dots, V_p; V)$ 的维度}

设 $\B_1, \B_2, \dots, \B_p$ 分别是 $V_1, V_2, \dots, V_p$ 中的基。由于线性变换由其在基上的作用定义,多线性函数 $F \in L(V_1, V_2, \dots, V_p; V)$ 由其在所有元组 
$$\bb^{1}_{j_1}, \bb^{2}_{j_2}, \dots, \bb^{p}_{j_p},\quad \bb^{k}_{j_k} \in \B_k$$
上的值定义。由于恰好有 
$$(\dim V_1)(\dim V_2) \dots (\dim V_p)$$ 
这样的元组,并且每个 $F(\bb^{1}_{j_1}, \bb^{2}_{j_2}, \dots, \bb^{p}_{j_p})$ 在(某个基下)由 $\dim V$ 个坐标确定。因此,我们可以得出 $F \in L(V_1, V_2, \dots, V_p; V)$ 由 $(\dim V_1)(\dim V_2) \dots (\dim V_p)(\dim V)$ 个项确定。换句话说,
$$ \dim L(V_1, V_2, \dots, V_p; V) = (\dim V_1)(\dim V_2) \dots (\dim V_p)(\dim V). $$
特别地,如果目标空间是标量域 $\FF$(即,如果我们处理多线性泛函),
$$ \dim L(V_1, V_2, \dots, V_p; \FF) = (\dim V_1)(\dim V_2) \dots (\dim V_p). $$
在 $L(V_1, V_2, \dots, V_p; \FF)$ 中找到一个基很容易。也就是说,设 $\B_k = \{\bb_{j}^{k}\}_{j=1}^{\dim V_k}$ 是 $V_k$ 中的一个基,设 $\B' = \{\tilde{\bb}_{j}^{k}\}_{j=1}^{\dim V_k}$ 是其对偶系统,$\tilde{\bb}_{j}^{k} \in V'_k$.~

\textbf{命题 5.2}~~
系统 $$\tilde{\bb}^{1}_{j_1} \otimes \tilde{\bb}^{2}_{j_2} \otimes \dots \otimes \tilde{\bb}^{p}_{j_p},\quad 1 \le j_k \le \dim V_k,\quad k = 1, 2, \dots, p$$
是空间 $L(V_1, V_2, \dots, V_p; \FF)$ 中的一个基。这里 $\tilde{\bb}^{1}_{j_1} \otimes \tilde{\bb}^{2}_{j_2} \otimes \dots \otimes \tilde{\bb}^{p}_{j_p}$ 是泛函的张量积,如 (5.1) 定义。

\textbf{证明}~~
我们想将 $F$ 表示为:
$$ (5.2)  \quad F = \sum_{j_1, j_2, \dots, j_p} \alpha^{j_1, j_2, \dots, j_p} \tilde{\bb}^{1}_{j_1} \otimes \tilde{\bb}^{2}_{j_2} \otimes \dots \otimes \tilde{\bb}^{p}_{j_p}$$
由于 $\tilde{\bb}_{j}(\bb_{l}) = \delta_{j,l}$,我们得到:
$$ (5.3)  \quad  \tilde{\bb}^{1}_{j_1} \otimes \tilde{\bb}^{2}_{j_2} \otimes \dots \otimes \tilde{\bb}^{p}_{j_p} (\bb^{1}_{j_1}, \bb^{2}_{j_2}, \dots, \bb^{p}_{j_p}) = 1 $$
并且
$$  (5.4)  \quad \tilde{\bb}^{1}_{j_1} \otimes \tilde{\bb}^{2}_{j_2} \otimes \dots \otimes \tilde{\bb}^{p}_{j_p} (\bb^{1}_{j'_1}, \bb^{2}_{j'_2}, \dots, \bb^{p}_{j'_p}) = 0 $$
对于任何不同于 $j_1, j_2, \dots, j_p$ 的指标集合 $j'_1, j'_2, \dots, j'_p$.

因此,将 (5.2) 应用于元组 $\{\bb^{1}_{j_1}, \bb^{2}_{j_2}, \dots, \bb^{p}_{j_p}\}$,我们得到:
$$ \alpha_{j_1, j_2, \dots, j_p} = F(\bb^{1}_{j_1}, \bb^{2}_{j_2}, \dots, \bb^{p}_{j_p}) ,$$
所以表示 (5.2) 是唯一的(如果存在的话)。

另一方面,定义 $\alpha_{j_1, j_2, \dots, j_p} := F(\bb^{1}_{j_1}, \bb^{2}_{j_2}, \dots, \bb^{p}_{j_p})$ 并使用 (5.3) 和 (5.4),我们可以看到等式 (5.2) 对所有形式为 $\{\bb^{1}_{j_1}, \bb^{2}_{j_2}, \dots, \bb^{p}_{j_p}\}$ 的元组都成立。因此,表示 (5.2) 确实成立,所以我们确实有一个基。

\subsection{5.2. 张量积}

\textbf{定义}~~
设 $V_1, V_2, \dots, V_p$ 是向量空间。空间 
$$V_1 \otimes V_2 \otimes \dots \otimes V_p$$
的\textbf{张量积}就是 $V'_1, V'_2, \dots, V'_p$ 的对偶空间的张量积 $L(V'_1, V'_2, \dots, V'_p; \FF)$ 的集合;这里 $V'_k$ 是 $V_k$ 的对偶。


\textbf{注记 5.3}~~
根据命题 5.2,如果 $\B_k = \{\bb^{k}_{j}\}_{j=1}^{\dim V_k}$ 是 $V_k$ 中的一个基,那么系统:
$$ (5.5) \quad \bb^{1}_{j_1} \otimes \bb^{2}_{j_2} \otimes \dots \otimes \bb^{p}_{j_p}, \quad 1 \le j_k \le \dim V_k, \quad k = 1, 2, \dots, p $$
是 $V_1 \otimes V_2 \otimes \dots \otimes V_p$ 中的一个基。

这里我们将向量 $\vv_k \in V_k$ 看作是 $V_k'$ 上的一个线性泛函;
向量的张量积 $\vv_1 \otimes \vv_2 \otimes \dots \otimes \vv_p$ 是根据 (5.1) 定义的。

\textbf{注记}~~
向量的张量积 $\vv_1 \otimes \vv_2 \otimes \dots \otimes \vv_p$ 在每个参数 $\vv_k$ 上显然是线性的。换句话说,映射 $(\vv_1, \vv_2, \dots, \vv_p) \mapsto \vv_1 \otimes \vv_2 \otimes \dots \otimes \vv_p$ 是一个取值于 $V_1 \otimes V_2 \otimes \dots \otimes V_p$ 的多线性泛函。我们将证明留给读者作为练习,见下面的问题 5.1。

\textbf{注记}~~
注意,向量张量积的集合 $\{\vv_1 \otimes \vv_2 \otimes \dots \otimes \vv_p : \vv_k \in V_k\}$ 严格小于 $V_1 \otimes V_2 \otimes \dots \otimes V_p$,见下面的问题 5.2。

\subsubsection{5.2.1. 将多线性函数提升到张量积上的线性变换}

\textbf{命题 5.4}~~
对于任何多线性函数 $F \in L(V_1, V_2, \dots, V_p; V)$,存在一个唯一的线性变换 $T : V_1 \otimes V_2 \otimes \dots \otimes V_p \to V$ 扩展 $F$,即满足:
$$ (5.6) \quad  F(\vv_1, \vv_2, \dots, \vv_p) = T \vv_1 \otimes \vv_2 \otimes \dots \otimes \vv_p,$$
对于所有向量 $\vv_k \in V_k,\quad 1 \le k \le p$ 的选择。

\textbf{注记}~~
如果 $T : V_1 \otimes V_2 \otimes \dots \otimes V_p \to V$ 是一个线性变换,那么显然函数 $F$, 
$$F(\vv_1, \vv_2, \dots, \vv_p) := T \vv_1 \otimes \vv_2 \otimes \dots \otimes \vv_p,$$
是 $L(V_1, V_2, \dots, V_p; V)$ 中的一个多线性函数。这直接源于表达式 $\vv_1 \otimes \vv_2 \otimes \dots \otimes \vv_p$ 在每个变量 $\vv_k$ 上是线性的。

\textbf{命题 5.4 的证明}~~
在基 (5.5) 上定义 $T$ 为:
$$ T \bb^{1}_{j_1} \otimes \bb^{2}_{j_2} \otimes \dots \otimes \bb^{p}_{j_p} = F(\bb^{1}_{j_1}, \bb^{2}_{j_2}, \dots, \bb^{p}_{j_p}) $$
然后通过线性将其扩展到整个空间 $V_1 \otimes V_2 \otimes \dots \otimes V_p$.~为了完成证明,我们需要证明 (5.6) 对所有向量 $\vv_k \in V_k, 1 \le k \le p$ 的选择都成立(我们现在知道只有当每个 $\vv_k$ 是 $\bb^{k}_{j_k}$ 之一时才成立)。

为了证明这一点,让我们将 $\vv_k$ 分解为:
$$ \vv_k = \sum_{j_k} \alpha^{k}_{j_k} \bb^{k}_{j_k}, \quad k = 1, 2, \dots, p .$$
使用每个变量的线性性质,我们得到:
$$ \vv_1 \otimes \vv_2 \otimes \dots \otimes \vv_p = \sum_{j_1, j_2, \dots, j_p} \alpha^{1}_{j_1} \alpha^{2}_{j_2} \dots \alpha^{p}_{j_p} \bb^{1}_{j_1} \otimes \bb^{2}_{j_2} \otimes \dots \otimes \bb^{p}_{j_p} ,$$
$$ F(\vv_1, \vv_2, \dots, \vv_p) = \sum_{j_1, j_2, \dots, j_p} \alpha^{1}_{j_1} \alpha^{2}_{j_2} \dots \alpha^{p}_{j_p} F(\bb^{1}_{j_1}, \bb^{2}_{j_2}, \dots, \bb^{p}_{j_p}) $$
所以根据 $T$ 的定义,恒等式 (5.6) 成立。

\textbf{5.2.2. 张量积的对偶}~~

正如人们可以很容易地看到的,张量积 $V_1 \otimes V_2 \otimes \dots \otimes V_p$ 的对偶是 $V'_1 \otimes V'_2 \otimes \dots \otimes V'_p$ 的张量积。

事实上,根据命题 5.4 和其后的注记,在多线性泛函 $L(V_1, V_2, \dots, V_p, \FF)$(即 $V'_1 \otimes V'_2 \otimes \dots \otimes V'_p$ 的元素)与线性变换 $T : V_1 \otimes V_2 \otimes \dots \otimes V_p \to \FF$(即 $V_1 \otimes V_2 \otimes \dots \otimes V_p$ 的对偶的元素)之间存在自然的\textbf{一一}对应关系。

注意,注记 5.3 和命题 5.2 中的基是对偶基 (分别为 $V_1 \otimes V_2 \otimes \dots \otimes V_p$ 和 $V'_1 \otimes V'_2 \otimes \dots \otimes V'_p$ )。了解对偶基可以让我们轻松地计算空间 $V_1 \otimes V_2 \otimes \dots \otimes V_p$ 和 $V'_1 \otimes V'_2 \otimes \dots \otimes V'_p$ 之间的\textbf{对偶性}(duality),即表达式 $\langle \xx, \xx' \rangle, \xx \in V_1 \otimes V_2 \otimes \dots \otimes V_p, \xx' \in V'_1 \otimes V'_2 \otimes \dots \otimes V'_p$.~

\subsection{5.3. 协变和逆变张量}

设 $X_1, X_2, \dots, X_p$ 是向量空间,设 $V_k$ 是 $X_k$ 或 $X'_k$, $k = 1, 2, \dots, p$.~对于多线性函数 $F \in L(V_1, V_2, \dots, V_p; V)$,我们说它对于变量 $\vv_k \in V_k$ 是\textbf{协变的},如果 $V_k = X_k$,并且对于这个变量是\textbf{逆变的},如果 $V_k = X'_k$.~

如果一个多线性函数在所有变量上都是协变的(逆变的),我们称该多线性函数是协变的(逆变的)。一般地,如果一个函数在 $r$ 个变量上是协变的,在 $s$ 个变量上是逆变的,我们称该多线性函数是 $r$-协变的~ $s$-逆变的(或简单地称为 $(r, s)$ 多线性函数,或称其次性为 $(r, s)$)。

因此,线性泛函可以解释为 1-协变张量(回忆一下,我们用\textbf{张量}这个词来指代目标空间是标量域 $\FF$ 的情况)。根据对偶性,向量可以解释为 1-逆变张量。

\textbf{注记}~~

一开始,这个术语可能看起来有点令人困惑:如果一个变量是向量(而不是泛函),它是一个协变变量,但却是一个逆变对象。但是请注意,我们这里说的不是“协变变量”:我们说的是,如果 $\vv_k \in X_k$,那么该\textbf{多线性函数在变量 $\vv_k$ 上是协变的}。

所以,协变对象不是 $\vv_k$,而是张量中我们放入它的“槽”(slot)!所以没有矛盾,我们将逆变对象放入协变槽,反之亦然。


有时,稍微滥用术语,人们会谈论协变(逆变)变量或参数。但通常的意思是相应的张量中的“槽”是协变的(逆变的),而不是作为对象的变量。

\subsubsection{5.3.1. 线性变换作为张量}
一个线性变换 $T : X_1 \to X_2$ 可以被解释为一个 1-协变 1-逆变张量。也就是说,双线性泛函 $F$, 
$$F(\xx_1, \xx'_2) := \langle T\xx_1, \xx'_2 \rangle,\quad \xx_1 \in X_1,\quad \xx'_2 \in X'_2$$
在第一个变量 $\xx_1$ 上是协变的,在第二个变量 $\xx'_2$ 上是逆变的。

反之,

\textbf{命题 5.5}~~
给定一个 1-1 张量 $F \in L(X_1, X'_2; \FF)$,存在一个唯一的线性变换 $T : X_1 \to X_2$ 使得
$$(5.7) \quad  F(\xx_1, \xx'_2) := \langle T\xx_1, \xx'_2 \rangle $$
对于所有 $\xx_1 \in X_2,\quad \xx'_2 \in X'_2$ 的选择成立。

\textbf{证明}~~
首先,请注意,由于引理 1.3,唯一性是平凡的推论,参见上面问题 3.1。所以我们只需要证明$T$的存在性。

设 $B_k = \{\bb^{k}_{j}\}_{j=1}^{\dim X_k}$ 是 $X_k$ 中的一个基,设 $B'_k = \{\tilde{\bb}^{k}_{j}\}_{j=1}^{\dim X_k}$ 是 $X'_k$ 中的对偶基,$k=1, 2$.~然后定义矩阵 $A = \{a_{k,j}\}_{k=1}^{\dim X_2}~_{j=1}^{\dim X_1}$ 为:
$$ a_{k,j} = F(\bb^{1}_{j}, \tilde{\bb}^{2,k}) $$
将 $T$ 定义为具有矩阵 $[T]_{\B_2, \B_1} = A$ 的算子。显然(参见注记 1.5):
$$(5.8)\quad \langle T \bb^{1}_{j}, \tilde{\bb}^{2}_{k} \rangle = a_{k,j} = F(\bb^{1}_{j}, \tilde{\bb}^{2}_{k}) $$
这暗示了等式 (5.7)。通过将 $\xx_1 = \sum_j \alpha_j \bb_j$ 和 $\xx'_2 = \sum_k \beta_k \bb'_k$ 分解,并利用每个参数的线性,可以很容易地看出这一点。

另一种更“高深”的解释是,(5.7) 两边的张量在 $X_1 \otimes X'_2$ 的基上(参见注记 5.3 关于基)是相同的,所以它们是相等的。更准确地说,人们应该将双线性形式提升为线性变换(泛函) $X_1 \otimes X'_2 \to \FF$(参见命题 5.4),并且由于变换在基上是相同的,所以它们是相等的。

也可以提出一种替代的、无坐标的证明 $T$ 的存在性,沿着对偶空间(见 3.1.3 节)的无坐标定义的思路。也就是说,如果我们固定 $\xx_1$,函数 $F(\xx_1, \xx'_2)$ 在 $\xx'_2$ 上是线性的,所以它是一个 $X'_2$ 上的线性泛函,即 $X_2$ 中的一个向量。

让我们称这个向量为 $T(\xx_1)$.~所以我们定义了一个变换 $T : X_1 \to X_2$.~可以很容易地通过基本上重复 3.1.3 节中的推理来证明 $T$ 是一个线性变换。等式 (5.7) 从 $T$ 的定义中自动得出。

\textbf{注记}~~
注意,我们也说 $F$ 从命题 5.5 定义的不是变换 $T$,而是它的伴随。先验地,不假设任何东西(如变量的顺序及其解释),我们就无法区分一个变换和它的伴随。

\textbf{注记}~~
注意,如果我们想遵循爱因斯坦记法,变换 $T$ 的矩阵 $A = [T]_{\B_2, \B_1}$ 的项 $a_{j,k}$ 应该写成 $a^j_k$,那么,如果 $x^k, k = 1, 2, \dots, \dim X_1$ 是向量 $\xx \in X_1$ 的坐标,那么 $\yy = T\xx$ 的第 $j$ 个坐标由下式给出:
$$ y^j = a^j_k x^k .$$
(这里我们跳过了求和符号,但我们指的是 $k$ 上的求和)。还请注意,我们保持了指标的位置,所以 $j$ 指标留在上面。指标 $k$ 没有出现在等式左侧,因为它在右侧被求和消掉(kill)了。

类似地,如果 $x'_j, j = 1, 2, \dots, \dim X_2$ 是向量 $\xx' \in X'_2$ 的坐标,那么 $\yy' = T'\xx'$ 的第 $k$ 个坐标由下式给出:
$$ y_k = a^j_k x_j .$$
(再次,跳过 $j$ 上的求和)。同样,由于我们保持了指标的位置,所以在 $y_k$ 中的指标 $k$ 是下标。

注意,由于 $\xx \in X_1$ 且 $\yy = T\xx \in X_2$ 是向量,根据爱因斯坦记法的约定,其坐标中的指标确实应该写成上标。

类似地,$\xx' \in X'_2$ 且 $\yy' = T'\xx' \in X'_1$ 是\textbf{余向量},所以其坐标中的指标应该写成下标。

爱因斯坦记法强调了上一注中提到的事实,即一个 1-协变 1-逆变张量同时给我们一个线性变换及其伴随:表达式 $a^j_k x^k$ 给出了 $T$ 的作用,而 $a^j_k x_j$ 给出了其伴随 $T'$ 的作用。

\subsubsection{5.3.2. 多线性变换作为张量}
更一般地,任何多线性变换都可以被解释为一个张量。也就是说,给定一个多线性变换 $F \in L(V_1, V_2, \dots, V_p; V)$,我们可以定义张量 $\tilde{F} \in L(V_1, V_2, \dots, V_p, V'; \FF)$ 为:
$$(5.9) \quad   \tilde{F}(\vv_1, \vv_2, \dots, \vv_p, \vv') = \langle F(\vv_1, \vv_2, \dots, \vv_p), \vv' \rangle, \quad \vv_k \in V_k, \vv' \in V'.$$

反之,

\textbf{命题 5.6}~~
给定一个张量 $\tilde{F} \in L(V_1, V_2, \dots, V_p, V'; \FF)$,存在一个唯一的多线性变换 $F \in L(V_1, V_2, \dots, V_p; V)$ 使得 (5.9) 成立。

\textbf{证明}~~
根据命题 5.4,张量 $\tilde{F}$ 可以扩展为一个线性变换(泛函) $\tilde{T} : V_1 \otimes V_2 \otimes \dots \otimes V_p \otimes V' \to \FF$,使得
$$ \tilde{F}(\vv_1, \vv_2, \dots, \vv_p, \vv') = \tilde{T}(\vv_1 \otimes \vv_2 \otimes \dots \otimes \vv_p \otimes \vv') $$
对于所有 $\vv_k \in V_k$, $\vv' \in V'$.~

如果 $\ww \in W := V_1 \otimes V_2 \otimes \dots \otimes V_p$ 且 $\vv' \in V'$,那么 
$$\ww \otimes \vv' \in V_1 \otimes V_2 \otimes \dots \otimes V_p \otimes V'.$$
因此,我们可以定义一个双线性泛函(张量) $G \in L(W, V'; \FF)$ 为:
$$ G(\ww, \vv') := \tilde{T}(\ww \otimes \vv) $$
根据命题 5.5,$G$ 产生一个线性变换,即存在一个唯一的线性变换 $T : W \to V$ 使得
$$ G(\ww, \vv') = \langle T\ww, \vv' \rangle \quad \forall \ww \in W, \forall \vv' \in V' $$
而线性变换 $T$ 通过 
$$F \in L(V_1, V_2, \dots, V_p; V)$$
定义为
$$ F(\vv_1, \vv_2, \dots, \vv_p) = T(\vv_1 \otimes \vv_2 \otimes \dots \otimes \vv_p), $$
见命题 5.4 后的注。

变换 $F$ 的唯一性,如同命题 5.5 中一样,是引理 1.3 的一个平凡推论。我们将细节留给读者作为练习。


这个章节展示了

\fbox{\begin{minipage}{0.9\textwidth}
张量是多线性代数中的通用对象,因为任何多线性变换都可以被解释为一个张量,反之亦然。
\end{minipage}}


\begin{exer} \textbf{练习}~~

5.1. 证明向量的张量积 $\vv_1 \otimes \vv_2 \otimes \dots \otimes \vv_p$ 在每个参数 $\vv_k$ 上是线性的。

5.2. 证明向量张量积的集合 $\{\vv_1 \otimes \vv_2 \otimes \dots \otimes \vv_p : \vv_k \in V_k\}$ 严格小于 $V_1 \otimes V_2 \otimes \dots \otimes V_p$.~

5.3. 证明命题 5.6 中的变换 $F$ 是唯一的。\end{exer}

\section{6. 张量的坐标变换公式}

多线性协变和逆变变量的区分的主要原因是,在改变基时,它们的坐标根据不同的规则变化。因此,协变和逆变向量的项也根据不同的规则变化。

在本节中,我们将详细研究这一点。请注意,坐标表示极其重要,原因是,例如所有数值计算(与理论研究不同)都是使用某种坐标系进行的。

\subsection{6.1. 张量的坐标表示}
设 $F$ 是一个 $r$-协变 $s$-逆变张量,$r+s=p$.~设 $\xx_1, \dots, \xx_r$ 是协变变量 ($\xx_k \in X_k$), $\ff_1, \dots, \ff_s$ 是逆变变量 ($\ff_k \in X'_k$)。让我们先写协变变量,所以张量将被写成 $F(\xx_1, \dots, \xx_r, \ff_1, \dots, \ff_s)$.~对于 $k=1, 2, \dots, p$,固定 $X_k$ 中的基 $\B_k = \{\bb^{(k)}_j\}^{\dim X_k}_{j=1}$,设 $\B'_k = \{\tilde{\bb}^{(k)}_j\}^{\dim X_k}_{j=1}$ 为 $X'_k$ 中的对偶基。

对于向量 $\xx_k \in X_k$,设 $x^j_{(k)}, j = 1, 2, \dots, \dim X_k$ 是其在基 $\B_k$ 下的坐标,类似地,如果 $\ff_k \in X'_k$,设 $f_k^{(k)}, j = 1, 2, \dots, \dim X_k$ 是其在对偶基 $\B'_k$ 下的坐标(注意,为了与爱因斯坦记法保持一致,向量的坐标用上标索引,协向量的坐标用下标索引)。

\textbf{命题 6.1}~~
记:
$$ (6.1)  \quad \phi_{k_1, \dots, k_s}^{j_1, \dots, j_r} := F(\bb^{(1)}_{j_1}, \dots, \bb^{(r)}_{j_r}, \tilde{\bb}^{(r+1)}_{k_1}, \dots, \tilde{\bb}^{(r+s)}_{k_s})$$
那么,在爱因斯坦记法下:
$$(6.2)  \quad  F(\xx_1, \dots, \xx_r, \ff_1, \dots, \ff_s) = \phi_{k_1, \dots, k_s}^{j_1, \dots, j_r} x_{(1)}^{j_1} \dots x_{(r)}^{j_r} f^{(1)}_{k_1} \dots f^{(s)}_{k_s}$$
(这里的求和是指对指标 $j_1, \dots, j_r$ 和 $k_1, \dots, k_s$)。

注意我们使用符号 $(1), \dots, (r)$ 和 $(1), \dots, (s)$ 来强调它们不是指标:括号中的数字仅仅表示参数的顺序。因此,(6.2) 的右侧没有剩下任何指标(所有指标都在求和中使用了),所以它只是一个数字(对于固定的 $\xx_k$ 和 $\ff_k$)。

\textbf{命题 6.1 的证明}~~
为了证明 (6.1) 意味着 (6.2),我们首先注意到 (6.1) 意味着 (6.2) 在 $\xx_j$ 和 $\ff_k$ 是对应基的元素时成立。通过将每个参数 $\xx_j$ 和 $\ff_k$ 分解到相应的基中,并使用每个参数的线性,我们可以很容易地得到 (6.2)。这个计算相当简单,但由于指标很多,公式可能会非常大,看起来可能非常吓人。


为了避免写出太多巨大的公式,我们将这个计算留给读者作为练习。

我们不希望读者感到被欺骗,所以我们提供了另一种更“高深”(抽象)的解释,它不需要任何计算!
也就是说,让我们注意到 (6.2) 等式两边的表达式定义了张量。根据命题 5.4,它们可以提升为张量积 $X_1 \otimes \dots \otimes X_r \otimes X'_{r+1} \otimes \dots \otimes X'_{r+s}$ 上的线性函数。

重新表述我们在此证明的开头讨论的内容,我们可以说 (6.1) 意味着函数在基 $$\bb^{(1)}_{j_1} \otimes \dots \otimes \bb^{(r)}_{j_r} \otimes \tilde{\bb}^{(r+1)}_{k_1} \otimes \dots \otimes \tilde{\bb}^{(r+s)}_{k_s}$$
上是相同的,所以函数(因此张量)是相等的。

张量 $F$ 的项 $\phi_{k_1, \dots, k_s}^{j_1, \dots, j_r}$ 被称为张量 $F$ 在基 $\B_k, k=1, 2, \dots, p$ 下的项。

现在,设 $\A_k$(以及 $\A'_k$)分别是 $X_k$(以及 $X'_k$)中的一个基。我们想研究当基从 $\B_k$ 改变到 $\A_k$ 时,张量 $F$ 的项如何变化。

\subsection{6.2. 爱因斯坦记法中的坐标变换公式}

首先,让我们考虑上面 1.1.1 节中更熟悉的向量和线性泛函的情况,但使用爱因斯坦记法将其写下来。设 $X$ 中有两个基 $\B$ 和 $\A$,设 $$A = [I]_{\A,\B}$$
是从 $\B$ 到 $\A$ 的坐标变换矩阵。对于向量 $\xx \in X$,设 $x^k$ 是其在基 $\B$ 下的坐标,设 $\tilde{x}^k$ 是其在基 $\A$ 下的坐标。
类似地,对于 $\ff \in X'$,设 $f_k$ 是其在基 $\B'$ 下的坐标,设 $\tilde{f}_k$ 是其在基 $\A'$ 下的坐标($\B'$ 和 $\A'$ 分别是 $\B$ 和 $\A$ 的对偶基)。

设 $(A)^j_k$ 为矩阵 $A$ 的项:为了与爱因斯坦记法保持一致,上标 $j$ 表示行的编号。那么我们可以将坐标变换公式写成:
$$ (6.3)  \quad \tilde{x}^j = (A)^j_k x^k.$$
类似地,设 $(A^{-1})^k_j$ 为 $A^{-1}$ 的项:再次,上标表示行的编号。那么我们可以将对偶空间的坐标变换公式写成:
$$ (6.4)  \quad \tilde{f}_j = (A^{-1})^k_j f_k;$$
这里的求和是在指标 $k$ 上(即沿着 $A^{-1}$ 的列),所以这种情况下的坐标变换矩阵确实是 $(A^{-1})^T$.~

让我们强调我们在这里没有证明任何东西:我们只是用爱因斯坦记法重写了第一章 1.1.1 节中的公式 (1.1)。

\textbf{注记}~~
虽然在接下来的内容中不需要,但让我们多玩玩爱因斯坦记法。也就是说,$$A^{-1}A = I \quad \text{和}\quad AA^{-1} = I$$
这两个方程可以分别用爱因斯坦记法重写为:
$$ (A)^j_k (A^{-1})^k_l = \delta_{j,l}\quad
\text{和}
\quad (A^{-1})^k_j (A)^j_l = \delta_{k,l}. $$


\subsection{6.3. 张量的坐标变换公式}


现在我们准备给出一般张量的坐标变换公式。

对于 $k = 1, 2, \dots, p := r+s$,设 $A_k = [I]_{\A,\B}$ 为坐标变换矩阵,设 $A^{-1}_k$ 为其逆矩阵。

像在 6.2 节中一样,我们用 $(A)^j_k$ 表示矩阵 $A$ 的项,约定上标给出行的编号。

\textbf{命题 6.2}~~
给定一个 $r$-协变 $s$-逆变张量 $F$,设 
$$\phi^{k_1, \dots, k_s}_{j_1, \dots, j_r}\quad \text{和} \quad \tilde{\phi}^{k_1, \dots, k_s}_{j_1, \dots, j_r}$$
分别是其在基 $\B_k$(旧的)和 $\A_k$(新的)下的项。在上面的记法中:
$$ \tilde{\phi}^{k_1, \dots, k_s}_{j_1, \dots, j_r} = \phi^{k'_1, \dots, k'_s}_{j'_1, \dots, j'_r} (A^{-1}_1)^{j'_1}_{j_1} \dots (A^{-1}_r)^{j'_r}_{j_r} (A_{r+1})^{k_1}_{k'_1} \dots (A_{r+s})^{k_s}_{k'_s} $$
(这里的求和是在指标 $j'_1, \dots, j'_r$ 和 $k'_1, \dots, k'_s$ 上)。

由于公式中有许多指标,这个命题看起来非常复杂。然而,如果理解了主要思想,公式就会变得相当简单且易于记忆。

为了解释主要思想,让我们稍微滥用语言,用“通俗英语”表达这个公式:

\fbox{\begin{minipage}{0.9\textwidth}
为了用“旧”张量项 $\phi^{k_1, \dots, k_s}_{j_1, \dots, j_r}$ 来表示“新”张量项 $\tilde{\phi}^{k_1, \dots, k_s}_{j_1, \dots, j_r}$,对于每个\textbf{协变}指标(下标)需要应用协变规则 (6.4),对于每个\textbf{逆变}指标(上标)需要应用逆变规则 (6.3)。
\end{minipage}}



\textbf{命题 6.2 的证明}~~
非正式地,证明的思路非常简单:我们一次只改变一个基,每次应用坐标变换公式 (6.3) 或 (6.4),取决于张量在相应变量上是协变还是逆变。

为了写出严格的正式证明,我们将使用关于 $r$ 和 $s$(张量的协变和逆变参数的数量)的归纳法。命题在 $r=1, s=0$ 和 $r=0, s=1$ 时成立,分别参见 (6.4) 或 (6.3)。

现在假设命题对某些 $p$ 和 $s$ 已经证明,我们来证明 $r+1, s$ 和 $r, s+1$ 的情况。

我们来处理后者,前者类似。主要思想是,我们首先改变 $p=r+s$ 个基并使用归纳假设;然后我们改变最后一个基并使用 (6.3)。

也就是说,设 $\hat{\phi}^{k_1, \dots, k_s, k_{s+1}}_{j_1, \dots, j_r}$ 是一个 $(r, s+1)$ 张量 $F$ 在基 $\A_1, \dots, \A_p, \B_{p+1}$ 下的项,$p=r+s$.~

让我们固定指标 $k_{s+1}$,并考虑张量 $F(\xx_1, \dots, \xx_r, \ff_1, \dots, \ff_s, \tilde{\bb}^{(r+s+1)}_{k_{s+1}})$ 的 $r$-协变 $s$-逆变函数(其中 $\xx_1, \dots, \xx_r, \ff_1, \dots, \ff_s$ 是变量)。
显然 
$$\phi^{k_1, \dots, k_s, k_{s+1}}_{j_1, \dots, j_r}\quad \text{和} \quad \hat{\phi}^{k_1, \dots, k_s, k_{s+1}}_{j_1, \dots, j_r}$$ 
是这个函数在基 $\B_1, \dots, \B_p$ 和 $\A_1, \dots, \A_p$ 下的项(你能看出为什么吗?)。这里指标 $k_{s+1}$ 是固定的。

根据归纳假设:
$$(6.5)\quad \hat{\phi}^{k_1, \dots, k_s, k_{s+1}}_{j_1, \dots, j_r} = \phi^{k'_1, \dots, k'_s, k_{s+1}}_{j'_1, \dots, j'_r} (A^{-1}_1)^{j'_1}_{j_1} \dots (A^{-1}_r)^{j'_r}_{j_r} (A_{r+1})^{k_1}_{k'_1} \dots (A_{r+s})^{k_s}_{k'_s} $$
注意,我们没有对指标 $k_{s+1}$ 做任何假设,所以 (6.5) 对所有 $k_{s+1}$ 都成立。

现在让我们固定指标 $j_1, \dots, j_r, k_1, \dots, k_s$,并考虑变量 $\ff_{s+1}$ 的 1-逆变张量 $$F(\aaa^{(1)}_{j_1}, \dots, \aaa^{(r)}_{j_r}, \tilde{\aaa}^{(r+1)}_{k_1}, \dots, \tilde{\aaa}^{(r+s)}_{k_s}, \ff_{s+1}).$$
这里 $\aaa^{(k)}_j$ 是基 $\A_k$ 中的向量,$\tilde{\aaa}^{(k)}_j$ 是对偶基 $\A'_k$ 中的向量。

再一次,很容易看出 
$$\hat{\phi}^{k_1, \dots, k_s, k_{s+1}}_{j_1, \dots, j_r}\quad \text{和} \quad \tilde{\phi}^{k_1, \dots, k_s, k_{s+1}}_{j_1, \dots, j_r}$$
$j_{s+1}=1, 2, \dots, \dim X_{p+1}$,是这个泛函在基 $\B_{p+1}$ 和 $\A_{p+1}$ 下的指标。根据 (6.3):
$$ \tilde{\phi}^{k_1, \dots, k_s, k_{s+1}}_{j_1, \dots, j_r} = \hat{\phi}^{k_1, \dots, k_s, k'_{s+1}}_{j_1, \dots, j_r} (A_{p+1})^{k_{s+1}}_{k'_{s+1}}, $$
由于我们没有对指标 $j_1, \dots, j_r, k_1, \dots, k_s$ 做任何假设,所以上述恒等式对它们的所有组合都成立。将此与 (6.5) 相结合,我们得到该命题对 $(r, s+1)$ 次张量成立。

$(r+1, s)$ 次张量的情况也是完全一样的处理方法:唯一的区别是最后我们得到一个 1-协变张量,并使用 (6.4) 而不是 (6.3)。

