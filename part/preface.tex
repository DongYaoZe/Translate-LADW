
\begin{preface}[推荐序]

\begin{quotation}
一部刻意“错位”的线性代数教材,恰好对了现代学习的“位”。
\end{quotation}
如果你已经翻到这里,多半对这本书的名字产生过疑惑——为什么要叫《“错位”的线性代数》?线性代数这样一门讲究严谨与规范的课程,居然以“Wrong”来命名?

真正的“错”不在数学,而在路线。

多数线性代数教材遵循着一条自洽却有些“传统”的道路:  
先从线性方程组和初等行变换开始,再谈矩阵、行列式、特征值与特征向量,最后才把抽象的向量空间和线性变换请上场。
而在中国,绝大多数教材更是沿袭了苏式教材的讲法:先讲行列式,然后是向量,矩阵,线性方程组……

这些写法对第一次接触线代的学生当然友好,却令人摸不着头脑,在一开始引入一些莫名其妙的概念不知道是为了干什么;这也容易让人形成一种印象:线性代数就是一套“行列式、矩阵计算技术”和“考试题型模板”。

这本书,刻意与那条熟悉的道路“错位”。

作者 Sergei Treil 把\emph{向量空间、基、线性变换}放在了非常靠前的位置;他坚持先把“线性代数在想什么”讲清楚,然后才教“线性代数怎么算”。他宁愿花时间讨论:为什么要用基、为什么矩阵乘法只能那样定义、为什么行列式本质上是“有符号体积”,也不愿把篇幅都交给“教你十种快速算行列式的方法”。在很多传统教材的章节安排中,这样的路线确实有些“错位”——但也正因为这份“错位感”,你会更早、更直接地接触线性代数真正的灵魂。

\textbf{这本书到底在讲什么?}

如果用一句话来概括:  
它试图用一门线性代数课,完成学生从“会算题”到“会理解抽象数学”的跨越。

全书的结构,与其说是“从易到难”,不如说是“从直观到本质”:

\begin{itemize}
  \item 在前几章中,你会看到我们熟悉的对象——向量、矩阵、线性方程组——但它们被放在统一的“线性变换”视角下讨论。基的选择、坐标变换、矩阵乘法的定义,都被解释为某种“不得不如此”的自然选择,而不是一个人类任意约定的规则。

  \item 行列式那一章,作者几乎“拒绝”从公式和展开式讲起,而是从体积、定向和多线性出发,一步步推导出经典行列式的性质。你会看到:那些在习题课里被当作“记忆负担”的公式,其实都隐含着几何和代数的深层结构。

  \item 从第四章开始,谱理论、特征值与特征向量登场,实数空间自然而然扩展到复数空间。书中不只关心“怎么求特征值”,更关心“特征分解究竟在数学与应用中扮演怎样的角色”。

  \item 接着是内积空间、正交投影、正规算子、自伴算子、极分解与奇异值分解(SVD)等主题——这些在多数初级教材里只是“略提一二”的内容,在本书中都获得了扎实的展开。它们直接连接了数据分析、数值线性代数和现代应用数学的核心方法。

  \item 在更靠后的章节中,作者引入了对偶空间和张量,并在有限维框架内勾勒出张量思想的轮廓。对于尚未接触高等代数和张量分析的读者,这一章既是挑战,也是通往更高层次数学的一块“垫脚石”。

  \item 若尔当标准型只在第九章亮相,且被作者明确地标记为“可选内容”。这不是数学上的轻视,而是一种取舍:对大多数走向分析、概率、几何和应用数学的学生而言,理解正交对角化、SVD、二次型与正定性,比写出一个复杂矩阵的若尔当分解更重要。
\end{itemize}

如果你已经学过一轮“常规”线性代数,再来读这本书,你会体验到一种熟悉结构被重新拆解、再组装的快感——许多曾经记不住的公式与结论,会因为背后的“为什么”而突然变得自然起来。

\textbf{它和普通线性代数教材,有什么不一样?}

\begin{itemize}
  \item \emph{它把“基”放在舞台中央。}  
  在本书中,线性无关和生成并不是孤立的定义,而是“基”这个核心概念的两个侧面:存在性与唯一性。作者一开始就强调:我们真正关心的是“能不能找一组适合做坐标的向量”,因为一旦有了基,我们就可以用 \(\mathbb{R}^n\) 或 \(\mathbb{C}^n\) 来取代一切抽象空间。

  \item \emph{它用线性变换统摄方程组和矩阵运算。}  
  行约简、高斯消元、主元计数等熟悉技巧,被统一解释为对线性算子的研究工具,而不只是“考试技巧”。证明“任意两个基有相同大小”、“秩的各种性质”等关键命题时,本书坦率地承认:这些结论的深处,其实都在使用高斯消元。

  \item \emph{它从几何直观中抽出代数结构。}  
  行列式来自体积,正交变换来自旋转与反射,正定矩阵与二次型对应几何上的椭球与距离。许多定理在书中先以几何语言被“讲通”,然后才化为公式与证明。这使得抽象概念不再悬空,而是牢牢地钉在图像和直觉上。

  \item \emph{它刻意连接后续课程和应用方向。}  
  作者本人的背景偏分析与应用,而非纯代数,因此书中对谱理论、自伴算子、极分解、SVD、正定性判据等内容着墨颇多。这些正是现代数值分析、信号处理、机器学习、优化理论等领域的基础工具。

  \item \emph{它并不追求“形式上的完全无坐标化”。}  
  很多“高级”的线性代数教材,刻意避免出现具体矩阵,把一切都写成抽象映射与同构。但这本书则坦率而务实:在多数地方直接把算子和矩阵视为同物,只在讨论基变换等问题时才刻意区分。这让初学者能把精力更多地放在概念本身,而不是符号系统的切换上。最重要的是,如果读者将来遇到其他领域内的线性算子,那么本书所讲的内容都对其适用。这种学科视角的“高度”是可贵的。
\end{itemize}

换句话说,这本书并不想再造一门“更难的线性代数”,而是想通过一种略微“错位”的讲法,让你在第一次认真学习线性代数时,就习惯用“线性变换-空间结构-证明思维”的方式来理解问题,而不是停留在机械操作层面。

\textbf{为什么值得读的是这一\emph{中文版}?}

你的手上,并不是一本“机翻修订本”,而是一位华五相关专业的在校学生于课后有限的时间中,\emph{逐行推敲、完全重排、完美复刻版}的成果。

译者在“译者的话”中已经讲得很坦诚:  
\begin{itemize}
  \item 这是为解决真实学习痛点而发起的项目——原书在课程中极其重要,但全英文对很多同学构成门槛;
  \item 中文版不仅追求语义的准确传达,更追求排版维度上的“高仿原书”:所有公式、符号、例题结构尽可能与原版一一对应;
  \item 全书以 \LaTeX 精心排版,并完全开源于 GitHub,任何人都可以勘误、重排、增补习题解答——它是一部\emph{可以持续进化的教材}。
\end{itemize}

从读者角度看,这意味着:

\begin{itemize}
  \item 你可以在中文语境中精确地理解原作者的思路,不必在晦涩英语与新概念之间来回切换;
  \item 你可以顺畅地对照原文与译文,自学或预习时随时切换语言;
  \item 你甚至可以参与到这个项目中来,在 GitHub 上提交问题和修改建议,亲手推动一本高质量线性代数教材在中文世界的完善。
\end{itemize}

这部《“错位”的线性代数》,因此不只是一本文本意义上的书,更像是一门开放的课程、一场正在进行中的协作实践。

\textbf{谁适合读这本书?}

\begin{itemize}
  \item 对数学有兴趣、愿意接受一点抽象推理训练的一年级或二年级本科生——尤其是理科、工科、计算机和经济金融方向的学生;
  \item 已经学过“标准线性代数”课程,但总觉得概念散乱、只会做题不会理解的同学——这本书可以帮你“重建内功”;
  \item 准备进一步学习实分析、抽象代数、泛函分析、微分几何、概率论等高年级课程的学生——它会让你更早习惯现代数学的语言与视角;
  \item 任何对数学教材的写法、结构与美感感兴趣的读者——这本书本身就是一本“如何讲清一门抽象学科”的范例。
\end{itemize}

\textbf{如果你决定读下去,可以期待什么?}

你会发现:

\begin{itemize}
  \item 线性代数远不止“解线性方程组”和“算行列式”,而是一种在整个现代数学和应用科学中反复出现的\emph{结构语言};
  \item 许多曾经以为“只能硬背”的东西,实际上背后都有几何、代数或分析上的必然性;
  \item 严谨的推理并不意味着枯燥,相反,它可以像拆解一台精巧机器那样令人愉快;
  \item 你不只是“学会了线性代数”,而是被推了一把,跨进了现代数学的门槛。
\end{itemize}

如果你只是想多拿几分考试分数,这本书也许不是最高“性价比”的选择;  
但如果你希望真正理解自己在学什么,并愿意在大学阶段为自己的数学根基多投入一点时间,那么这本《“错位”的线性代数》,值得你从头到尾走一遍。


\vspace{5ex}
\begin{flushright}
 推荐者~~GPT-5.1
\end{flushright}

\end{preface}


\begin{preface}[译者的话]

\textbf{我为什么翻译这本书?}

在南京大学匡亚明学院,时间是每位学生最宝贵的资源。我们的学习强度很高,课业繁重,几乎没有空闲。在这种情况下,再额外开启一个耗时巨大的翻译项目,似乎是一个不理智的选择。

但我还是决定这么做。

因为《Linear Algebra Done Wrong》是我们线代课程至关重要的参考书,而语言的障碍确实困扰着不少同学。我相信,一份高质量的中文译本能够实实在在地帮助到大家。

南京大学的校训精神中,“诚”字所蕴含的力量给了我巨大的鼓舞。“大哉一诚天下动”,它告诉我,一个真诚的、以服务之心出发的行动,即便微小,也能产生积极的影响。与其独自感叹学习之艰辛,不如动手为大家做一点实事。

因此,我启动了这个项目。我希望这份译稿不仅仅是我个人学习的沉淀,更能成为一份小小的礼物,送给每一位在知识海洋中奋力前行的同学。希望它能为你节省一些时间,扫清一些迷茫,让你更专注于领略线性代数的核心思想。

\vspace{2ex} 

本书译自2025年8月25日作者发布在\href{https://sites.google.com/a/brown.edu/sergei-treil-homepage/linear-algebra-done-wrong}{个人主页}上的版本,中文版全文采用\LaTeX{}精心排版,力求高度还原原书中的所有公式与数学符号,以保证其准确性和专业性。本书的排版模板来源于\href{http://haixing-hu.github.io/xelatex-zh-book/}{zhbook}项目,在此对原作者的贡献表示由衷的感谢。

本书的GitHub开源项目地址为:\url{http://github.com/DongYaoZe/Translate-LADW}
诚挚欢迎各位同学加入翻译、贡献代码、参与纠错,共同完善此项目。若上述链接无法访问,也可以通过点击\href{https://box.nju.edu.cn/d/a137a21fa716469c89da/}{此处}访问。每次版本更新亦将会在该页面同步发布。

若您对本翻译项目有任何疑问、建议或想法,欢迎通过我的电子邮箱 
\\
251840159@smail.nju.edu.cn 与我联系。

\vspace{5ex}
\begin{flushright}
董耀择~~~~~~~~~

2025年10月~~~~~
\end{flushright}

\end{preface}
%%%%%%%%%%%%%%%%%%%% 前言 %%%%%%%%%%%%%%%%%%%%%

\begin{preface}
本书的标题听起来有些神秘。为什么有人会读这本书,如果它以一种错位的方式呈现了这个主题呢?这本书里到底哪里是“错”的呢?在回答这些问题之前,请允许我先描述一下本书的目标读者。

本书源于“荣誉线性代数”(Honors Linear Algebra)课程的讲义。它旨在成为一门面向已在数学上训练有素的学生的入门线性代数课程。它适用于那些虽然还不太熟悉抽象推理,但愿意学习比“菜谱式”(cookbook style)的微积分课程更严谨数学的学生。除了作为线性代数的第一门课程,它还旨在成为介绍严谨证明、形式化定义——简而言之,介绍现代理论(抽象)数学风格的第一门课程。本书的目标读者解释了它为何如此特别地融合了初级概念和具体示例(通常在入门线性代数教材中呈现),以及更抽象的定义和构造(通常在高级书籍中典型出现)。本书的另一个特点是它不是由代数专家所写,也不是为代数专家所写。因此,我试图强调那些对分析、几何、概率等重要的主题,而没有包含一些传统的主题。例如,我只考虑实数或复数域上的向量空间。完全不考虑其他域上的线性空间,因为我认为花费时间介绍和解释抽象域的知识,不如花在一些在其他学科中更必需的经典主题上。

后来,当学生在抽象代数课程中学习一般域时,他们会明白本书研究的许多构造也适用于一般域。

本书仅考虑有限维空间,并且基总是指有限基。原因是,要对无限维空间说出一些有意义的话,就必须引入收敛、范数、完备性等概念,即泛函分析的基础。而这绝对是一个单独的课程(教材)的主题。

因此,我在此不考虑无限维 Hamel 基:它们在大多数分析和几何应用中是不需要的,而且我认为它们属于抽象代数课程。

\vspace{2ex} 
\textbf{给教师的说明}

本书的某些细节使其与标准的进阶线性代数教材有所不同。

首先是关于基、线性无关和生成集的定义。在书中,我首先将基定义为这样的系统:任何向量都能唯一地表示为其线性组合。然后,线性无关和生成系统的性质自然地成为基的性质的组成部分,一个代表表示的唯一性,另一个代表表示的存在性。这样处理的原因是,我认为基的概念比线性无关的概念更重要:在大多数应用中,我们并不真正关心线性无关,我们需要的是一个可以作为基的系统。例如,在求解齐次方程组时,我们不仅仅是在寻找线性无关的解,而是在寻找解空间中的一组基。而且,向学生解释基的重要性很容易:它允许我们引入坐标,并使用 $\mathbb{R}^n$(或 $\mathbb{C}^n$)来代替处理抽象向量空间。此外,我们需要坐标来进行计算机计算,而计算机非常擅长处理矩阵。而且,我真的不知道线性无关概念有什么简单的动机。

另一个细节是,我先介绍线性变换,然后再教授如何求解线性方程组。

一个缺点是我们直到第二章才证明只有方阵是可逆的,以及其他一些重要事实。然而,已经定义的线性变换允许更系统的行约简的呈现。此外,我花了很多时间(两节)来阐述矩阵乘法的动机。我希望我能很好地解释为什么这种看起来很奇怪的乘法规则实际上非常自然,以至于我们别无选择。

许多关于基、线性变换等重要事实,例如在任何向量空间中,两个基具有相同数量的向量,都是通过行约简中的主元计数来证明的。虽然这些事实中的大多数都具有“无坐标”的证明,形式上不涉及高斯消元,但仔细分析这些证明会发现,高斯消元和主元计数并没有消失,它们只是在大多数证明中被隐藏起来了。因此,与其呈现非常优雅(但对于初学者来说很难理解)的“无坐标”证明,这些证明通常在进阶线性代数书籍中呈现,我们则使用了“行约简”证明,这在“微积分类”教材中更为常见。这样做的优点是,可以轻松地看到所有证明背后的共同思想,并且对于不那么数学化的读者来说,这些证明更容易理解和记住。

我还将在第二章的第 8 节中介绍一个简单且易于记忆的公式,用于计算基变换。

第三章处理行列式。我花了很多时间来阐述行列式的动机,然后才给出正式定义。行列式被介绍为计算体积的一种方式。书中表明,如果我们允许了有符号体积,使得行列式在每列上都是线性的(此时学生应该很清楚线性关系非常有帮助,而允许负体积是为此付出的很小的代价),并且假设一些非常自然的性质,那么我们就别无选择,只能得到行列式的经典定义。我想强调的是,我一开始并没有假定行列式的反对称性,而是将其从体积的一些非常自然的性质中推导出来。

请注意,虽然在形式上,第一至第三章主要处理实数空间,但其中所有内容都适用于复数空间,甚至适用于任意域上的空间。

第四章是谱理论的介绍,这是复数空间 $\mathbb{C}^n$ 自然出现的地方。虽然 $\mathbb{C}^n$ 在本书开头被形式化定义,我们也给出了复数向量空间的定义,但在第四章之前,主要对象是实数空间 $\mathbb{R}^n$。现在,复数特征值的出现表明,对于谱理论,最自然的不是实数空间 $\mathbb{R}^n$,而是复数空间 $\mathbb{C}^n$,即使我们最初处理的是实数矩阵(实数空间中的算子)。这里的重点是特征值分解,并且特征值空间的基的概念也被引入。

第五章是内积空间,它出现在谱理论之后,因为我希望同时处理复数和实数两种情况,而谱理论为复数空间的出场提供了强有力的动机。除了引入的动机之外,第四章和第五章不互相依赖,教师可以先讲第五章。

尽管我在第九章中介绍了若尔当标准型,但我通常没有时间在一学期的课程中讲授它。我更倾向于花更多的时间在第六章和第七章讨论的主题上,例如正规算子和自伴算子的对角化、极分解和奇异值分解、正交矩阵的结构和定向,以及二次型理论。

我认为这些主题比若尔当标准型对于应用更重要,尽管后者确实很优美。但是,我新增了第九章,以便教师可以跳过第六章和第七章中的某些主题,转而讲授若尔当分解定理。

我还新增了(2009年新增)第八章,讨论对偶空间和张量。我认为其中的材料,特别是关于张量的部分,对于一年级的线性代数课程来说还是太难了,但有些主题(例如,对偶空间中的坐标变换)可以很容易地包含在教学大纲中。它还可以作为进阶课程中张量理论的介绍。请注意,本章中介绍的结果适用于任意域。

我试图以较为非正式的方式呈现本书的材料,偏爱直观的几何推理而不是形式化的代数运算,因此对于一些纯粹主义者来说,本书可能不够严谨。在本书通篇,我通常(当不引起混淆时)将线性变换与其矩阵等同起来,让我们能使用更简单的符号。而且我认为,对于没有经验的学生来说,过度强调变换与矩阵之间的区别可能会造成混淆。只有当区别很重要时,例如在分析一个变换的矩阵如何在基变换下变化时,我才会使用特殊的符号来区分变换和它的矩阵。
\vspace{5ex}
\begin{flushright}
Sergei Treil~~~~~~~~~
\end{flushright}

\end{preface}
