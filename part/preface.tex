
%%%%%%%%%%%%%%%%%%%% 前言 %%%%%%%%%%%%%%%%%%%%%

\begin{preface}
本书的标题听起来有些神秘。为什么有人会读这本书,如果它以一种错误的方式呈现了这个主题呢?这本书里到底哪里是“错误”的呢?在回答这些问题之前,请允许我先描述一下本书的目标读者。

本书源于“荣誉线性代数”(Honors Linear Algebra)课程的讲义。它旨在成为一门面向数学上训练有素的学生的入门线性代数课程。它适用于那些虽然还没有很熟悉抽象推理,但愿意比“菜谱式”(cookbook style)的微积分课程学习更严谨数学的学生。除了作为线性代数的第一门课程,它还旨在成为介绍严谨证明、形式化定义——简而言之,介绍现代理论(抽象)数学风格的第一门课程。本书的目标读者解释了它为何如此特别地融合了初级概念和具体示例(通常在入门线性代数教材中呈现),以及更抽象的定义和构造(通常在高级书籍中典型出现)。本书的另一个特点是它不是由代数专家所写,也不是为代数专家所写。因此,我试图强调那些对分析、几何、概率等重要的主题,而没有包含一些传统的主题。例如,我只考虑实数或复数域上的向量空间。完全不考虑其他域上的线性空间,因为我认为花费时间介绍和解释抽象域的知识,不如花在一些在其他学科中更必需的经典主题上。

后来,当学生在抽象代数课程中学习一般域时,他们会明白本书研究的许多构造也适用于一般域。

本书仅考虑有限维空间,并且基总是指有限基。原因是,要对无限维空间说出一些有意义的话,就必须引入收敛、范数、完备性等概念,即泛函分析的基础。而这绝对是一个单独的课程(教材)的主题。

因此,我在此不考虑无限维 Hamel 基:它们在大多数分析和几何应用中是不需要的,而且我认为它们属于抽象代数课程。

\textbf{给教师的说明}

本书的某些细节使其与标准的进阶线性代数教材有所不同。

首先是关于基、线性无关和生成集的定义。在书中,我首先将基定义为这样的系统:任何向量都能唯一地表示为线性组合。然后,线性无关和生成系统的性质自然地成为基的性质的组成部分,一个代表唯一性,另一个代表表示的存在性。这样处理的原因是,我认为基的概念比线性无关的概念更重要:在大多数应用中,我们并不真正关心线性无关,我们需要的是一个系统作为基。例如,在求解齐次方程组时,我们不仅仅是在寻找线性无关的解,而是在寻找解空间中的一组基。而且,向学生解释基的重要性很容易:它们允许我们引入坐标,并使用 $\mathbb{R}^n$(或 $\mathbb{C}^n$)来代替处理抽象向量空间。此外,我们需要坐标来进行计算机计算,而计算机非常擅长处理矩阵。而且,我真的不知道线性无关概念有什么简单的动机。

另一个细节是,我先介绍线性变换,然后再教授如何求解线性方程组。

一个缺点是我们直到第二章才证明只有方阵才能是可逆的,以及其他一些重要事实。然而,已经定义的线性变换允许更系统的行约简的呈现。此外,我花了很多时间(两节)来阐述矩阵乘法的动机。我希望我能很好地解释了为什么这种看起来很奇怪的乘法规则实际上非常自然,以至于我们别无选择。

许多关于基、线性变换等重要事实,例如在任何向量空间中,两个基具有相同数量的向量,都是通过行约简中的主元计数来证明的。虽然这些事实中的大多数都具有“无坐标”的证明,形式上不涉及高斯消元,但仔细分析这些证明会发现,高斯消元和主元计数并没有消失,它们只是在大多数证明中隐藏起来了。因此,与其呈现非常优雅(但对于初学者来说很难理解)的“无坐标”证明,这些证明通常在进阶线性代数书籍中呈现,我们则使用了“行约简”证明,这在“微积分类”教材中更为常见。这样做的优点是,可以轻松地看到所有证明背后的共同思想,并且对于不那么数学化的读者来说,这些证明更容易理解和记住。我还将在第二章的第 8 节中介绍一个简单且易于记忆的公式,用于计算基变换。

第三章处理行列式。我花了很多时间来阐述行列式的动机,然后才给出正式定义。行列式被介绍为计算体积的一种方式。书中表明,如果我们允许有符号体积,使得行列式在每列上都是线性的(此时学生应该很清楚线性关系非常有帮助,而允许负体积是为此付出的很小的代价),并且假设一些非常自然的性质,那么我们就别无选择,只能得到行列式的经典定义。我想强调的是,我一开始并没有假定行列式的反对称性,而是从体积的其他非常自然的性质中推导出来的。请注意,虽然在形式上,第一至第三章主要处理实数空间,但其中所有内容都适用于复数空间,甚至适用于任意域上的空间。

第四章是谱理论的介绍,这就是复数空间 $\mathbb{C}^n$ 自然出现的地方。它在本书开头被形式化定义,复数向量空间的定义也给出了,但在第四章之前,主要对象是实数空间 $\mathbb{R}^n$。现在,复数特征值的出现表明,对于谱理论,最自然的不是实数空间 $\mathbb{R}^n$,而是复数空间 $\mathbb{C}^n$,即使我们最初处理的是实数矩阵(实数空间中的算子)。这里的重点是特征值分解,并且特征值空间的基的概念也被引入。

第五章是内积空间,它出现在谱理论之后,因为我希望同时处理复数和实数两种情况,而谱理论为复数空间提供了强有力的动机。除了动机之外,第四章和第五章不互相依赖,教师可以先讲第五章。

尽管我在第九章中介绍了 Jordan 标准型,但我通常没有时间在一学期的课程中讲授它。我更喜欢花更多时间讲授第六章和第七章中的主题,例如法线算子和自伴随算子的特征值分解、极坐标和奇异值分解、正交矩阵的结构和方向,以及二次型理论。我认为这些主题比 Jordan 标准型对于应用更重要,尽管后者确实很优美。但是,我包含了第九章,以便教师可以跳过第六章和第七章中的某些主题,转而讲授 Jordan 分解定理。我还包括了(2009年新增)第八章,讨论对偶空间和张量。我认为其中的材料,特别是关于张量的部分,对于一年级的线性代数课程来说有点太难了,但有些主题(例如,对偶空间中的坐标变换)可以很容易地包含在教学大纲中。它还可以作为进阶课程中张量理论的介绍。请注意,本章中介绍的结果适用于任意域。

我试图以比较非正式的方式呈现本书的材料,偏爱直观的几何推理而不是形式化的代数运算,因此对于一些纯粹主义者来说,本书可能不够严谨。在本书通篇,我通常(当不引起混淆时)将线性变换与其矩阵等同起来。这允许使用更简单的符号,而且我认为对于没有经验的学生来说,过度强调变换与矩阵之间的区别可能会造成混淆。只有当区别至关重要时,例如在分析一个变换的矩阵如何在基变换下变化时,我才会使用特殊的符号来区分变换和它的矩阵。
\vspace{5ex}
\begin{flushright}
Sergei Treil~~~~~~~~~
\end{flushright}





\end{preface}

\begin{preface}[译者的话]

\textbf{我为什么翻译这本书?}

在南京大学匡亚明学院,时间是每位学生最宝贵的资源。我们的学习强度很高,课业繁重,几乎没有空闲。在这种情况下,再额外开启一个耗时巨大的翻译项目,似乎是一个不理智的选择。

但我还是决定这么做。

因为《Linear Algebra Done Wrong》是我们线代课程至关重要的参考书,而语言的障碍确实困扰着不少同学。我相信,一份高质量的中文译本能够实实在在地帮助到大家。

南京大学的校训精神中,“诚”字所蕴含的力量给了我巨大的鼓舞。“大哉一诚天下动”,它告诉我,一个真诚的、以服务之心出发的行动,即便微小,也能产生积极的影响。与其独自感叹学习之艰辛,不如动手为大家做一点实事。

因此,我启动了这个项目。我希望这份译稿不仅仅是我个人学习的沉淀,更能成为一份小小的礼物,送给每一位在知识海洋中奋力前行的同学。希望它能为你节省一些时间,扫清一些迷茫,让你更专注于领略线性代数的核心思想。

\vspace{2ex} 

本书中文版全文采用\LaTeX{}精心排版,力求高仿复刻原书中的所有公式与数学符号,以保证其准确性和专业性。本书的排版模板来源于\href{http://haixing-hu.github.io/xelatex-zh-book/}{zhbook}项目,在此对原作者的贡献表示由衷的感谢。

本书的GitHub开源项目地址为:\url{http://github.com/DongYaoZe/Translate-LADW}
诚挚欢迎各位同学加入翻译、贡献代码、参与纠错,共同完善此项目。

若您对本翻译项目有任何疑问、建议或想法,欢迎通过我的电子邮箱 

251840159@smail.nju.edu.cn 与我联系。

\vspace{5ex}
\begin{flushright}
董耀择~~~~~~~~~

2025年10月~~~~~
\end{flushright}

\end{preface}