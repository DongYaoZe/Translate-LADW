

\chapter{第六章~~内积空间中算子的结构}

在本章中,我们再次假设所有空间都是有限维的。同样,我们只处理复数或实数空间,内积空间的理论不适用于任意域上的空间。当没有提及我们所处的空间时,所有结果都适用于复数和实数空间。

为了避免重复书写基本相同的公式,我们将使用复数情况的记号:在实数情况下,它给出正确但有时稍显复杂的公式。

\section{1. 算子的上三角(舒尔)表示}

\textbf{定理 1.1}~~设 $A: X \to X$ 是作用在复内积空间中的算子。存在一个标准正交基 $\{\uu_1, \uu_2, \dots, \uu_n\}$ 在 $X$ 中,使得 $A$ 在该基下的矩阵是上三角矩阵。

换句话说,任何 $n \times n$ 矩阵 $A$ 都可以表示为 $A = UTU^*$,其中 $U$ 是酉矩阵,而 $T$ 是上三角矩阵。

\textbf{证明}~~我们使用 $\dim X$ 的数学归纳法来证明定理。如果 $\dim X = 1$,则定理是平凡的,因为任何 $1 \times 1$ 矩阵都是上三角矩阵。

假设我们已经证明了当 $\dim X = n-1$ 时定理成立,并且我们想证明对于 $\dim X = n$ 时定理成立。

设 $\lambda_1$ 是 $A$ 的一个特征值,设 $\uu_1$, $\|\uu_1\| = 1$ 是相应的特征向量,$A\uu_1 = \lambda_1 \uu_1$.~记 $E = \uu_1^\perp$,并设 $\{\vv_2, \dots, \vv_n\}$ 是 $E$ 中的某个标准正交基(显然 $\dim E = \dim X - 1 = n-1$),那么 $\{\uu_1, \vv_2, \dots, \vv_n\}$ 是 $X$ 中的一个标准正交基。在这一基下,$A$ 的矩阵具有形式$$ (1.1)\quad
\begin{pmatrix} \lambda_1 & * & \dots & * \\ 0 & & & \\ \vdots & & A_1 & \\ 0 & & & \end{pmatrix};$$
这里 $\lambda_1$ 下方的所有元素都是零,而 $*$ 表示我们不关心 $\lambda_1$ 右侧的元素。

我们足够关心右下角的 $(n-1) \times (n-1)$ 块,以便给它命名:我们将其记为 $A_1$.~

注意,$A_1$ 定义了 $E$ 中的一个线性变换,并且由于 $\dim E = n-1$,归纳假设表明存在一个标准正交基(我们将其记为 $\{\uu_2, \dots, \uu_n\}$),使得 $A_1$ 在该基下的矩阵是上三角矩阵。

所以,$A$的矩阵在正交基$\uu_1,\uu_2,\dots, \uu_n$具有形式(1.1),其中矩阵$A1$是上三角矩阵。
因此,$A$ 在该标准正交基 $\{\uu_1, \uu_2, \dots, \uu_n\}$ 下的矩阵也是上三角矩阵。

\textbf{注记}~~注意,在证明中引入的子空间 $E = \uu_1^\perp$ 对于 $A$ 不是不变的,即 $AE \subseteq E$ 不一定成立。这意味着 $A_1$ 不是 $A$ 的一部分,它是从 $A$ 构建的某个算子。

还需注意,$AE \subseteq E$ 当且仅当所有标记为 $*$ 的元素(即除了 $\lambda_1$ 之外的第一行的所有元素)都为零。

\textbf{注记}~~注意,即使我们从一个实数矩阵 $A$ 开始,矩阵 $U$ 和 $T$ 也可以是复数的。旋转矩阵 $$\begin{pmatrix} \cos \alpha & -\sin \alpha \\ \sin \alpha & \cos \alpha \end{pmatrix}, \quad \alpha \neq k\pi, k \in \mathbb{Z}$$
与实数上三角矩阵不是酉等价的(甚至不是相似的)。因为该矩阵的特征值是复数,而上三角矩阵的特征值是对角线元素。

\textbf{注记}~~ 定理 1.1 的一个类似版本可以陈述并证明用于任意向量空间,而不要求它具有内积。在这种情况下,定理声称在某个基下任何算子都有上三角形式。可以通过模仿定理 1.1 的证明来完成。另一种方法是为 $V$ 装备内积,方法是固定一个基并声明它是标准正交基,见第 5 章第 2.4 节。

注意,内积空间版本(定理 1.1)比向量空间版本更强大,因为它说明我们总能找到一个标准正交基,而不仅仅是一个基。

下面的定理是定理 1.1 的实数版本:

\textbf{定理 1.2}~~设 $A: X \to X$ 是作用在\textbf{实数}内积空间中的算子。假设 $A$ 的所有特征值都是实数(意味着 $A$ 恰好有 $n = \dim X$ 个实数特征值,计入重数)。那么存在 $X$ 中的一个标准正交基 $\{\uu_1, \uu_2, \dots, \uu_n\}$,使得 $A$ 在该基下的矩阵是上三角矩阵。

换句话说,任何具有所有实数特征值的实数 $n \times n$ 矩阵 $A$ 都可以表示为 $A = UTU^* = U T U^T$,其中 $U$ 是正交矩阵,而 $T$ 是实数上三角矩阵。

\textbf{证明}~~为了证明定理,我们只需要分析定理 1.1 的证明。让我们假设(我们可以无损于一般性地这样做)算子(矩阵)$A$ 作用在 $\RR^n$ 上。

假设定理对 $(n-1) \times (n-1)$ 矩阵成立。在定理 1.1 的证明中,设 $\lambda_1$ 是 $A$ 的一个实特征值,$\uu_1 \in \RR^n$, $\|\uu_1\| = 1$ 是相应的特征向量,设 $\{\vv_2, \dots, \vv_n\}$ 是 $\RR^n$ 中的一个标准正交系统,使得 $\{\uu_1, \vv_2, \dots, \vv_n\}$ 是 $\RR^n$ 中的一个标准正交基。

在该基下 $A$ 的矩阵具有形式 (1.1),其中 $A_1$ 是某个实数矩阵。

如果我们能证明矩阵 $A_1$ 只有实数特征值,那么我们就完成了证明。确实,根据归纳假设,存在 $E = \uu_1^\perp$ 中的一个标准正交基 $\{\uu_2, \dots, \uu_n\}$,使得 $A_1$ 在该基下的矩阵是上三角矩阵,因此 $A$ 在 $\{\uu_1, \uu_2, \dots, \uu_n\}$ 基下的矩阵也是上三角矩阵。

为了证明 $A_1$ 只有实数特征值,让我们注意到 
$$\det(A - \lambda I) = ( \lambda_1 - \lambda ) \det( A_1 - \lambda )$$
(例如,通过第一行的代数余子式展开),所以 $A_1$ 的任何特征值也是 $A$ 的特征值。但是 $A$ 只有实数特征值!

\begin{exer} \textbf{练习}

1.1. 利用算子的上三角表示,给出行列式是乘积,迹是计算重数的特征值之和这一事实的另一种证明。\end{exer}

\section{2. 自伴随和正规算子的谱定理}

在本章中,我们处理的是酉等价于对角矩阵的矩阵(算子)。

让我们回忆一下,如果一个算子满足 $A = A^*$,则称其为\textbf{自伴随}的。在某个标准正交基下的自伴随算子(即满足 $A^* = A$ 的矩阵)称为\textbf{埃尔米特矩阵}。
术语“自伴随”和“埃尔米特”基本上是同义的。通常人们在谈论算子(变换)时说自伴随,在谈论矩阵时说 埃尔米特。我们将尝试遵循这个约定,但由于我们经常不区分算子和它们的矩阵,所以有时会混合使用这两个术语。

\textbf{定理 2.1}~~设 $A = A^*$ 是内积空间 $X$(空间可以是复数或实数)中的一个自伴随算子。那么 $A$ 的所有特征值都是实数,并且 $X$ 中存在 $A$ 的特征向量的标准正交基。

这个定理可以用矩阵形式重述如下:

\textbf{定理 2.2}~~设 $A = A^*$ 是一个自伴随(因此是方阵)矩阵。那么 $A$ 可以表示为 
$$A = UDU^*,$$
其中 $U$ 是酉矩阵,$D$ 是具有实数项的对角矩阵。

而且,如果矩阵 $A$ 是实数的,矩阵 $U$ 可以选择为实数的(即正交的)。

\textbf{证明}~~为了证明定理 2.1 和定理 2.2,我们首先对内积空间 $X$(或实数空间 $X$)应用定理 1.1(或定理 1.2)来找到一个标准正交基,使得 $A$ 在该基下的矩阵是上三角矩阵。现在让我们问自己一个问题:什么样的上三角矩阵是自伴随的?

答案是显而易见的:上三角矩阵是自伴随的当且仅当它是具有实数项的对角矩阵。定理 2.1(以及因此定理 2.2)得证。

\textbf{注记}~~注意,在许多教科书中只考虑实数矩阵,并且定理 2.2 通常被称为“\textbf{对称矩阵的谱定理}”。然而,我们应该强调,定理 2.2 的结论对于\textbf{复数}对称矩阵是不成立的:该定理适用于 埃尔米特 矩阵,特别是\textbf{实数}对称矩阵。

让我们给出一个 $A=A^*$ 的算子的特征值是实数的独立证明。设 $A=A^*$ 且 $A\xx=\lambda \xx$, $\xx \neq 0$.~那么 
$$(A\xx, \xx) = (\lambda \xx, \xx) = \lambda(\xx, \xx) = \lambda\|\xx\|^2.$$
另一方面,
$$(A\xx, \xx) = (\xx, A^*\xx) = (\xx, A\xx) = (\xx, \lambda \xx) = \bar{\lambda}(\xx, \xx) = \bar{\lambda}\|\xx\|^2,$$(这里我们用了 $(\xx, \lambda \yy) = \bar{\lambda}(\xx, \yy)$)。所以 $\lambda\|\xx\|^2 = \bar{\lambda}\|\xx\|^2$.~由于 $\xx \neq 0$,$\|\xx\|^2 \neq 0$,我们可以得出 $\lambda = \bar{\lambda}$,所以 $\lambda$ 是实数。

从定理 2.1 也可以得出,自伴随算子的特征子空间是相互正交的。让我们给出一个该结果的独立证明。

\textbf{命题 2.3}~~设 $A = A^*$ 是一个自伴随算子,设 $\uu, \vv$ 是它的特征向量,$A\uu = \lambda \uu$, $A\vv = \mu \vv$.~那么,如果 $\lambda \neq \mu$,则特征向量 $\uu$ 和 $\vv$ 是正交的。

\textbf{证明}~~这个命题虽然可以从谱定理(定理 1.1)得出,但我们在这里给出一个直接的证明。即,$$(A\uu, \vv) = (\lambda \uu, \vv) = \lambda(\uu, \vv).$$
另一方面,$$(A\uu, \vv) = (\uu, A^*\vv) = (\uu, A\vv) = (\uu, \mu \vv) = \mu(\uu, \vv)$$
(最后一个等式成立是因为自伴随算子的特征值是实数),所以 $\lambda(\uu, \vv) = \mu(\uu, \vv)$.~如果 $\lambda \neq \mu$,则这只有在 $(\uu, \vv) = 0$ 时才可能。

现在让我们尝试找到哪些矩阵是酉等价于一个对角矩阵。可以很容易地检验出,对于对角矩阵 $D$, 
$$D^*D = DD^*.$$
因此,如果 $A$ 在某个标准正交基下的矩阵是对角矩阵,那么 $A^*A = AA^*$.~

\textbf{定义}~~称算子(矩阵)$N$ 是\textbf{正规}的,如果 $N^*N = NN^*$.~

显然,任何自伴随算子($A^*A = AA^*$)都是正规的。同样,任何酉算子 $U: X \to X$ 也是正规的,因为 $U^*U = UU^* = I$.~

注意,正规算子是作用在同一个空间上的算子,而不是从一个空间到另一个空间。所以,如果 $U$ 是作用在一个空间到另一个空间上的酉算子,我们就不能说 $U$ 是正规的。

\textbf{定理 2.4}~~任何复数向量空间中的正规算子 $N$ 都有一个标准正交的特征向量基。

换句话说,任何满足 $N^*N = NN^*$ 的矩阵 $N$ 都可以表示为 $$N = UDU^*,$$
其中 $U$ 是酉矩阵,$D$ 是对角矩阵。

\textbf{注记}~~注意,在上述定理中,即使 $N$ 是实数矩阵,我们也没有声称矩阵 $U$ 和 $D$ 是实数。而且,可以很容易地证明,如果 $D$ 是实数,那么 $N$ 必须是自伴随的。


\textbf{定理 2.4 的证明}~~为了证明定理 2.4,我们应用定理 1.1 来得到一个标准正交基,使得 $N$ 在该基下的矩阵是上三角矩阵。为了完成定理的证明,我们只需要证明一个上三角正规矩阵必须是对角矩阵。

我们将使用矩阵维数的数学归纳法来证明这一点。$1 \times 1$ 矩阵的情况是平凡的,因为任何 $1 \times 1$ 矩阵都是对角矩阵。

假设我们已经证明了任何 $(n-1) \times (n-1)$ 的上三角正规矩阵都是对角矩阵,并且我们想证明对于 $n \times n$ 矩阵也成立。设 $N$ 是一个 $n \times n$ 上三角正规矩阵。我们可以将其写成
$$N = \begin{pmatrix} a_{1,1} & a_{1,2} & \dots & a_{1,n} \\ 0 & & & \\ \vdots & & N_1 & \\ 0 & & & \end{pmatrix}$$
其中 $N_1$ 是一个 $(n-1) \times (n-1)$ 的上三角矩阵。

让我们比较 $N^*N$ 和 $NN^*$ 的左上角元素(第一行第一列)。直接计算表明 $$(N^*N)_{1,1} = \bar{a}_{1,1}a_{1,1} = |a_{1,1}|^2,$$
而 
$$(NN^*)_{1,1} = |a_{1,1}|^2 + |a_{1,2}|^2 + \dots + |a_{1,n}|^2.$$
所以,$(N^*N)_{1,1} = (NN^*)_{1,1}$ 当且仅当 $a_{1,2} = \dots = a_{1,n} = 0$.~因此,矩阵 $N$ 具有形式
$$N = \begin{pmatrix} a_{1,1} & 0 & \dots & 0 \\ 0 & & & \\ \vdots & & N_1 & \\ 0 & & & \end{pmatrix}.$$
从上述表示可以得出 $$N^*N =  \begin{pmatrix}|a_{1,1}|^2 & 0 & \dots & 0 \\ 0 & & & \\ \vdots & & N_1^*N_1 & \\ 0 & & & \end{pmatrix}, \quad NN^* =\begin{pmatrix} |a_{1,1}|^2 & 0 & \dots & 0 \\ 0 & & & \\ \vdots & & N_1 N_1^* & \\ 0 & & & \end{pmatrix}.$$
所以 $N_1^*N_1 = N_1N_1^*$.~这意味着矩阵 $N_1$ 也是正规的,并且根据归纳假设它是对角矩阵。所以矩阵 $N$ 也是对角矩阵。

以下命题给出了正规算子的一个非常有用的刻画。

\textbf{命题 2.5}~~算子 $N: X \to X$ 是正规的当且仅当 
$$\|N\xx\| = \|N^*\xx\| \quad \forall \xx \in X.$$

\textbf{证明}~~设 $N$ 是正规的,$N^*N = NN^*$.~那么 
$$\|N\xx\|^2 = (N\xx, N\xx) = (N^*N\xx, \xx) = (NN^*\xx, \xx) = (N^*\xx, N^*\xx) = \|N^*\xx\|^2,$$
所以 $\|N\xx\| = \|N^*\xx\|$.~
现在设 
$$\|N\xx\| = \|N^*\xx\| \quad \forall \xx \in X.$$
极化恒等式(第 5 章引理 1.9)暗示对于所有 $\xx, \yy \in X$,
\begin{equation} \notag
\begin{split}
(N^*N\xx, \yy) = (N\xx, N\yy) =&\   \frac{1}{4} \sum_{\alpha = \pm 1, \pm i} \alpha \|N\xx + \alpha N\yy\|^2  \\
=&\ \frac{1}{4} \sum_{\alpha = \pm 1, \pm i} \alpha \|N(\xx + \alpha \yy)\|^2   \\
=&\  \frac{1}{4} \sum_{\alpha = \pm 1, \pm i} \alpha \|N^*(\xx + \alpha \yy)\|^2  \\
=&\  \frac{1}{4} \sum_{\alpha = \pm 1, \pm i} \alpha \|N^* \xx + \alpha N^* \yy)\|^2  \\
=&\ (N^*\xx, N^*\yy) = (NN^*\xx, \yy). 
 \end{split}
\end{equation}
因此(见推论 1.6),$N^*N = NN^*$.~

\begin{exer} \textbf{练习}~

2.1. 判断正误:

a) 任何酉算子 $U: X \to X$ 都是正规的。

b) 矩阵是酉的当且仅当它是可逆的。

c) 如果两个矩阵酉等价,那么它们也相似。

d) 两个自伴随算子之和是自伴随的。

e) 酉算子的伴随是酉的。

f) 正规算子的伴随是正规的。

g) 如果一个线性算子的所有特征值都是 1,那么该算子必须是酉的或正交的。

h) 如果一个正规算子的所有特征值都是 1,那么该算子是恒等算子。

i) 线性算子可能保持范数但不保持内积。

2.2. 判断正误:两个正规算子之和是正规的?证明你的结论。

2.3. 证明一个酉等价于对角矩阵的矩阵是正规的。

2.4. 正交对角化矩阵 
$$\begin{pmatrix} 3 & 2 \\ 2 & 3 \end{pmatrix}.$$
找出 $A$ 的所有平方根,即找出所有满足 $B^2 = A$ 的矩阵 $B$.~
\textbf{注记:} $A$ 的所有平方根都是自伴随的。

2.5. 判断正误:任何自伴随矩阵都有一个自伴随的平方根。证明你的结论。

2.6. 正交对角化矩阵 $$A = \begin{pmatrix} 7 & 2 \\ 2 & 4 \end{pmatrix},$$
即将其表示为 $A = UDU^*$,其中 $D$ 是对角矩阵,$U$ 是酉矩阵。

在 $A$ 的所有平方根中,找出具有正特征值的平方根。你可以将 $B$ 表示为乘积形式。

2.7. 判断正误:

a) 两个自伴随矩阵的乘积是自伴随的。

b) 如果 $A$ 是自伴随的,那么 $A^k$ 是自伴随的。证明你的结论。

2.8. 设 $A$ 是 $m \times n$ 矩阵。证明:

a) $A^*A$ 是自伴随的。

b) $A^*A$ 的所有特征值都是非负的。

c) $A^*A + I$ 是可逆的。

2.9. 如果陈述为真,则证明;如果陈述为假,则给出反例:

a) 如果 $A$ 是自伴随的,那么 $A + \ii I$ 是可逆的。

b) 如果 $U$ 是酉的,$U + \frac{3}{4}I$ 是可逆的。

c) 如果矩阵 $A$ 是实数的,那么 $A - \ii I$ 是可逆的。

2.10. \textbf{正交对角化}旋转矩阵 
$$R_\alpha = \begin{pmatrix} \cos \alpha & -\sin \alpha \\ \sin \alpha & \cos \alpha \end{pmatrix},$$
其中 $\alpha$ 不是 $\pi$ 的整数倍。注意,在这种情况下你会得到复数特征值。

2.11. \textbf{正交对角化}矩阵 
$$A = \begin{pmatrix} \cos \alpha & \sin \alpha \\ \sin \alpha & -\cos \alpha \end{pmatrix}.$$
\textbf{提示:} 你会得到实数特征值。此外,三角恒等式 $\sin^2 x = 2 \sin x \cos x$, $\sin^2 x = (1 - \cos 2x)/2$, $\cos^2 x = (1 + \cos 2x)/2$(应用于 $x = \alpha/2$)将有助于简化特征向量的表达式。

2.12. 你能从几何上描述上一问题中矩阵 $A$ 所代表的线性变换吗?它有一个非常简单的几何解释。

2.13. 证明一个具有模为 1 的特征值(即所有特征值满足 $|\lambda_k| = 1$)的正规算子是酉的。
\textbf{提示:} 考虑对角化。

2.14. 证明一个具有实数特征值的正规算子是自伴随的。


2.15. 举例说明定理 2.2 的结论对于复数对称矩阵不成立。 即:

    a) 构建一个(可对角化的)$2 \times 2$ 复数对称矩阵,它不容许一个正交的特征向量基;
    
    b) 构建一个 $2 \times 2$ 复数对称矩阵,它不能被对角化。
\end{exer}


\section{3. 极分解与奇异值分解}

\subsection{3.1. 正定算子~~平方根}

\textbf{定义}~~称自伴随算子 $A: X \to X$ 为\textbf{正定}的,如果 
$$(A\xx, \xx) > 0 \quad \forall \xx \neq 0,$$
称其为\textbf{半正定}的,如果 
$$(A\xx, \xx) \geq 0 \quad \forall \xx \in X.$$

我们将使用记号 $A > 0$ 表示正定算子,$A \geq 0$ 表示半正定算子。

下面的定理描述了正定和半正定算子。

\textbf{定理 3.1}~~设 $A = A^*$.~那么

1. $A > 0$ 当且仅当 $A$ 的所有特征值都是正的。

2. $A \geq 0$ 当且仅当 $A$ 的所有特征值都是非负的。

\textbf{证明}~~通过选取一个标准正交基,使得 $A$ 在该基下的矩阵是对角矩阵(见定理 2.1),我们可以无损于一般性地证明。要完成证明,只需注意到,对于对角矩阵,当且仅当其对角线元素都为正(非负)时,该矩阵才是正定(半正定)的。

\textbf{推论 3.2}~~设 $A = A^* \geq 0$ 是一个半正定算子。存在一个唯一的半正定算子 $B$,使得 $B^2 = A$.~

这样的 $B$ 被称为 $A$ 的(正)\textbf{平方根},并记作 $\sqrt{A}$ 或 $A^{1/2}$.~

\textbf{证明}~~我们来证明 $\sqrt{A}$ 的存在性。设 $\{\vv_1, \vv_2, \dots, \vv_n\}$ 是 $A$ 的特征向量的标准正交基,并设 $\lambda_1, \lambda_2, \dots, \lambda_n$ 是相应的特征值。注意,由于 $A \geq 0$,所有 $\lambda_k \geq 0$.~

在基 $\{\vv_1, \vv_2, \dots, \vv_n\}$ 下,$A$ 的矩阵是对角矩阵 $\text{diag}\{\lambda_1, \lambda_2, \dots, \lambda_n\}$,对角线上是 $\lambda_1, \lambda_2, \dots, \lambda_n$.~定义 $B$ 在同一基下的矩阵为 $\text{diag}\{\sqrt{\lambda_1}, \sqrt{\lambda_2}, \dots, \sqrt{\lambda_n}\}$.~

显然,$B = B^* \geq 0$ 且 $B^2 = A$.~

为了证明 $B$ 的唯一性,让我们假设存在一个算子 $C = C^* \geq 0$ 使得 $C^2 = A$.~设 $\{\uu_1, \uu_2, \dots, \uu_n\}$ 是 $C$ 的特征向量的标准正交基,并设 $\mu_1, \mu_2, \dots, \mu_n$ 是相应的特征值(注意 $\mu_k \geq 0 \quad \forall k$)。$C$ 在该基下的矩阵是对角矩阵 $\text{diag}\{\mu_1, \mu_2, \dots, \mu_n\}$,因此 $A = C^2$ 在同一基下的矩阵是 $\text{diag}\{\mu_1^2, \mu_2^2, \dots, \mu_n^2\}$.~这暗示 $A$ 的任何特征值 $\lambda$ 都必须是 $\mu_k^2$ 的形式,并且,更重要的是,如果 $A\xx = \lambda \xx$,那么 $C\xx = \sqrt{\lambda} \xx$.~

因此,在上面的基 $\{\vv_1, \vv_2, \dots, \vv_n\}$ 下,$C$ 的矩阵是对角矩阵 $\text{diag}\{\sqrt{\lambda_1}, \sqrt{\lambda_2}, $ $\dots, \sqrt{\lambda_n}\}$,即 $B = C$.~

\subsection{3.2. 算子的模~~奇异值}
考虑算子 $A: X \to Y$.~它的\textbf{埃尔米特平方} $A^*A$ 是作用在 $X$ 上的半正定算子。确实,$$(A^*A)^* = A^*(A^*)^* = A^*A$$
并且 $$(A^*A\xx, \xx) = (A\xx, A\xx) = \|A\xx\|^2 \geq 0 \quad \forall \xx \in X.$$
因此,存在一个(唯一的)半正定算子 $R = \sqrt{A^*A}$.~这个算子 $R$ 被称为算子 $A$ 的\textbf{模},通常记为 $|A|$.~

$A$ 的模显示了算子 $A$ 的“大小”:

\textbf{命题 3.3}~~对于线性算子 $A: X \to Y$,
$$\| |A| \xx \| = \|A\xx\| \quad \forall \xx \in X.$$

\textbf{证明}~~对于任何 $\xx \in X$,
\begin{equation} \notag
\begin{split}
\| |A| \xx \|^2 =&\  (|A|\xx, |A|\xx) = (|A|^*|A|\xx, \xx) = ( |A|^2 \xx, \xx )  \\
=&\    (A^*A\xx, \xx) = (A\xx, A\xx) = \|A\xx\|^2.
\end{split}
\end{equation}

\textbf{推论 3.4}~~
$$\text{Ker } A = \text{Ker } |A| = (\text{Ran } |A|)^\perp.$$

\textbf{证明}~~第一个等式直接来自命题 3.3,第二个等式来自恒等式 $\text{Ker } T = (\text{Ran } T^*)^\perp$($|A|$ 是自伴随的)。

\textbf{定理 3.5(算子的极分解)}~~设 $A: X \to X$ 是一个算子(方阵)。那么 $A$ 可以表示为 
$$A = U|A|,$$
其中 $U$ 是酉算子。

\textbf{注记}~~酉算子 $U$ 通常不是唯一的。正如从定理的证明中可以看出,$U$ 仅在 $A$ 可逆时才唯一。

\textbf{注记}~~极分解 $A = U|A|$ 也适用于作用在一个空间到另一个空间上的算子 $A: X \to Y$.~但在这种情况下,我们只能保证 $U$ 是从 $\text{Ran } |A| = (\text{Ker } A)^\perp$ 到 $Y$ 的一个等距同构。

如果 $\dim X \leq \dim Y$,则此等距同构可以扩展为从整个 $X$ 到 $Y$ 的等距同构(如果 $\dim X = \dim Y$,则它将是一个酉算子)。

\textbf{定理 3.5 的证明}~~考虑向量 $\xx \in \text{Ran } |A|$.~那么向量 $\xx$ 可以表示为 $\xx = |A|\vv$ 对于某个向量 $\vv \in X$.~

定义 $U_0 \xx := A\vv$.~根据命题 3.3 
$$\|U_0 \xx\| = \|A\xx\| = \||A|\vv\| = \|\xx\|,$$
所以看起来 $U$ 是从 $\text{Ran } |A|$ 到 $X$ 的一个等距同构。

但首先我们需要证明 $U_0$ 是良好定义的。设 $\vv_1$ 是另一个使得 $\xx = |A|\vv_1$ 的向量。但是 $\xx = |A|\vv = |A|\vv_1$ 意味着 $\vv - \vv_1 \in \text{Ker } |A| = \text{Ker } A$(参见推论 3.4),所以 $A\vv = A\vv_1$,这意味着 $U_0 \xx$ 是良好定义的。

根据构造,$A = U_0|A|$.~我们将检查 $U_0$ 是一个线性变换的证明留给读者。

为了将 $U_0$ 扩展为酉算子 $U$,我们找到一个酉变换 $U_1 : \text{Ker } A \to (\text{Ran } A)^\perp = \text{Ker } A^*$。这样做总是可能的,因为对于方阵,$\text{dim Ker } A = \text{dim Ker } A^*$(根据秩定理)。

很容易验证,$U = U_0 + U_1$ 是一个酉算子,并且 $A = U |A|$.

\subsection{3.3. 奇异值~~施密特分解}

\textbf{定义}~~$|A|$ 的特征值被称为 $A$ 的\textbf{奇异值}(singular value)。换句话说,如果 $\lambda_1, \lambda_2, \dots, \lambda_n$ 是 $A^*A$ 的特征值,那么 $\sqrt{\lambda_1}, \sqrt{\lambda_2}, \dots, \sqrt{\lambda_n}$ 就是 $A$ 的奇异值。

\textbf{注记}~~在许多文献中,奇异值被定义为 $A^*A$ 的特征值的非负平方根,而不提及算子 $|A|$.~

我认为算子 $|A|$ 的概念很重要,所以上面已经介绍了。然而,算子 $|A|$ 的概念对于后续内容(定义舒尔和奇异值分解)不是必需的。此外,正如下面将要显示的,算子 $|A|$ 可以很容易地从奇异值分解构造出来。

设 $A: X \to Y$ 是一个算子,并设 $\sigma_1, \sigma_2, \dots, \sigma_n$ 是 $A$ 的奇异值(计入重数)。假设 $\sigma_1, \sigma_2, \dots, \sigma_r$ 是 $A$ 的\textbf{非零}奇异值(计入重数)。这意味着,特别地,$\sigma_k = 0$ 对于 $k > r$.~


根据奇异值的定义,数字 $\sigma_1^2, \sigma_2^2, \dots, \sigma_n^2$ 是 $A^*A$ 的特征值。设 $\{\vv_1, \vv_2, \dots, \vv_n\}$ 是 $A^*A$ 的特征向量的标准正交基,$A^*A\vv_k = \sigma_k^2 \vv_k$.~

\textbf{命题 3.6}~~系统 
$$\{\ww_k := \frac{1}{\sigma_k} A\vv_k,\quad k = 1, 2, \dots, r\}$$
是一个标准正交系统。

\textbf{证明}~~$$(A\vv_j, A\vv_k) = (A^*A\vv_j, \vv_k) = (\sigma_j^2 \vv_j, \vv_k) = \sigma_j^2 (\vv_j, \vv_k) = \begin{cases} 0, & j \neq k \\ \sigma_j^2, & j = k \end{cases},$$因为 $\{\vv_1, \vv_2, \dots, \vv_r\}$ 是一个标准正交系统。

在上述命题的记号中,算子 $A$ 可以表示为
$$(3.1)\quad A = \sum_{k=1}^r \sigma_k \ww_k \vv_k^*,$$
或者等价地
$$(3.2)\quad A\xx = \sum_{k=1}^r \sigma_k (\xx, \vv_k) \ww_k.$$
确实,我们知道 $\{\vv_1, \vv_2, \dots, \vv_n\}$ 是 $X$ 的一个标准正交基。那么将 $\xx = \vv_j$ 代入 (3.2) 的右侧,我们得到 
$$\sum_{k=1}^r \sigma_k (\vv_j, \vv_k) \ww_k = \sigma_j (\vv_j, \vv_j) \ww_j = \sigma_j \ww_j = A\vv_j$$
如果 $j=1, 2, \dots, r$,并且 
$$\sum_{k=1}^r \sigma_k (\vv_k^* \vv_j) \ww_k = \oo = A\vv_j$$ 对于 $j > r$.~所以,(3.1) 中左右两侧的算子在基 $\{\vv_1, \vv_2, \dots, \vv_n\}$ 上是相同的,因此它们是相等的。

\textbf{定义}~~上述分解 (3.1)(或 (3.2))被称为算子 $A$ 的\textbf{施密特分解}(Schmidt decomposition)。

\textbf{注记}~~施密特分解不是唯一的。为什么?


\textbf{引理 3.7}~~设$A$可以被表示为 
$$A = \sum_{k=1}^r \sigma_k \ww_k \vv_k^*,$$
其中 $\sigma_k > 0$ 并且 $\{\vv_1, \vv_2, \dots, \vv_r\}$, $\{\ww_1, \ww_2, \dots, \ww_r\}$ 是某些标准正交系。

那么这个表示给出了 $A$ 的施密特分解。

\textbf{证明}~~我们只需要证明 $\vv_1, \vv_2, \dots, \vv_r$ 是 $A^*A$ 的特征向量,$A^*A\vv_k = \sigma_k^2 \vv_k$.~由于 $\{\ww_1, \ww_2, \dots, \ww_r\}$ 是标准正交系统,$$\ww_k^* \ww_j = (\ww_j, \ww_k) = \delta_{k,j} := \begin{cases} 0, & j \neq k \\ 1, & j = k \end{cases},$$
因此 
$$A^*A = \sum_{k=1}^r \sigma_k^2 \vv_k \vv_k^*.$$
由于 $\{\vv_1, \vv_2, \dots, \vv_r\}$ 是标准正交系统,
$$A^*A\vv_j = \sum_{k=1}^r \sigma_k^2 \vv_k \vv_k^* \vv_j = \sigma_j^2 \vv_j,$$
因此 $\vv_k$ 是 $A^*A$ 的特征向量。

\textbf{推论 3.8}~~设 $$A = \sum_{k=1}^r \sigma_k \ww_k \vv_k^*$$
是 $A$ 的施密特分解。那么 
$$A^* = \sum_{k=1}^r \sigma_k \vv_k \ww_k^*$$
是 $A^*$ 的施密特分解。

\subsection{3.4. 施密特分解的矩阵表示~~奇异值分解}

施密特分解可以写成一个很好的矩阵形式。即,假设 $A: \FF^n \to \FF^m$(这里 $\FF$ 总是 $\CC$ 或 $\RR$;我们可以通过选取 $X$ 和 $Y$ 中的标准正交基并处理这些基下的坐标来完成)。设 $\sigma_1, \sigma_2, \dots, \sigma_r$ 是 $A$ 的非零奇异值(计入重数),并设 
$$A = \sum_{k=1}^r \sigma_k \ww_k \vv_k^*$$
是 $A$ 的施密特分解。

如你所见,这个等式可以重写为
$$(3.3)\quad A = \tilde{W} \tilde{\Sigma} \tilde{V}^*,$$
其中 $\tilde{\Sigma} = \text{diag}\{\sigma_1, \sigma_2, \dots, \sigma_r\}$ 并且 $\tilde{V}$ 和 $\tilde{W}$ 是分别以 $\vv_1, \vv_2, \dots, \vv_r$ 和 $\ww_1, \ww_2, \dots, \ww_r$ 为列的矩阵。(你能说出每个矩阵的大小吗?)

注意,由于 $\{\vv_1, \vv_2, \dots, \vv_r\}$ 和 $\{\ww_1, \ww_2, \dots, \ww_r\}$ 是标准正交系统,矩阵 $\tilde{V}$ 和 $\tilde{W}$ 是等距同构。还需注意 $r = \text{rank } A$,见下面的练习 3.1。

如果矩阵 $A$ 是可逆的,那么 $m=n=r$,矩阵 $\tilde{V}$, $\tilde{W}$ 是酉的,并且 $\tilde{\Sigma}$ 是一个可逆的对角矩阵。

事实证明,总是可以写出一个类似的表示(3.3),用酉矩阵 $V$ 和 $W$ 来代替 $\tilde{V}$ 和 $\tilde{W}$,并且在许多情况下,处理这样的表示会更方便。
为了写出这个表示,我们首先需要将系统 $\{\vv_1, \vv_2, \dots, \vv_r\}$ 和 $\{\ww_1, \ww_2, \dots, \ww_r\}$ \textbf{补全}为 $\FF^n$ 和 $\FF^m$ 中的正交基。

回想一下,要将 $\{\vv_1, \vv_2, \dots, \vv_r\}$ 补全为 $\FF^n$ 中的标准正交基,只需找到 $\text{Ker } V^*$ 的一个标准正交基 $\{\vv_{r+1}, \dots, \vv_n\}$;那么系统 $\{\vv_1, \vv_2, \dots, \vv_n\}$ 将是 $\FF^n$ 中的一个标准正交基。并且人们总是能通过格拉姆-施密特正交化从任意系统得到一个标准正交基。

然后 $A$ 可以表示为
$$(3.4)\quad A = W \Sigma V^*,$$
其中 $V \in M^\FF_{n \times n}$ 和 $W \in M^\FF_{m \times m}$ 是以 $\vv_1, \vv_2, \dots, \vv_n$ 和 $\ww_1, \ww_2, \dots, \ww_m$ 为列的酉矩阵,而 $\Sigma \in M^{\RR_+}_{ m \times n}$ 是一个“对角”矩阵
$$(3.5)\quad \sigma_{j,k} = \begin{cases} \sigma_k, & j=k \leq r \\ 0, & \text{其他} \end{cases}.$$
也就是说,为了得到矩阵 $\Sigma$,你需要取对角矩阵 $\text{diag}\{\sigma_1, \sigma_2, \dots, \sigma_r\}$ 并通过在“南方”和“东方”添加额外的零将其变成一个 $m \times n$ 矩阵。

\textbf{定义 3.9}~~对于矩阵 $A \in M^\FF_{m \times n}$(这里 $\FF$ 总是 $\CC$ 或 $\RR$),其\textbf{奇异值分解} (singular value decomposition, SVD) 是形如 (3.4) 的分解,即分解 $A = W \Sigma V^*$,其中 $W \in M^\FF_{n \times n}$ 和 $V \in M^\FF_{m \times m}$ 是酉矩阵,而 $\Sigma \in M^{\RR_+}_{ m \times n}$ 是“对角”矩阵(意思是 $\sigma_{k,k} \geq 0$ 对于所有 $k = 1, 2, \dots, \min\{m, n\}$,并且 $\sigma_{j,k} = 0$ 对于所有 $j \neq k$)。

(3.3)这种表示经常称作\textbf{约简 SVD}(reduced or compact SVD) .~更精确地说,约简 SVD 是一个表示 $A = \tilde{W} \tilde{\Sigma} \tilde{V}^*$,其中 $\tilde{\Sigma} \in M^{\RR_+}_{ r \times r}$, $r \leq \min\{m, n\}$ 是一个对角矩阵,其对角线元素严格为正,而 $\tilde{W} \in M^\FF_{n \times r}$, $\tilde{V} \in M^\FF_{m \times r}$ 是等距同构;而且,我们要求 $\tilde{W}$ 和 $\tilde{V}$ 中至少有一个不是方阵。

\textbf{注记 3.10}~~很容易看出,如果 $A = W \Sigma V^*$ 是 $A$ 的奇异值分解,那么 $\sigma_k := \sigma_{k,k}$ 是 $A$ 的奇异值,即 $\sigma_k^2$ 是 $A^*A$ 的特征值。而且,$V$ 的列 $\vv_k$ 是 $A^*A$ 的相应特征向量,$A^*A\vv_k = \sigma_k^2 \vv_k$.~还要注意,如果 $\sigma_k \neq 0$,那么 $\ww_k = \frac{1}{\sigma_k} A\vv_k$.~

所有这些都意味着任何奇异值分解 $A = W \Sigma V^*$ 都可以通过本节上面描述的构造从施密特分解 (3.2) 得到。

对于不可逆矩阵 $A$,约简奇异值分解可以解释为施密特分解 (3.2) 的矩阵形式。对于可逆矩阵 $A$,施密特分解的矩阵形式给出了奇异值分解。

\textbf{注记 3.11}~~$A = W \Sigma V^*$ 的奇异值分解的另一种解释是,$\Sigma$ 是 $A$ 在(标准正交)基 $\{\vv_1, \vv_2, \dots, \vv_n\}$ 和 $\{\ww_1, \ww_2, \dots, \ww_n\}$ 下的矩阵,即 $\Sigma = [A]_{B,A}$.~

我们将在后面使用这个解释。

\subsubsection{3.4.1. 从奇异值分解到极坐标分解}

注意,如果我们知道方阵 $A$ 的奇异值分解 $A = W \Sigma V^*$,我们可以写出 $A$ 的极坐标分解:
$$(3.6) \quad A = W \Sigma V^* = (WV^*) (V \Sigma V^*) = U|A|$$
其中 $|A| = V \Sigma V^*$ 并且 $U = WV^*$.~

为了说明这确实是一个极坐标分解,让我们注意到 $V\Sigma V^*$ 是一个自伴随的、半正定的算子,并且 
$$A^*A = (W\Sigma V^*)^*(W\Sigma V^*) = V \Sigma^* W^* W \Sigma V^* = V \Sigma^*\Sigma V^* = V (\Sigma^* \Sigma) V^* = (V \Sigma V^*)(V \Sigma V^*) = (|A|)^2.$$
所以根据 $|A|$ 的定义(它是 $A^*A$ 的唯一半正定平方根),我们可以看出 $|A| = V \Sigma V^*$.~变换 $WV^*$ 显然是一个酉变换,因为它是由两个酉变换相乘得到的,所以(3.6)确实给出了 $A$ 的一个极坐标分解。

请注意,此推理仅适用于方阵,因为如果 $A$ 不是方阵,则乘积 $V\Sigma$ 是未定义的(维度不匹配,你能看出为什么吗?)。

\begin{exer} \textbf{练习}~

3.1. 证明矩阵 $A$ 的非零奇异值的数量(计入重数)与其秩相等。


3.2. 为以下矩阵 $A$ 找出施密特分解 $A = \sum_{k=1}^r s_k \ww_k \vv_k^*$:

$$\begin{pmatrix} 2 & 3 \\ 0 & 2 \end{pmatrix},\quad \begin{pmatrix} 7 & 1 & 0 \\ 0 & 0 & 5 \\ 5 & 0 & 5 \end{pmatrix}, \quad \begin{pmatrix} 1 & 1 & 0 \\ 1 & 2 & 2 \\ 0 & -1 & 1 \end{pmatrix}.$$

3.3. 设 $A$ 是一个可逆矩阵,设 $A = W \Sigma V^*$ 是它的奇异值分解。求 $A^*$ 和 $A^{-1}$ 的奇异值分解。

3.4. 为以下矩阵 $A$ 找出奇异值分解 $A = W \Sigma V^*$,其中 $V$ 和 $W$ 是酉矩阵:

a) $A = \begin{pmatrix} -3 & 1 \\ 6 & -2 \\ 6 & -2 \end{pmatrix}$;

b) $A = \begin{pmatrix} 3 & 2 & 2 \\ 2 & 3 & -2 \end{pmatrix}$.~

3.5. 找出矩阵 
$$A = \begin{pmatrix} 2 & 3 \\ 0 & 2 \end{pmatrix}$$ 
的奇异值分解。并用它来找出:

a) $\max_{\|\xx\| \leq 1} \|A\xx\|$ 以及最大值达到的向量;

b) $\min_{\|\xx\|=1} \|A\xx\|$ 以及最小值达到的向量;

c) $A$ 对 $\RR^2$ 中的闭单位球 $B = \{\xx \in \RR^2 : \|\xx\| \leq 1\}$ 的像 $A(B)$.~几何上描述 $A(B)$.~

3.6. 证明对于方阵 $A$,$|\det A| = \det |A|$.~

3.7. 判断正误:

a) 矩阵的奇异值也是该矩阵的特征值。

b) 矩阵 $A$ 的奇异值是 $A^*A$ 的特征值。

c) 如果 $s$ 是矩阵 $A$ 的一个奇异值,而 $c$ 是一个标量,那么 $|c|s$ 是 $cA$ 的奇异值。

d) 任何线性算子的奇异值都是非负的。

e) 自伴随矩阵的奇异值与其特征值相等。

3.8. 设 $A$ 是一个 $m \times n$ 矩阵。证明 $A^*A$ 和 $AA^*$ 的\textbf{非零}特征值(计入重数)是相同的。你能说出 $A^*A$ 的零特征值和 $AA^*$ 的零特征值何时具有相同的重数吗?

3.9. 设 $s$ 是算子 $A$ 的最大奇异值,设 $\lambda$ 是 $A$ 具有最大绝对值的特征值。证明 $|\lambda| \leq s$.~

3.10. 证明矩阵的秩等于其非零奇异值的数量(计入重数)。


3.11. 证明算子范数 $\|A\|$ 与 Frobenius 范数 $\|A\|_2$ 相等当且仅当该矩阵秩为 1。
\textbf{提示:} 上一个问题可能有所帮助。

3.12. 对于矩阵 $$A = \begin{pmatrix} 2 & -3 \\ 0 & 2 \end{pmatrix},$$
描述单位球的逆像,即所有 $\xx \in \RR^2$ 使得 $\|A\xx\| \leq 1$ 的集合。使用奇异值分解。\end{exer}


\section{4. 奇异值分解的应用}

正如我们上面讨论的,奇异值分解(SVD)本质上是对两个不同标准正交基的对角化。由于这里有两个不同的基,我们无法从其奇异值分解中得知一个算子的光谱性质。例如,奇异值分解 (3.5) 中的 $\Sigma$ 对角线上的元素并不是 $A$ 的特征值。注意,对于 $A = W \Sigma V^*$(如 (3.5) 所示),通常有 $A^n \ne W \Sigma^n V^*$,因此这种对角化并不能帮助我们计算矩阵的函数。

然而,正如下面的例子所示,奇异值能够很好地揭示线性变换的所谓\textbf{度量性质}(metric property)。

最后说明:进行奇异值分解需要找到埃尔米特(自伴)矩阵 $A^*A$ 的特征值和特征向量。为了找到特征值,我们通常计算特征多项式,找到它的根,等等。这看起来是一个相当复杂的过程,尤其考虑到对于五次及更高次的方程,并没有求根公式。

然而,存在非常有效的数值方法,可以计算出埃尔米特矩阵的特征值和特征向量,精确到任意给定的精度。这些方法不涉及计算特征多项式及其根。它们通过迭代过程直接计算近似的特征值和特征向量。由于埃尔米特矩阵具有标准正交的特征向量基,这些方法效果非常好。

我们在此不讨论这些方法,这超出了本书的范围。但是,你可以相信我,存在非常有效的数值方法来计算埃尔米特矩阵的特征值和特征向量,以及找到奇异值分解。这些方法非常有效,计算量也只比求解线性方程组略多一些。



\subsection{4.1. 单位球的像}

例如,考虑以下问题:设 $A : \RR^n \to \RR^m$ 是一个线性变换,令 $B = \{ \xx \in \RR^n : \|\xx\| \le 1 \}$ 是 $\RR^n$ 中的闭单位球。我们希望描述 $A(B)$,即我们想弄清楚单位球在仿射变换下是如何被映射的。

让我们先考虑最简单的情况,即 $A$ 是一个对角矩阵 $A = \text{diag}\{\sigma_1, \sigma_2, \dots, \sigma_n\}$,且 $\sigma_k > 0$ 对 $k = 1, 2, \dots, n$ 成立。那么对于 $\xx = (x_1, x_2, \dots, x_n)^T$ 和 $\yy = (y_1, y_2, \dots, y_n)^T = A\xx$,我们有 $y_k = \sigma_k x_k$(等价地,$x_k = y_k/\sigma_k$)对 $k = 1, 2, \dots, n$ 成立。因此,
$$\yy = (y_1, y_2, \dots, y_n)^T = A\xx~~\text{其中}~~\|\xx\| \le 1,$$
当且仅当坐标 $y_1, y_2, \dots, y_n$ 满足不等式
$$\frac{y_1^2}{\sigma_1^2} + \frac{y_2^2}{\sigma_2^2} + \dots + \frac{y_n^2}{\sigma_n^2} = \sum_{k=1}^n \frac{y_k^2}{\sigma_k^2} \le 1$$
(这只是不等式 $\|\xx\|^2 = \sum_{k} |x_k|^2 \le 1$)。

满足上述不等式点的集合被称为\textbf{椭球体}。
如果 $n = 2$,这是一个半轴长为 $\sigma_1$ 和 $\sigma_2$ 的椭圆;如果 $n = 3$,它是一个半轴长为 $\sigma_1, \sigma_2$ 和 $\sigma_3$ 的椭球体。在 $\RR^n$ 中,这个集合的几何形状也容易可视化,我们称之为半轴长为 $\sigma_1, \sigma_2, \dots, \sigma_n$ 的椭球体。向量 $\ee_1, \ee_2, \dots, \ee_n$ 或更确切地说,它们对应的直线,被称为椭球体的\textbf{主轴}。

奇异值分解本质上说明了,在任意内积空间中,任何算子都可以通过一对标准正交基变得对角化(见注记 3.11)。即,考虑奇异值分解 (3.1) 中的正交基 $\A = \{\vv_1, \vv_2, \dots, \vv_n\}$ 和 $\B = \{\ww_1, \ww_2, \dots, \ww_n\}$.~那么 $A$ 在这些基下的矩阵是 
$$\left[A\right]_{\B,\A} = \text{diag}\{\sigma_n : n=1, 2, \dots, n\}.$$
假设所有 $\sigma_k > 0$,并重复上述推理,很容易证明任何点 $\yy = A\xx \in A(B)$ 当且仅当它满足不等式:
$$\frac{y_1^2}{\sigma_1^2} + \frac{y_2^2}{\sigma_2^2} + \dots + \frac{y_n^2}{\sigma_n^2} = \sum_{k=1}^n \frac{y_k^2}{\sigma_k^2} \le 1.$$
其中 $y_1, y_2, \dots, y_n$ 是 $\yy$ 在标准正交基 $\B = \{\ww_1, \ww_2, \dots, \ww_n\}$ 下的坐标,而不是标准基下的坐标。类似地,$(x_1, x_2, \dots, x_n)^T = [\xx]_\A$.~

但这本质上是同一个椭球体,只是“旋转”了(具有不同但仍正交的主轴)!

还有一个替代的解释呈现在下面。

考虑“对角”矩阵 $\Sigma$ 的一般情况,形式如 (3.5)。很容易看出,单位球 $B$ 的像 $\Sigma B$ 是一个椭球体(不是在整个空间中,而是在 $\text{Ran } \Sigma$ 中),其半轴长为 $\sigma_1, \sigma_2, \dots, \sigma_r$.~

现在考虑一般情况,$A = W \Sigma V^*$,其中 $W, V$ 是酉算子。酉变换不改变单位球(因为它们保持范数),所以 $V^*(B) = B$.~我们知道 $\Sigma(B)$ 是 $\text{Ran } \Sigma$ 中的一个椭球体,半轴长为 $\sigma_1, \sigma_2, \dots, \sigma_r$.~酉变换不改变物体的几何形状,所以 $W(\Sigma(B))$ 也是一个椭球体,具有相同的半轴长。
从分解 $A = W \Sigma V^*$(利用 $W$ 和 $V^*$ 都是可逆的事实)可以很容易看出,$W$ 将 $\text{Ran } \Sigma$ 映射到 $\text{Ran } A$,因此我们可以得出结论:

\fbox{\begin{minipage}{0.9\textwidth}
闭单位球 $B$ 的像 $A(B)$ 是 $\text{Ran } A$ 中的一个椭球体,其半轴长为 $\sigma_1, \sigma_2, \dots, \sigma_r$.~这里 $r$ 是非零奇异值的数量,即 $A$ 的秩。
\end{minipage}}


\subsection{4.2. 线性变换的算子范数}

给定一个线性变换 $A : X \to Y$,我们考虑以下优化问题:在闭单位球 $B = \{ \xx \in X : \|\xx\| \le 1 \}$ 上,求 $\|A\xx\|$ 的最大值。

再次,奇异值分解允许我们解决这个问题。对于具有非负项的对角矩阵 $A$,最大值正好是最大的对角项。确实,设 $s_1, s_2, \dots, s_r$ 是 $A$ 的非零对角项,设 $s_1$ 是最大的。由于对于 $\xx = (x_1, x_2, \dots, x_n)^T$

$$(4.1)\quad A\xx = \sum_{k=1}^r s_k x_k \ee_k,$$
我们可以得出 
$$\|A\xx\|^2 = \sum_{k=1}^r s_k^2 |x_k|^2 \le s_1^2 \sum_{k=1}^r |x_k|^2 = s_1^2 \cdot \|\xx\|^2,$$
因此 $\|A\xx\| \le s_1 \|\xx\|$.~另一方面,$\|A\ee_1\| = \|s_1 \ee_1\| = s_1 \|\ee_1\|$,因此 $s_1$ 确实是闭单位球 $B$ 上 $\|A\xx\|$ 的最大值。
注意,在上述推理中,我们没有假设矩阵 $A$ 是方阵;我们只假设“主对角线”外的所有元素都为 0,因此公式 (4.1) 成立。

为了处理一般情况,我们考虑奇异值分解 (3.5),$A = W \Sigma V^*$,其中 $W, V$ 是酉算子,$\Sigma$ 是具有非负项的对角矩阵。由于酉变换不改变范数,我们可以得出 

\fbox{\begin{minipage}{0.9\textwidth}
$\|A\xx\|$ 在单位球 $B$ 上的最大值是 $\Sigma$ 的最大对角项,即 $A$ 的最大奇异值。
\end{minipage}}


\textbf{定义}~~ 量 $\max\{\|A\xx\| : \xx \in X, \|\xx\| \le 1\}$ 被称为 $A$ 的\textbf{算子范数},记作 $\|A\|$.~

很容易看出 $\|A\|$ 满足范数的所有性质:

1. $\|\alpha A\| = |\alpha| \cdot \|A\|$;

2. $\|A + B\| \le \|A\| + \|B\|$;

3. 对所有 $A$ 都有 $\|A\| \ge 0$;

4. $\|A\| = 0$ 当且仅当 $A = \oo$,

因此它确实是 $X$ 到 $Y$ 的线性变换空间上的一个范数。

算子范数的一个主要性质是不等式 
$$\|A\xx\| \le \|A\| \cdot \|\xx\|,$$
这很容易从范数的齐次性 $\|\xx\|$ 推导出来。

事实上,可以证明算子范数 $\|A\|$ 是满足 
$$\|A\xx\| \le C \|\xx\| \quad \forall \xx \in X$$ 
的最佳(最小)非负数 $C$.~这通常被用作算子范数的定义。

在线性变换空间上,我们已经有了一个范数,即弗罗贝尼乌斯范数,或称希尔伯特-施密特范数 $\|A\|_2$:$$\|A\|_2^2 = \text{trace}(A^*A).$$
所以,让我们来研究这两个范数是如何比较的。

设 $s_1, s_2, \dots, s_r$ 是 $A$ 的非零奇异值(计重数),设 $s_1$ 是它们中最大的。那么 $s_1^2, s_2^2, \dots, s_r^2$ 是 $A^*A$ 的非零特征值(同样计重数)。回想一下,迹等于特征值之和,我们得出 
$$\|A\|_2^2 = \text{trace}(A^*A) = \sum_{k=1}^r s_k^2.$$
另一方面,我们知道算子范数 $\|A\|$ 等于其最大奇异值,即 $\|A\| = s_1$.~因此,我们可以得出 $\|A\| \le \|A\|_2$,即

\fbox{\begin{minipage}{0.9\textwidth}
矩阵的算子范数不能大于其弗罗贝尼乌斯范数。
\end{minipage}}
\\
这个陈述也可以用柯西-施瓦茨不等式直接证明,并且这种证明在一些教材中已经给出。我们这里展示的证明的美妙之处在于,它不需要任何计算,并阐明了不等式背后的原因。

\subsection{4.3. 矩阵的条件数}

假设我们有一个可逆矩阵 $A$,并且我们想解方程 $A\xx = \bb$.~解当然是 $\xx = A^{-1}\bb$,但我们想研究如果我们只知道近似数据时会发生什么。

在现实生活中,数据是通过某些实验获得的。但即使我们有精确的数据,计算机计算过程中的舍入误差也可能产生类似效应,扭曲数据。

让我们考虑最简单的模型,假设方程的右侧有一个小的误差。这意味着,我们求解的是 $$A\xx = \bb + \Delta \bb,$$
而不是 $A\xx = \bb$.~其中 $\Delta \bb$ 是右侧 $\bb$ 的一个小扰动。因此,我们得到近似解 $\xx + \Delta \xx$,而 $A(\xx + \Delta \xx) = \bb + \Delta \bb$.~我们假设 $A$ 是可逆的,所以 $\Delta \xx = A^{-1} \Delta \bb$.~

我们想知道解中的相对误差 $\|\Delta \xx\| / \|\xx\|$ 与右侧的相对误差 $\|\Delta \bb\| / \|\bb\|$ 相比有多大。很容易看出:
$$\frac{\|\Delta \xx\|}{\|\xx\|} = \frac{\|A^{-1} \Delta \bb\|}{\|\xx\|} = \frac{\|A^{-1} \Delta \bb\|}{\|\bb\|} \frac{\|\bb\|}{\|\xx\|} = \frac{\|A^{-1} \Delta \bb\|}{\|\bb\|} \frac{\|A\xx\|}{\|\xx\|}.$$
由于 $\|A^{-1} \Delta \bb\| \le \|A^{-1}\| \cdot \|\Delta \bb\|$ 且 $\|A\xx\| \le \|A\| \cdot \|\xx\|$,我们可以得出:
$$\frac{\|\Delta \xx\|}{\|\xx\|} \le \|A^{-1}\| \cdot \|A\| \cdot \frac{\|\Delta \bb\|}{\|\bb\|}.$$

$\|A\| \cdot \|A^{-1}\|$ 这个量被称为矩阵的\textbf{条件数}(condition number)。它估计了解的相对误差 $\xx$ 在多大程度上取决于右侧 $\bb$ 的相对误差。

让我们看看这个量与奇异值是如何关联的。设 $s_1, s_2, \dots, s_n$ 是 $A$ 的奇异值,并且假设 $s_1$ 是最大奇异值,$s_n$ 是最小奇异值。我们知道算子(算子)范数等于其最大奇异值,所以 
$$\|A\| = s_1,\quad  \|A^{-1}\| = \frac{1}{s_n},$$
因此 
$$\|A\| \cdot \|A^{-1}\| = \frac{s_1}{s_n}.$$
换句话说,

\fbox{\begin{minipage}{0.9\textwidth}
矩阵的条件数等于最大和最小奇异值之比。
\end{minipage}}

我们上面推导出 $\|\Delta \xx\| / \|\xx\| \le \|A^{-1}\| \cdot \|A\| \cdot \|\Delta \bb\| / \|\bb\|$.~不难看出,这个估计是尖锐的,即可以选择右侧 $\bb$ 和误差 $\Delta \bb$,使得等式成立:
$$\frac{\|\Delta \xx\| } {\|\xx\|} = \|A^{-1}\| \cdot \|A\| \cdot \frac{\|\Delta \bb\| }{ \|\bb\|}.$$
我们只需取 $\bb = \ww_1$ 且 $\Delta \bb = \alpha \ww_n$,其中 $\ww_1$ 和 $\ww_n$ 分别是奇异值分解 $A = W \Sigma V^*$ 中 $W$ 的第一列和最后一列,$\alpha \ne 0$ 是任意标量。这里,像往常一样,奇异值假设为非递增排序 $s_1 \ge s_2 \ge \dots \ge s_n$,所以 $s_1$ 是最大的,而 $s_n$ 是最小的。

我们将细节留给读者作为练习。

如果一个矩阵的条件数不是太大,则称该矩阵是\textbf{良态}的(well conditioned)。如果条件数很大,则称该矩阵是\textbf{病态}的(ill conditioned)。这里的“大”取决于具体问题:你能在多大程度上确定你的右侧数据,对解需要多高的精度等等。

\subsection{4.4. 矩阵的有效秩}

理论上,矩阵的秩很容易计算:只需对矩阵进行行变换并计算主元即可。然而,在实际应用中,情况并非如此简单。主要原因是,我们通常不知道精确的矩阵,只知道其近似值,精度有限。

此外,即使我们知道精确的矩阵,大多数计算机程序在计算过程中也会引入舍入误差,因此我们实际上无法区分一个零主元和一个非常小的主元。

一种简单粗暴的工作方法是这样的:在计算秩(以及与之相关的其他对象,如列空间、核等)时,只需设置一个容差(一个小的数),如果主元小于容差,就将其视为零。
这种方法的优点在于它的简单性,因为它非常容易编程。然而,主要的缺点是无法看出容差的作用。例如,如果我们设置容差为 $10^{-6}$,我们失去了什么?$10^{-8}$ 会好多少?虽然上述方法对良态矩阵效果很好,但在一般情况下并不可靠。

一个更好的方法是使用奇异值。它需要更多的计算,但能给出更好、更易于解释的结果。在这种方法中,我们也设定一个小的数作为容差,然后进行奇异值分解。之后,我们简单地将小于容差的奇异值视为零。这种方法的优点在于我们可以清楚地看到我们在做什么。奇异值是椭球体 $A(B)$($B$ 是闭单位球)的半轴长,因此通过设置容差,我们只是决定了椭球体应该有多“细”才被认为是“扁平”的。

\subsection{4.5. 摩尔-彭罗斯(伪)逆}

正如我们在第 5 章第 4 节中所讨论的,在方程 $A\xx = \bb$ 没有解的情况下,最小二乘解给了我们“次优”的解决方案(当方程有解时,它也给出了 $A\xx = \bb$ 的解)。

注意,最小二乘解并未解决唯一性问题:方程 $A^*A\xx = A^*\bb$ 的解不一定唯一。一个自然区别开来的解是具有最小范数的解;这样的解确实是唯一的,并且可以通过取任意一个解,然后将其投影到 $(\text{Ker } A^*A)^\perp = (\text{Ker } A)^\perp$ 上得到(参见第 5 章的问题 4.5 和 4.6)。

不难看出,如果 $A = \tilde{W} \tilde{\Sigma} \tilde{V}^*$ 是 $A$ 的\textbf{约简}奇异值分解,那么最小范数最小二乘解 $\xx_0$ 由下式给出:
$$(4.2)\quad \xx_0 = \tilde{V} \tilde{\Sigma}^{-1} \tilde{W}^* \bb.$$ 
确实,$\xx_0$ 是 $A\xx = \bb$ 的一个最小二乘解(即 $A\xx = P_{\text{Ran } A} \bb$ 的解):
$$A\xx_0 = \tilde{W} \tilde{\Sigma} \tilde{V}^* (\tilde{V} \tilde{\Sigma}^{-1} \tilde{W}^* \bb) = \tilde{W} \tilde{\Sigma} \tilde{\Sigma}^{-1} \tilde{W}^* \bb = \tilde{W} \tilde{W}^* \bb = P_{\text{Ran } A} \bb;$$
在链的最后一个等式中,我们使用了 $\tilde{W} \tilde{W}^* = P_{\text{Ran } \tilde{W}}$($P_{\text{Ran } \tilde{W}} = \tilde{W} (\tilde{W}^* \tilde{W})^{-1} \tilde{W}^* = \tilde{W} \tilde{W}^*$)并且 $\text{Ran } \tilde{W} = \text{Ran } A$(见问题 4.4)。

$A\xx = P_{\text{Ran } A} \bb$ 的一般解由 
$$\xx = \xx_0 + \yy,\quad \yy \in \text{Ker } A$$
给出,因此 $\xx_0$ 确实是 $A\xx = P_{\text{Ran } A} \bb$ 的唯一最小范数解,或者等价地,是 $A\xx = \bb$ 的最小范数最小二乘解。

\textbf{定义 4.1.} 算子 $A^+ := \tilde{V} \tilde{\Sigma}^{-1} \tilde{W}^*$,其中 $A = \tilde{W} \tilde{\Sigma} \tilde{V}^*$ 是 $A$ 的\textbf{约简}奇异值分解,称为 $A$ 的\textbf{摩尔-彭罗斯逆}(或\textbf{摩尔-彭罗斯伪逆})(Moore–Penrose (pseudo)inverse)。换句话说,\textbf{摩尔-彭罗斯逆}是给出 $A\xx = \bb$ 的唯一最小二乘解的算子。

\textbf{注记 4.2.} 文献中通常将摩尔-彭罗斯逆定义为一个矩阵 $A^+$,它满足:

1. $AA^+A = A$;

2. $A^+AA^+ = A^+$;

3. $(AA^+)^* = AA^+$;

4. $(A^+A)^* = A^+A$.~

很容易验证算子 $A^+ := \tilde{V} \tilde{\Sigma}^{-1} \tilde{W}^*$ 满足上述性质 1-4。

还可以(尽管有点难)证明满足性质 1-4 的算子 $A^+$ 是唯一的。
确实,通过用 $A^+$ 左乘或右乘等式 1,我们得到 $(A^+A)^2 = A^+A$ 和 $(AA^+)^2 = AA^+$;与性质 3 和 4 一起,这意味着 $A^+A$ 和 $AA^+$ 是正交投影(见第 5 章问题 5.6)。

显然,$\text{Ker } A \subset \text{Ker } A^+A$.~另一方面,等式 1 暗示 $\text{Ker } A^+A \subset \text{Ker } A$(为什么?),所以 $\text{Ker } A^+A = \text{Ker } A$.~但这表明 $A^+A$ 是 $(\text{Ker } A)^\perp = \text{Ran } A^*$ 上的正交投影,
$$A^+A = P_{\text{Ran } A^*}.$$

性质 1 也暗示了 $AA^+\yy = \yy$ 对所有 $\yy \in \text{Ran } A$ 成立。由于 $AA^+$ 是一个正交投影,我们得出 $\text{Ran } A \subset \text{Ran } AA^+$.~相反的包含关系 $\text{Ran } AA^+ \subset \text{Ran } A$ 是平凡的,所以 $AA^+$ 是 $\text{Ran } A$ 上的正交投影,
$$AA^+ = P_{\text{Ran } A}.$$

知道了 $A^+A$ 和 $AA^+$,我们可以将性质 2 重写为 
$$P_{\text{Ran } A^*} A^+ = A^+ \quad \text{或} \quad A^+ P_{\text{Ran } A} = A^+.$$
结合上述恒等式,我们得到 $$P_{\text{Ran } A^*} A^+ P_{\text{Ran } A} = A^+.$$

最后,对于 $A$ 的目标空间中的任何 $\bb$,令 
$$\xx_0 := A^+\bb = P_{\text{Ran } A^*} A^+ \bb \in \text{Ran } A^*$$ 
并且 
$$A\xx_0 = AA^+\bb = P_{\text{Ran } A} \bb,$$
即 $\xx_0$ 是 $A\xx = \bb$ 的一个最小二乘解。由于 $\xx_0 \in \text{Ran } A^* = (\text{Ker } A)^\perp$, $\xx_0$ 如前所述,是最小范数的最小二乘解。但是,如我们之前所示,这样的最小范数解由 (4.2) 给出,所以 $A^+ = \tilde{V} \tilde{\Sigma}^{-1} \tilde{W}^*$.~

\begin{exer} \textbf{练习}~~

4.1. 求以下矩阵的范数和条件数:

a) $A = \begin{pmatrix} 4 & 0 \\ 1 & 3 \end{pmatrix}$.~
    
b) $A = \begin{pmatrix} 5 & 3 \\ -3 & 3 \end{pmatrix}$.~

对于 a) 部分的矩阵 $A$,给出一个右侧 $\bb$ 和误差 $\Delta \bb$ 的例子,使得 
$$\frac{\|\Delta \xx\|} {\|\xx\| }= \|A\| \cdot \|A^{-1}\| \cdot \frac{\|\Delta \bb\| }{\|\bb\|};$$
这里 $A\xx = \bb$ 且 $A(\xx + \Delta \xx) = \bb + \Delta \bb$.~

4.2. 设 $A$ 是一个正常算子,其特征值为 $\lambda_1, \lambda_2, \dots, \lambda_n$(计重数)。证明 $A$ 的奇异值是 $|\lambda_1|, |\lambda_2|, \dots, |\lambda_n|$.~

4.3. 求矩阵 $$A = \begin{pmatrix} 2 & 1 & 1 \\ 1 & 2 & 1 \\ 1 & 1 & 2 \end{pmatrix}$$ 的奇异值、范数和条件数。
你可以基本上不经计算完成此问题,如果你能回答以下问题:

a) 某个子空间 $E$ 上的正交投影 $P_E$ 的奇异值是多少?
    
b) 跨越向量 $(1, 1, 1)^T$ 的子空间的零空间的矩阵是什么?

c) 算子 $T$ 和 $aT + bI$ (其中 $a$ 和 $\bb$ 是标量)的特征值之间有什么关系?

当然,你也可以直接进行计算。

4.4. 设 $A = \tilde{W} \tilde{\Sigma} \tilde{V}^*$ 是 $A$ 的约简奇异值分解。证明 $\text{Ran } A = \text{Ran } \tilde{W}$,然后通过取伴随矩阵证明 $\text{Ran } A^* = \text{Ran } \tilde{V}$.~

4.5. 用奇异值分解 $A = W \Sigma V^*$ 表示摩尔-彭罗斯逆 $A^+$ 的公式。

4.6. (提霍诺夫正则化):证明摩尔-彭罗斯逆 $A^+$ 可以计算为极限:
$$A^+ = \lim_{\varepsilon \to 0^+} (A^*A + \varepsilon I)^{-1} A^* = \lim_{\varepsilon \to 0^+} A^*(AA^* + \varepsilon I)^{-1}.$$\end{exer}

\section{5. 正交矩阵的结构}

一个行列式为 1 的正交矩阵 $U$ 通常被称为\textbf{旋转}(rotation)。下面的定理解释了这个名称。

\textbf{定理 5.1.} 设 $U$ 是 $\RR^n$ 中的一个正交算子,且 $\det U = 1$.~
\footnote{
对于一个正交矩阵$U$,它的行列式为$\pm 1$.
}
则存在一个标准正交基 $\vv_1, \vv_2, \dots, \vv_n$,使得 $U$ 在该基下的矩阵具有分块对角形式:

$$\begin{pmatrix} R_{\phi_1} & & & & \\ & R_{\phi_2} & & & \\ & & \ddots & & \\ & & & R_{\phi_k} & \\ & & & & I_{n-2k} \end{pmatrix},$$
其中 $R_{\phi_k}$ 是 $2$ 维旋转矩阵,
$$R_{\phi_k} = \begin{pmatrix} \cos \phi_k & -\sin \phi_k \\ \sin \phi_k & \cos \phi_k \end{pmatrix},$$
而 $I_{n-2k}$ 表示 $(n-2k) \times (n-2k)$ 的单位矩阵。

\textbf{证明}~~ 我们知道,如果 $p$ 是一个实系数多项式,并且 $\lambda$ 是它的复根,$p(\lambda) = 0$,那么 $\bar{\lambda}$ 也是 $p$ 的根,$p(\bar{\lambda}) = 0$(这可以通过将 $\bar{\lambda}$ 代入 $p(z) = \sum_{k=0}^n a_k z^k$ 来很容易地验证)。

因此,实矩阵 $A$ 的所有复特征值可以配对成 $\lambda_k, \bar{\lambda}_k$.~

我们知道,酉矩阵的特征值绝对值都为 1,所以 $A$ 的所有复特征值都可以写成 $\lambda_k = \cos \alpha_k + \ii \sin \alpha_k$ 和 $\bar{\lambda}_k = \cos \alpha_k - \ii \sin \alpha_k$.~

固定一对复特征值 $\lambda$ 和 $\bar{\lambda}$,设 $\uu \in \CC^n$ 是 $U$ 的特征向量,$U\uu = \lambda \uu$.~那么 $U\bar{\uu} = \bar{\lambda} \bar{\uu}$.~现在,将 $\uu$ 分解为实部和虚部,即定义 
$$\xx := \text{Re } \uu = \frac{\uu + \bar{\uu}}{2},\quad \yy := \text{Im } \uu = \frac{\uu - \bar{\uu}}{2\ii}$$
(注意,$\xx, \yy$ 是实向量,即所有项均为实数的向量)。那么 $\uu = \xx + \ii \yy$.
% (我们在此处定义 $\uu$ 使得 $\uu = \xx+\ii \yy$)。
那么 
$$U \xx = U \frac{\uu + \bar{\uu}}{2} = \frac{1}{2}(U\uu + U\bar{\uu}) = \frac{1}{2}(\lambda \uu + \bar{\lambda} \bar{\uu}) = \text{Re}(\lambda \uu).$$
类似地,$$U \yy = U \frac{\uu - \bar{\uu}}{2\ii} = \frac{1}{2\ii}(U\uu - U\bar{\uu}) = \frac{1}{2\ii}(\lambda \uu - \bar{\lambda} \bar{\uu}) = \text{Im}(\lambda \uu).$$
由于 $\lambda = \cos \alpha + \ii \sin \alpha$,我们有
$$\lambda \uu = (\cos \alpha + \ii \sin \alpha)(\xx + \ii \yy) = ((\cos \alpha)\xx - (\sin \alpha)\yy) + \ii((\cos \alpha)\yy + (\sin \alpha)\xx).$$
所以 $$U \xx = \text{Re}(\lambda \uu) = (\cos \alpha)\xx - (\sin \alpha)\yy,\quad U \yy = \text{Im}(\lambda \uu) = (\cos \alpha)\yy + (\sin \alpha)\xx.$$

换句话说,$U$ 将由向量 $\xx, \yy$ 生成的二维子空间 $E_\lambda$ 保持不变,即 $E_\lambda$ 是 $U$ 的不变子空间,且 $U$ 在该子空间上的限制矩阵是旋转矩阵 
$$R_{-\alpha} = \begin{pmatrix} \cos \alpha & \sin \alpha \\ -\sin \alpha & \cos \alpha \end{pmatrix}.$$
% (注意:这里的旋转方向与定理中的 $R_{\phi_k}$ 可能相反,取决于如何定义旋转角度。如果我们按照定理中的约定,矩阵将是 $R_\alpha$)。
注意,向量 $\uu$ 和 $\bar{\uu}$(酉矩阵对应于不同特征值的特征向量)是正交的,所以根据勾股定理 
$$\|\xx\| = \|\yy\| = \frac{1}{\sqrt{2}}\|\uu\|,$$
很容易检查 $\xx \perp \yy$,所以 $\xx, \yy$ 是 $E_\lambda$ 中的一个正交基。如果我们乘以每个向量 $\xx, \yy$ 相同的非零数,我们不会改变线性变换的矩阵,所以我们可以无妨碍地假设 $\|\xx\| = \|\yy\| = 1$,即 $\xx, \yy$ 是 $E_\lambda$ 中的一个标准正交基。

让我们将标准正交向量组 $\vv_1 = \xx, \vv_2 = \yy$ 补充成 $\RR^n$ 中的一个标准正交基。由于 $UE_\lambda \subset E_\lambda$,即 $E_\lambda$ 是 $U$ 的不变子空间,在该基下的 $U$ 的矩阵具有分块三角形式:
$$\begin{pmatrix} R_{-\alpha} & * \\ \oo & U_1 \end{pmatrix},$$
其中 $\oo$ 表示一个 $(n-2) \times 2$ 的零块。

由于旋转矩阵 $R_{-\alpha}$ 是可逆的,我们有 $U E_\lambda = E_\lambda$.~因此 
$$U^* E_\lambda = U^{-1} E_\lambda = E_\lambda,$$所以我们构造的基下的 $U$ 的矩阵实际上是分块对角形式:
$$\begin{pmatrix} R_{-\alpha} & \oo \\ \oo & U_1 \end{pmatrix}.$$
由于 $U$ 是酉的,
$$I = U^*U = \begin{pmatrix} I & \oo \\ \oo & U_1^* U_1 \end{pmatrix},$$
所以,由于 $U_1$ 是方阵,它也是酉的。

如果 $U_1$ 有复特征值,我们可以应用相同的过程将其大小减 2,直到我们剩下只具有实特征值的块。实特征值只能是 $+1$ 或 $-1$,所以在一个标准正交基下,$U$ 的矩阵具有以下形式:
$$\begin{pmatrix} R_{-\alpha_1} & & & & & \\ & R_{-\alpha_2} & & & & \\ & & \ddots & & & \\ & & & R_{-\alpha_d} & & \\ & & & & -I_r & \\ & & & & & I_l \end{pmatrix};$$
这里 $I_r$ 和 $I_l$ 分别是 $r \times r$ 和 $l \times l$ 的单位矩阵。由于 $\det U = 1$,特征值 $-1$ 的重数(即 $r$)必须是偶数。

注意,$2 \times 2$ 矩阵 $-I_2$ 可以解释为通过角度 $\pi$ 的旋转。因此,上述矩阵具有定理结论中的形式,其中 $\phi_k = -\alpha_k$ 或 $\phi_k = \pi$.~

让我们给出定理 5.1 的另一个解释。定义 $T_j$ 为在由向量 $\vv_j, \vv_{j+1}$ 张成的平面中的一次 $\phi_j$ 旋转。那么定理 5.1 简单地说 $U$ 是旋转 $T_j, j = 1, 2, \dots, k$ 的复合。注意,由于旋转 $T_j$ 在相互正交的平面上作用,它们是可交换的,也就是说,复合的顺序并不重要。因此,该定理可以解释为:

\fbox{\begin{minipage}{0.9\textwidth}
任何 $\RR^n$ 中的旋转都可以表示为最多 $n/2$ 个可交换的平面旋转的复合。
\end{minipage}}


如果一个正交矩阵的行列式为 $-1$,其结构由以下定理描述。

\textbf{定理 5.2.} 设 $U$ 是 $\RR^n$ 中的一个正交算子,且 $\det U = -1$.~则存在一个标准正交基 $\vv_1, \vv_2, \dots, \vv_n$,使得 $U$ 在该基下的矩阵具有分块对角形式:
$$\begin{pmatrix} R_{\phi_1} & & & & & \\ & R_{\phi_2} & & & & \\ & & \ddots & & & \\ & & & R_{\phi_k} & & \\ & & & & I_r & \\ & & & & & -1 \end{pmatrix},$$
其中 $r = n - 2k - 1$,并且 $R_{\phi_k}$ 是 $2$ 维旋转矩阵,
$$R_{\phi_k} = \begin{pmatrix} \cos \phi_k & -\sin \phi_k \\ \sin \phi_k & \cos \phi_k \end{pmatrix},$$
而 $I_{n-2k}$ 表示 $(n-2k) \times (n-2k)$ 的单位矩阵。

我们把证明留给读者作为练习。对定理 5.1 证明的修改是很明显的。

注意,从上述定理可以得出,一个行列式为 $-1$ 的 $2 \times 2$ 正交矩阵总是反射。

现在让我们固定一个标准正交基,例如 $\RR^n$ 中的标准基。我们称一个\textbf{初等旋转}(elementary rotation)\footnote{
这个术语并没有被广泛接受。
} 
是在 $x_j$ - $x_k$ 平面中的一次旋转,即一个只改变坐标 $x_j$ 和 $x_k$ 的线性变换,并且它在这两个坐标上像平面旋转一样作用。

\textbf{定理 5.3.} 任何旋转 $U$(即 $\det U = 1$ 的正交变换)都可以表示为最多 $n(n-1)/2$ 个初等旋转的乘积。

为了证明这个定理,我们需要以下简单的引理。

\textbf{引理 5.4.} 设 $\xx = (x_1, x_2)^T \in \RR^2$.~存在一个 $\RR^2$ 的旋转 $R_\alpha$,它将向量 $\xx$ 移动到向量 $(a, 0)^T$,其中 $a = \sqrt{x_1^2 + x_2^2}$.~

证明是基本的,我们将其作为读者练习。你可以画一张图或者写出 $R_\alpha$ 的公式。

\textbf{引理 5.5.} 设 $\xx = (x_1, x_2, \dots, x_n)^T \in \RR^n$.~存在 $n-1$ 个初等旋转 $R_1, R_2, \dots, R_{n-1}$,使得 $R_{n-1} \dots R_2 R_1 \xx = (a, 0, 0, \dots, 0)^T$,其中 $a = \sqrt{x_1^2 + x_2^2 + \dots + x_n^2}$.~

\textbf{证明}~~ 引理证明的思路非常简单。我们使用一个初等旋转 $R_1$(在 $x_{n-1}$ - $x_n$ 平面中)来“消去” $\xx$ 的最后一个坐标(引理 5.4 保证了这样的旋转存在)。然后使用一个初等旋转 $R_2$(在 $x_{n-2}$ - $x_{n-1}$ 平面中)来“消去” $R_1 \xx$ 的第 $n-1$ 个坐标(旋转 $R_2$ 不改变最后一个坐标,所以 $R_2 R_1 \xx$ 的最后一个坐标保持为零),以此类推。

为了进行正式证明,我们将使用数学归纳法。$n=1$ 的情况是平凡的,因为 $\RR^1$ 中的任何向量都具有所需的形状。$n=2$ 的情况由引理 5.4 处理。

现在假设引理对 $n-1$ 成立,我们将证明对 $n$ 成立。根据引理 5.4,存在一个 $2 \times 2$ 旋转矩阵 $R_\alpha$,使得 
$$R_\alpha \begin{pmatrix} x_{n-1} \\ x_n \end{pmatrix} = \begin{pmatrix} a_{n-1} \\ 0 \end{pmatrix},$$其中 $a_{n-1} = \sqrt{x_{n-1}^2 + x_n^2}$.~那么如果我们定义 $n \times n$ 的初等旋转 $R_1$ 为:
$$R_1 = \begin{pmatrix} I_{n-2} & \oo \\ \oo & R_\alpha \end{pmatrix}$$
($I_{n-2}$ 是 $(n-2) \times (n-2)$ 的单位矩阵),那么 $$R_1 \xx = (x_1, x_2, \dots, x_{n-2}, a_{n-1}, 0)^T.$$

我们假设引理的结论对 $n-1$ 成立,因此存在 $n-2$ 个初等旋转(我们称它们为 $R_2, R_3, \dots, R_{n-1}$),它们在 $\RR^{n-1}$ 中(仅作用于坐标 $x_1, x_2, \dots, x_{n-1}$)将向量 $(x_1, x_2, \dots, x_{n-1}, a_{n-1})^T \in \RR^{n-1}$ 变换为向量 $(a, 0, \dots, 0)^T \in \RR^{n-1}$.~换句话说,
$$R_{n-1} \dots R_3 R_2 (x_1, x_2, \dots, x_{n-1}, a_{n-1})^T = (a, 0, \dots, 0)^T.$$

我们可以总假设初等旋转 $R_2, R_3, \dots, R_{n-1}$ 在 $\RR^n$ 中作用,只需假设它们不改变最后一个坐标。那么 $$R_{n-1} \dots R_3 R_2 R_1 \xx = (a, 0, \dots, 0)^T \in \RR^n.$$

现在我们来证明 $a = \sqrt{x_1^2 + x_2^2 + \dots + x_n^2}$.~这可以通过直接计算轻松验证,但我们采用间接推理。我们知道正交变换保持范数,并且我们知道 $a \ge 0$.~但是,那么我们就没有任何选择,唯一的可能性就是 $a = \sqrt{x_1^2 + x_2^2 + \dots + x_n^2}$.~

\textbf{引理 5.6.} 设 $A$ 是一个具有实数项的 $n \times n$ 矩阵。存在初等旋转 $R_1, R_2, \dots, R_N$,$N \le n(n-1)/2$,使得矩阵 $B = R_N \dots R_2 R_1 A$ 是上三角的,并且其所有对角线元素,除了最后一个 $B_{n,n}$ 之外,都是非负的。

\textbf{证明}~~ 我们将使用数学归纳法。$n=1$ 的情况是平凡的,因为我们可以说任何 $1 \times 1$ 矩阵都具有所需的形状。

让我们考虑 $n=2$ 的情况。设 $\aaa_1$ 是 $A$ 的第一列。根据引理 5.4,存在一个旋转 $R$,它可以“消去” $\aaa_1$ 的第二个坐标,使得第一个坐标非负。然后矩阵 $B = RA$ 具有所需的形状。

现在假设引理对 $(n-1) \times (n-1)$ 矩阵成立,我们想对 $n \times n$ 矩阵证明它。对于 $n \times n$ 矩阵 $A$,设 $\aaa_1$ 是它的第一列。根据引理 5.5,我们可以找到 $n-1$ 个初等旋转(例如 $R_1, R_2, \dots, R_{n-1}$),它们将 $\aaa_1$ 变换为 $(a, 0, \dots, 0)^T$.~那么矩阵 $R_{n-1} \dots R_2 R_1 A$ 具有以下分块三角形式:
$$R_{n-1} \dots R_2 R_1 A = \begin{pmatrix} a & * \\ \oo & A_1 \end{pmatrix},$$
其中 $A_1$ 是一个 $(n-1) \times (n-1)$ 的块。

我们假设引理对 $n-1$ 成立,所以 $A_1$ 可以通过最多 $(n-1)(n-2)/2$ 个旋转变换成所需的上三角形式。注意,这些旋转作用在 $\RR^{n-1}$ 中(只作用于坐标 $x_2, x_3, \dots, x_n$),但我们可以总假设它们作用在整个 $\RR^n$ 中,只需假设它们不改变第一个坐标。那么,这些旋转不会改变 $R_{n-1} \dots R_2 R_1 A$ 的第一列向量 $(a, 0, \dots, 0)^T$.~因此,矩阵 $A$ 可以通过最多 $n-1 + (n-1)(n-2)/2 = n(n-1)/2$ 个初等旋转变换成所需的上三角形式。

\textbf{定理 5.3 的证明}~~ 根据引理 5.5,存在初等旋转 $R_1, R_2, \dots, R_N$,使得矩阵 $U_1 = R_N \dots R_2 R_1 U$ 是上三角的,并且除最后一个对角线元素 $B_{n,n}$ 外,所有对角线元素都是非负的。

注意,矩阵 $U_1$ 是正交的。任何正交矩阵都是正常的,我们知道一个上三角矩阵只有在它是对角矩阵时才是正常的。因此,$U_1$ 是一个对角矩阵。

我们知道正交矩阵的特征值只能是 $1$ 或 $-1$,所以我们只能在 $U_1$ 的对角线上有 $1$ 或 $-1$.~但是,我们知道 $U_1$ 的所有对角线元素,除了最后的之外,都是非负的,所以 $U_1$ 的所有对角线元素,除了最后的之外,都是 $1$.~最后一个对角线元素可以是 $\pm 1$.~

由于初等旋转的行列式为 1,我们可以得出 $\det U_1 = \det U = 1$,所以最后一个对角线元素也必须是 1。因此 $U_1 = I$,所以 $U$ 可以表示为初等旋转的乘积 $U = R_1^{-1} R_2^{-1} \dots R_N^{-1}$.~这里我们使用了初等旋转的逆也是初等旋转这一事实。

\section{6. 方向}

\subsection{6.1. 动机}
图\ref{fig:04} 和图\ref{fig:05} 分别展示了 $\RR^2$ 和 $\RR^3$ 中的 3 个标准正交基。在每张图中,基 b) 可以通过一次旋转从标准基 a) 获得,而不可能通过旋转将标准基 a) 变成基 c)(使得 $\ee_k$ 变成 $\vv_k$ 对所有 $k$ 成立)。

\begin{figure}[ht]
  \centering  \includegraphics[width=0.7\linewidth]{figures/Figure4.PNG}
  \caption{$\RR^2$上的方向}
  \label{fig:04} 
\end{figure}

\begin{figure}[ht]
  \centering  \includegraphics[width=0.7\linewidth]{figures/Figure5.PNG}
  \caption{$\RR^3$上的方向}
  \label{fig:05} 
\end{figure}

你可能以前听过“方向”这个词,并且可能知道基 a) 和 b) 具有正方向,而基 c) 的方向是负的。你可能还知道一些确定方向的规则,例如物理学中的右手定则。所以,如果你能“看到”一个基,比如在 $\RR^3$ 中,你大概可以判断它的方向是什么。

但如果只给出向量 $\vv_1, \vv_2, \vv_3$ 的坐标呢?当然,你可以尝试画一张图来可视化向量,然后看看方向是什么。但这并非总是一件容易的事。更重要的是,你如何“告诉”计算机呢?

事实证明,有一个更简单的方法。让我们来解释一下。我们需要检查是否有可能通过旋转标准基 $\ee_1, \ee_2, \ee_3$ 来得到基 $\vv_1, \vv_2, \vv_3$ 在 $\RR^3$ 中。存在一个唯一的线性变换 $U$,使得 $$U\ee_k = \vv_k,\quad k=1, 2, 3;$$
它的矩阵(在标准基下)是由列向量 $\vv_1, \vv_2, \vv_3$ 组成的。它是一个正交矩阵(因为它将一个标准正交基变换为另一个标准正交基),所以我们需要看看它何时是旋转。定理 5.1 和 5.2 给出了答案:矩阵 $U$ 是旋转当且仅当 $\det U = 1$.~注意(对于 $3 \times 3$ 矩阵),如果 $\det U = -1$,那么 $U$ 是围绕某个轴的旋转与该旋转平面(即该轴的垂直平面)上的反射的复合。

这为下面的形式化定义提供了动机。

\subsection{6.2. 形式定义}

设 $\A $ 和 $\B$ 是\textbf{实}向量空间 $X$ 中的两个基。如果坐标变换矩阵 $[I]_{\B,\A}$ 的行列式为正,我们说基 $\A$ 和 $\B$ 具有\textbf{相同}的方向;如果行列式为负,我们说它们具有\textbf{不同}的方向。

注意,由于 $[I]_{\A,\B} = [I]_{\B,\A}^{-1}$,可以在定义中使用矩阵 $[I]_{\A,\B}$.~

我们通常假设 $\RR^n$ 中的标准基 $\ee_1, \ee_2, \dots, \ee_n$ 具有正方向。在一个抽象空间中,只需要固定一个基,并令其方向为正。

如果在 $\RR^n$ 中,一个标准正交基 $\vv_1, \vv_2, \dots, \vv_n$ 具有正方向(即与标准基相同的方向),那么定理 5.1 和 5.2 说明基 $\vv_1, \vv_2, \dots, \vv_n$ 是通过一次旋转从标准基获得的。

\red{这里也有张图。}

\subsection{6.3. 基的连续变换与方向}

\textbf{定义}~~ 我们说基 $\A = \{\aaa_1, \aaa_2, \dots, \aaa_n\}$ 可以\textbf{连续地}变换为基 $\B = \{\bb_1, \bb_2, \dots, \bb_n\}$,如果存在一个基的连续族 $\mathcal{V}(t) = \{\vv_1(t), \vv_2(t), \dots, \vv_n(t)\}, t \in [a, b]$,使得 $$\vv_k(a) = \aaa_k,\quad \vv_k(b) = \bb_k,\quad k = 1, 2, \dots, n.$$
“连续的基族”意味着向量函数 $\vv_k(t)$ 是连续的(它们在某个基下的坐标是连续函数),并且,至关重要的是,系统 $\vv_1(t), \vv_2(t), \dots, \vv_n(t)$ 在所有 $t \in [a, b]$ 都是一个基。

注意,通过进行变量替换,我们可以总假设(如果需要)$[a, b] = [0, 1]$.~

\textbf{定理 6.1.} 两个基  $\A = \{\aaa_1, \aaa_2, \dots, \aaa_n\}$ 和 $\B = \{\bb_1, \bb_2, \dots, \bb_n\}$ 具有相同方向,当且仅当其中一个基可以连续地变换为另一个基。

\textbf{证明}~~ 假设基 $\A$ 可以连续地变换为基 $\B$,设 $\mathcal{V}(t), t \in [a, b]$ 是一个连续的基族,执行这个变换。考虑一个矩阵函数 $V(t)$,其列向量是 $\vv_k(t)$ 在基 $\A$ 下的坐标向量 $[\vv_k(t)]_\A$.~

显然,$V(t)$ 的元素是连续函数,并且 $V(a) = I$, $V(b) = [I]_{\A,\B}$.~注意,因为 $\mathcal{V}(t)$ 始终是一个基,$\det V(t)$ 永远不为零。那么,介值定理断言 $\det V(a)$ 和 $\det V(b)$ 具有相同的符号。由于 $\det V(a) = \det I = 1$,我们可以得出 $[I]_{\A,\B}$ 的行列式 $$\det[I]_{\A,\B} = \det V(b) > 0,$$
所以基 $\A$ 和 $\B$ 具有相同方向。

为了证明反向蕴含,即定理的“仅当”部分,需要证明单位矩阵 $I$ 可以通过可逆矩阵连续地变换为任何满足 $\det B > 0$ 的矩阵 $B$.~换句话说,需要证明存在一个连续的矩阵函数 $V(t)$ 在区间 $[a, b]$ 上,使得对所有 $t \in [a, b]$ 矩阵 $V(t)$ 是可逆的,并且 
$$V(a) = I,\quad V(b) = B.$$
我们将证明这个事实留给读者作为练习。有几种方法可以证明这一点,其中一种在下面的问题 6.2-6.5 中概述。

\begin{exer} \textbf{练习}~~

6.1. 设 $R_\alpha$ 是 $\alpha$ 角的旋转,其在标准基下的矩阵为 
$$\begin{pmatrix} \cos \alpha & -\sin \alpha \\ \sin \alpha & \cos \alpha \end{pmatrix}.$$
求 $R_\alpha$ 在基 $\vv_1, \vv_2$,其中 $\vv_1 = \ee_2, \vv_2 = \ee_1$ 下的矩阵。

6.2. 设 $$R_\alpha = \begin{pmatrix} \cos \alpha & -\sin \alpha \\ \sin \alpha & \cos \alpha \end{pmatrix}$$ 是旋转矩阵。证明 $2 \times 2$ 单位矩阵 $I_2$ 可以通过可逆矩阵连续变换为 $R_\alpha$.~

6.3. 设 $U$ 是一个 $n \times n$ 正交矩阵,且 $\det U > 0$.~证明 $n \times n$ 单位矩阵 $I_n$ 可以通过可逆矩阵连续变换为 $U$.~
\textbf{提示:} 使用前一个问题和旋转在 $\RR^n$ 中的表示(作为平面旋转的乘积),见第 5 节。

6.4. 设 $A$ 是一个 $n \times n$ 正定埃尔米特矩阵,$A = A^* > \oo$.~证明 $n \times n$ 单位矩阵 $I_n$ 可以通过可逆矩阵连续变换为 $A$.~
\textbf{提示:} 对角矩阵怎么样?

6.5. 使用极分解和上面问题 6.3、6.4,完成定理 6.3 的“仅当”部分的证明。
\end{exer}


