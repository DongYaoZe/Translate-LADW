
\chapter{第九章~~高级谱理论
}

\section{1. 凯莱-哈密顿定理}

\textbf{定理 1.1}~~ (凯莱-哈密顿,Cayley–Hamilton)
设 $A$ 是一个方阵,其特征多项式为 $p(\lambda) = \det(A - \lambda I)$.~那么,$p(A) = \oo$.~

\textbf{一个错误的证明}~~
这个证明看起来异常简单:将 $\lambda$ 替换为 $A$ 代入特征多项式的定义中,我们得到
$$p(A) = \det(A - AI) = \det(\oo) = 0.$$

但这是一个错误的证明!为了明白为什么错误,让我们分析一下定理的内容。定理说明,如果我们计算特征多项式 
$$\det(A - \lambda I) = p(\lambda) = \sum_{k=0}^n c_k \lambda^k,$$
然后用矩阵 $A$ 替换 $\lambda$ 得到 
$$p(A) := \sum_{k=0}^n c_k A^k = c_0 I + c_1 A + \dots + c_n A^n,$$
那么结果将是零矩阵。

我们并不清楚,为何仅仅执行了$A$而不是将 $\lambda$ 代入行列式 $\det(A - \lambda I)$ 就得到了相同的结果。
而且,很容易看出,除了 $1 \times 1$ 矩阵的平凡情况外,我们会得到不同的对象。即,$A - AI$ 是零矩阵,它的行列式只是数字 $0$.~

但是 $p(A)$ 是一个矩阵,而定理声称这个矩阵是零矩阵。因此,我们是在比较苹果和橘子。尽管两种情况下我们都得到了零,但它们是不同的零:数字零和零矩阵!

让我们给出另一个基于分析思想的证明。


\textbf{一个“连续的”证明}
\footnote{这个证明阐述了一个重要的思想,即\textbf{通常只考虑典型、一般的状况就足够了}。虽然这超出了本书的范围,但我们仍提及一下,而不深入细节:一个一般的(即典型的)矩阵是可对角化的。}

该证明基于几个观察。首先,对于对角矩阵,该定理是平凡的,因此对于与对角矩阵相似的矩阵(即可对角化矩阵)也是平凡的(见下文问题 1.1)。

第二个观察是,任何矩阵都可以被可对角化矩阵近似(任意精确)。
由于任何算子在某个标准正交基下都可以表示为上三角矩阵(见第 6 章定理 1.1),我们可以不失一般性地假设 $A$ 是一个上三角矩阵。

我们可以通过微扰 $A$ 的对角线元素(任意小)来使它们全部不同,因此扰动后的矩阵 $\tilde{A}$ 是可对角化的(三角矩阵的特征值是其对角线元素,见第 4 章第 1.7 节,并且根据第 4 章推论 2.3,一个具有 $n$ 个不同特征值的 $n \times n$ 矩阵是可对角化的)。

如我刚才提到的,我们可以任意小地扰动 $A$ 的对角线元素,所以 Frobenius 范数 $\|A - \tilde{A}\|_2$ 可以任意小。因此,我们可以找到一个可对角化矩阵序列 $A_k$ 使得 $A_k \to A$ 当 $k \to \infty$(例如,使得 $\|A_k - A\|_2 \to 0$ 当 $k \to \infty$)。可以证明,特征多项式 $p_k(\lambda) = \det(A_k - \lambda I)$ 收敛于 $A$ 的特征多项式 $p(\lambda) = \det(A - \lambda I)$.~因此,$$p(A) = \lim_{k \to \infty} p_k(A_k).$$
但是正如我们上面讨论的,对于可对角化矩阵,凯莱-哈密顿定理是平凡的,所以 $p_k(A_k) = \oo$.~因此,$p(A) = \lim_{k \to \infty} \oo = \oo$.~

这个证明是为那些熟悉分析(即连续性、收敛性等的严格处理)中的想法的读者准备的。
\footnote{这里我指的是\textbf{分析},即连续性、收敛性等概念的严格处理,而不是微积分,微积分在其目前的教学方式下,仅仅是一系列技巧的集合。}
这样的读者应该能够填补所有细节,并且对他/她来说,这个证明应该看起来非常简单和自然。

然而,对于那些还不熟悉这些想法的读者来说,这个证明肯定会显得奇怪。它甚至可能看起来像是某种作弊,尽管,让我重申,这是一个完全正确且严谨的证明(取决于分析中的一些标准事实)。
因此,让我们给出定理的另一个证明,这也是其他线性代数教科书中的“标准”证明之一。


\textbf{一个“标准”的证明}~~
我们知道,见第 6 章定理 6.1.1,任何方阵都与一个上三角阵是酉等价的。由于对于任何多项式 $p$,我们有 $p(UAU^{-1}) = Up(A)U^{-1}$,并且酉等价矩阵的特征多项式是相同的,所以我们只需要证明该定理对于上三角矩阵成立。

因此,设 $A$ 是一个上三角矩阵。我们知道三角矩阵的对角线元素与其特征值相等,所以设 $\lambda_1, \lambda_2, \dots, \lambda_n$ 是 $A$ 的特征值,按其在对角线上的顺序排列,即
$$A = \begin{pmatrix} \lambda_1 &  & & *\\ & \lambda_2 & & \\ & & \ddots & \\\oo & & & \lambda_n \end{pmatrix}.$$
$A$ 的特征多项式 $p(z) = \det(A - zI)$ 可以表示为
$$p(z) = (\lambda_1 - z)(\lambda_2 - z) \dots (\lambda_n - z) = (-1)^n (z - \lambda_1)(z - \lambda_2) \dots (z - \lambda_n),$$
所以
$$p(A) = (-1)^n (A - \lambda_1 I)(A - \lambda_2 I) \dots (A - \lambda_n I).$$


定义子空间 $E_k := \text{span}\{\ee_1, \ee_2, \dots, \ee_k\}$,其中 $\ee_1, \ee_2, \dots, \ee_n$ 是 $\CC^n$ 中的标准基。由于 $A$ 的矩阵是上三角的,子空间 $E_k$ 是算子 $A$ 的\textbf{不变子空间}(invariant subspace),即 $AE_k \subset E_k$(表示 $\vv \in E_k$ 的所有向量 $A\vv$ 都属于 $E_k$)。此外,由于对于任何 $\vv \in E_k$ 和任何 $\lambda$,
$$(A - \lambda I)\vv = A\vv - \lambda \vv \in E_k,$$
因为 $A\vv$ 和 $\lambda \vv$ 都属于 $E_k$.~因此 $(A - \lambda I)E_k \subset E_k$,即 $E_k$ 是 $A - \lambda I$ 的不变子空间。

我们还可以对子空间 $(A - \lambda_k I)E_k$ 说得更多。即,$(A - \lambda_k I)\ee_k \in \text{span}\{\ee_1, \ee_2, \dots, \ee_{k-1}\}$,因为 $A - \lambda_k I$ 矩阵的第 $k$ 列的前 $k-1$ 个元素可能非零。另一方面,对于 $j < k$,有 $(A - \lambda_k)\ee_j \in E_j \subset E_k$(因为 $E_j$ 是 $A - \lambda_k I$ 的不变子空间)。

取任意向量 $\vv \in E_k$.~根据 $E_k$ 的定义,它可以表示为向量 $\ee_1, \ee_2, \dots, \ee_k$ 的线性组合。由于所有向量 $\ee_1, \ee_2, \dots, \ee_k$ 都被 $A - \lambda_k I$ 映射到 $E_{k-1}$ 中的某个向量,我们可以得出结论:
$$(1.1) \quad (A - \lambda_k I)\vv \in E_{k-1} \quad \forall \vv \in E_k.$$
取任意向量 $\xx \in \CC^n = E_n$.~通过归纳地应用 (1.1),令 $k = n, n-1, \dots, 1$,我们得到
\begin{equation} \notag
\begin{split}
&\ \xx_1 := (A - \lambda_n I)\xx \in E_{n-1},\\
&\ \xx_2 := (A - \lambda_{n-1} I)\xx_1 = (A - \lambda_{n-1} I)(A - \lambda_n I)\xx \in E_{n-2},\\
&\ \dots \\
&\ \xx_n := (A - \lambda_2 I)\xx_{n-1} = (A - \lambda_2 I) \dots (A - \lambda_{n-1} I)(A - \lambda_n I)\xx \in E_1.
\end{split}\end{equation}
最后一个包含关系意味着 $\xx_n = \alpha \ee_1$.~但是 $(A - \lambda_1 I)\ee_1 = 0$,所以
$$\oo = (A - \lambda_1 I)\xx_n = (A - \lambda_1 I)(A - \lambda_2 I) \dots (A - \lambda_n I)\xx.$$
因此,$p(A)\xx = \oo$ 对所有 $\xx \in \CC^n$ 成立,这意味着 $p(A) = \oo$.~

\begin{exer} \textbf{练习}

1.1 (可对角化矩阵的凯莱-哈密顿定理)。如上节讨论,凯莱-哈密顿定理说明,如果 $A$ 是一个方阵,其特征多项式为 
$$p(\lambda) = \det(A - \lambda I) = \sum_{k=0}^n c_k \lambda^k,$$
则 $p(A) := \sum_{k=0}^n c_k A^k = \oo$(我们假设 $A^0 = I$)。

证明该定理在 $A$ 与一个对角矩阵相似的特殊情况下的情况,即 $A = SDS^{-1}$.~

\textbf{提示}:如果 $D = \text{diag}\{\lambda_1, \lambda_2, \dots, \lambda_n\}$ 且 $p$ 是任意多项式,你能计算 $p(D)$ 吗?那么 $p(A)$ 呢?\end{exer}

\section{2. 谱映射定理}

\subsection{2.1. 算子的多项式}
同样需要回忆一下,对于一个方阵(算子)$A$ 和一个多项式 $p(z) = \sum_{k=0}^N a_k z^k$,算子 $p(A)$ 是通过将独立变量替换为 $A$ 来定义的,
$$p(A) := \sum_{k=0}^N a_k A^k = a_0 I + a_1 A + a_2 A^2 + \dots + a_N A^N,$$
这里我们约定 $A^0 = I$.~

我们知道,一般而言,矩阵乘法不是可交换的,即通常 $AB \neq BA$,所以顺序很重要。然而,
$$A^k A^j = A^j A^k = A^{k+j},$$
由此很容易证明对于任意多项式 $p$ 和 $q$,
$$p(A)q(A) = q(A)p(A) = R(A),$$
其中 $R(z) = p(z)q(z)$.~

这意味着,当只处理一个算子 $A$ 的多项式时,不必担心不可交换性,就像 $A$ 是一个独立的(标量)变量一样。特别地,如果一个多项式 $p(z)$ 可以表示为单项式的乘积
$$p(z) = a(z - z_1)(z - z_2) \dots (z - z_N),$$
其中 $z_1, z_2, \dots, z_N$ 是 $p$ 的根,那么 $p(A)$ 可以表示为
$$p(A) = a(A - z_1 I)(A - z_2 I) \dots (A - z_N I).$$

\subsection{2.2. 谱映射定理}
回顾一下,一个方阵(算子)$A$ 的\textbf{谱} $\sigma(A)$ 是 $A$ 的所有特征值(不计重数)的集合。

\textbf{定理 2.1} ~~(谱映射定理)
对于一个方阵 $A$ 和任意多项式 $p$,
$$\sigma(p(A)) = p(\sigma(A)).$$
换句话说,$\mu$ 是 $p(A)$ 的一个特征值,当且仅当 $\mu = p(\lambda)$ 对某个 $A$ 的特征值 $\lambda$ 成立。

注意,如表述所示,这个定理没有说明特征值的重数。

\textbf{注记}~~需要注意的是,一个包含关系是平凡的。即,如果 $\lambda$ 是 $A$ 的一个特征值,$Ax = \lambda \xx$ 对某个 $\xx \neq \oo$ 成立,那么 $A^k \xx = \lambda^k \xx$,并且 $p(A)\xx = p(\lambda)\xx$,所以 $p(\lambda)$ 是 $p(A)$ 的一个特征值。这意味着包含关系 $p(\sigma(A)) \subset \sigma(p(A))$ 是平凡的。


如果我们考虑上述定理中的 $\mu=0$ 的特殊情况,我们得到以下推论。

\textbf{推论 2.2}~~
设 $A$ 是一个方阵,其特征值为 $\lambda_1, \lambda_2, \dots, \lambda_n$,且 $p$ 是一个多项式。那么,$p(A)$ 是可逆的,当且仅当 $$p(\lambda_k) \neq 0\quad \forall k = 1, 2, \dots, n.$$

\textbf{定理 2.1 的证明}~~
如上所述,包含关系 
$$p(\sigma(A)) \subset \sigma(p(A))$$
是平凡的。

为了证明另一个包含关系 $\sigma(p(A)) \subset p(\sigma(A))$,取一个点 $\mu \in \sigma(p(A))$.~令 $q(z) = p(z) - \mu$,则 $q(A) = p(A) - \mu I$.~由于 $\mu \in \sigma(p(A))$,算子 $q(A) = p(A) - \mu I$ 是不可逆的。

我们将多项式 $q(z)$ 表示为单项式的乘积:
$$q(z) = a(z - z_1)(z - z_2) \dots (z - z_N).$$
那么,如第 2.1 节所讨论的,我们可以将 $q(A)$ 表示为
$$q(A) = a(A - z_1 I)(A - z_2 I) \dots (A - z_N I).$$
算子 $q(A)$ 是不可逆的,因此其中一个因子 $A - z_k I$ 必须是不可逆的(因为可逆变换的乘积总是可逆的)。这意味着 $z_k \in \sigma(A)$.~

另一方面,$z_k$ 是 $q$ 的一个根,所以 
$$0 = q(z_k) = p(z_k) - \mu,$$
因此 $\mu = p(z_k)$.~
所以我们证明了包含关系 $\sigma(p(A)) \subset p(\sigma(A))$.~

\begin{exer} \textbf{练习}

2.1. 一个算子 $A$ 被称为\textbf{幂零}的,如果 $A^k = \oo$ 对某个 $k$ 成立。证明如果 $A$ 是幂零的,那么 $\sigma(A) = \{0\}$(即 $0$ 是 $A$ 的唯一特征值)。

你能不使用谱映射定理来做到这一点吗?\end{exer}

\section{3. 广义特征子空间~~代数重数的几何意义}

\subsection{3.1. 不变子空间}

\textbf{定义}~~
设 $A: V \to V$ 是向量空间 $V$ 上的一个算子(线性变换)。子空间 $E$ 被称为算子 $A$ 的\textbf{不变子空间}(或简而言之,$A$-不变)如果 $AE \subset E$,即如果 $\vv \in E$ 的所有向量 $A\vv$ 都属于 $E$.~

如果 $E$ 是 $A$-不变的,那么 
$$A^2 E = A(AE) \subset AE \subset E,$$
即 $E$ 是 $A^2$-不变的。

类似地,我们可以证明(例如,通过归纳法),如果 $AE \subset E$,那么 $$A^k E \subset E \quad \forall k \geq 1.$$
这意味着 $P(A)E \subset E$ 对于任何多项式 $p$,即:

\fbox{\begin{minipage}{0.9\textwidth}
任何 $A$-不变子空间 $E$ 都是 $p(A)$ 的不变子空间。
\end{minipage}}


如果 $E$ 是一个 $A$-不变子空间,那么对于所有 $\vv \in E$,结果 $A\vv$ 也属于 $E$.~因此,我们可以将 $A$ 作为作用在 $E$ 上的算子来处理,而不是作用在整个空间 $V$ 上。形式上,对于一个 $A$-不变子空间 $E$,我们定义 $A$ 到 $E$ 上的\textbf{限制} 
$A|_E : E \to E$ 为 
$$(A|_E)\vv = A\vv \quad \forall \vv \in E.$$
这里我们改变了算子的定义域和目标空间,但将值赋给自变量的规则保持不变。

我们将需要以下简单引理:

\textbf{引理 3.1}
设 $p$ 是一个多项式,且 $E$ 是一个 $A$-不变子空间。则 $p(A|_E) = p(A)|_E$.~

\textbf{证明}:证明是平凡的。

如果 $E_1, E_2, \dots, E_r$ 是 $A$-不变子空间的基,并且 $A_k := A|_{E_k}$ 是相应的限制,那么由于 $AE_k = A_k E_k \subset E_k$,算子 $A_k$ 独立地相互作用(不交互),为了分析 $A$ 的作用,我们可以分别分析算子 $A_k$.~

特别地,如果我们选择每个子空间 $E_k$ 中的一个基,并将它们连接起来得到 $V$ 中的一个基(见第 4 章定理 2.6),那么在基下算子 $A$ 将具有以下块对角形式:
$$A = \begin{pmatrix} A_1 & & & \oo \\ & A_2 & & \\ & & \ddots & \\ \oo & & & A_r \end{pmatrix}.$$
(当然,这里我们对 $V$ 中的基进行了正确的排序,首先在一个基 $E_1$ 中,然后是一个基 $E_2$ 等等)。

我们现在的目标是选择不变子空间 $E_1, E_2, \dots, E_r$ 的基,使得限制 $A_k$ 具有简单的结构。在这种情况下,我们将得到一个基,其中 $A$ 的矩阵具有简单的结构。

特征子空间 $\ker(A - \lambda_k I)$ 将是好的候选者,因为 $A$ 在特征子空间 $\ker(A - \lambda_k I)$ 上的限制仅仅是 $\lambda_k I$.~不幸的是,如我们所知,特征子空间并不总是构成一个基(当且仅当 $A$ 可对角化时,它们才构成一个基,参见第 4 章定理 2.1)。

然而,所谓的广义特征子空间将起作用。

\subsection{3.2. 广义特征子空间}

\textbf{定义 3.2}~~

向量 $\vv$ 被称为\textbf{广义特征向量}(generalized eigenvector)(对应于特征值 $\lambda$),如果 $(A - \lambda I)^k \vv = 0$ 对某个 $k \geq 1$ 成立。

所有广义特征向量与 $0$ 的集合被称为\textbf{广义特征子空间}(generalized eigenspace)(对应于特征值 $\lambda$)。

换句话说,广义特征子空间 $E_\lambda$ 可以表示为
$$(3.1) \quad E_\lambda = \bigcup_{k \geq 1} \ker(A - \lambda I)^k.$$
子空间序列 $\ker(A - \lambda I)^k, k = 1, 2, 3, \dots$ 是一个递增的子空间序列,即 
$$\ker(A - \lambda I)^k \subset \ker(A - \lambda I)^{k+1} \quad \forall k \geq 1.$$

(3.1)这个表示  看起来并不太简单,因为它涉及无限并集。然而,子空间 $\ker(A - \lambda I)^k$ 的序列会\textbf{稳定},即 
$$\ker(A - \lambda I)^k = \ker(A - \lambda I)^{k+1} \quad \forall k \geq k_\lambda,$$
所以实际上可以取有限并集。

为了证明核序列的稳定性,让我们注意到,如果对于有限维子空间 $E$ 和 $F$,我们有 $E \subsetneq F$(真子集符号 $E \subsetneq F$ 表示 $E \subset F$ 但 $E \neq F$),那么 $\dim E < \dim F$.~

由于 $\dim \ker(A - \lambda I)^k \leq \dim V < \infty$,它不能无限增长,所以某处 
$$\ker(A - \lambda I)^k = \ker(A - \lambda I)^{k+1}.$$

其余部分遵循以下引理。

\textbf{引理 3.3}~~
如果对某个 $k$,
$$\ker(A - \lambda I)^k = \ker(A - \lambda I)^{k+1}.$$则
$$\ker(A - \lambda I)^{k+r} = \ker(A - \lambda I)^{k+r+1} \quad \forall r \geq 0.$$

\textbf{证明}~~设 $\vv \in \ker(A - \lambda I)^{k+r+1}$,即 $(A - \lambda I)^{k+r+1} \vv = \oo$.~那么 $$\ww := (A - \lambda I)^r \vv \in \ker(A - \lambda I)^{k+1}.$$
但我们知道 $\ker(A - \lambda I)^k = \ker(A - \lambda I)^{k+1}$,所以 $\ww \in \ker(A - \lambda I)^k$,这意味着 $(A - \lambda I)^k \ww = 0$.~回忆 $\ww$ 的定义,我们得到 
$$(A - \lambda I)^{k+r} \vv = (A - \lambda I)^k \ww = \oo,$$
所以 $\vv \in \ker(A - \lambda I)^{k+r}$.~我们证明了 $\ker(A - \lambda I)^{k+r+1} \subset \ker(A - \lambda I)^{k+r}$.~反向包含关系是平凡的。

\textbf{定义}~~
序列 $\ker(A - \lambda I)^k$ 稳定的数字 $d = d(\lambda)$,即满足 $$\ker(A - \lambda I)^{d-1} \subsetneq \ker(A - \lambda I)^d = \ker(A - \lambda I)^{d+1}$$
的数字 $d$,被称为特征值 $\lambda$ 的\textbf{深度}(depth)。

从深度的定义可以得出,对于广义特征子空间 $E_\lambda$,
$$(3.2) \quad (A - \lambda I)^{d(\lambda)} \vv = \oo \quad \forall \vv \in E_\lambda.$$

现在总结一下我们对广义特征子空间的了解。

a) $E_\lambda$ 是 $A$ 的一个不变子空间,$AE_\lambda \subset E_\lambda$.~

b) 如果 $d(\lambda)$ 是特征值 $\lambda$ 的深度,那么 
$$((A - \lambda I)|_{E_\lambda})^{d(\lambda)} = (A|_{E_\lambda} - \lambda I_{E_\lambda})^{d(\lambda)} = \oo.$$(这只是 (3.2) 的另一种写法)

c) $\sigma(A|_{E_\lambda}) = \{\lambda\}$,因为算子 $A|_{E_\lambda} - \lambda I_{E_\lambda}$ 是幂零的,见 2,而幂零算子的谱只包含一个点 $0$,见问题 2.1。

现在我们准备陈述本节的主要结果。
设 $A: V \to V$.~

\textbf{定理 3.4}~~
设 $\sigma(A)$ 包含 $r$ 个点 $\lambda_1, \lambda_2, \dots, \lambda_r$,设 $E_k := E_{\lambda_k}$ 为相应的广义特征子空间。那么子空间系统 $E_1, E_2, \dots, E_r$ 是 $V$ 的一个基(的子空间)。

\textbf{注记 3.5}~~
如果我们连接所有广义特征子空间 $E_k$ 的基,那么根据第 4 章定理 2.6,我们将得到空间 $V$ 的一个基。在该基下,算子 $A$ 的矩阵具有块对角形式:
$A = \text{diag}\{A_1, A_2, \dots, A_r\}$,
其中 $A_k := A|_{E_k}$,$E_k = E_{\lambda_k}$.~也很容易看出(见 (3.2))算子 $N_k := A_k - \lambda_k I_{E_k}$ 是幂零的,$N_k^{d_k} = 0$.~

\textbf{定理 3.4 的证明}~~
设 $m_k$ 是特征值 $\lambda_k$ 的重数,所以 $p(z) = \PPP rod_{k=1}^r (z - \lambda_k)^{m_k}$ 是 $A$ 的特征多项式。定义 
$$p_k(z) = \frac{p(z)}{(z - \lambda_k)^{m_k}} = \PPP rod_{j \neq k} (z - \lambda_j)^{m_j}.$$

\textbf{引理 3.6}~~
$$(3.3) \quad (A - \lambda_k I)^{m_k}|_{E_k} = \oo,$$

\textbf{证明}~~有两种简单的证明方法。第一个是注意到 $m_k \geq d_k$,其中 $d_k$ 是特征值 $\lambda_k$ 的深度,并利用事实 
$$(A - \lambda_k I)^{d_k}|_{E_k} = (A|_{E_k} - \lambda_k I_{E_k})^{m_k} = \oo,$$
其中 $A_k := A|_{E_k}$(广义特征子空间的性质 2)。

第二种可能性是注意到根据谱映射定理,见推论 2.2,算子 $p_k(A)|_{E_k} = p_k(A_k)$ 是可逆的。根据凯莱-哈密顿定理(定理 1.1),
$$\oo = p(A) = (A - \lambda_k I)^{m_k} p_k(A),$$
将所有算子限制到 $E_k$ 上我们得到 $$\oo = p(A_k) = (A_k - \lambda_k I_{E_k})^{m_k} p_k(A_k),$$
所以 
$$(A_k - \lambda_k I_{E_k})^{m_k} = p(A_k) p_k(A_k)^{-1} = \oo  p_k(A_k)^{-1} = \oo.$$

为了证明定理,定义 
$$q(z) = \sum_{k=1}^r p_k(z).$$
由于 $p_k(\lambda_j) = 0$ 对 $j \neq k$ 且 $p_k(\lambda_k) \neq 0$,我们可以得出 $q(\lambda_k) \neq 0$ 对所有 $k$ 成立。因此,根据谱映射定理,见推论 2.2,算子 
$$B = q(A)$$ 是可逆的。

注意,$BE_k \subset E_k$(任何 $A$-不变子空间也必然是 $p(A)$-不变的)。由于 $B$ 是一个可逆算子,$\dim(BE_k) = \dim E_k$,这与 $BE_k \subset E_k$ 一起意味着 $BE_k = E_k$.~将最后一个恒等式乘以 $B^{-1}$ 得到 $B^{-1}E_k = E_k$,即 $E_k$ 是 $B^{-1}$ 的不变子空间。

还需注意,从 (3.3) 可知,$$p_k(A)|_{E_j} = \oo \quad \forall j \neq k,$$
因为 $p_k(A)|_{E_j} = p_k(A_j)$ 并且 $p_k(A_j)$ 包含因子 $(A_j - \lambda_j I_{E_j})^{m_j} = \oo$.~

定义算子 $\PPP _k$ 为 $$\PPP _k = B^{-1} p_k(A).$$

\textbf{引理 3.7}~~
对于上述定义的算子 $\PPP _k$:

a) $\PPP _1 + \PPP _2 + \dots + \PPP _r = I$;

b) $\PPP _k|_{E_j} = \oo$ 对 $j \neq k$;

c) $\text{Ran } \PPP _k \subset E_k$;

d) 更进一步, $\PPP _k \vv = \vv \quad \forall \vv \in E_k$,所以实际上 $\text{Ran } \PPP _k = E_k$.~

\textbf{证明}~~属性 1 是平凡的:$$\sum_{k=1}^r \PPP _k = B^{-1} \sum_{k=1}^r \PPP _k p_k(A) = B^{-1} B = I.$$
属性 2 来自 (3.3)。实际上,$p_k(A)$ 包含因子 $(A - \lambda_j)^{m_j}$,将其限制到 $E_j$ 上为零。因此,$p_k(A)|_{E_j} = \oo$,从而 $\PPP _k|_{E_j} = B^{-1} p_k(A)|_{E_j} = \oo$.~

为了证明属性 3,回忆根据凯莱-哈密顿定理 $p(A) = \oo$.~由于 $p(z) = (z - \lambda_k)^{m_k} p_k(z)$,我们得到对于 $\ww = p_k(A)\vv$,
$$(A - \lambda_k I)^{m_k} \ww = (A - \lambda_k I)^{m_k} p_k(A) \vv = p(A) \vv = \oo.$$
这意味着,$\text{Ran } p_k(A)$ 中的任何向量$\ww$都被 $(A - \lambda_k I)$ 的某个幂化为了零,根据定义,这意味着 $\text{Ran } p_k(A) \subset E_k$.~

为了证明最后一个属性,让我们注意到从 (3.3) 可以得出,对于 $\vv \in E_k$,
$$p_k(A) \vv = \sum_{j=1}^r p_j(A) \vv = B \vv,$$
这使得 $\PPP _k \vv = B^{-1} B \vv = \vv$.~

现在我们准备完成定理的证明。取 $\vv \in V$,定义 $\vv_k = \PPP _k \vv$.~那么根据引理 3.7 的陈述 c),$\vv_k \in E_k$,并且根据陈述 a),
$$\vv = \sum_{k=1}^r \vv_k,$$
所以 $\vv$ 允许表示为线性组合。

为了证明这个表示是唯一的,我们可以注意到,如果 $\vv$ 被表示为 $\vv = \sum_{k=1}^r \vv_k$,其中 $\vv_k \in E_k$,那么根据引理 3.7 的陈述 b) 和 d),
$$\PPP _k \vv = \PPP _k \left( \vv_1 + \vv_2 + \dots + \vv_r \right) = \PPP _k \vv_k = \vv_k.$$

\subsection{3.3. 代数重数的几何意义}

\textbf{命题 3.8}~~
一个特征值的代数重数等于对应广义特征子空间的维数。

\textbf{证明}~~根据注记 3.5,如果我们连接广义特征子空间 $E_k = E_{\lambda_k}$ 的基得到整个空间的一个基,那么在该基下算子 $A$ 的矩阵具有块对角形式 $\text{diag}\{A_1, A_2, \dots, A_r\}$,其中 $A_k := A|_{E_k}$.~算子 $N_k = A_k - \lambda_k I_{E_k}$ 是幂零的,所以 $\sigma(N_k) = \{0\}$.~因此,算子 $A_k$(回忆 $A_k = N_k - \lambda_k I$)的谱只包含一个特征值 $\lambda_k$,其代数重数为 $n_k = \dim E_k$.~重数等于 $n_k$,因为一个有限维空间 $V$ 中的算子有恰好 $\dim V$ 个特征值(计入重数),而 $A_k$ 只有一个特征值。

注意,我们可以自由选择 $E_k$ 中的基,所以我们选择它们使得相应的块 $A_k$ 是上三角的。那么 
$$\det(A - \lambda I) = \PPP rod_{k=1}^r \det(A_k - \lambda I_{E_k}) = \PPP rod_{k=1}^r (\lambda_k - \lambda)^{n_k}.$$
但这表示特征值 $\lambda_k$ 的代数重数是 $n_k = \dim E_{\lambda_k}$.~

\subsection{3.4. 一个重要的应用}
以下推论对于微分方程非常重要。

\textbf{推论 3.9}~~
任何算子 $A$ 在 $V$ 中都可以表示为 $A = D + N$,其中 $D$ 是可对角化的(即在某个基下是对角形的)且 $N$ 是幂零的 ($N^m = \oo$ 对某个 $m$),并且 $DN = ND$.~

\textbf{证明}~~如上所述,见注记 3.5,如果我们连接广义特征子空间 $E_k$ 的基得到整个空间的一个基,那么在该基下 $A$ 具有块对角形式 $A = \text{diag}\{A_1, A_2, \dots, A_r\}$,其中 $A_k := A|_{E_k}$.~算子 $N_k = A_k - \lambda_k I_{E_k}$ 是幂零的,并且算子 $D = \text{diag}\{\lambda_1 I_{E_1}, \lambda_2 I_{E_2}, \dots, \lambda_r I_{E_r}\}$ 是对角的(在该基下)。
还需注意,$\lambda_k I_{E_k} N_k = N_k \lambda_k I_{E_k}$(恒等算子与任何算子可交换),所以块对角算子 $N = \text{diag}\{N_1, N_2, \dots, N_r\}$ 与 $D$ 可交换,$DN = ND$.~因此,定义 $N$ 为块对角算子 $N = \text{diag}\{N_1, N_2, \dots, N_r\}$,我们得到所需的分解。

这个推论允许我们计算算子的函数。让我们回顾一下,如果 $p$ 是一个 $d$ 次多项式,那么 $p(a + x)$ 可以用泰勒公式计算:
$$p(a + x) = \sum_{k=0}^d \frac{p^{(k)}(a)}{k!} x^k.$$
这是一个代数恒等式,意味着对于每个多项式 $p$,我们可以通过对 $a$ 和 $x$ 进行形式代数运算而不关心它们的性质来验证该公式的正确性。

由于算子 $D$ 和 $N$ 可交换,$DN = ND$,因此它们遵循与普通(标量)变量相同的规则,我们可以写(通过用 $D$ 替换 $a$ 并用 $N$ 替换 $x$):
$$p(A) = p(D + N) = \sum_{k=0}^d \frac{p^{(k)}(D)}{k!} N^k.$$
这里,为了计算导数 $p^{(k)}(D)$,我们首先通过(普通的微积分)规则计算多项式 $p(x)$ 的 $k$ 阶导数,然后将 $x$ 替换为 $D$.~

但是由于 $N$ 是幂零的,$N^m = \oo$ 对某个 $m$ 成立,只有前 $m$ 项可能非零,所以
$$p(A) = p(D + N) = \sum_{k=0}^{m-1} \frac{p^{(k)}(D)}{k!} N^k.$$
如果 $m$ 远小于 $d$,这个公式将使 $p(A)$ 的计算容易得多。

同样的方法也适用于 $p$ 不是多项式,而是无穷幂级数的情况。
对于一般的幂级数,我们必须注意所有级数的收敛性,所以我们不能说这个公式对任意幂级数 $p(x)$ 都成立。然而,如果幂级数的收敛半径是 $\infty$,那么一切都正常工作。特别是,如果 $p(x) = e^x$,那么使用 $(e^x)' = e^x$ 的事实,我们得到:
$$e^A = e^{D + N} = \sum_{k=0}^{m-1} \frac{e^{(k)}(D)}{k!} N^k = \sum_{k=0}^{m-1} \frac{e^D}{k!} N^k = e^D \sum_{k=0}^{m-1} \frac{1}{k!} N^k.$$
这个公式在微分方程中具有重要的应用。

请注意,$ND = DN$ 的事实在这里是至关重要的!

\section{4. 幂零算子的结构}

回想一下,向量空间 $V$ 中的一个算子 $A$ 被称为\textbf{幂零}的,如果 $A^k = 0$ 对某个指数 $k$ 成立。

在上一节中,我们证明了(见注记 3.5),如果我们连接所有广义特征子空间 $E_k = E_{\lambda_k}$ 的基得到空间 $V$ 的一个基,那么算子 $A$ 在该基下的矩阵具有块对角形式 $\text{diag}\{A_1, A_2, \dots, A_r\}$,并且算子 $A_k$ 可以表示为 $A_k = \lambda_k I + N_k$,其中 $N_k$ 是幂零算子。

在每个广义特征子空间 $E_k$ 中,我们想选择一个基,使得 $A_k$ 在该基下的矩阵具有最简单的形式。
由于单位算子在任何基下的矩阵都是单位矩阵,我们需要找到一个基,使得幂零算子 $N_k$ 具有简单的形式。

由于我们可以分别处理每个 $N_k$,我们将需要考虑以下问题:

对于一个幂零算子 $A$,找到一个基,使得该算子在该基下的矩阵是简单的。

让我们看看,一个矩阵具有简单形式意味着什么。很容易看出以下矩阵
$$(4.1) \quad \begin{pmatrix} 0 & 1 & & & 0 \\ & 0 & 1 & & \\ & & \ddots & \ddots & \\ & & & 0 & 1 \\ 0 & & & & 0 \end{pmatrix}$$
是幂零的。

这些矩阵(以及 $1 \times 1$ 的零矩阵)将是我们的“构建模块”。即,我们将证明对于任何幂零算子,都可以找到一个基,使得算子在该基下的矩阵具有块对角形式 $\text{diag}\{A_1, A_2, \dots, A_r\}$,其中每个 $A_k$ 要么是形式 (4.1) 的块,要么是 $1 \times 1$ 的零块。

让我们看看我们应该寻找什么。
假设一个算子 $A$ 在基 $\vv_1, \vv_2, \dots, \vv_p$ 下的矩阵是 (4.1) 的形式。那么
$$(4.2) \quad A\vv_1 = \oo$$
并且
$$(4.3) \quad A\vv_{k+1} = \vv_k, \quad k = 1, 2, \dots, p-1.$$
因此,我们必须寻找满足上述关系 (4.2)、(4.3) 的向量链 $\vv_1, \vv_2, \dots, \vv_p$.~

\subsection{4.1. 广义特征向量的循环}

\textbf{定义}~~
设 $A$ 是一个幂零算子。满足关系 (4.2)、(4.3) 的非零向量 $\vv_1, \vv_2, \dots, \vv_p$ 的链被称为 $A$ 的\textbf{广义特征向量循环}(cycle of generalized eigenvectors)。向量 $\vv_1$ 被称为循环的\textbf{初始向量}(initial vector),向量 $\vv_p$ 被称为循环的\textbf{末端向量}(end vector),并且数字 $p$ 被称为循环的\textbf{长度}(length)。

\textbf{注记}~~对于任意算子,也可以做出类似的定义。那么所有向量 $\vv_k$ 必须属于同一个广义特征子空间 $E_\lambda$,并且它们必须满足恒等式
$$(A - \lambda I)\vv_1 = \oo, \quad (A - \lambda I)\vv_{k+1} = \vv_k, \quad k = 1, 2, \dots, p-1.$$



\textbf{定理 4.1}~~
设 $A$ 是一个幂零算子,设 $\C_1, \C_2, \dots, \C_r$ 是其广义特征向量的循环,$\C_k = \{\vv^{k}_{1}, \vv^{k}_{2}, \dots, \vv^{k}_{p_k}\}$,其中 $p_k$ 是循环 $\C_k$ 的长度。假设初始向量 $\vv^{1}_{1}, \vv^{2}_{1}, \dots, \vv^{r}_{1}$ 是线性无关的。那么没有向量属于两个循环,并且所有向量组成的集合是线性无关的。

\textbf{证明}~~设 $n = p_1 + p_2 + \dots + p_r$ 是所有循环中向量的总数
\footnote{
我们只是数每个循环中的向量,并将所有数字相加。我们不关心是否有些循环有共同的向量,我们将其计入它所属的每个循环(当然,根据定理,这是不可能的,但一开始我们不能假设这一点)
}。我们将使用 $n$ 的归纳法。如果 $n=1$,则定理是平凡的。

现在假设定理对于所有算子和所有循环集合成立,只要所有循环中向量的总数严格小于 $n$.~

不失一般性,我们可以假设向量 $\vv^{k}_{j}$ 构成整个空间 $V$ 的基,否则我们可以考虑算子 $A$ 限制在不变子空间 $\text{span}\{\vv^{k}_{j} : k = 1, 2, \dots, r, 1 \leq j \leq p_k\}$ 上。

考虑子空间 $\text{Ran } A$.~从关系 (4.2)、(4.3) 可知,向量 $\vv^{k}_{j} : k = 1, 2, \dots, r, 1 \leq j \leq p_k - 1$ 构成 $\text{Ran } A$ 的基。注意,如果 $p_k > 1$,那么系统 $\vv^{k}_{1}, \vv^{k}_{2}, \dots, \vv^{k}_{p_k-1}$ 是一个循环,并且 $A$ 使长度为 1 的任何循环的向量化零。

因此,我们有有限数量的循环,并且这些循环的初始向量是线性无关的,所以归纳假设适用,并且向量 $\vv^{k}_{j} : k = 1, 2, \dots, r, 1 \leq j \leq p_k - 1$ 是线性无关的。

由于这些向量也张成 $\text{Ran } A$,我们在那里有了一个基。因此,$$\text{rank } A = \dim \text{Ran } A = n - r.$$
(我们有$n$个向量,并从每个循环 $\C_k$ 中移除了一个向量 $\vv_{p_k}^k$,其中 $k = 1, 2, \dots, r$.~因此,我们在基 $\vv_j^k : k = 1, 2, \dots, r, 1 \le j \le p_k - 1$ 中有 $n-r$ 个向量)。
另一方面,$A \vv^{k}_{1} = 0$ 对 $k = 1, 2, \dots, r$ 成立,并且由于这些向量是线性无关的,$\dim \text{Ker } A \geq r$.~根据秩定理(第 2 章定理 7.1),
$$\dim V = \text{rank } A + \dim \text{Ker } A = (n - r) + \dim \text{Ker } A \geq (n - r) + r = n,$$
所以 $\dim V \geq n$.~

另一方面,$V$ 由 $n$ 个向量张成,因此向量 $\vv^{k}_{j} : k = 1, 2, \dots, r, 1 \leq j \leq p_k$ 构成一个基,所以它们是线性无关的。


\subsection{4.2. 幂零算子的若尔当标准形}

\textbf{定理 4.2}~~
设 $A: V \to V$ 是一个幂零算子。那么 $V$ 有一个由 $A$ 的广义特征向量的循环组成的集合构成的基。

\textbf{证明}~~我们将使用 $n = \dim V$ 的归纳法。对于 $n=1$,定理是平凡的。

假设定理对于任何作用在维度小于 $n$ 的空间中的算子都成立。考虑子空间 $X = \text{Ran } A$.~

$X$ 是算子 $A$ 的一个不变子空间,所以我们可以考虑限制 $A|_X$.~

由于 $A$ 是不可逆的,$\dim \text{Ran } A < \dim V$,所以根据归纳假设存在广义特征向量的循环 $\C_1, \C_2, \dots, \C_r$,使得它们的并集是 $X$ 中的一个基。设 $\C_k = \{\vv^{k}_{1}, \vv^{k}_{2}, \dots, \vv^{k}_{p_k}\}$,其中 $\vv^{k}_{1}$ 是循环的初始向量。

由于末端向量 $\vv^{k}_{p_k}$ 属于 $\text{Ran } A$,可以找到一个向量 $\vv^{k}_{p_k+1}$ 使得 $A \vv_{p_k+1} = \vv^{k}_{p_k}$.~因此,我们可以将每个循环 $\C_k$ 扩展成一个更大的循环 $\tilde{\C}_k = \{\vv^{k}_{1}, \vv^{k}_{2}, \dots, \vv^{k}_{p_k}, \vv^{k}_{p_k+1}\}$.~
由于循环 $\tilde{\C}_k, k = 1, 2, \dots, r$ 的初始向量 $\vv^{k}_{1}$ 是线性无关的,上述定理 4.1 暗示了这些循环的并集是一个线性无关系统。

根据循环的定义,我们有 $\vv^{k}_{1} \in \text{Ker } A$,并且我们假设初始向量 $\vv^{k}_{1}, k = 1, 2, \dots, r$ 是线性无关的。让这个系统扩展成 $\text{Ker } A$ 中的一个基,即找到向量 $\uu_1, \uu_2, \dots, \uu_q$,使得系统 $\{\vv^{1}_{1}, \vv^{2}_{1}, \dots, \vv^{r}_{1}, \uu_1, \uu_2, \dots, \uu_q\}$ 是 $\text{Ker } A$ 中的一个基(可能发生的情况是系统 $\vv^{k}_{1}, k = 1, 2, \dots, r$ 已经是 $\text{Ker } A$ 中的一个基,在这种情况下,我们让 $q = 0$ ,并没有添加任何内容)。

向量 $\uu_j$ 可以被视为长度为 1 的循环,因此我们有一个循环集合 $\{\tilde{\C}_1, \tilde{\C}_2, \dots, \tilde{\C}_r, \uu_1, \uu_2, \dots, \uu_q\}$,其初始向量是线性无关的。
所以,我们可以应用定理 4.1 得到所有这些循环的并集是一个线性无关系统。

为了证明它是一个基,让我们计算维度。我们知道循环 $\C_1, \C_2, \dots, \C_r$ 总共有 $\dim \text{Ran } A = \text{rank } A$ 个向量。每个循环 $\tilde{\C}_k$ 是从 $\C_k$ 通过添加 1 个向量得到的,所以所有循环 $\tilde{\C}_k$ 中的向量总数是 $\text{rank } A + r$.~

我们知道 $\dim \text{Ker } A = r + q$(因为 $\{\vv^{1}_{1}, \vv^{2}_{1}, \dots, \vv^{r}_{1}, \uu_1, \uu_2, \dots, \uu_q\}$ 是那里的一个基)。我们将循环 $\tilde{\C}_1, \tilde{\C}_2, \dots, \tilde{\C}_r$ 添加了额外的 $q$ 个向量,所以我们得到了 $$\text{rank } A + r + q = \text{rank } A + \dim \text{Ker } A = \dim V$$
个线性无关的向量。但是 $\dim V$ 个线性无关的向量构成一个基。

\textbf{定义}~~
由幂零算子 $A$ 的广义特征向量循环的并集构成的基(其存在由定理 4.2 保证)称为 $A$ 的\textbf{若尔当标准基}(Jordan canonical basis)。

注意,这样的基不是唯一的。

\textbf{推论 4.3}~~
设 $A$ 是一个幂零算子。存在一个基(若尔当标准基),使得该算子在该基下的矩阵是块对角的 $\text{diag}\{A_1, A_2, \dots, A_r\}$,其中所有 $A_k$(可能有一个例外)是形式 (4.1) 的块,并且其中一个块 $A_k$ 可以是零。

算子在若尔当标准基下的矩阵称为该算子的 \textbf{若尔当标准形}。我们稍后会看到,如果我们要约定块的顺序(即约定基中的向量顺序),那么若尔当标准形是唯一的。

\textbf{定理 4.3 的证明}~~
根据定理 4.2,可以找到一个由广义特征向量循环的并集构成的基。大小为 $p$ 的循环产生一个 $p \times p$ 的对角块,形式为 (4.1),而长度为 1 的循环对应于一个 $1 \times 1$ 的零块。我们可以将这些 $1 \times 1$ 的零块连接成一个大的零块(因为非对角线元素是 0)。

\subsection{4.3. 点图~~若尔当标准形的唯一性}

有一种很好的方法来可视化定理 4.2 和推论 4.3,即所谓的\textbf{点图}。这种方法还可以让我们回答许多自然问题,例如“由推论 4.3 给出的块对角表示是否是唯一的?”

当然,如果我们字面上处理这个问题,答案是“否”,因为我们可以随时更改块的顺序。但是,如果我们排除这种微不足道的情况,例如约定某种块的顺序(比如,如果我们把所有非零块按降序排列,然后放零块),那么这个表示是唯一的,还是不是唯一的?

\begin{figure}[ht]
  \centering  \includegraphics[width=1.0\linewidth]{figures/Figure6.PNG}
  \caption{幂零算子的点图和相应的若尔当标准形}
  \label{fig:06} 
\end{figure}

为了更好地理解第 4.1 节中描述的幂零算子的结构,让我们绘制所谓的点图。即,假设我们有一个基,它是广义特征向量循环的并集。
让我们用一个点的数组来表示基,这样每一列代表一个循环。第一行由循环的初始向量组成,我们按照长度的降序排列这些列(循环),将最长的放在左边。

在图\ref{fig:06} 中,我们有一个幂零算子的点图,以及它的若尔当标准形。这个点图表明,基由一个长度为 5 的循环,一个长度为 3 的循环,两个长度为 2 的循环,以及 2 个长度为 1 的循环组成。长度为 5 的循环对应于矩阵的 $5 \times 5$ 块,长度为 3 的循环对应于一个 $3 \times 3$ 非零块,两个长度为 2 的循环对应于两个 $2 \times 2$ 块。三个长度为 1 的循环对应于对角线上的两个零项。这里,在每个块中,我们只给出主对角线和它上面的对角线;矩阵的所有其他项都是零。

如果我们约定块的顺序,那么点图与若尔当标准形(对于幂零算子)之间存在一对一的对应关系。因此,关于若尔当标准形唯一性的问题等同于关于点图唯一性的问题。

为了回答这个问题,让我们分析算子 $A$ 如何变换点图。
由于算子 $A$ 使循环的初始向量化零,并将循环中的向量 $\vv_{k+1}$ 移到向量 $\vv_k$,我们可以看到算子 $A$ 通过删除图的第一行(顶部)来作用于其点图。

新的点图对应于 $\text{Ran } A$ 中的若尔当标准基,并且允许我们写出限制 $A|_{\text{Ran } A}$ 的若尔当标准形。

类似地,不难看出算子 $A^k$ 删除图的前 $k$ 行。因此,如果对于所有 $k$,我们都知道维度 $\dim \text{Ker }(A^k)$,我们就知道了算子 $A$ 的点图。
即,第一行的点的数量是 
$$\dim \text{Ker } A,$$第二行的点的数量是 
$$\dim \text{Ker }(A^2) - \dim \text{Ker } A,$$
而第 $k$ 行的点的数量是 $$\dim \text{Ker }(A^k) - \dim \text{Ker }(A^{k+1}).$$

但这意味着,最初使用若尔当标准基定义的点图,并不取决于该标准基的特定选择。因此,点图是唯一的!这意味着如果我们约定块的顺序,那么若尔当标准形是唯一的。

\subsection{4.4. 计算若尔当标准基}
让我们简单谈谈如何为幂零算子计算若尔当标准基。设 $p_1$ 是最大的整数,使得 $A^{p_1} \neq \oo$(因此 $A^{p_1+1} = \oo$)。从上面对点图的分析可以看出,$p_1$ 是最长循环的长度。

计算算子 $A^k, k = 1, 2, \dots, p_1$,并计数 $\dim \text{Ker }(A^k)$,我们可以构造 $A$ 的点图。现在我们想用向量替换点,并找到一个构成循环并集的基。

我们从找到最长的循环开始(因为我们知道点图,我们知道有多少个循环,以及每个循环的长度)。
考虑 $\text{Ran } (A^{p_1})$ 中的一个基。将该基中的向量命名为 $\vv^{1}_{1}, \vv^{2}_{1}, \dots, \vv^{r_1}_{1}$,这些将是循环的初始向量。然后我们通过求解方程 
$$A^{p_1} \vv^{k}_{p_1} = \vv^{k}_{1}, k = 1, 2, \dots, r_1$$ 
来找到循环的末端向量 $\vv^{1}_{p_1}, \vv^{2}_{p_1}, \dots, \vv^{r_1}_{p_1}$.~
通过连续地将算子 $A$ 应用于末端向量 $\vv^{k}_{p_1}$,我们得到循环中的所有向量 $\vv^{k}_{j}$.~
因此,我们构造了所有最大长度的循环。

设 $p_2$ 是剩余循环中最大循环的长度。
考虑 $\text{Ran } (A^{p_2})$ 子空间,并设 $\dim \text{Ran}(A^{p_2}) = r_2$.~由于$\text{Ran }(A^{p_1}) \subset \text{Ran }(A^{p_2})$,我们可以将基 $\vv^{1}_{1}, \vv^{2}_{1}, \dots, \vv^{r_1}_{1}$ 扩展成$\text{Ran }(A^{p_2})$ 中的一个基 $\vv^{1}_{1}, \vv^{2}_{1}, \dots, \vv^{r_1}_{1}, \vv^{r_1+1}_{1}, \dots, \vv^{r_2}_{1}$.~然后我们通过求解方程 
$$A^{p_1} \vv^{k}_{p_2} = \vv^{k}_{1}, \quad k = r_1+1, r_1+2, \dots, r_2$$ 
来找到循环 $\C_{r_1+1}, \dots, \C_{r_2}$ 的末端向量,从而构造长度为 $p_2$ 的循环。

设 $p_3$ 表示剩余循环中最大循环的长度。然后,通过将基 $\vv^{1}_{1}, \vv^{2}_{1}, \dots, \vv^{r_2}_{1}$ 在 $\text{Ker }(A^{p_2})$ 中扩展成 $\text{Ker }(A^{p_3})$ 中的一个基,我们构造长度为 $p_3$ 的循环,依此类推。

最后还有一个说明:如上所述,如果我们知道点图,我们就知道标准形,所以一旦我们找到了一个若尔当标准基,我们就不需要计算该基下算子 $A$ 的矩阵:我们已经知道了!

\section{5.若尔当分解定理}

\textbf{定理 5.1}~~
给定一个算子 $A$,存在一个基(若尔当标准基),使得该算子在该基下的矩阵具有块对角形式,块的形式为
$$(5.1) \quad \begin{pmatrix} \lambda & 1 & & & \\ & \lambda & 1 & & \\ & & \ddots & \ddots & \\ & & & \lambda & 1 \\ & & & & \lambda \end{pmatrix}$$
其中 $\lambda$ 是 $A$ 的一个特征值。这里我们假设大小为 1 的块就是 $\lambda$.~

定理 5.1 中的块对角形式被称为算子的\textbf{若尔当标准形}。相应的基被称为算子的\textbf{若尔当标准基}。

\textbf{定理 5.1 的证明}~~
根据定理 3.4 和注记 3.5,如果我们连接广义特征子空间 $E_k = E_{\lambda_k}$ 的基得到整个空间的一个基,那么该基下 $A$ 的矩阵具有块对角形式 $\text{diag}\{A_1, A_2, \dots, A_r\}$,其中 $A_k = A|_{E_k}$.~
算子 $N_k = A_k - \lambda_k I_{E_k}$ 是幂零的,所以根据定理 4.2(更精确地说,根据推论 4.3),可以在 $E_k$ 中找到一个基,使得 $N_k$ 在该基下的矩阵是 $N_k$ 的若尔当标准形。为了得到 $A_k$ 在该基下的矩阵,只需在主对角线上用 $\lambda_k$ 替换 0 即可。



\subsection{5.1. 关于计算若尔当标准基的注记}
首先,让我们回顾一下,计算特征值是最难的部分,但我们在这里不讨论这部分,并假设特征值已经计算出来。

对于每个特征值 $\lambda$,我们计算子空间 $\text{Ker }(A - \lambda I)^k, k = 1, 2, \dots$,直到子空间序列稳定。实际上,由于我们有一个递增的子空间序列($\text{Ker }(A - \lambda I)^k \subset \text{Ker }(A - \lambda I)^{k+1}$),所以只需要跟踪它们的维度(或算子 $(A - \lambda I)^k$ 的秩)。
对于特征值 $\lambda$,设 $m = m_\lambda$ 是子空间序列 $\text{Ker }(A - \lambda I)^k$ 稳定的数字,即 $m$ 满足 
$$\dim \text{Ker }(A - \lambda I)^{m-1} < \dim \text{Ker }(A - \lambda I)^m = \dim \text{Ker }(A - \lambda I)^{m+1}.$$
那么 $E_\lambda = \text{Ker }(A - \lambda I)^m$ 是对应于特征值 $\lambda$ 的广义特征子空间。

在计算了所有广义特征子空间之后,有两种可能的行动方案。第一种方法是找到每个广义特征子空间中的一个基,因此算子 $A$ 在该基下的矩阵具有块对角形式 $\text{diag}\{A_1, A_2, \dots, A_r\}$,其中 $A_k = A|_{E_{\lambda_k}}$.~然后我们可以分别处理每个矩阵 $A_k$.~算子 $N_k = A_k - \lambda_k I$ 是幂零的,所以通过应用第 4.4 节中描述的算法,我们可以得到 $N_k$ 的若尔当标准表示,通过在主对角线上用 $\lambda_k$ 替换 0,我们得到块 $A_k$ 的若尔当标准表示。这种方法的优点是我们在处理更小的块。但是我们需要找到算子在新基下的矩阵,这涉及到矩阵求逆和矩阵乘法。

另一种方法是通过直接处理算子 $A$ 来找到每个广义特征子空间 $E_{\lambda_k}$ 中的若尔当标准基,而无需先将其分成块。同样,我们在第 4.4 节中概述的算法可以稍作修改。即,在为广义特征子空间 $E_{\lambda_k}$ 计算若尔当标准基时,而不是考虑子空间 $\text{Ran } (A^k - \lambda_k I)^j$,我们应该考虑子空间 $(A - \lambda_k I)^j E_{\lambda_k}$.~

(THE ~END)

