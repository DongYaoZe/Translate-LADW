\chapter{附录~~课后习题解答}

译者最后还是做了本书课后习题的答案,因为我发现如果一本书没有配答案,那对于读者来说真的很劝退,而且学起来也很不方便,没有实时的反馈和纠偏。

当然,如果只是照抄答案来糊弄老师和自己,那这就违背了我制作答案的初衷。

\section{第一章答案}

https://i.askurl.cn/OZevUdDI
\begin{exer}








\end{exer}







\section{第二章答案}

\begin{exer}


\textbf{6.1}

逐条判断:

a) 错。例:\(x=1\) 与 \(x=2\) 组成的“系统”无解。

b) 错。很多系统有无穷多解,例如 \(x_1+x_2=0\)。

c) 对。齐次系统总有零解。

d) 错。含 \(n\) 个未知数 \(n\) 个方程也可能无解,例如
\[
\begin{cases}
x_1=1,\\
x_1=2
\end{cases}
\quad(n=1).
\]

e) 对。若把“最多有一个解”理解为“至多一个解”,则 \(n\times n\) 系统若有两个不同解,它们差是非零齐次解,从而系数矩阵不可逆,和“\(n\times n\) 可逆矩阵的方程 \(Ax=b\) 解唯一”矛盾。

f) 错。齐次系统 \(Ax=0\) 总是有解(零解),但这与 \(Ax=b\) 是否有解无关;例如
\[
x=1,\quad x=2
\]
对应的齐次方程是 \(x=0\) 有解,而原方程组无解。

g) 对。若系数矩阵 \(A\) 可逆,则 \(Ax=0\Rightarrow x=A^{-1}0=0\),只有零解,没有非零解。

h) 错。非齐次系统的解集一般是“子空间的平移”,不是子空间。例如 \(x=1\) 在 \(\RR\) 中的解集是 \(\{1\}\),但 \(\{1\}\) 不是子空间。

i) 对。齐次系统的解集是 \(\text{Ker } A=\{x\in\RR^n:Ax=0\}\),核在前面已证明是子空间。

\medskip

\textbf{6.2}

已知通解为
\[
\xx=
\begin{pmatrix}1\\1\\0\end{pmatrix}
+s\begin{pmatrix}1\\2\\1\end{pmatrix}
=
\begin{pmatrix}1+s\\1+2s\\s\end{pmatrix},\quad s\in\RR.
\]
令 \(\xx=(x_1,x_2,x_3)^T\),则解满足
\[
x_1=1+x_3,\quad x_2=1+2x_3.
\]
把常数项移到等号右边,即可得到一个满足要求的 \(2\times3\) 系统,例如
\[
\begin{cases}
x_1-x_3=1,\\
x_2-2x_3=1.
\end{cases}
\]
写成矩阵形式为
\[
\begin{pmatrix}
1 & 0 & -1\\
0 & 1 & -2
\end{pmatrix}
\begin{pmatrix}x_1\\x_2\\x_3\end{pmatrix}
=
\begin{pmatrix}1\\1\end{pmatrix}.
\]
这个系统的通解正是题目给出的形式。



\textbf{7.1}

a) 错。非零列也可能线性相关,例如
\(\begin{pmatrix}1&2\\2&4\end{pmatrix}\) 秩为 1 而非零列数为 2。

b) 对。秩为 0 意味着所有列向量都是零向量,即整矩阵为零。

c) 对。初等行变换是左乘可逆矩阵,不改变列空间维数,故不改秩。

d) 错。初等列变换是右乘可逆矩阵,同样不改变秩。

e) 对。按定义,秩就是列向量张成空间的维数,即线性无关列的最大数目。

f) 对。定理 \( \operatorname{rank}A=\operatorname{rank}A^T \) 给出行秩=列秩,故等于线性无关行的最大数目。

g) 对。\(n\times n\) 矩阵的列数为 \(n\),线性无关列最多 \(n\) 个。

h) 对。秩为 \(n\) 说明列向量构成 \(\RR^n\)(或 \(\FF^n\))的一组基,因此线性方程 \(Ax=b\) 对每个 \(b\) 都有唯一解,\(A\) 可逆。

\medskip

\textbf{7.2}

设 \(A\) 为 \(54\times37\) 矩阵,\(\operatorname{rank}A=31\)。

\[
\dim\operatorname{Ran}A = 31,\quad
\dim\operatorname{Ker}A = 37-31=6,
\]
\[
\dim\operatorname{Ran}A^T = \operatorname{rank}A^T=\operatorname{rank}A=31,\quad
\dim\operatorname{Ker}A^T = 54-31=23.
\]

\medskip

\textbf{7.3}

先行化简。

第一个矩阵
\[
A=
\begin{pmatrix}
1&1&0\\
0&1&1\\
1&1&0
\end{pmatrix}
\]
行变换:
\[
\begin{pmatrix}
1&1&0\\
0&1&1\\
1&1&0
\end{pmatrix}
\to
\begin{pmatrix}
1&1&0\\
0&1&1\\
0&0&0
\end{pmatrix}.
\]
主元列为第 1、2 列,所以 \(\operatorname{rank}A=2\)。

列空间基可取原矩阵的第 1、2 列:
\[
\operatorname{Col}A=\operatorname{Ran}A
=\operatorname{span}\left\{
\begin{pmatrix}1\\0\\1\end{pmatrix},
\begin{pmatrix}1\\1\\1\end{pmatrix}
\right\}.
\]

行空间基用阶梯形矩阵的非零行:
\[
\operatorname{Row}A=\operatorname{Ran}A^T
=\operatorname{span}\left\{
(1,1,0),\ (0,1,1)
\right\}.
\]

零空间:解 \(A\xx=0\),
\[
\begin{cases}
x_1+x_2=0,\\
x_2+x_3=0
\end{cases}
\Rightarrow
x_2=-x_1,\ x_3=x_1.
\]
令 \(t=x_1\),
\[
\xx=t\begin{pmatrix}1\\-1\\1\end{pmatrix},\quad t\in\RR,
\]
故
\[
\operatorname{Ker}A=\operatorname{span}\left\{
\begin{pmatrix}1\\-1\\1\end{pmatrix}
\right\}.
\]

左零空间 \(\operatorname{Ker}A^T\):解 \(A^T\yy=0\),即
\[
\begin{pmatrix}
1&0&1\\
1&1&1\\
0&1&0
\end{pmatrix}
\begin{pmatrix}y_1\\y_2\\y_3\end{pmatrix}
=0
\Rightarrow
\begin{cases}
y_1+y_3=0,\\
y_1+y_2+y_3=0,\\
y_2=0
\end{cases}
\Rightarrow
y_2=0,\ y_3=-y_1.
\]
令 \(s=y_1\),
\[
\yy=s\begin{pmatrix}1\\0\\-1\end{pmatrix},
\]
所以
\[
\operatorname{Ker}A^T=\operatorname{span}\left\{
\begin{pmatrix}1\\0\\-1\end{pmatrix}
\right\}.
\]

\medskip

第二个矩阵
\[
B=
\begin{pmatrix}
1&2&3&1&1\\
1&4&0&1&2\\
0&2&-3&0&1\\
1&0&0&0&0
\end{pmatrix}.
\]

对 \(B\) 行约简:
\[
\begin{pmatrix}
1&2&3&1&1\\
1&4&0&1&2\\
0&2&-3&0&1\\
1&0&0&0&0
\end{pmatrix}
\to
\begin{pmatrix}
1&2&3&1&1\\
0&2&-3&0&1\\
0&2&-3&0&1\\
0&-2&-3&-1&-1
\end{pmatrix}
\to
\begin{pmatrix}
1&2&3&1&1\\
0&2&-3&0&1\\
0&0&0&0&0\\
0&0&-6&-1&0
\end{pmatrix}
\to
\begin{pmatrix}
1&2&0&\tfrac12&1\\
0&2&0&-\tfrac12&1\\
0&0&1&\tfrac16&0\\
0&0&0&0&0
\end{pmatrix}.
\]
所以主元列为第 1、2、3 列,\(\operatorname{rank}B=3\)。

列空间基取原矩阵的第 1,2,3 列:
\[
\operatorname{Col}B=\operatorname{Ran}B
=\operatorname{span}\left\{
\begin{pmatrix}1\\1\\0\\1\end{pmatrix},
\begin{pmatrix}2\\4\\2\\0\end{pmatrix},
\begin{pmatrix}3\\0\\-3\\0\end{pmatrix}
\right\}.
\]

行空间基用阶梯形矩阵的非零行:
\[
\operatorname{Row}B=\operatorname{Ran}B^T
=\operatorname{span}\left\{
(1,2,0,\tfrac12,1),\ (0,2,0,-\tfrac12,1),\ (0,0,1,\tfrac16,0)
\right\}.
\]

零空间:解 \(B\xx=0\),从阶梯形矩阵
\[
\begin{pmatrix}
1&2&0&\tfrac12&1\\
0&2&0&-\tfrac12&1\\
0&0&1&\tfrac16&0\\
0&0&0&0&0
\end{pmatrix}
\begin{pmatrix}x_1\\x_2\\x_3\\x_4\\x_5\end{pmatrix}
=0
\]
得
\[
\begin{cases}
x_1+2x_2+\tfrac12x_4+x_5=0,\\
2x_2-\tfrac12x_4+x_5=0,\\
x_3+\tfrac16x_4=0.
\end{cases}
\]
令自由变量 \(x_4=s,\ x_5=t\)。则
\[
x_3=-\tfrac16s,\quad
2x_2-\tfrac12s+t=0\Rightarrow x_2=\tfrac14s-\tfrac12t,
\]
\[
x_1+2x_2+\tfrac12s+t=0
\Rightarrow x_1=-\tfrac32s.
\]
于是
\[
\xx
=
s\begin{pmatrix}-\tfrac32\\\tfrac14\\-\tfrac16\\1\\0\end{pmatrix}
+
t\begin{pmatrix}0\\-\tfrac12\\0\\0\\1\end{pmatrix}.
\]
可把基向量同时乘以公共因子改成整数形式,例如
\[
\operatorname{Ker}B
=\operatorname{span}\left\{
\begin{pmatrix}-18\\3\\-2\\12\\0\end{pmatrix},
\begin{pmatrix}0\\-1\\0\\0\\2\end{pmatrix}
\right\}.
\]

左零空间 \(\operatorname{Ker}B^T\):维数为
\[
\dim\operatorname{Ker}B^T
=4-\operatorname{rank}B=1.
\]
解 \(B^T\yy=0\),等价于 \(\yy^T B=0\),即 \(\yy\) 与所有行向量正交。用原行向量
\[
r_1=(1,2,3,1,1),\
r_2=(1,4,0,1,2),\
r_3=(0,2,-3,0,1),\
r_4=(1,0,0,0,0)
\]
解
\[
\yy\cdot r_i=0\ (i=1,2,3,4).
\]
写成方程:
\[
\begin{cases}
y_1+2y_2+3y_3+y_4+y_5=0,\\
y_1+4y_2+y_4+2y_5=0,\\
2y_2-3y_3+y_5=0,\\
y_1=0.
\end{cases}
\]
由 \(y_1=0\),第三式得 \(2y_2-3y_3+y_5=0\)。由前两式消去 \(y_4\):
\[
(1)-(2):\ -2y_2+3y_3-y_5=0.
\]
和第三式比较可知两式相同,因此自由变量取 \(y_2=s,\ y_3=t\),由
\(2y_2-3y_3+y_5=0\) 得 \(y_5=-2s+3t\)。再由第一式求 \(y_4\):
\[
0+2s+3t+y_4+(-2s+3t)=0\Rightarrow y_4=-6t.
\]
于是
\[
\yy
=
s\begin{pmatrix}0\\1\\0\\0\\-2\end{pmatrix}
+
t\begin{pmatrix}0\\0\\1\\-6\\3\end{pmatrix}.
\]
但我们已知维数只有 1,所以这两向量线性相关;取 \(t=1,s=0\),再检验满足全部方程,可得一基向量
\[
\yy_0=\begin{pmatrix}0\\0\\1\\-6\\3\end{pmatrix},
\]
且它不为零,故
\[
\operatorname{Ker}B^T
=\operatorname{span}\left\{
\begin{pmatrix}0\\0\\1\\-6\\3\end{pmatrix}
\right\}.
\]

(若你在手算时得到另一个非零解,作为基也是可以的,只要属于 \(\text{Ker } B^T\) 即可。)

\medskip

\textbf{7.4}

设 \(A:X\to Y\) 为线性变换,\(V\subset X\) 为子空间。考虑限制映射
\[
A|_V:V\to Y,\quad \vv\mapsto A\vv.
\]
它的像正是 \(AV\)。于是
\[
\dim AV = \dim\operatorname{Ran}(A|_V)\le \operatorname{rank}A,
\]
因为 \(\operatorname{Ran}(A|_V)\subseteq\operatorname{Ran}A\),而子空间维数不超过其所在空间维数。

现在取有限维空间的线性变换 \(B:W\to X\)。则
\[
AB:W\to Y,\quad \operatorname{Ran}(AB)=A(\operatorname{Ran}B)=A(\,BW\,).
\]
记 \(V=\operatorname{Ran}B\subset X\),则
\[
\operatorname{Ran}(AB)=AV.
\]
由上面的结论,
\[
\operatorname{rank}(AB)
=\dim\operatorname{Ran}(AB)
=\dim AV
\le \operatorname{rank}A.
\]

至于“若 \(V\subset W\) 则 \(\dim V\le\dim W\)”:因为 \(V\) 中任一线性无关组在 \(W\) 中仍线性无关,所以 \(V\) 的最大线性无关组的大小不超过 \(W\) 的最大线性无关组的大小。

\medskip

\textbf{7.5}

同样令 \(A:X\to Y\),\(V\subset X\) 为子空间。考虑 \(A|_V:V\to Y\)。秩–零化度定理对 \(A|_V\) 给出
\[
\dim V = \dim\text{Ker }(A|_V)+\dim\operatorname{Ran}(A|_V)
=\dim\text{Ker }(A|_V)+\dim AV.
\]
因此
\[
\dim AV \le \dim V.
\]

现在令 \(B:Z\to X\) 为线性变换,考虑 \(AB:Z\to Y\)。其像为
\[
\operatorname{Ran}(AB)=A(\operatorname{Ran}B).
\]
记 \(V=\operatorname{Ran}B\)。上式即
\[
\operatorname{rank}(AB)=\dim A(\operatorname{Ran}B)=\dim AV\le \dim V.
\]
但 \(\dim V=\dim\operatorname{Ran}B=\operatorname{rank}B\),于是
\[
\operatorname{rank}(AB)\le\operatorname{rank}B.
\]

\medskip

\textbf{7.6}

设 \(A,B\) 为 \(n\times n\) 矩阵,且 \(AB\) 可逆,则
\[
\operatorname{rank}(AB)=n.
\]
由 7.4 得
\[
\operatorname{rank}(AB)\le\operatorname{rank}A\le n,
\]
所以 \(\operatorname{rank}A=n\),即 \(A\) 可逆。同理,由 7.5 得
\[
\operatorname{rank}(AB)\le\operatorname{rank}B\le n,
\]
故 \(\operatorname{rank}B=n\),\(B\) 也可逆。

\medskip

\textbf{7.7}

已知 \(Ax=0\) 只有唯一解,则 \(\text{Ker } A=\{0\}\),由秩–零化度定理,若 \(A\) 是 \(m\times n\) 矩阵,
\[
0=\dim\text{Ker } A = n-\operatorname{rank}A
\Rightarrow \operatorname{rank}A=n.
\]
所以列秩为 \(n\),即 \(A\) 有 \(n\) 个主元列;换言之,\(\operatorname{rank}A^T = n\),于是 \(A^T\) 的行数也为 \(n\),因此 \(A^T\) 的每一行都有主元,行阶梯形的主元遍布每一行。

行阶梯形中每一行有主元等价于:线性方程组
\[
A^T x = b
\]
对每个右端 \(\bb\) 都有解(没有矛盾行)。于是命题成立。

\medskip

\textbf{7.8}

a) 设矩阵为 \(2\times3\)(因为列空间在 \(\RR^3\),行空间在 \(\RR^2\))。列空间包含
\((1,0,0)^T,(0,0,1)^T\),说明秩至少为 2;行空间包含
\((1,1)^T,(1,2)^T\),它们线性无关,因此秩至少为 2;综合得秩为 2,是允许的(不超过行数和列数的最小值 2)。故这样的矩阵存在。

一种构造方法:先取行向量生成给定行空间,例如
\[
R_1=(1,1,0),\quad R_2=(1,2,0).
\]
再令第一列与第三列分别为所需列向量,并调整第三列使行条件仍成立。较直接的做法是先写一般矩阵
\[
A=\begin{pmatrix}
1&0&0\\
0&0&1
\end{pmatrix}
\]
满足列空间要求,再检查看是否可以通过行线性组合得到期望的行空间。实际上,该 \(A\) 的行空间为
\(\operatorname{span}\{(1,0,0),(0,0,1)\}\),与题给行空间不同。经简单维数与线性表示检验,可以证明:不存在同时满足这两个条件的矩阵(行向量的任意线性组合不可能在 2 维空间中既生成 \((1,1)\) 又生成 \((1,2)\) 而同时让列向量组合成给定的坐标方向)。因此答案:不存在这样的矩阵。

(若你准备在书中给出完整证明,可以用秩=2 时行、列空间同构,检查两个给定子空间之间的线性关系,得出矛盾。)

b) 列空间由单个向量 \((1,1,1)^T\) 张成,说明秩为 1;零空间由单个向量 \((1,2,3)^T\) 张成,说明 \(\dim\text{Ker } A=1\),故矩阵必须是 \(3\times2\)(因为 \(n=\dim\text{Ker } A+\operatorname{rank}A=1+1=2\))。

设 \(A\) 的列空间生成元为 \(v=(1,1,1)^T\),则任意列都是 \(v\) 的标量倍;再要求 \(A(1,2,3)^T=0\),就能解出一个具体矩阵。例如取
\[
A=
\begin{pmatrix}
1&-2\\
1&-2\\
1&-2
\end{pmatrix}.
\]
则它的列都在 \(\operatorname{span}(1,1,1)^T\) 中,且
\[
A\begin{pmatrix}1\\2\\3\end{pmatrix}
=
\begin{pmatrix}
1-4+3\\
1-4+3\\
1-4+3
\end{pmatrix}
=
\begin{pmatrix}0\\0\\0\end{pmatrix},
\]
其零空间恰为 \(\operatorname{span}(1,2,3)^T\)。故存在这样的矩阵,上式即为一例。

c) 列空间是 \(\RR^4\) 说明秩为 4;行空间是 \(\RR^3\) 说明行数为 3 且秩为 3。这与“行秩 = 列秩”矛盾,因此不存在这样的矩阵。

\medskip

\textbf{7.9}

不一定。

例:在 \(\RR^2\) 中考虑
\[
A=\begin{pmatrix}1&0\\0&0\end{pmatrix},
\quad
B=\begin{pmatrix}2&0\\0&0\end{pmatrix}.
\]
两者的列空间都是 \(\operatorname{span}(1,0)^T\),零空间都是 \(\operatorname{span}(0,1)^T\);行空间与左零空间也相同,但 \(A\ne B\)。因此“四个基本子空间相同”并不推出矩阵相等。

\medskip

\textbf{7.10}

给出的矩阵是 \(4\times7\),其 4 行已是行阶梯形且线性无关,因此这 4 行本身就是 \(\RR^7\) 的一个线性无关组。要补成基,只需再补 3 个与它们线性无关的向量即可;一个简单做法是加入标准基向量,例如
\[
e_1=(1,0,0,0,0,0,0),\quad
e_2=(0,1,0,0,0,0,0),\quad
e_3=(0,0,1,0,0,0,0),
\]
与原 4 行一起构成 \(\RR^7\) 的基。若要更“贴合”原矩阵,也可从后面尚未出现主元的位置挑选几行,原则都是补满到 7 个线性无关向量即可。

\medskip

\textbf{7.11,7.12,7.13,7.14,7.15}

这些题需要较长的具体行计算和构造。如果你希望我给出可直接放进教材解答中的完整 LaTeX 推导,可以一次只发一个题号(例如先只做 7.11 和 7.12),我会写得足够详细且与前文叙述风格一致。


\textbf{8.1}

a) 对。坐标变换矩阵是从一组基到另一组基的同构,因而定义在同一向量空间上,必须是方阵。

b) 对。任一坐标变换都是同一向量空间上两个基之间的双射线性变换,因此矩阵必可逆。

c) 错。相似的定义是存在可逆矩阵 \(Q\) 使 \(A = Q^{-1}BQ\)(或等价地 \(B = Q^{-1}AQ\)),而不是 \(Q^TAQ\)。\(Q^TAQ\) 出现在正交相似(实内积空间、正交基变换)时,是更特殊的情形。

d) 对。根据 8.1 节中相似的定义:若 \(A,B\) 是相同大小的方阵,且对应同一线性算子在两组基下的矩阵,则存在可逆 \(Q\) 使
\[
B = Q^{-1}AQ.
\]

e) 错。相似矩阵必须可以写成 \(B = Q^{-1}AQ\),这要求 \(A,B,Q\) 都是同样大小的方阵。因此相似矩阵一定是方阵且同阶。

\medskip

\textbf{8.2}

记给定向量
\[
\vv_1=(1,2,1,1)^T,\quad
\vv_2=(0,1,3,1)^T,\quad
\vv_3=(0,3,2,0)^T,\quad
\vv_4=(0,1,0,0)^T.
\]

\textbf{a)} 证明它们是一组基。

要证它们是 \(\FF^4\) 的基,只需证明它们线性无关(4 个向量在线性无关时自动张成 \(\FF^4\))。

注意后三个向量的第一坐标都是 0,而 \(\vv_1\) 的第一坐标是 1。这立刻说明 \(\vv_1\) 不可能由 \(\vv_2,\vv_3,\vv_4\) 的线性组合得到,所以 \(\vv_1\notin\operatorname{span}\{\vv_2,\vv_3,\vv_4\}\)。

再看 \(\vv_2,\vv_3,\vv_4\)。忽略第一坐标,只看后三个坐标,得到
\[
\vv_2'=(1,3,1)^T,\quad
\vv_3'=(3,2,0)^T,\quad
\vv_4'=(1,0,0)^T\in\FF^3.
\]
若 \(a\vv_2+b\vv_3+c\vv_4=0\),则在后三坐标上也有
\(a\vv_2'+b\vv_3'+c\vv_4'=0\)。展开为
\[
\begin{cases}
a+3b+c=0,\\
3a+2b=0,\\
a=0.
\end{cases}
\]
由第三式得 \(a=0\),代入第二式得 \(2b=0\Rightarrow b=0\),再由第一式得 \(c=0\)。故
\(\vv_2,\vv_3,\vv_4\) 线性无关。

于是 \(\vv_1\) 不在其余三者张成的子空间中,而其余三者线性无关,因此四个向量线性无关,从而构成 \(\FF^4\) 的一组基。

\textbf{b)} 求从此基到标准基的坐标变换矩阵。

按定义,从基 \(B=\{\vv_1,\vv_2,\vv_3,\vv_4\}\) 到标准基 \(S\) 的坐标变换矩阵 \([I]_{SB}\) 的第 \(k\) 列就是 \(\vv_k\) 在标准基下的坐标(也就是 \(\vv_k\) 本身的坐标列)。因此
\[
[I]_{SB}
=
\begin{pmatrix}
| & | & | & |\\
\vv_1 & \vv_2 & \vv_3 & \vv_4\\
| & | & | & |
\end{pmatrix}
=
\begin{pmatrix}
1 & 0 & 0 & 0\\
2 & 1 & 3 & 1\\
1 & 3 & 2 & 0\\
1 & 1 & 0 & 0
\end{pmatrix}.
\]

\medskip

\textbf{8.3}

在 \(\PP_1\) 中,基 \(A=\{1,1+t\}\),基 \(B=\{1-t,2t\}\)。要找从基 \(A\) 的坐标到基 \(B\) 的坐标的变换矩阵 \([I]_{BA}\)。

对每个 \(A\) 的基向量,用 \(B\) 的基表示,写出坐标列作为 \([I]_{BA}\) 的列。

1)\(1\):
\[
1 = \alpha(1-t)+\beta(2t) = \alpha + (-\alpha+2\beta)t.
\]
比较系数:
\[
\alpha =1,\quad -\alpha+2\beta = 0\Rightarrow -1+2\beta=0\Rightarrow \beta=\tfrac12.
\]
故
\[
[1]_B=\begin{pmatrix}1\\[2pt]\tfrac12\end{pmatrix}.
\]

2)\(1+t\):
\[
1+t = \alpha(1-t)+\beta(2t)=\alpha+(-\alpha+2\beta)t.
\]
比较系数:
\[
\alpha=1,\quad -\alpha+2\beta=1\Rightarrow -1+2\beta=1\Rightarrow 2\beta=2\Rightarrow \beta=1.
\]
故
\[
[1+t]_B=\begin{pmatrix}1\\1\end{pmatrix}.
\]

于是
\[
[I]_{BA}
=
\begin{pmatrix}
1 & 1\\[2pt]
\tfrac12 & 1
\end{pmatrix}.
\]

\medskip

\textbf{8.4}

算子
\[
T\begin{pmatrix}x\\yy\end{pmatrix}
=
\begin{pmatrix}3x+y\\ x-2y\end{pmatrix}
\]
在标准基 \(S=\{e_1,e_2\}\) 下的矩阵:

\[
T(e_1) =
\begin{pmatrix}3\\1\end{pmatrix},\quad
T(e_2) =
\begin{pmatrix}1\\-2\end{pmatrix},
\]
故
\[
[T]_S =
\begin{pmatrix}
3 & 1\\
1 & -2
\end{pmatrix}.
\]

现在取基
\[
B=\left\{
\bb_1=\begin{pmatrix}1\\1\end{pmatrix},
\bb_2=\begin{pmatrix}1\\2\end{pmatrix}
\right\}.
\]

首先求从 \(B\) 到标准基的坐标变换矩阵 \([I]_{SB}\):它的列就是 \(\bb_1,\bb_2\),所以
\[
[I]_{SB}=
\begin{pmatrix}
1 & 1\\
1 & 2
\end{pmatrix}.
\]
其逆为
\[
[I]_{BS}=[I]_{SB}^{-1}
=
\begin{pmatrix}
2 & -1\\
-1 & 1
\end{pmatrix}.
\]

\(T\) 在基 \(B\) 下的矩阵按一般公式为
\[
[T]_B=[I]_{BS}\,[T]_S\,[I]_{SB}.
\]
先算
\[
[T]_S[I]_{SB}
=
\begin{pmatrix}
3 & 1\\
1 & -2
\end{pmatrix}
\begin{pmatrix}
1 & 1\\
1 & 2
\end{pmatrix}
=
\begin{pmatrix}
4 & 5\\
-1 & -3
\end{pmatrix}.
\]
再左乘 \([I]_{BS}\):
\[
[T]_B
=
\begin{pmatrix}
2 & -1\\
-1 & 1
\end{pmatrix}
\begin{pmatrix}
4 & 5\\
-1 & -3
\end{pmatrix}
=
\begin{pmatrix}
9 & 13\\
-5 & -8
\end{pmatrix}.
\]

也可以直接计算:
\[
T(\bb_1)=T(1,1)^T=(4,-1)^T,\quad
T(\bb_2)=T(1,2)^T=(5,-3)^T,
\]
再把它们在基 \(B\) 下分解,结果与上矩阵一致。

\medskip

\textbf{8.5}

若 \(A,B\) 相似,则存在可逆矩阵 \(Q\) 使
\[
B = Q^{-1}AQ.
\]
于是
\[
\operatorname{trace}B
= \operatorname{trace}(Q^{-1}AQ).
\]
使用 \(\operatorname{trace}(XY)=\operatorname{trace}(YX)\)(对任意尺寸允许乘法的矩阵成立),得
\[
\operatorname{trace}(Q^{-1}AQ)
= \operatorname{trace}(AQ Q^{-1})
= \operatorname{trace}(A I)
= \operatorname{trace}A.
\]
所以 \(\operatorname{trace}A=\operatorname{trace}B\)。

\medskip

\textbf{8.6}

设
\[
A=\begin{pmatrix}1&3\\2&2\end{pmatrix},\quad
B=\begin{pmatrix}0&2\\4&2\end{pmatrix}.
\]

相似矩阵必须同阶,且具有相同的特征多项式(因此有相同的迹与行列式)。

先比较迹与行列式:
\[
\operatorname{trace}A = 1+2=3,\quad
\det A = 1\cdot2-3\cdot2 = -4;
\]
\[
\operatorname{trace}B = 0+2=2,\quad
\det B = 0\cdot2-2\cdot4 = -8.
\]

迹已经不同,因此它们不可能相似。





\end{exer}








\section{第三章答案}

\begin{exer}


下面只给解答内容,方便你直接嵌到解答册里(不使用 \verb|\begin{itemize}|,加粗用 \verb|\textbf{}|)。

\medskip

\textbf{3.1}

\(A\) 为 \(n\times n\) 矩阵,把 \(A\) 的每一\emph{行}都乘以 3,相当于把向量组的每一列都乘以 3;按行列式的多重线性,
\[
\det(3A)=3^n\det A.
\]
当 \(n=1\) 时才退化为 \(\det(3A)=3\det A\)。

\medskip

\textbf{3.2}

行列式对每一列都是线性的。

a) \(B\) 的第一列是 \(2\) 倍 \(A\) 的第一列,第二列是 \(3\) 倍,第三列是 \(5\) 倍,因此
\[
\det B =2\cdot 3\cdot 5\;\det A=30\det A.
\]

b) 记
\[
A=[\,a_1\ a_2\ a_3\,],
\quad
B=[\,3a_1,\ 4a_2+5a_1,\ 5a_3\,].
\]
按第二列的线性展开:
\[
\det B=\det(3a_1,\,4a_2+5a_1,\,5a_3)
     =\det(3a_1,\,4a_2,\,5a_3)+\det(3a_1,\,5a_1,\,5a_3).
\]
第二项中第一、二列成比例,行列式为 0,所以
\[
\det B=\det(3a_1,\,4a_2,\,5a_3)=3\cdot4\cdot5\;\det A=60\det A.
\]

\medskip

\textbf{3.3}

\[
D_1=
\begin{vmatrix}
0 & 1 & 2\\
-1&0 &-3\\
2 &3 &0
\end{vmatrix}.
\]
对第一行按列展开:
\[
D_1=0
-1\cdot
\begin{vmatrix}
-1&-3\\
2&0
\end{vmatrix}
+2\cdot
\begin{vmatrix}
-1&0\\
2 &3
\end{vmatrix}
=
-\bigl((-1)\cdot0-(-3)\cdot2\bigr)
+2\bigl((-1)\cdot3-0\cdot2\bigr)
=-6-6=-12.
\]

\[
D_2=
\begin{vmatrix}
1&2&3\\
4&5&6\\
7&8&9
\end{vmatrix}.
\]
用行运算:用第二行减去第一行的 4 倍、第三行减去第一行的 7 倍(行列式不变):
\[
\sim
\begin{vmatrix}
1&2&3\\
0&-3&-6\\
0&-6&-12
\end{vmatrix}.
\]
第二、三行成比例,故
\[
D_2=0.
\]

\[
D_3=
\begin{vmatrix}
1 & 0 & -2 & 3\\
-3& 1 & 1  & 2\\
0 & 4 & -1 & 1\\
2 & 3 & 0  & 1
\end{vmatrix}.
\]
对第一列展开:
\[
D_3
=
1\cdot
\begin{vmatrix}
1 & 1 & 2\\
4 & -1 & 1\\
3 & 0 & 1
\end{vmatrix}
+(-3)\cdot(-1)^{2+1}
\begin{vmatrix}
0 & -2 & 3\\
4 & -1 & 1\\
3 & 0 & 1
\end{vmatrix}
+2\cdot(-1)^{4+1}
\begin{vmatrix}
0 & -2 & 3\\
1 & 1  & 2\\
4 & -1 & 1
\end{vmatrix}.
\]
记这三个 \(3\times3\) 行列式分别为 \(M_1,M_2,M_3\)。

第一项:
\[
M_1=
\begin{vmatrix}
1 & 1 & 2\\
4 & -1 & 1\\
3 & 0 & 1
\end{vmatrix}
=
1\begin{vmatrix}-1&1\\0&1\end{vmatrix}
-1\begin{vmatrix}4&1\\3&1\end{vmatrix}
+2\begin{vmatrix}4&-1\\3&0\end{vmatrix}
=(-1)+(-1)+2\cdot3=4.
\]

第二项:
\[
M_2=
\begin{vmatrix}
0 & -2 & 3\\
4 & -1 & 1\\
3 & 0 & 1
\end{vmatrix}
=
0
-(-2)\begin{vmatrix}4&1\\3&1\end{vmatrix}
+3\begin{vmatrix}4&-1\\3&0\end{vmatrix}
=2\cdot1+3\cdot3=11.
\]

第三项:
\[
M_3=
\begin{vmatrix}
0 & -2 & 3\\
1 & 1  & 2\\
4 & -1 & 1
\end{vmatrix}
=
0
-(-2)\begin{vmatrix}1&2\\4&1\end{vmatrix}
+3\begin{vmatrix}1&1\\4&-1\end{vmatrix}
=2(-7)+3(-5)=-29.
\]

代回:
\[
D_3
=1\cdot4+3\cdot11-2\cdot(-29)
=4+33+58=95.
\]

最后一个:
\[
D_4=\begin{vmatrix}1&x\\1&y\end{vmatrix}=1\cdot y-x\cdot1=y-x.
\]

\medskip

\textbf{3.4}

若 \(A\) 是 \(n\times n\) 反对称矩阵,则 \(A^T=-A\)。利用 \(\det A=\det(A^T)\) 和 \(\det(\alpha A)=\alpha^n\det A\) 得
\[
\det A=\det(A^T)=\det(-A)=(-1)^n\det A.
\]
若 \(n\) 为奇数,则 \((-1)^n=-1\),于是
\[
\det A=-\det A\quad\Rightarrow\quad \det A=0.
\]
若 \(n\) 为偶数,上式只给出 \(\det A=\det A\),不能推出为 0,实际上也不为 0:例如
\[
A=
\begin{pmatrix}
0&1\\
-1&0
\end{pmatrix}
\]
是 \(2\times2\) 的反对称矩阵,而 \(\det A=1\ne0\)。因此结论只对奇数维成立。

\medskip

\textbf{3.5}

若 \(A\) 幂零,则存在正整数 \(k\) 使 \(A^k=0\)。对方阵有
\[
\det(A^k)=(\det A)^k.
\]
又 \(A^k=0\) 的行列式为 0,所以
\[
(\det A)^k=\det(A^k)=\det 0=0,
\]
从而 \(\det A=0\)。

\medskip

\textbf{3.6}

若 \(A\) 与 \(B\) 相似,则存在可逆矩阵 \(Q\) 使
\[
B=Q^{-1}AQ.
\]
使用乘积行列式公式:
\[
\det B=\det(Q^{-1}AQ)
=\det(Q^{-1})\det(A)\det(Q)
=\frac1{\det Q}\det A\det Q
=\det A.
\]

\medskip

\textbf{3.7}

若 \(Q\) 为正交矩阵,则 \(Q^TQ=I\)。取行列式:
\[
\det(Q^TQ)=\det I=1.
\]
左边
\[
\det(Q^TQ)=\det(Q^T)\det(Q)=(\det Q)^2.
\]
于是
\[
(\det Q)^2=1\quad\Rightarrow\quad \det Q=\pm1.
\]

\medskip

\textbf{3.8}

计算
\[
V=
\begin{vmatrix}
1 & x & x^2\\
1 & y & y^2\\
1 & z & z^2
\end{vmatrix}.
\]
用列运算(不改变行列式):用第二列减第一列、第三列减第一列的倍数:
\[
\begin{aligned}
V
&=
\begin{vmatrix}
1 & x & x^2\\
1 & y & y^2\\
1 & z & z^2
\end{vmatrix}
\sim
\begin{vmatrix}
1 & x-1\cdot x & x^2- x\cdot x\\
1 & y-x & y^2-xy\\
1 & z-x & z^2-xz
\end{vmatrix}\\
&=
\begin{vmatrix}
1 & 0 & 0\\
1 & y-x & (y-x)(y+x)\\
1 & z-x & (z-x)(z+x)
\end{vmatrix}.
\end{aligned}
\]
对第一行展开:
\[
V=1\cdot
\begin{vmatrix}
y-x & (y-x)(y+x)\\
z-x & (z-x)(z+x)
\end{vmatrix}.
\]
从这两列中各提取一个公因子:
\[
V=(y-x)(z-x)
\begin{vmatrix}
1 & y+x\\
1 & z+x
\end{vmatrix}
=(y-x)(z-x)\bigl((z+x)-(y+x)\bigr)
=(y-x)(z-x)(z-y).
\]
通常把因子按 \((z-x)(z-y)(y-x)\) 排列,只是顺序不同,相差一个偶置换的符号,这里可以直接写成
\[
V=(z-x)(z-y)(y-x).
\]

\medskip

\textbf{3.9}

三角形 \(ABC\) 的面积等于从 \(A\) 出发的向量 \(\overrightarrow{AB},\overrightarrow{AC}\) 构成平行四边形面积的一半。平行四边形面积的绝对值为
\[
\left|
\det
\begin{pmatrix}
x_2-x_1 & y_2-y_1\\
x_3-x_1 & y_3-y_1
\end{pmatrix}
\right|.
\]
故
\[
\operatorname{Area}(ABC)
=
\frac12
\left|
\det
\begin{pmatrix}
x_2-x_1 & y_2-y_1\\
x_3-x_1 & y_3-y_1
\end{pmatrix}
\right|.
\]

现在把题给的 \(3\times3\) 行列式化简。考虑
\[
D=
\begin{vmatrix}
1 & x_1 & y_1\\
1 & x_2 & y_2\\
1 & x_3 & y_3
\end{vmatrix}.
\]
用第二、三行分别减去第一行(行列式不变):
\[
D=
\begin{vmatrix}
1 & x_1 & y_1\\
0 & x_2-x_1 & y_2-y_1\\
0 & x_3-x_1 & y_3-y_1
\end{vmatrix}.
\]
对第一列展开得
\[
D=
1\cdot
\begin{vmatrix}
x_2-x_1 & y_2-y_1\\
x_3-x_1 & y_3-y_1
\end{vmatrix}.
\]
因此
\[
\operatorname{Area}(ABC)
=
\frac12 |D|
=
\frac12
\left|
\begin{vmatrix}
1 & x_1 & y_1\\
1 & x_2 & y_2\\
1 & x_3 & y_3
\end{vmatrix}
\right|.
\]

\medskip

\textbf{3.10}

设 \(A\) 和 \(C\) 是方阵,大小可以不同。先看
\[
M_1=
\begin{pmatrix}
I & *\\
\oo & A
\end{pmatrix}.
\]
对上方块 \(I\) 的行,可以加下方块 \(A\) 的线性组合而不改变行列式;特别地,我们可以把右上角的 \(*\) 消为 0,得到
\[
M_1\sim
\begin{pmatrix}
I & \oo\\
\oo & A
\end{pmatrix}.
\]
后者是块对角矩阵,其行列式为两个对角块行列式之积:
\[
\det M_1 = (\det I)(\det A)=\det A.
\]

同理,
\[
M_2=
\begin{pmatrix}
A & *\\
\oo & I
\end{pmatrix}
\]
可以通过对下方块 \(I\) 的行加上上方块 \(A\) 的线性组合,把右上角 \(*\) 消为 0,得到块对角矩阵 \(\begin{pmatrix}A&0\\0&I\end{pmatrix}\),故
\[
\det M_2=\det A.
\]

再看第三个:
\[
M_3=
\begin{pmatrix}
I & \oo\\
* & A
\end{pmatrix}.
\]
这次对\emph{列}做运算:可以用右边的列(属于块 \(A\))的合适线性组合去掉左下角的 \(*\),由列线性不变性仍得
\[
M_3\sim
\begin{pmatrix}
I & \oo\\
\oo & A
\end{pmatrix},
\]
故 \(\det M_3=\det A\)。

第四个矩阵同理:
\[
M_4=
\begin{pmatrix}
A & \oo\\
* & I
\end{pmatrix}
\sim
\begin{pmatrix}
A & \oo\\
\oo & I
\end{pmatrix},
\]
从而 \(\det M_4=\det A\)。

\medskip

\textbf{3.11}

设
\[
M=
\begin{pmatrix}
A & B\\
\oo & C
\end{pmatrix}.
\]
按提示分解:
\[
M=
\begin{pmatrix}
I & B\\
\oo & C
\end{pmatrix}
\begin{pmatrix}
A & \oo\\
\oo & I
\end{pmatrix}.
\]
由 3.10,第一因子的行列式是 \(\det C\),第二因子的行列式是 \(\det A\)。使用乘积公式:
\[
\det M
=
\det
\begin{pmatrix}
A & B\\
\oo & C
\end{pmatrix}
=
(\det C)(\det A)
=(\det A)(\det C).
\]

\medskip

\textbf{3.12}

设 \(A\) 是 \(m\times n\) 矩阵,\(B\) 是 \(n\times m\) 矩阵。考虑块矩阵
\[
M=
\begin{pmatrix}
\oo & A\\
-B & I
\end{pmatrix}
\quad \text{和}\quad
P=
\begin{pmatrix}
I & \oo\\
B & I
\end{pmatrix}.
\]
两者都是方阵(大小为 \((m+n)\times(m+n)\))。计算它们的乘积:
\[
MP=
\begin{pmatrix}
\oo & A\\
-B & I
\end{pmatrix}
\begin{pmatrix}
I & \oo\\
B & I
\end{pmatrix}
=
\begin{pmatrix}
AB & A\\
\oo & I
\end{pmatrix}.
\]
由 3.10,
\[
\det(MP)=\det
\begin{pmatrix}
AB & A\\
\oo & I
\end{pmatrix}
=\det(AB)\cdot\det I=\det(AB).
\]
另一方面,\(\det(MP)=\det M\cdot\det P\)。再由 3.10,
\[
\det P=\det
\begin{pmatrix}
I & \oo\\
B & I
\end{pmatrix}
=\det I=1.
\]
所以
\[
\det M=\det(MP)=\det(AB).
\]
也就是
\[
\det
\begin{pmatrix}
\oo & A\\
-B & I
\end{pmatrix}
=\det(AB).
\]


\textbf{4.1}

\(\sigma\) 把 \((1,2,3,4,5)\) 变成 \((5,4,1,2,3)\),即
\[
\sigma(1)=5,\ \sigma(2)=4,\ \sigma(3)=1,\ \sigma(4)=2,\ \sigma(5)=3,
\]
写成循环
\[
\sigma=(1\,5\,3)(2\,4).
\]

a) \(\sigma\) 是一个 3-循环和一个 2-循环的乘积。一个 \(k\)-循环可写成 \(k-1\) 个换位,故
\[
\text{sign}(\sigma)=(-1)^{(3-1)+(2-1)}=(-1)^3=-1,
\]
所以 \(\sigma\) 是奇排列。

b) 计算 \(\sigma^2\):
\[
\sigma^2(1)=\sigma(5)=3,\quad
\sigma^2(2)=\sigma(4)=2,\quad
\sigma^2(3)=\sigma(1)=5,\quad
\sigma^2(4)=\sigma(2)=4,\quad
\sigma^2(5)=\sigma(3)=1.
\]
因此
\[
\sigma^2=(1\,3\,5), 
\]
在有序组上表现为
\[
(1,2,3,4,5)\mapsto{\sigma^2}(3,2,5,4,1).
\]

c) 由循环分解,\(\sigma^{-1}=(1\,3\,5)(2\,4)\)(只需把每个循环反向)。显式地,
\[
\sigma^{-1}(1)=3,\ \sigma^{-1}(2)=4,\ \sigma^{-1}(3)=5,\ \sigma^{-1}(4)=2,\ \sigma^{-1}(5)=1,
\]
所以
\[
(1,2,3,4,5)\mapsto{\sigma^{-1}}(3,4,5,2,1).
\]

d) 排列的符号满足 \(\text{sign}(\sigma^{-1})=\text{sign}(\sigma)\),因为
\[
1=\text{sign}(\operatorname{id})
=\text{sign}(\sigma\sigma^{-1})
=\text{sign}(\sigma)\,\text{sign}(\sigma^{-1}).
\]
既然 \(\text{sign}(\sigma)=-1\),只可能有 \(\text{sign}(\sigma^{-1})=-1\)。因此 \(\sigma^{-1}\) 也是奇排列。

\medskip

\textbf{4.2}

设 \(P\) 是一个 \(n\times n\) 排列矩阵。

a) 将 \(P\) 作用在标准基向量上:若第 \(j\) 列的唯一的 1 在第 \(i\) 行,则
\[
P e_j = e_i.
\]
因此 \(P\) 对向量的作用就是\emph{重新排列坐标}。对任意
\[
x=(x_1,\dots,x_n)^T,\quad Px=(x_{\sigma(1)},\dots,x_{\sigma(n)})^T,
\]
其中 \(\sigma\) 是某个 \(\{1,\dots,n\}\) 的排列。这解释了“排列矩阵”的名字:它按排列 \(\sigma\) 重新排列坐标。

b) 由于 \(\sigma\) 是双射,这个线性变换是从 \(\RR^n\) 到自身的一一对应,故必然可逆。更具体地说,\(\sigma\) 有逆排列 \(\sigma^{-1}\),对应的排列矩阵 \(P_{\sigma^{-1}}\) 满足
\[
P_{\sigma^{-1}}P_\sigma = P_{\sigma}P_{\sigma^{-1}} = I.
\]
因此
\[
P^{-1}
=
P_{\sigma^{-1}},
\]
它正是把坐标“排回原位”的排列矩阵;矩阵上表现为:把 \(P\) 的行(和列)互换回原来的位置,也就是
\[
P^{-1}=P^T.
\]

c) 与 \(P\) 对应的排列记为 \(\sigma\)。那么
\[
P^k \quad\text{对应排列}\quad \sigma^k.
\]
在有限集合 \(\{1,\dots,n\}\) 上,\(\sigma\) 生成的循环都是有限长度的。记每个循环的长度分别为 \(l_1,\dots,l_r\),取
\[
N=\operatorname{lcm}(l_1,\dots,l_r)>0.
\]
则对每个 \(j\),循环上的元素在 \(\sigma^N\) 作用下都回到原位,因此
\[
\sigma^N=\operatorname{id},\quad P^N=I.
\]

\medskip

\textbf{4.3}

用行列式解释。记 \(n=9\),考虑单位矩阵 \(I_n\)。一方面
\[
\det I_n = 1.
\]
另一方面,用排列定义写出
\[
\det I_n = \sum_{\sigma\in S_n} \text{sign}(\sigma)\,a_{1,\sigma(1)}\dots a_{n,\sigma(n)}.
\]
对单位矩阵,只有当 \(\sigma=\operatorname{id}\) 时,每个 \(a_{k,\sigma(k)}=1\),否则有某个位置为 0;因此
\[
\det I_n = \text{sign}(\operatorname{id}) = 1,
\]
这乍看好像没用。但现在取任意一个奇排列 \(\tau\)。左乘对应的排列矩阵等价于对行做重排,不改变行列式的绝对值,只改变符号,但对 \(I_n\) 来说 \(\det I_n=1\)。于是
\[
1=\det I_n = \det(P_\tau I_n) = \det P_\tau = \text{sign}(\tau) = -1,
\]
显然矛盾——原因在于“只取一个排列”不对,我们得看“全部排列的和”。

更好的做法:把所有排列分成两类:偶排列集合 \(E\) 与奇排列集合 \(O\),并用
\[
\sum_{\sigma\in S_n} \text{sign}(\sigma) = |E|-|O|
\]
来观察。令
\[
A=I_n.
\]
如前所述,只有恒等排列一项不为 0;但是我们也可以把 \(A\) 的行(或列)先做任意一个\emph{奇}排列 \(\tau\),得到矩阵
\[
B=P_\tau I_n.
\]
一方面,\(B\) 仍然是排列矩阵,所以
\[
\det B=\text{sign}(\tau)=-1.
\]
另一方面,用排列公式
\[
\det B
=
\sum_{\sigma\in S_n}\text{sign}(\sigma)\,b_{1,\sigma(1)}\dots b_{n,\sigma(n)}.
\]
和单位矩阵情况类似,只有一项不为 0,而这唯一一项来自排列 \(\tau^{-1}\);于是
\[
\det B = \text{sign}(\tau^{-1}) = \text{sign}(\tau)=-1.
\]
这与前一条自洽。

真正使用行列式解法的标准论证是:考虑
\[
P=\prod_{1\le i<j\le 9}(x_i-x_j).
\]
这是\emph{反对称}多项式:任意交换两个变量,它变号不变模。排列奇偶性恰好是:偶排列保持符号,奇排列翻转符号。把所有排列加在一起得到 0,从而可以看出偶排列和奇排列数目相同。更简单地,用线性代数语言描述为:在 \(S_9\) 上的函数空间里,“符号函数”是一个非零函数,它的总和
\[
\sum_{\sigma\in S_9}\text{sign}(\sigma)
\]
可理解为 \(\det\) 在某个特殊矩阵上的值,也必为 0,于是偶排列数与奇排列数相等,从而总排列数为偶数。

\medskip

一个简洁结论可以直接陈述为:

设 \(N_+\) 为偶排列个数,\(N_-\) 为奇排列个数。若我们取矩阵
\[
A=I_9,
\]
则
\[
\det A = \sum_{\sigma\in S_9}\text{sign}(\sigma)\,a_{1,\sigma(1)}\dots a_{9,\sigma(9)}
       = N_+-N_-.
\]
另一方面,\(\det A=1\),但对 \(I_9\) 来说只有恒等置换那一项非零,因此实际上 \(N_+-N_-=1\)。再考虑对某个行交换的矩阵,进行同样分析,可得 \(N_+-N_-=-1\),二式联立给出 \(N_+=N_-\)。故共 \(9!\) 个排列中奇偶各半。

(根据你书的精确写法,可以保留最简那一种行列式论证;上面的长解释可大幅压缩。)

\medskip

\textbf{4.4}

若 \(\sigma\) 是奇排列,则 \(\text{sign}(\sigma)=-1\)。

\[
\text{sign}(\sigma^2)
=
\text{sign}(\sigma)^2
=(-1)^2=1,
\]
所以 \(\sigma^2\) 是偶排列。

对逆排列,
\[
1=\text{sign}(\operatorname{id})
=\text{sign}(\sigma\sigma^{-1})
=\text{sign}(\sigma)\,\text{sign}(\sigma^{-1}).
\]
因此
\[
\text{sign}(\sigma^{-1})=\text{sign}(\sigma)=-1,
\]
所以 \(\sigma^{-1}\) 也是奇排列。

\medskip

\textbf{4.5}

按公式
\[
\det A=
\sum_{\sigma\in S_n}\text{sign}(\sigma)\,
a_{1,\sigma(1)}a_{2,\sigma(2)}\dots a_{n,\sigma(n)}.
\]

共有 \(n!\) 个排列。对每个排列 \(\sigma\),要计算一个长度为 \(n\) 的乘积
\[
a_{1,\sigma(1)}\cdots a_{n,\sigma(n)}
\]
需要 \(n-1\) 次乘法。于是总乘法次数为
\[
n!\,(n-1).
\]

把所有 \(n!\) 个乘积相加需要 \(n!-1\) 次加法。故用定义式直接计算一个 \(n\times n\) 行列式,约需
\[
\text{乘法次数}\;=\;n!\,(n-1),\quad
\text{加法次数}\;=\;n!-1.
\]



下面只给各题的计算和结论,方便你直接嵌入解答,不用列表环境。

\medskip

\textbf{5.1}

第一个行列式:
\[
\begin{vmatrix}
0 & 1 & 1\\
1 & 2 & -5\\
6 & 4 & -3
\end{vmatrix}
=
0\cdot\begin{vmatrix}2&-5\\4&-3\end{vmatrix}
-1\cdot\begin{vmatrix}1&-5\\6&-3\end{vmatrix}
+1\cdot\begin{vmatrix}1&2\\6&4\end{vmatrix}.
\]
\[
\begin{vmatrix}1&-5\\6&-3\end{vmatrix}
=1\cdot(-3)-(-5)\cdot6=-3+30=27,
\quad
\begin{vmatrix}1&2\\6&4\end{vmatrix}
=1\cdot4-2\cdot6=4-12=-8.
\]
所以
\[
\det=
-27-8=-35.
\]

第二个行列式,做行变换:
\[
\begin{vmatrix}
1 & -2 & 3 & -12\\
-5 & 12 & -14 & 19\\
-9 & 22 & -20 & 31\\
-4 & 9 & -14 & 15
\end{vmatrix}
\sim
\begin{vmatrix}
1 & -2 & 3 & -12\\
0 & 2 & 1 & -41\\
0 & 4 & 7 & 77\\
0 & 1 & -2 & -33
\end{vmatrix}.
\]
将第 2 行、第 3 行互换(记号 \(\det\) 变号一次):
\[
\det=-\begin{vmatrix}
1 & -2 & 3 & -12\\
0 & 4 & 7 & 77\\
0 & 2 & 1 & -41\\
0 & 1 & -2 & -33
\end{vmatrix}.
\]
沿第一列展开:
\[
\det
=-1\cdot
\begin{vmatrix}
4 & 7 & 77\\
2 & 1 & -41\\
1 & -2 & -33
\end{vmatrix}.
\]
再对这个 \(3\times3\) 用第一行展开:
\[
\begin{vmatrix}
4 & 7 & 77\\
2 & 1 & -41\\
1 & -2 & -33
\end{vmatrix}
=
4\begin{vmatrix}1&-41\\-2&-33\end{vmatrix}
-7\begin{vmatrix}2&-41\\1&-33\end{vmatrix}
+77\begin{vmatrix}2&1\\1&-2\end{vmatrix}.
\]
\[
\begin{vmatrix}1&-41\\-2&-33\end{vmatrix}
=1\cdot(-33)-(-41)\cdot(-2)=-33-82=-115,
\]
\[
\begin{vmatrix}2&-41\\1&-33\end{vmatrix}
=2\cdot(-33)-(-41)\cdot1=-66+41=-25,
\]
\[
\begin{vmatrix}2&1\\1&-2\end{vmatrix}
=2\cdot(-2)-1\cdot1=-4-1=-5.
\]
于是
\[
4(-115)-7(-25)+77(-5)
=-460+175-385=-670.
\]
所以原行列式
\[
\det=-(-670)=670.
\]

\medskip

\textbf{5.2}

第一个:
\[
\begin{vmatrix}
1 & 2 & 0\\
1 & 1 & 5\\
1 & -3 & 0
\end{vmatrix}
\]
选择第 1 列展开:
\[
=
1\begin{vmatrix}1&5\\-3&0\end{vmatrix}
-1\begin{vmatrix}2&0\\-3&0\end{vmatrix}
+1\begin{vmatrix}2&0\\1&5\end{vmatrix}.
\]
\[
\begin{vmatrix}1&5\\-3&0\end{vmatrix}=1\cdot0-5\cdot(-3)=15,
\quad
\begin{vmatrix}2&0\\-3&0\end{vmatrix}=0,
\quad
\begin{vmatrix}2&0\\1&5\end{vmatrix}=10.
\]
故
\[
\det=15+10=25.
\]

第二个:
\[
\begin{vmatrix}
4 & -6 & -4 & 4\\
2 & 1 & 0 & 0\\
0 & -3 & 1 & 3\\
-2 & 2 & -3 & -5
\end{vmatrix}
\]
沿第二行(两个零)展开:
\[
=2\cdot(-1)^{2+1}
\begin{vmatrix}
-6 & -4 & 4\\
-3 & 1 & 3\\
2 & -3 & -5
\end{vmatrix}
+1\cdot(-1)^{2+2}
\begin{vmatrix}
4 & -4 & 4\\
0 & 1 & 3\\
-2 & -3 & -5
\end{vmatrix}.
\]
即
\[
=-2
\begin{vmatrix}
-6 & -4 & 4\\
-3 & 1 & 3\\
2 & -3 & -5
\end{vmatrix}
+
\begin{vmatrix}
4 & -4 & 4\\
0 & 1 & 3\\
-2 & -3 & -5
\end{vmatrix}.
\]

先算
\[
M_1=
\begin{vmatrix}
-6 & -4 & 4\\
-3 & 1 & 3\\
2 & -3 & -5
\end{vmatrix}
\]
沿第一行展开:
\[
M_1
=-6\begin{vmatrix}1&3\\-3&-5\end{vmatrix}
-(-4)\begin{vmatrix}-3&3\\2&-5\end{vmatrix}
+4\begin{vmatrix}-3&1\\2&-3\end{vmatrix}.
\]
\[
\begin{vmatrix}1&3\\-3&-5\end{vmatrix}=1\cdot(-5)-3\cdot(-3)=-5+9=4,
\]
\[
\begin{vmatrix}-3&3\\2&-5\end{vmatrix}=(-3)\cdot(-5)-3\cdot2=15-6=9,
\]
\[
\begin{vmatrix}-3&1\\2&-3\end{vmatrix}=(-3)\cdot(-3)-1\cdot2=9-2=7.
\]
所以
\[
M_1=-6\cdot4+4\cdot9+4\cdot7=-24+36+28=40.
\]

再算
\[
M_2=
\begin{vmatrix}
4 & -4 & 4\\
0 & 1 & 3\\
-2 & -3 & -5
\end{vmatrix}
\]
沿第二行展开:
\[
M_2
=1\cdot(-1)^{2+2}\begin{vmatrix}4&4\\-2&-5\end{vmatrix}
+3\cdot(-1)^{2+3}\begin{vmatrix}4&-4\\-2&-3\end{vmatrix}
\]
\[
=\begin{vmatrix}4&4\\-2&-5\end{vmatrix}
-3\begin{vmatrix}4&-4\\-2&-3\end{vmatrix}.
\]
\[
\begin{vmatrix}4&4\\-2&-5\end{vmatrix}=4\cdot(-5)-4\cdot(-2)=-20+8=-12,
\]
\[
\begin{vmatrix}4&-4\\-2&-3\end{vmatrix}=4\cdot(-3)-(-4)\cdot(-2)=-12-8=-20.
\]
故
\[
M_2=-12-3(-20)=-12+60=48.
\]

于是原行列式
\[
\det=-2\cdot40+48=-80+48=-32.
\]

\medskip

\textbf{5.3}

记
\[
A=
\begin{pmatrix}
0 & 0 & \cdots & 0 & a_0\\
-1 & 0 & \cdots & 0 & a_1\\
0 & -1 & \cdots & 0 & a_2\\
\vdots & \vdots & \ddots & \vdots & \vdots\\
0 & 0 & \cdots & -1 & a_{n-1}
\end{pmatrix},
\quad
A+tI=
\begin{pmatrix}
t & 0 & \cdots & 0 & a_0\\
-1 & t & \cdots & 0 & a_1\\
0 & -1 & \cdots & 0 & a_2\\
\vdots & \vdots & \ddots & \vdots & \vdots\\
0 & 0 & \cdots & -1 & a_{n-1}+t
\end{pmatrix}.
\]
对第一行按代数余子式展开。第一行只有两项 \(t\)(在位置 \((1,1)\))和 \(a_0\)(在位置 \((1,n)\))非零:
\[
\det(A+tI)
=t\det B
+(-1)^{1+n}a_0\det C,
\]
其中 \(B\) 是去掉第 1 行第 1 列得到的 \((n-1)\times(n-1)\) 矩阵,
\[
B=
\begin{pmatrix}
t & 0 & \cdots & 0 & a_1\\
-1 & t & \cdots & 0 & a_2\\
\vdots & \vdots & \ddots & \vdots & \vdots\\
0 & 0 & \cdots & -1 & a_{n-1}+t
\end{pmatrix},
\]
而 \(C\) 是去掉第 1 行第 \(n\) 列得到的 \((n-1)\times(n-1)\) 矩阵:
\[
C=
\begin{pmatrix}
-1 & 0 & \cdots & 0\\
0 & -1 & \cdots & 0\\
\vdots & \vdots & \ddots & \vdots\\
0 & 0 & \cdots & -1
\end{pmatrix}
=-I_{n-1}.
\]
因此
\[
\det C=\det(-I_{n-1})=(-1)^{n-1},
\]
从而
\[
(-1)^{1+n}\det C=(-1)^{1+n}(-1)^{n-1}=(-1)^{2n}=1.
\]
所以
\[
\det(A+tI)=t\det B + a_0.
\]

注意 \(B\) 与 \(A+tI\) 具有相同的三对角结构,只是维数减少为 \(n-1\),并且最后一列为 \(a_1,a_2,\dots,a_{n-2},a_{n-1}+t\)。由归纳假设,
\[
\det B
=t^{n-1}+a_{n-1}t^{n-2}+\cdots+a_2 t + a_1.
\]
于是
\[
\det(A+tI)
=t\bigl(t^{n-1}+a_{n-1}t^{n-2}+\cdots+a_2 t + a_1\bigr)+a_0
=t^n+a_{n-1}t^{n-1}+\cdots+a_1 t + a_0.
\]
也就是说
\[
\det(A+tI)=t^n+a_{n-1}t^{n-1}+a_{n-2}t^{n-2}+\dots+a_1 t + a_0.
\]

\medskip

\textbf{5.4}

利用 \(2\times2\) 的代数余子式公式
\[
\begin{pmatrix}
a & b\\
c & d
\end{pmatrix}^{-1}
=\frac1{ad-bc}
\begin{pmatrix}
d & -b\\
-c & a
\end{pmatrix}.
\]

第一个矩阵
\[
A_1=
\begin{pmatrix}
1 & 2\\
3 & 4
\end{pmatrix},\quad
\det A_1=1\cdot4-2\cdot3=-2,
\]
\[
A_1^{-1}
=\frac1{-2}
\begin{pmatrix}
4 & -2\\
-3 & 1
\end{pmatrix}
=
\begin{pmatrix}
-2 & 1\\
\frac32 & -\frac12
\end{pmatrix}.
\]

第二个矩阵
\[
A_2=
\begin{pmatrix}
19 & -17\\
3 & -2
\end{pmatrix},\quad
\det A_2=19\cdot(-2)-(-17)\cdot3=-38+51=13,
\]
\[
A_2^{-1}
=\frac1{13}
\begin{pmatrix}
-2 & 17\\
-3 & 19
\end{pmatrix}.
\]

第三个矩阵
\[
A_3=
\begin{pmatrix}
1 & 0\\
3 & 5
\end{pmatrix},\quad
\det A_3=1\cdot5-0\cdot3=5,
\]
\[
A_3^{-1}
=\frac1{5}
\begin{pmatrix}
5 & 0\\
-3 & 1
\end{pmatrix}
=
\begin{pmatrix}
1 & 0\\
-\frac35 & \frac15
\end{pmatrix}.
\]

第四个矩阵
\[
A_4=
\begin{pmatrix}
1 & 1 & 0\\
2 & 1 & 2\\
0 & 1 & 1
\end{pmatrix}.
\]
先算 \(\det A_4\):
\[
\det A_4
=1\begin{vmatrix}1&2\\1&1\end{vmatrix}
-1\begin{vmatrix}2&2\\0&1\end{vmatrix}
+0\cdot(\cdots)
=(1\cdot1-2\cdot1)-(2\cdot1-2\cdot0)
=(-1)-2=-3.
\]
余子式及代数余子式:
\[
C_{11}=(-1)^{1+1}\det\begin{pmatrix}1&2\\1&1\end{pmatrix}
=1\cdot(-1)=-1,
\]
\[
C_{12}=(-1)^{1+2}\det\begin{pmatrix}2&2\\0&1\end{pmatrix}
=-1\cdot2=-2,
\]
\[
C_{13}=(-1)^{1+3}\det\begin{pmatrix}2&1\\0&1\end{pmatrix}
=1\cdot2=2,
\]
\[
C_{21}=(-1)^{2+1}\det\begin{pmatrix}1&0\\1&1\end{pmatrix}
=-1\cdot1=-1,
\]
\[
C_{22}=(-1)^{2+2}\det\begin{pmatrix}1&0\\0&1\end{pmatrix}
=1\cdot1=1,
\]
\[
C_{23}=(-1)^{2+3}\det\begin{pmatrix}1&1\\0&1\end{pmatrix}
=-1\cdot1=-1,
\]
\[
C_{31}=(-1)^{3+1}\det\begin{pmatrix}1&0\\1&2\end{pmatrix}
=1\cdot2=2,
\]
\[
C_{32}=(-1)^{3+2}\det\begin{pmatrix}1&0\\2&2\end{pmatrix}
=-1\cdot2=-2,
\]
\[
C_{33}=(-1)^{3+3}\det\begin{pmatrix}1&1\\2&1\end{pmatrix}
=1\cdot(-1)=-1.
\]
代数余子式矩阵
\[
C=
\begin{pmatrix}
-1 & -2 & 2\\
-1 & 1 & -1\\
2 & -2 & -1
\end{pmatrix},
\quad
C^T=
\begin{pmatrix}
-1 & -1 & 2\\
-2 & 1 & -2\\
2 & -1 & -1
\end{pmatrix}.
\]
于是
\[
A_4^{-1}
=\frac1{\det A_4}C^T
=-\frac13
\begin{pmatrix}
-1 & -1 & 2\\
-2 & 1 & -2\\
2 & -1 & -1
\end{pmatrix}
=
\frac13
\begin{pmatrix}
1 & 1 & -2\\
2 & -1 & 2\\
-2 & 1 & 1
\end{pmatrix}.
\]

\medskip

\textbf{5.5}

设
\[
D_n=\det
\begin{pmatrix}
1 & -1 &  &        &        &   \\
1 & 1  & -1&        &        &   \\
  & 1  & 1 & \ddots &        &   \\
\vdots &   &\ddots&\ddots&-1 &   \\
       &   &       &1&1&-1\\
       &   &       & &1&1
\end{pmatrix}.
\]
对最后一行按代数余子式展开。最后一行只有两项非零:倒数第二列的 \(1\)(位置 \((n,n-1)\))和最后一列的 \(1\)(位置 \((n,n)\)):
\[
D_n
=1\cdot(-1)^{n+(n-1)}\det M_{n,n-1}
+1\cdot(-1)^{n+n}\det M_{n,n},
\]
即
\[
D_n
=-\det M_{n,n-1}+\det M_{n,n}.
\]

观察 \(M_{n,n}\):删去最后一行、最后一列,得到的正是前 \(n-1\) 行、\(n-1\) 列的同样三对角矩阵,所以
\[
\det M_{n,n}=D_{n-1}.
\]

再看 \(M_{n,n-1}\):删去最后一行、第 \(n-1\) 列。写出其结构,可验证在做一次沿着倒数第二行(即原来的第 \(n-1\) 行)展开后,会出现 \(D_{n-2}\),且符号处理后给出
\[
\det M_{n,n-1}=-D_{n-2}.
\]
代回上式:
\[
D_n
=-(-D_{n-2})+D_{n-1}
=D_{n-1}+D_{n-2}.
\]
于是 \(D_n\) 满足递推
\[
D_n=D_{n-1}+D_{n-2},
\]
初始值 \(D_1=1\), \(D_2=\det\begin{pmatrix}1&-1\\1&1\end{pmatrix}=2\),因此
\[
D_n
\]
是以 \(1,2,3,5,8,13,21,\dots\) 为首项的斐波那契数列。

(若你希望完整地把 \(\det M_{n,n-1}=-D_{n-2}\) 的行列式展开写出来,可以让我单独补一段专门的推导。)

\medskip

\textbf{5.6}

只给关键结论和结构,方便与你书的证明衔接。

a) 对 \(n=1\):
\[
\begin{vmatrix}
1 & c_0\\
1 & c_1
\end{vmatrix}
=c_1-c_0,
\]
右侧公式为 \((c_1-c_0)\),成立。

对 \(n=2\):
\[
\begin{vmatrix}
1 & c_0 & c_0^2\\
1 & c_1 & c_1^2\\
1 & c_2 & c_2^2
\end{vmatrix}
=(c_1-c_0)(c_2-c_0)(c_2-c_1),
\]
直接展开或用行列变换都可验证等式成立。

b) 视 \(c_n\) 为变量 \(x\),其他 \(c_0,\dots,c_{n-1}\) 固定。行列式是关于最后一行 \((1,x,x^2,\dots,x^n)\) 的多项式,线性代数中的一般事实:行列式关于每一行都是线性的,因此关于这一行的每个坐标是线性的;因为这些坐标本身是 \(1,x,\dots,x^n\),可见得到一个次数不超过 \(n\) 的多项式:
\[
\Delta(x)=A_0+A_1x+\cdots+A_n x^n,
\]
系数 \(A_k\) 只依赖于 \(c_0,\dots,c_{n-1}\)。

c) 当 \(x=c_j\)(\(0\le j\le n-1\))时,第 \(n\) 行与第 \(j\) 行完全相同,因此行列式为 0。于是 \(x=c_0,\dots,c_{n-1}\) 都是多项式 \(\Delta(x)\) 的根,因而
\[
\Delta(x)=A_n (x-c_0)(x-c_1)\cdots(x-c_{n-1}),
\]
其中 \(A_n\) 为最高次项系数。

d) 用归纳假设对 \(n-1\) 维范德蒙德行列式应用公式。将 \(\Delta(x)\) 中最高次 \(x^n\) 的系数计算出来:展开
\[
\Delta(x)=\det
\begin{pmatrix}
1 & c_0 & c_0^2 & \dots & c_0^n\\
\vdots & \vdots & \vdots & & \vdots\\
1 & c_{n-1} & c_{n-1}^2 & \dots & c_{n-1}^n\\
1 & x & x^2 & \dots & x^n
\end{pmatrix},
\]
对最后一列按代数余子式展开,最高次 \(x^n\) 的系数就是最后一列中 \(x^n\) 的系数 1 乘以与之对应的代数余子式,即
\[
A_n=\det
\begin{pmatrix}
1 & c_0 & \dots & c_0^{n-1}\\
\vdots & \vdots & & \vdots\\
1 & c_{n-1} & \dots & c_{n-1}^{n-1}
\end{pmatrix}.
\]
根据归纳假设,这正是
\[
A_n=\prod_{0\le j<k\le n-1}(c_k-c_j).
\]

另一方面,从 c) 得
\[
\Delta(x)=A_n(x-c_0)\cdots(x-c_{n-1}).
\]
将 \(x=c_n\) 代入,得到原 \(n+1\) 维范德蒙德行列式:
\[
\Delta(c_n)
=\prod_{0\le j<k\le n-1}(c_k-c_j)\cdot\prod_{j=0}^{n-1}(c_n-c_j)
=\prod_{0\le j<k\le n}(c_k-c_j),
\]
这就是所需公式。

\medskip

\textbf{5.7}

对一个 \(n\times n\) 矩阵用代数余子式展开(例如沿第一行):
\[
\det A=\sum_{j=1}^n a_{1j}C_{1j},
\]
其中 \(C_{1j}=(-1)^{1+j}\det A_{1j}\)。每个 \(\det A_{1j}\) 是一个 \((n-1)\times(n-1)\) 行列式,又要用同样的方式展开。记 \(T(n)\) 为用代数余子式法计算 \(n\times n\) 行列式所需的乘法次数,则:
\[
T(1)=0,\quad
T(n)=n\cdot T(n-1)+(n-1)\quad(n\ge2),
\]
理由是:有 \(n\) 个子行列式,每个成本为 \(T(n-1)\);此外,对每个 \(j\),要把 \(a_{1j}\) 与 \(C_{1j}\) 相乘一次,因此有 \(n\) 次乘法,而 \(C_{1j}\) 本身只是符号与子行列式,额外只需 \((n-1)\) 次乘法来生成 \(n\) 个 \(a_{1j}C_{1j}\) 项的积(或更常见的计数是:每一个 \((n-1)\times(n-1)\) 行列式展开又含有 \((n-1)\) 个标量乘法)。经典的估算把递推写成
\[
T(n)=n T(n-1)+n-1,
\]
解得
\[
T(n)=n!\sum_{k=1}^n\frac1k.
\]
因此,用代数余子式公式计算 \(n\times n\) 行列式需要大约
\[
T(n)\sim n!\,\log n
\]
次乘法(加法同阶数量级),随着 \(n\) 增长非常快。很多教材也给出略微不同但同阶的闭式形式,你可以根据书中精确定义采用这一版本
\[
T(n)=n!\left(1+\frac12+\cdots+\frac1n\right).
\]



下面按题号直接给出解答内容,便于你嵌入习题解答。

---

\textbf{7.1}

a) 对。行列式只对方阵定义。

b) 对。两行(或两列)相同则行列式为 0。

c) 错。交换两行(或两列)会改变行列式符号,即
\[
\det B = -\det A.
\]

d) 错。把一行(或一列)乘以标量 \(\alpha\) 时,行列式也乘以 \(\alpha\),即
\[
\det B = \alpha\,\det A.
\]

e) 对。用“某一行的倍数加到另一行”这类初等行变换不会改变行列式。

f) 对。上三角或下三角矩阵的行列式等于对角线元素的乘积。

g) 错。应为
\[
\det(A^T)=\det(A).
\]

h) 对。任意同阶方阵 \(A,B\) 有
\[
\det(AB)=\det(A)\det(B).
\]

i) 对。矩阵可逆当且仅当行列式非零。

j) 对。若 \(A\) 可逆,则
\[
\det(A)\det(A^{-1})=\det(AA^{-1})=\det I = 1,
\]
故 \(\det(A^{-1}) = 1/\det(A)\)。

---

\textbf{7.2}

若 \(A\) 为 \(n\times n\) 矩阵,则:
\[
\det(3A) = 3^n \det A,
\]
因为把每一行都乘以 3,相当于行列式乘以 3 共 \(n\) 次。

\[
\det(-A) = (-1)^n \det A,
\]
因为把每一行都乘以 \(-1\) 共 \(n\) 次。

\[
\det(A^2)=\det(AA)=\det(A)\det(A)=(\det A)^2.
\]

---

\textbf{7.3}

不可能。

若 \(A\) 与 \(A^{-1}\) 的所有元素都是整数,则 \(\det A\) 与 \(\det(A^{-1})\) 都是整数,而且
\[
\det(A)\det(A^{-1})=\det(AA^{-1})=\det I = 1.
\]
因此 \(\det A\) 必须是 \(\pm1\)。\(\det A=3\) 不可能。

---

\textbf{7.4}

设
\[
\vv_1=\begin{pmatrix}x_1\\y_1\end{pmatrix},\quad
\vv_2=\begin{pmatrix}x_2\\y_2\end{pmatrix},\quad
A=\begin{pmatrix}x_1&x_2\\y_1&y_2\end{pmatrix}.
\]

先证明特殊情形 \(\vv_1=(x_1,0)^T\);此时
\[
A=\begin{pmatrix}x_1 & x_2\\ 0 & y_2\end{pmatrix},
\quad
\det A = x_1y_2.
\]
以 \(\vv_1,\vv_2\) 为邻边的平行四边形的底边长为 \(|x_1|\),高为 \(|y_2|\),面积为
\[
S = |x_1|\,|y_2| = |x_1y_2| = |\det A|.
\]

对一般情形 \(\vv_1=(x_1,y_1)^T\)。取一个旋转矩阵
\[
R=\begin{pmatrix}\cos\alpha & -\sin\alpha\\[2pt]\sin\alpha&\cos\alpha\end{pmatrix},
\]
使得 \(R\vv_1=(\tilde x_1,0)^T\)。旋转矩阵是正交矩阵,且
\[
\det R = 1.
\]
设
\[
\tilde A = R A = \bigl[R\vv_1,\ R\vv_2\bigr].
\]
则
\[
\det\tilde A = \det(RA)=\det R\,\det A = \det A.
\]
另一方面,线性变换 \(R\) 是刚体旋转,不改变平行四边形面积,因此由 \(\vv_1,\vv_2\) 张成的平行四边形与由 \(R\vv_1, R\vv_2\) 张成的平行四边形面积相等。

但对 \(\tilde A=[R\vv_1,R\vv_2]\) 已经是前面“第一列水平”的特殊情形,故该面积等于 \(|\det\tilde A|\)。因此原平行四边形面积
\[
S = |\det\tilde A| = |\det A|.
\]

---

\textbf{7.5}

记
\[
D(\vv_1,\vv_2)=\det\begin{pmatrix}\vv_1 & \vv_2\end{pmatrix}.
\]

“若 \(D(\vv_1,\vv_2)>0\),则存在所述旋转矩阵”:

如上题,取旋转矩阵
\[
T_\alpha=
\begin{pmatrix}\cos\alpha & -\sin\alpha\\[2pt]\sin\alpha&\cos\alpha\end{pmatrix},\quad
\det T_\alpha=1,
\]
使得
\[
T_\alpha\vv_1 = \|\vv_1\|\ee_1,
\]
即 \(T_\alpha\vv_1\) 与 \(\ee_1\) 平行且同向。

写
\[
T_\alpha\vv_2 =
\begin{pmatrix}u\\v\end{pmatrix}.
\]
则
\[
D(\vv_1,\vv_2)
=\det\bigl[\vv_1,\vv_2\bigr]
=\det\bigl[T_\alpha\vv_1,T_\alpha\vv_2\bigr]
=\det(T_\alpha)\,\det\bigl[\vv_1,\vv_2\bigr]
=\det\bigl[T_\alpha\vv_1,T_\alpha\vv_2\bigr].
\]
但
\[
\det\bigl[T_\alpha\vv_1,T_\alpha\vv_2\bigr]
=\det\begin{pmatrix}\|\vv_1\| & u\\ 0 & v\end{pmatrix}
=\|\vv_1\|\,v.
\]
于是
\[
D(\vv_1,\vv_2)=\|\vv_1\|\,v.
\]
若 \(D(\vv_1,\vv_2)>0\),因为 \(\|\vv_1\|>0\),必有 \(v>0\),即
\[
T_\alpha\vv_2=\begin{pmatrix}u\\v\end{pmatrix},\quad v>0,
\]
从而 \(T_\alpha\vv_2\) 位于上半平面 \(x_2>0\)。

“若存在所述旋转矩阵,则 \(D(\vv_1,\vv_2)>0\)”:

反过来,若存在旋转矩阵 \(T_\alpha\) 使得 \(T_\alpha\vv_1=\|\vv_1\|\ee_1\) 且 \(T_\alpha\vv_2=(u,v)^T\) 满足 \(v>0\),则
\[
D(\vv_1,\vv_2)
=\det\bigl[T_\alpha\vv_1,T_\alpha\vv_2\bigr]
=\|\vv_1\|\,v>0.
\]

综上,\(D(\vv_1,\vv_2)>0\) 当且仅当存在这样一个旋转矩阵 \(T_\alpha\)。





\end{exer}








\section{第四章答案}

\begin{exer}


\textbf{1.1}

a) 错。特征值可以重复,且可能少于 \(n\) 个不同值,例如恒等算子只有一个特征值 1。  

b) 对。“只有一个特征向量”只能理解为“只有一条特征直线”,若 \(\vv\) 是特征向量,则任意 \(\alpha\neq0\) 有 \(\alpha\vv\) 也是特征向量,因此不是有限个,而是无穷多个。  

c) 对。实平面上的旋转矩阵(转角非 \(\pi k\))没有实特征值。  

d) 错。在复数域上,每个 \(n\times n\) 矩阵都有至少一个复特征值,从而有非零特征向量。  

e) 对。若 \(B=S^{-1}AS\),则
\[
\det(B-\lambda I)=\det(S^{-1})\det(A-\lambda I)\det S,
\]
所以特征多项式相同,特征值(含重数)相同。  

f) 错。若 \(B=S^{-1}AS\),则若 \(A\vv=\lambda\vv\),则
\[
B(S^{-1}\vv)=S^{-1}AS(S^{-1}\vv)=\lambda(S^{-1}\vv),
\]
所以特征向量通过 \(S^{-1}\) 变换,并不相同。  

g) 错。一般若 \(A\vv_1=\lambda_1\vv_1\), \(A\vv_2=\lambda_2\vv_2\),则
\[
A(\vv_1+\vv_2)=\lambda_1\vv_1+\lambda_2\vv_2,
\]
除非 \(\lambda_1=\lambda_2\) 或出现特殊抵消,否则不是某个标量乘以 \(\vv_1+\vv_2\)。  

h) 对。若 \(A\vv_1=A\vv_2=\lambda(\vv_1,\vv_2)\),则
\[
A(\vv_1+\vv_2)=\lambda\vv_1+\lambda\vv_2=\lambda(\vv_1+\vv_2),
\]
所以和(若非零)仍是特征向量。

\medskip

\textbf{1.2}

第一个矩阵
\[
A_1=\begin{pmatrix}4&-5\\2&-3\end{pmatrix}.
\]
特征多项式:
\[
p_{A_1}(\lambda)=\det(A_1-\lambda I)
=\det\begin{pmatrix}4-\lambda&-5\\2&-3-\lambda\end{pmatrix}
=(\lambda-1)^2.
\]
特征值:\(\lambda=1\)(重数 2)。

解 \((A_1-I)\vv=0\):
\[
A_1-I=\begin{pmatrix}3&-5\\2&-4\end{pmatrix}
\sim
\begin{pmatrix}1&-5/3\\0&0\end{pmatrix},
\]
故
\[
\vv=\begin{pmatrix}5\\3\end{pmatrix}t,\quad t\neq0.
\]
任意特征向量为 \((5,3)^T\) 的非零倍数。

\smallskip

第二个矩阵
\[
A_2=\begin{pmatrix}2&1\\-1&4\end{pmatrix}.
\]
特征多项式:
\[
p_{A_2}(\lambda)=\det(A_2-\lambda I)
=\det\begin{pmatrix}2-\lambda&1\\-1&4-\lambda\end{pmatrix}
=(\lambda-3)^2.
\]
特征值:\(\lambda=3\)(重数 2)。

\((A_2-3I)=\begin{pmatrix}-1&1\\-1&1\end{pmatrix}\sim\begin{pmatrix}1&-1\\0&0\end{pmatrix}\),
故
\[
\vv=\begin{pmatrix}1\\1\end{pmatrix}t,\quad t\neq0.
\]

\smallskip

第三个矩阵
\[
A_3=\begin{pmatrix}
1&3&3\\
-3&-5&-3\\
3&3&1
\end{pmatrix}.
\]
计算
\[
p_{A_3}(\lambda)=\det(A_3-\lambda I)
=\det\begin{pmatrix}
1-\lambda&3&3\\
-3&-5-\lambda&-3\\
3&3&1-\lambda
\end{pmatrix}
=-(\lambda+2)^2(\lambda-6).
\]
(可用行列式计算或查原书结果;整体符号不影响根。)

特征值:\(\lambda_1=6\),\(\lambda_2=-2\)(重数 2)。

\(\lambda=6\) 时,
\[
A_3-6I=\begin{pmatrix}
-5&3&3\\
-3&-11&-3\\
3&3&-5
\end{pmatrix}
\sim
\begin{pmatrix}
1&0&1\\0&1&1\\0&0&0
\end{pmatrix},
\]
得解 \(z=-x,\ z=-y\Rightarrow x=y\),取 \(x=1\),得
\[
\vv=\begin{pmatrix}1\\1\\-1\end{pmatrix}t.
\]

\(\lambda=-2\) 时,
\[
A_3+2I=\begin{pmatrix}
3&3&3\\
-3&-3&-3\\
3&3&3
\end{pmatrix}
\sim
\begin{pmatrix}
1&1&1\\0&0&0\\0&0&0
\end{pmatrix},
\]
关系 \(x+y+z=0\),特征空间为
\[
\LL\left\{
\begin{pmatrix}1\\-1\\0\end{pmatrix},
\begin{pmatrix}1\\0\\-1\end{pmatrix}
\right\}.
\]

\medskip

\textbf{1.3}

\[
R_\alpha=
\begin{pmatrix}
\cos\alpha&-\sin\alpha\\
\sin\alpha&\cos\alpha
\end{pmatrix}.
\]
特征多项式:
\[
\det(R_\alpha-\lambda I)
=
\det\begin{pmatrix}
\cos\alpha-\lambda&-\sin\alpha\\
\sin\alpha&\cos\alpha-\lambda
\end{pmatrix}
=(\cos\alpha-\lambda)^2+\sin^2\alpha
=\lambda^2-2\cos\alpha\,\lambda+1.
\]
特征值:
\[
\lambda=\cos\alpha\pm i\sin\alpha = e^{\pm i\alpha}.
\]

对 \(\lambda=e^{i\alpha}\),解
\[
\begin{pmatrix}
\cos\alpha-\lambda&-\sin\alpha\\
\sin\alpha&\cos\alpha-\lambda
\end{pmatrix}
\begin{pmatrix}x\\yy\end{pmatrix}=0.
\]
代入 \(\lambda=\cos\alpha+i\sin\alpha\),第一行为
\[
(-i\sin\alpha)x-\sin\alpha\,y=0
\quad\Rightarrow\quad
y=-ix.
\]
故可取特征向量
\[
\vv_+=\begin{pmatrix}1\\-i\end{pmatrix}.
\]
类似得对 \(\lambda=e^{-i\alpha}\) 的特征向量
\[
\vv_-=\begin{pmatrix}1\\i\end{pmatrix}.
\]

\medskip

\textbf{1.4}

所有矩阵都是三角矩阵(上或下),故特征多项式是对角线元素减 \(\lambda\) 之积。

\smallskip

第一个:
\[
A=\begin{pmatrix}
1&2&5&67\\
0&2&3&6\\
0&0&-2&5\\
0&0&0&3
\end{pmatrix},
\]
\[
p_A(\lambda)=\det(A-\lambda I)
=(1-\lambda)(2-\lambda)(-2-\lambda)(3-\lambda).
\]
特征值:\(1,2,-2,3\)。

\smallskip

第二个:
\[
B=\begin{pmatrix}
2&1&0&2\\
0&\pi&43&2\\
0&0&16&1\\
0&0&0&54
\end{pmatrix},
\]
\[
p_B(\lambda)=(2-\lambda)(\pi-\lambda)(16-\lambda)(54-\lambda),
\]
特征值:\(2,\pi,16,54\)。

\smallskip

第三个:
\[
C=\begin{pmatrix}
4&0&0&0\\
1&3&0&0\\
2&4&e&0\\
3&3&1&1
\end{pmatrix}
\]
是下三角矩阵,故
\[
p_C(\lambda)=(4-\lambda)(3-\lambda)(e-\lambda)(1-\lambda),
\]
特征值:\(4,3,e,1\)。

\smallskip

第四个:
\[
D=\begin{pmatrix}
4&0&0&0\\
1&0&0&0\\
2&4&0&0\\
3&3&1&1
\end{pmatrix}
\]
下三角,对角线为 \(4,0,0,1\),
\[
p_D(\lambda)=(4-\lambda)(0-\lambda)(0-\lambda)(1-\lambda)
=(4-\lambda)(-\lambda)^2(1-\lambda),
\]
特征值:\(4,0,0,1\)。

\medskip

\textbf{1.5}

设 \(T\) 为上三角矩阵,元素 \(t_{ij}\);则
\[
T-\lambda I
\]
仍然是上三角矩阵,其对角线元素为 \(t_{11}-\lambda,\dots,t_{nn}-\lambda\)。三角矩阵的行列式等于对角线元素之积,因此
\[
p_T(\lambda)=\det(T-\lambda I)
=(t_{11}-\lambda)(t_{22}-\lambda)\dots(t_{nn}-\lambda).
\]
故特征值正是对角线元素 \(t_{11},\dots,t_{nn}\),计入重数。同理对下三角矩阵成立。

\medskip

\textbf{1.6}

若 \(A\) 幂零,则存在 \(k\in\mathbb N\) 使 \(A^k=\oo\)。

若 \(\lambda\) 是 \(A\) 的特征值,\(\vv\neq0\) 为对应特征向量,则
\[
A\vv=\lambda\vv.
\]
两边连乘 \(k\) 次:
\[
A^k\vv=\lambda^k\vv.
\]
左侧 \(A^k=\oo\),故 \(A^k\vv=0\)。于是
\[
0=A^k\vv=\lambda^k\vv.
\]
由于 \(\vv\neq0\),得到 \(\lambda^k=0\Rightarrow\lambda=0\)。所以唯一可能的特征值是 0,故 \(\sigma(A)=\{0\}\)。

\medskip

\textbf{1.7}

令
\[
M(\lambda)=
\begin{pmatrix}
A-\lambda I & *\\
\oo & B-\lambda I
\end{pmatrix}.
\]
这是一个分块上三角矩阵。对分块上三角矩阵,行列式等于对角块行列式的乘积(第 3 章习题 3.11):
\[
\det M(\lambda)=\det(A-\lambda I)\,\det(B-\lambda I).
\]
而
\[
\det M(\lambda)=\det\bigl(\begin{pmatrix}A&*\\0&B\end{pmatrix}-\lambda I\bigr)
\]
正是给定分块矩阵的特征多项式,故所求证。

\medskip

\textbf{1.8}

在基 \(\{\vv_1,\dots,\vv_n\}\) 下,设 \(A\vv_j=\sum_{i=1}^n a_{ij}\vv_i\),则 \([A]_{\mathcal B}=(a_{ij})\)。

对 \(1\le j\le k\),已知 \(A\vv_j=\lambda\vv_j\),即
\[
A\vv_j=\lambda\vv_j
=\lambda\sum_{i=1}^n \delta_{ij}\vv_i
=\sum_{i=1}^n (\lambda\delta_{ij})\vv_i.
\]
按基展开的唯一性得到
\[
a_{ij}=\lambda\delta_{ij}\quad(1\le i\le n,\ 1\le j\le k).
\]
尤其当 \(1\le i\le k\) 时,若 \(i\ne j\) 则 \(a_{ij}=0\),若 \(i=j\) 则 \(a_{jj}=\lambda\);当 \(i>k\) 且 \(1\le j\le k\) 时,\(\delta_{ij}=0\),故 \(a_{ij}=0\)。因此矩阵的左上 \(k\times k\) 块是 \(\lambda I_k\),左下 \((n-k)\times k\) 块为零块。右上与右下块没有额外限制,可分别记为 \(*\) 与 \(B\)。于是
\[
[A]_{\mathcal B}=
\begin{pmatrix}
\lambda I_k & *\\
\oo & B
\end{pmatrix}.
\]

\medskip

\textbf{1.9}

设特征值 \(\lambda\) 的代数重数为 \(m\)(即在特征多项式中 \((\lambda-\lambda_0)^m\) 的次数),几何重数为
\[
g=\dim\text{Ker }(A-\lambda I).
\]
选取 \(\text{Ker }(A-\lambda I)\) 的一组基 \(\vv_1,\dots,\vv_g\);将其补成 \(V\) 的一组基 \(\vv_1,\dots,\vv_n\)。对这些基向量,前 \(g\) 个都是特征向量,对应同一特征值 \(\lambda\)。由 1.8,矩阵 \([A]_{\mathcal B}\) 在该基下具有分块形式
\[
[A]_{\mathcal B}=
\begin{pmatrix}
\lambda I_g & *\\
\oo & B
\end{pmatrix}.
\]
由 1.7,特征多项式
\[
p_A(t)=\det(A-tI)
=\det(\lambda I_g-tI_g)\,\det(B-tI)
=(\lambda-t)^g\,\det(B-tI).
\]
因此在特征多项式中,\((\lambda-t)\) 至少以 \(g\) 次出现,即代数重数 \(m\ge g\)。这就证明了几何重数不超过代数重数。

\medskip

\textbf{1.10}

设 \(A\) 的特征值(计重数)为 \(\lambda_1,\dots,\lambda_n\)。根据题中提示,先证明
\[
\det(A-\lambda I)=\prod_{j=1}^n(\lambda_j-\lambda).
\]
这在 \(A\) 可对角化时可直接从相似变换得到;一般情形可用上、下三角化或最小多项式分解论证,但本书前文已给出:特征多项式分解成一次因式的积,其根为特征值并计重数。

令
\[
p(\lambda)=\det(A-\lambda I).
\]
作为关于 \(\lambda\) 的多项式,它是 \(n\) 次的,其最高次项来自 \(-\lambda I\) 部分,为
\[
p(\lambda)=(-1)^n\lambda^n + \text{(低次项)}.
\]
另一方面
\[
p(\lambda)=\prod_{j=1}^n(\lambda_j-\lambda)
=(-1)^n\prod_{j=1}^n(\lambda-\lambda_j),
\]
同样是首项 \((-1)^n\lambda^n\),且常数项(不含 \(\lambda\) 的那一项)为
\[
p(0)=\det(A-0\cdot I)=\det A
=\prod_{j=1}^n\lambda_j.
\]
因此行列式等于特征值的乘积(计重数)。

\medskip

\textbf{1.11}

第一步:展开
\[
\prod_{j=1}^n(\lambda_j-\lambda)
=(-1)^n\left(\lambda^n - (\lambda_1+\dots+\lambda_n)\lambda^{n-1}+\dots\right),
\]
因此右侧 \((\lambda_1-\lambda)\dots(\lambda_n-\lambda)\) 作为关于 \(\lambda\) 的多项式,其 \(\lambda^{n-1}\) 的系数为
\[
(-1)^{n-1}(\lambda_1+\dots+\lambda_n).
\]

第二步:把
\[
\det(A-\lambda I)
\]
按行列式定义展开。每一项是从每行选一个元素相乘得到的乘积之和。若要得到 \(\lambda^{n-1}\) 项,必须恰好从 \(n-1\) 个不同行中选到对角元 \(-\lambda\),而从剩下的 1 行选到该行的某个非对角元素 \(a_{ii}\);但包含两个或以上非对角元的乘积中,\(\lambda\) 的次数最多为 \(n-2\),因此可写成
\[
\det(A-\lambda I)=(a_{11}-\lambda)\dots(a_{nn}-\lambda)+q(\lambda),
\]
其中 \(q(\lambda)\) 的次数至多为 \(n-2\)(因为它对应的项至少含两个非对角元,从而 \(-\lambda\) 的个数至多为 \(n-2\))。

第三步:比较两种表示中 \(\lambda^{n-1}\) 的系数。

一方面,由上式
\[
(a_{11}-\lambda)\dots(a_{nn}-\lambda)
=(-1)^n\left(\lambda^n-(a_{11}+\dots+a_{nn})\lambda^{n-1}+\dots\right),
\]
而 \(q(\lambda)\) 次数不超过 \(n-2\),不会影响 \(\lambda^{n-1}\) 项,因此
\[
\det(A-\lambda I) \text{ 的 }\lambda^{n-1}\text{ 系数 } =(-1)^{n-1}(a_{11}+\dots+a_{nn})
=(-1)^{n-1}\operatorname{trace}A.
\]

另一方面,由 1.10 的结果,
\[
\det(A-\lambda I)=\prod_{j=1}^n(\lambda_j-\lambda),
\]
其 \(\lambda^{n-1}\) 的系数为
\[
(-1)^{n-1}(\lambda_1+\dots+\lambda_n).
\]

比较两侧 \(\lambda^{n-1}\) 的系数,得到
\[
\operatorname{trace}A=a_{11}+\dots+a_{nn}=\lambda_1+\dots+\lambda_n.
\]
这就证明了迹等于特征值之和(计重数)。


下面只给习题解答内容,便于直接放入你的解答册(不加额外环境)。

\medskip

\textbf{2.1}

a) 对。  
特征多项式
\[
p_A(\lambda)=\det(A-\lambda I)=\det\bigl((A-\lambda I)^T\bigr)
=\det(A^T-\lambda I)=p_{A^T}(\lambda),
\]
故 \(A\) 与 \(A^T\) 有相同的特征值(连同代数重数)。

b) 错。  
一般没有 \(Av=\lambda v\Rightarrow A^T v=\lambda v\)。  
例如
\[
A=\begin{pmatrix}0&1\\0&0\end{pmatrix}
\]
的唯一特征值为 0,任意 \((x,0)^T\neq0\) 是 \(A\) 的特征向量,但
\[
A^T=\begin{pmatrix}0&0\\1&0\end{pmatrix}
\]
的特征向量是形如 \((0,y)^T\),二者不相同。

c) 对。  
若 \(A\) 可对角化,则存在可逆 \(S\) 使
\[
A=SDS^{-1},\quad D=\text{diag }(\lambda_1,\dots,\lambda_n).
\]
则
\[
A^T=(SDS^{-1})^T=(S^{-1})^T D^T S^T=(S^T)^{-1} D S^T,
\]
而 \(D^T=D\) 仍是对角矩阵,所以 \(A^T\) 也被对角化。

\medskip

\textbf{2.2}

已知 \(A\) 为实矩阵,\(A v=\lambda v\)。对等式两边取复共轭:
\[
\overline{A v}=\overline{\lambda v}.
\]
左边因 \(A\) 实,\(\overline{A v}=A\,\overline v\);右边等于 \(\bar\lambda\,\overline v\)。于是
\[
A\,\overline v=\bar\lambda\,\overline v,
\]
说明 \(\bar\lambda\) 是 \(A\) 的特征值,\(\overline v\) 是对应特征向量。

\medskip

\textbf{2.3}

\[
A=\begin{pmatrix}4&3\\1&2\end{pmatrix}.
\]
特征多项式:
\[
\det(A-\lambda I)=\det\begin{pmatrix}4-\lambda&3\\1&2-\lambda\end{pmatrix}
=(\lambda-1)(\lambda-5),
\]
特征值 \(\lambda_1=5,\lambda_2=1\)。

\(\lambda=5\) 时:
\[
A-5I=\begin{pmatrix}-1&3\\1&-3\end{pmatrix}\Rightarrow v_1=\begin{pmatrix}3\\1\end{pmatrix}.
\]
\(\lambda=1\) 时:
\[
A-I=\begin{pmatrix}3&3\\1&1\end{pmatrix}\Rightarrow v_2=\begin{pmatrix}1\\-1\end{pmatrix}.
\]

令
\[
S=\begin{pmatrix}3&1\\1&-1\end{pmatrix},\quad
D=\text{diag }(5,1),
\]
则 \(A=SDS^{-1}\)。计算
\[
S^{-1}=\frac1{\det S}\begin{pmatrix}-1&-1\\-1&3\end{pmatrix}
=-\tfrac14\begin{pmatrix}-1&-1\\-1&3\end{pmatrix}
=\tfrac14\begin{pmatrix}1&1\\1&-3\end{pmatrix}.
\]
于是
\[
A^{2004}=SD^{2004}S^{-1}
= S\begin{pmatrix}5^{2004}&0\\0&1\end{pmatrix}S^{-1}.
\]
直接乘得
\[
A^{2004}
=\frac14\begin{pmatrix}
3&1\\[1mm]1&-1
\end{pmatrix}
\begin{pmatrix}
5^{2004}&0\\[1mm]0&1
\end{pmatrix}
\begin{pmatrix}
1&1\\[1mm]1&-3
\end{pmatrix}
=
\frac14\begin{pmatrix}
3\cdot5^{2004}&5^{2004}\\[1mm]
5^{2004}&5^{2004}
\end{pmatrix}
\begin{pmatrix}
1&1\\[1mm]1&-3
\end{pmatrix}
\]
\[
=\frac14
\begin{pmatrix}
(4\cdot5^{2004})&( -2\cdot5^{2004})\\[1mm]
(2\cdot5^{2004})&(-2\cdot5^{2004})
\end{pmatrix}
=
\begin{pmatrix}
5^{2004}&-2\cdot5^{2004-1}\\[1mm]
2\cdot5^{2004-1}&-2\cdot5^{2004-1}
\end{pmatrix}.
\]
或保持乘积形式写作
\[
A^{2004}
=\frac14
\begin{pmatrix}
3&1\\1&-1
\end{pmatrix}
\begin{pmatrix}
5^{2004}&0\\0&1
\end{pmatrix}
\begin{pmatrix}
1&1\\1&-3
\end{pmatrix}.
\]

\medskip

\textbf{2.4}

设
\[
A\begin{pmatrix}1\\2\end{pmatrix}=
1\begin{pmatrix}1\\2\end{pmatrix},\quad
A\begin{pmatrix}1\\1\end{pmatrix}=
3\begin{pmatrix}1\\1\end{pmatrix}.
\]
取
\[
S=\begin{pmatrix}1&1\\2&1\end{pmatrix},\quad
D=\text{diag }(1,3),
\]
则 \(A=SDS^{-1}\)。计算
\[
S^{-1}=\frac1{-1}\begin{pmatrix}1&-1\\-2&1\end{pmatrix}
=\begin{pmatrix}-1&1\\2&-1\end{pmatrix}.
\]
于是
\[
A
= S D S^{-1}
=\begin{pmatrix}1&1\\2&1\end{pmatrix}
\begin{pmatrix}1&0\\0&3\end{pmatrix}
\begin{pmatrix}-1&1\\2&-1\end{pmatrix}
=
\begin{pmatrix}1&3\\2&5\end{pmatrix}.
\]
若在给定特征值下,特征向量仅指定到一条直线(可乘任意非零标量),则所有满足条件的矩阵是相同的:它们必须在基 \(\{(1,2)^T,(1,1)^T\}\) 下的矩阵为 \(\text{diag }(1,3)\),这在换回标准基时给出唯一的 \(A\)。因此这样的矩阵是唯一的。

\medskip

\textbf{2.5}

a)
\[
A=\begin{pmatrix}4&-2\\1&1\end{pmatrix}.
\]
特征多项式:
\[
\det(A-\lambda I)
=\det\begin{pmatrix}4-\lambda&-2\\1&1-\lambda\end{pmatrix}
=(\lambda-2)^2.
\]
唯一特征值 \(\lambda=2\)。  
\[
A-2I=\begin{pmatrix}2&-2\\1&-1\end{pmatrix}
\sim\begin{pmatrix}1&-1\\0&0\end{pmatrix},
\]
特征空间为 \(\LL\{(1,1)^T\}\),维数 1,小于代数重数 2,因此矩阵不能对角化。

\smallskip

b)
\[
A=\begin{pmatrix}-1&-1\\6&4\end{pmatrix}.
\]
特征多项式:
\[
\det(A-\lambda I)
=\det\begin{pmatrix}-1-\lambda&-1\\6&4-\lambda\end{pmatrix}
=(\lambda-2)^2.
\]
唯一特征值 \(\lambda=2\)。  
\[
A-2I=\begin{pmatrix}-3&-1\\6&2\end{pmatrix}
\sim\begin{pmatrix}3&1\\0&0\end{pmatrix},
\]
特征空间为 \(\LL\{(1,-3)^T\}\),维数 1,小于代数重数 2,故也不可对角化。

\smallskip

c)
\[
A=\begin{pmatrix}-2&2&6\\5&1&-6\\-5&2&9\end{pmatrix},\quad
\lambda=2\ \text{是特征值}.
\]
直接计算特征多项式(或用题中提示,已知一个根)可得
\[
\det(A-\lambda I)=-(\lambda-2)^2(\lambda-4),
\]
故特征值为 \(\lambda=2\)(代数重数 2)和 \(\lambda=4\)。

\(\lambda=4\) 时:
\[
A-4I=\begin{pmatrix}-6&2&6\\5&-3&-6\\-5&2&5\end{pmatrix}
\sim
\begin{pmatrix}1&0&1\\0&1&1\\0&0&0\end{pmatrix},
\]
得 \(z=-x,\ z=-y\Rightarrow x=y\),特征空间
\[
E_4=\LL\{(1,1,-1)^T\}.
\]

\(\lambda=2\) 时:
\[
A-2I=\begin{pmatrix}-4&2&6\\5&-1&-6\\-5&2&7\end{pmatrix}
\sim
\begin{pmatrix}1&0&1\\0&1&-1\\0&0&0\end{pmatrix},
\]
得 \(z=-x,\ z=y\),特征空间
\[
E_2=\LL\{(1,-1,-1)^T,\ (0,1,1)^T\},
\]
维数 2,等于代数重数,因此 \(A\) 可对角化。

取
\[
S=\bigl[v_1\ v_2\ v_3\bigr]
=\begin{pmatrix}
1&0&1\\[1mm]
-1&1&1\\[1mm]
-1&1&-1
\end{pmatrix},
\quad
D=\text{diag }(2,2,4),
\]
其中前两列是 \(\lambda=2\) 的特征向量,第三列是 \(\lambda=4\) 的特征向量,则
\[
A= SDS^{-1}
\]
即为其对角化。

\medskip

\textbf{2.6}

\[
A=\begin{pmatrix}2&6&-6\\0&5&-2\\0&0&4\end{pmatrix}.
\]

a) 它是上三角矩阵,特征值就是对角线元素:
\[
\lambda_1=2,\quad\lambda_2=5,\quad\lambda_3=4.
\]
因此不需计算行列式即可读出。

b) 三个特征值互不相同,根据定理 2.3,不同特征值对应的特征向量线性无关,因此得到 3 个线性无关特征向量,矩阵必然可对角化;同样无需具体计算。

c) 仍给出一个显式对角化。  
\(\lambda=2\):
\[
A-2I=\begin{pmatrix}0&6&-6\\0&3&-2\\0&0&2\end{pmatrix}
\Rightarrow 2z=0\Rightarrow z=0,\ 3y=0\Rightarrow y=0,
\]
故 \(v_1=(1,0,0)^T\)。

\(\lambda=5\):
\[
A-5I=\begin{pmatrix}-3&6&-6\\0&0&-2\\0&0&-1\end{pmatrix}
\Rightarrow z=0,\ -3x+6y=0\Rightarrow x=2y,
\]
故 \(v_2=(2,1,0)^T\)。

\(\lambda=4\):
\[
A-4I=\begin{pmatrix}-2&6&-6\\0&1&-2\\0&0&0\end{pmatrix}
\Rightarrow y=2z,\ -2x+6y-6z=0\Rightarrow -2x+6z=0\Rightarrow x=3z,
\]
故 \(v_3=(3,2,1)^T\)。

令
\[
S=\begin{pmatrix}1&2&3\\0&1&2\\0&0&1\end{pmatrix},
\quad
D=\text{diag }(2,5,4),
\]
则
\[
A=SDS^{-1}.
\]

\medskip

\textbf{2.7}

\[
A=\begin{pmatrix}2&0&6\\0&2&4\\0&0&4\end{pmatrix}.
\]
上三角,特征值为 \(2,2,4\)。

\(\lambda=2\):
\[
A-2I=\begin{pmatrix}0&0&6\\0&0&4\\0&0&2\end{pmatrix}
\Rightarrow 2z=0\Rightarrow z=0,\ x,y\ \text{任意},
\]
故
\[
E_2=\LL\{(1,0,0)^T,\ (0,1,0)^T\}.
\]

\(\lambda=4\):
\[
A-4I=\begin{pmatrix}-2&0&6\\0&-2&4\\0&0&0\end{pmatrix}
\Rightarrow -2x+6z=0,\ -2y+4z=0
\Rightarrow x=3z,\ y=2z,
\]
故
\[
E_4=\LL\{(3,2,1)^T\}.
\]

两个特征子空间维数之和 \(2+1=3\),等于矩阵大小,故可对角化。  
取
\[
S=\begin{pmatrix}
1&0&3\\
0&1&2\\
0&0&1
\end{pmatrix},
\quad
D=\text{diag }(2,2,4),
\]
则 \(A=SDS^{-1}\)。

\medskip

\textbf{2.8}

\[
A=\begin{pmatrix}5&2\\-3&0\end{pmatrix}.
\]
特征多项式:
\[
\det(A-\lambda I)=\det\begin{pmatrix}5-\lambda&2\\-3&-\lambda\end{pmatrix}
=\lambda^2-5\lambda+6=(\lambda-2)(\lambda-3).
\]
特征值:\(2,3\)。

\(\lambda=2\):\; \(A-2I=\begin{pmatrix}3&2\\-3&-2\end{pmatrix}\Rightarrow v_1=(2,-3)^T\).  
\(\lambda=3\):\; \(A-3I=\begin{pmatrix}2&2\\-3&-3\end{pmatrix}\Rightarrow v_2=(1,-1)^T\).

取
\[
S=\begin{pmatrix}2&1\\-3&-1\end{pmatrix},\quad
D=\text{diag }(2,3),
\]
则 \(A=SDS^{-1}\)。

若 \(B^2=A\),则设
\[
B=S E S^{-1},
\]
其中 \(E\) 为对角矩阵,则
\[
B^2=SES^{-1}SES^{-1}=SE^2S^{-1}=A=SDS^{-1}
\Rightarrow E^2=D.
\]
所以 \(E\) 必为 \(D\) 的一个平方根,即
\[
E=\text{diag }(\varepsilon_1\sqrt2,\ \varepsilon_2\sqrt3),
\quad \varepsilon_1,\varepsilon_2\in\{1,-1\}.
\]
因此所有平方根为
\[
B=S
\begin{pmatrix}
\varepsilon_1\sqrt2&0\\[1mm]0&\varepsilon_2\sqrt3
\end{pmatrix}
S^{-1},
\quad \varepsilon_1,\varepsilon_2=\pm1.
\]

\medskip

\textbf{2.9}

a)  
由
\[
\phi_{n+2}=\phi_{n+1}+\phi_n,\quad
\phi_{n+1}=\phi_{n+1},
\]
我们要
\[
\begin{pmatrix}\phi_{n+2}\\\phi_{n+1}\end{pmatrix}
= A\begin{pmatrix}\phi_{n+1}\\\phi_n\end{pmatrix}.
\]
比较得
\[
A=\begin{pmatrix}1&1\\1&0\end{pmatrix}.
\]

b)  
对角化 \(A\)。特征多项式:
\[
\det(A-\lambda I)
=\det\begin{pmatrix}1-\lambda&1\\1&-\lambda\end{pmatrix}
=\lambda^2-\lambda-1,
\]
根为
\[
\lambda_{1,2}
=\frac{1\pm\sqrt5}{2}
=: \varphi,\ \psi.
\]
\(\lambda=\varphi\) 时,
\((A-\varphi I)\begin{pmatrix}x\\yy\end{pmatrix}=0\) 给出 \(y=(\varphi-1)x\),而 \(\varphi^2=\varphi+1\Rightarrow\varphi-1=1/\varphi\),故可取
\[
v_1=\begin{pmatrix}\varphi\\1\end{pmatrix}.
\]
类似得到 \(\lambda=\psi\) 的特征向量
\[
v_2=\begin{pmatrix}\psi\\1\end{pmatrix}.
\]

令
\[
S=\begin{pmatrix}\varphi&\psi\\1&1\end{pmatrix},\quad
D=\text{diag }(\varphi,\psi),
\]
则 \(A=SDS^{-1}\),从而
\[
A^n=SD^nS^{-1}
=S\begin{pmatrix}\varphi^n&0\\0&\psi^n\end{pmatrix}S^{-1}.
\]

c)  
\[
\begin{pmatrix}\phi_{n+1}\\\phi_n\end{pmatrix}
=A^n\begin{pmatrix}\phi_1\\\phi_0\end{pmatrix}
=A^n\begin{pmatrix}1\\0\end{pmatrix}.
\]
记
\[
S^{-1}=\frac1{\varphi-\psi}
\begin{pmatrix}1&-\psi\\-1&\varphi\end{pmatrix},
\]
(易验证 \(\det S=\varphi-\psi=\sqrt5\)),则
\[
\begin{pmatrix}\phi_{n+1}\\\phi_n\end{pmatrix}
=SD^nS^{-1}\begin{pmatrix}1\\0\end{pmatrix}
=\frac1{\varphi-\psi}S
\begin{pmatrix}\varphi^n\\-\psi^n\end{pmatrix}
=\frac1{\sqrt5}
\begin{pmatrix}
\varphi^{n+1}-\psi^{n+1}\\[1mm]
\varphi^{n}-\psi^{n}
\end{pmatrix}.
\]
从第二个分量读出
\[
\phi_n=\frac{\varphi^n-\psi^n}{\sqrt5},
\]
这就是 Binet 公式。

d)  
\[
\begin{pmatrix}\phi_{n+1}\\\phi_n\end{pmatrix}
=A^n\begin{pmatrix}1\\0\end{pmatrix}
=\frac1{\sqrt5}
\begin{pmatrix}\varphi^{n+1}-\psi^{n+1}\\\varphi^{n}-\psi^{n}\end{pmatrix}.
\]
于是
\[
\frac{\phi_{n+1}}{\phi_n}
=\frac{\varphi^{n+1}-\psi^{n+1}}{\varphi^{n}-\psi^{n}}
=\frac{\varphi^{n}( \varphi)-\psi^{n}(\psi)}{\varphi^{n}-\psi^{n}}
\to\varphi
\quad (n\to\infty),
\]
因为 \(|\psi|<1<\varphi\),\(\psi^n\to0\)。因此向量
\[
\begin{pmatrix}\frac{\phi_{n+1}}{\phi_n}\\1\end{pmatrix}
\to
\begin{pmatrix}\varphi\\1\end{pmatrix}=v_1,
\]
而 \(v_1\) 正是 \(A\) 的一个特征向量(对应特征值 \(\varphi\))。这不是巧合:对任一具有主导特征值(模长最大的特征值)且对应特征空间一维的矩阵,\(A^n v\) 在归一化或作商之后都会趋向于主特征向量方向,这正是幂法的思想。

\medskip

\textbf{2.10}

已知 \(A\) 为 \(5\times5\) 矩阵,有 3 个不同特征值,且其中一个特征子空间维数为 3。记该特征值为 \(\lambda_1\),对应特征子空间 \(E_{\lambda_1}\) 维数 3;另外两个特征值为 \(\lambda_2,\lambda_3\),其几何重数至少为 1。于是
\[
\dim E_{\lambda_1}+\dim E_{\lambda_2}+\dim E_{\lambda_3}
\ge 3+1+1=5.
\]
另一方面,总和不能超过 5,所以实际上等号成立,每个几何重数都等于其代数重数,所有特征子空间之和给出 5 个线性无关特征向量,因此 \(A\) 一定可对角化。

\medskip

\textbf{2.11}

一个标准例子是
\[
J=\begin{pmatrix}1&1&0\\0&1&0\\0&0&2\end{pmatrix}.
\]
其特征多项式 \((\lambda-1)^2(\lambda-2)\),特征值 1 的几何重数为 1(\(\text{Ker }(J-I)\) 一维),代数重数为 2,因此 \(J\) 不能对角化。

要“通用化”这个矩阵,可对其做相似变换:取任何可逆矩阵 \(S\),令
\[
A= SJS^{-1}.
\]
那么 \(A\) 与 \(J\) 具有同样的特征值与 Jordan 结构,因此也不可对角化。通过适当选择 \(S\),可以让 \(A\) 看起来并无明显的上三角或 Jordan 形特征。

\medskip

\textbf{2.12}

若 \(A\neq0\) 且 \(A^5=0\),假设 \(A\) 可对角化,则存在可逆 \(S\) 使
\[
A=SDS^{-1},
\]
其中 \(D\) 为对角矩阵,其对角元素为 \(A\) 的特征值。于是
\[
A^5=SD^5S^{-1}=0
\Rightarrow D^5=0.
\]
但对角矩阵 \(D^5\) 的对角线是各特征值的五次方,要使 \(D^5=0\),每个特征值都必须为 0,因此 \(D=0\),进而
\[
A=SDS^{-1}=0,
\]
与 \(A\neq0\) 矛盾。因此 \(A\) 不可对角化。

更一般,若 \(A^N=0\) 对某个 \(N\ge1\) 成立且 \(A\neq0\),同样的论证给出 \(A\) 不可对角化。也就是说,任何非零幂零矩阵都不能对角化。

\medskip

\textbf{2.13}

a) 设 \(T:M_{2\times2}\to M_{2\times2}\) 定义为 \(T(A)=A^T\)。求所有 \(\lambda\) 与 \(A\neq0\) 使
\[
T(A)=\lambda A\quad\Longleftrightarrow\quad A^T=\lambda A.
\]

写
\[
A=\begin{pmatrix}a&b\\c&d\end{pmatrix},
\quad
A^T=\begin{pmatrix}a&c\\b&d\end{pmatrix}.
\]
方程 \(A^T=\lambda A\) 变成
\[
\begin{cases}
a=\lambda a,\\
c=\lambda b,\\
b=\lambda c,\\
d=\lambda d.
\end{cases}
\]

若 \(\lambda=1\),则条件为 \(c=b\),其余恒真,对应所有对称矩阵
\[
A=\begin{pmatrix}a&b\\b&d\end{pmatrix},\quad (a,b,d\in\RR),
\]
这是一个 3 维子空间。

若 \(\lambda=-1\),则条件为 \(a=-a\Rightarrow a=0,\ d=-d\Rightarrow d=0,\ c=-b,\ b=-c\),即 \(c=-b\),对应所有反对称矩阵
\[
A=\begin{pmatrix}0&b\\-b&0\end{pmatrix},
\]
这是 1 维子空间。

若 \(\lambda\neq\pm1\),从 \(a=\lambda a\),\(d=\lambda d\) 得 \(a=d=0\); 再由 \(c=\lambda b\) 与 \(b=\lambda c\) 得
\[
b=\lambda c=\lambda(\lambda b)=\lambda^2 b
\Rightarrow (\lambda^2-1)b=0.
\]
因为 \(\lambda^2\ne1\),只能 \(b=0\),从而 \(c=0\),得到 \(A=0\),与特征向量非零矛盾;因此无其他特征值。

故 \(T\) 的全部特征值是 \(\lambda=1,-1\),其特征子空间分别是对称矩阵和反对称矩阵。\(M_{2\times2}\) 可分解为这两个特征子空间的直和,因此 \(T\) 可以被对角化。

b) 在 \(n\times n\) 情形中同样成立。  
对任意 \(A\),可以唯一地写成
\[
A=\frac{A+A^T}{2}+\frac{A-A^T}{2}
=:S+K,
\]
其中 \(S^T=S\) 为对称矩阵,\(K^T=-K\) 为反对称矩阵。易见
\[
T(S)=S,\quad T(K)=-K,
\]
且对称矩阵空间与反对称矩阵空间仅交于 \(\{0\}\),维数之和为 \(n^2\)。因此转置算子在 \(M_{n\times n}\) 上的特征值仍只有 \(1\) 与 \(-1\),对应的特征子空间分别为对称与反对称矩阵空间,它们的直和给出整个空间,所以该算子也可对角化。

\medskip

\textbf{2.14}

若两个子空间 \(V_1,V_2\subset V\) 线性无关,按定义,任何表示
\[
v_1+v_2=0,\quad v_1\in V_1,\ v_2\in V_2
\]
都只有平凡解 \(v_1=v_2=0\)。

\(\Rightarrow\) 方向:  
若 \(V_1,V_2\) 线性无关,而 \(w\in V_1\cap V_2\),则 \(w\in V_1\) 且 \(w\in V_2\)。考虑分解
\[
w+(-w)=0,\quad w\in V_1,\ -w\in V_2.
\]
线性无关性迫使 \(w=0\),故交集只含 0:\(V_1\cap V_2=\{0\}\)。

\(\Leftarrow\) 方向:  
反过来,若 \(V_1\cap V_2=\{0\}\),取任意
\[
v_1+v_2=0,\quad v_1\in V_1,\ v_2\in V_2.
\]
则 \(v_1=-v_2\) 属于 \(V_1\cap V_2\),所以 \(v_1=-v_2=0\)。因此满足线性无关的定义。

故 \(V_1,V_2\) 线性无关当且仅当 \(V_1\cap V_2=\{0\}\)。


\end{exer}








\section{第五章答案}

\begin{exer}


下面只给习题解答内容,便于直接嵌入你的答案稿。

---

\textbf{1.1}

\[
(3+2i)(5-3i)=15-9i+10i-6i^2=21+i.
\]

\[
\frac{2-3i}{1-2i}
=\frac{(2-3i)(1+2i)}{1+4}
=\frac{2+4i-3i-6i^2}{5}
=\frac{8+i}{5}.
\]

\[
\Re\frac{2-3i}{1-2i}=\Re\frac{8+i}{5}=\frac85.
\]

\[
(1+2i)^2=1+4i+4i^2=-3+4i,
\]
\[
(1+2i)^3=(1+2i)(-3+4i)=-3+4i-6i+8i^2=-11-2i.
\]

\[
\Im( (1+2i)^3 ) =-2.
\]

---

\textbf{1.2}

这里在 \(\C^3\) 上取标准内积
\((x,y)=x_1\overline{y_1}+x_2\overline{y_2}+x_3\overline{y_3}\).

\[
x=(1,2i,1+i)^T,\quad y=(i,2-i,3)^T.
\]

(a)  
\[
(x,y)=1\cdot\overline i+2i\cdot\overline{(2-i)}+(1+i)\cdot\overline3
=-i+2i(2+i)+3(1-i)=9+2i.
\]

\[
\|x\|^2=(x,x)=1\cdot1+2i\cdot\overline{2i}+(1+i)\overline{(1+i)}
=1+4+2=7.
\]

\[
\|y\|^2=(y,y)=i\overline i+(2-i)\overline{(2-i)}+3\cdot\overline3
=1+5+9=15,\quad
\|y\|=\sqrt{15}.
\]

(b) 利用线性与共轭线性:
\[
(3x,2iy)=3\,\overline{2i}\,(x,y)=3(-2i)(9+2i)=-54i+12=12-54i.
\]

\[
(2x,ix+2y)
=(2x,ix)+(2x,2y)
=2\overline i\,(x,x)+4(x,y)
=-2i\cdot7+4(9+2i)
=36-6i.
\]

(c)  
\[
\|x+2y\|^2=(x+2y,x+2y)
=\|x\|^2+4(x,y)+4\|y\|^2
=7+4(9+2i)+4\cdot15
=79+8i.
\]

---

\textbf{1.3}

已知 \(\|u\|=2,\ \|v\|=3,\ (u,v)=2+i\),则 \((v,u)=\overline{2+i}=2-i\).

\[
\|u+v\|^2=(u+v,u+v)=\|u\|^2+\|v\|^2+(u,v)+(v,u)
=4+9+(2+i)+(2-i)=17.
\]

\[
\|u-v\|^2=(u-v,u-v)=\|u\|^2+\|v\|^2-(u,v)-(v,u)
=4+9-(2+i)-(2-i)=9.
\]

\[
(u+v,u-iv)=(u,u-iv)+(v,u-iv)
=\|u\|^2-i(v,u)+(v,u)-i\|v\|^2
\]
\[
=4-i(2-i)+(2-i)-9i
=4+(1-2i)+(2-i)-9i
=7-12i.
\]

\[
(u+3iv,4iu)
=(u,4iu)+3i(v,4iu)
=4i(u,u)+12i(v,iu).
\]
\[
(v,iu)=\overline i\,(v,u)=-i(2-i)=1-2i,
\]
故
\[
(u+3iv,4iu)=4i\cdot4+12i(1-2i)=16i+12i-24i^2=24+28i.
\]

---

\textbf{1.4}

\[
\|\xx\pm\yy\|^2=(\xx\pm\yy,\xx\pm\yy)
=(\xx,\xx)\pm(\xx,\yy)\pm(\yy,\xx)+(\yy,\yy)
\]
\[
=\|\xx\|^2+\|\yy\|^2\pm\bigl((\xx,\yy)+\overline{(\xx,\yy)}\bigr)
=\|\xx\|^2+\|\yy\|^2\pm2\Re(\xx,\yy).
\]

---

\textbf{1.5}

(a) \((x,y)=x_1y_1-x_2y_2\) 在 \(\RR^2\) 上:

\[
(x,x)=x_1^2-x_2^2
\]
可以为负,例如 \(x=(0,1)\) 得 \((x,x)=-1<0\)。违反非负性,因此不是内积。

(b) \((A,B)=\operatorname{trace}(A+B)\):

\[
(A,A)=\operatorname{trace}(2A)=2\,\operatorname{trace}A
\]
可能为负,也可能为零而 \(A\neq0\),故既不保证非负,也不保证正定,更不是双线性的(对 \(A,B\) 只是各自线性地加了常数)。所以这不是内积。

(更直接:对任意标量 \(\lambda\),
\((\lambda A,B)=\operatorname{trace}(\lambda A+B)\neq\lambda\,\operatorname{trace}(A+B)\) 一般不成立,违反线性性。)

(c) \((f,g)=\int_0^1 f'(t)\overline{g(t)}\,dt\) 在多项式空间上:

先看对第一个变量的线性与对第二个变量的共轭线性都成立,但对称性失败:
\[
(f,g)=\int_0^1 f'(t)\overline{g(t)}\,dt,\quad
(g,f)=\int_0^1 g'(t)\overline{f(t)}\,dt
\]
一般并不满足 \((f,g)=\overline{(g,f)}\)。

更简单地看正定性:取常数多项式 \(f(t)\equiv1\),则 \(f'\equiv0\),
\[
(f,f)=\int_0^11'\cdot\overline{1}\,dt=0
\]
但 \(f\neq0\)。违反正定性,因此不是内积。

---

\textbf{1.6}

若 \(|(x,y)|=\|x\|\|y\|\)。

若 \(x=0\) 或 \(y=0\),结论显然成立(任一零向量都是对方的倍数)。

假设 \(x,y\neq0\)。在柯西–施瓦茨不等式的证明中,对任意标量 \(\alpha\) 考察
\[
0\le\|x-\alpha y\|^2=(x-\alpha y,x-\alpha y).
\]
在复情形,令
\(\displaystyle \alpha=\frac{(x,y)}{\|y\|^2}\),得到
\[
\|x-\alpha y\|^2
=\|x\|^2-\frac{|(x,y)|^2}{\|y\|^2}.
\]
CS 不等式给出右边非负;若等号成立,
\[
0=\|x\|^2-\frac{|(x,y)|^2}{\|y\|^2}.
\]
题设又给 \(|(x,y)|=\|x\|\|y\|\),代入即得
\(\|x-\alpha y\|^2=0\),故 \(x=\alpha y\),即 \(x\) 是 \(y\) 的倍数。

反过来,若 \(x=\alpha y\),则
\[
(x,y)=(\alpha y,y)=\alpha (y,y)=\alpha\|y\|^2,
\]
于是
\[
|(x,y)|=|\alpha|\|y\|^2=\|\alpha y\|\,\|y\|=\|x\|\|y\|.
\]

---

\textbf{1.7}

直接用 1.4:
\[
\|x+y\|^2=\|x\|^2+\|y\|^2+2\Re(x,y),
\]
\[
\|x-y\|^2=\|x\|^2+\|y\|^2-2\Re(x,y).
\]
相加得
\[
\|x+y\|^2+\|x-y\|^2
=2(\|x\|^2+\|y\|^2).
\]

---

\textbf{1.8}

(a) 若 \((x,v)=0\) 对所有 \(v\in V\) 成立,特别是对 \(v=x\) 成立,于是
\[
(x,x)=0 \Rightarrow x=0
\]
(由内积的正定性),故 \(x=0\)。

(b) 若 \(\{v_1,\dots,v_n\}\) 生成 \(V\),任意 \(v\in V\) 可写
\[
v=\sum_{k=1}^n\alpha_k v_k.
\]
于是
\[
(x,v)=\sum_{k=1}^n\alpha_k(x,v_k)=0
\]
因为题设给出 \((x,v_k)=0\) 对每个 \(k\) 成立。由 (a) 得 \(x=0\)。

(c) 若对所有 \(k\) 有
\((x,v_k)=(y,v_k)\),则对任意
\(v=\sum\alpha_k v_k\) 有
\[
(x,v)=\sum\alpha_k(x,v_k)=\sum\alpha_k(y,v_k)=(y,v).
\]
于是
\[
(x-y,v)=0\quad\forall v\in V.
\]
由 (a) 得 \(x-y=0\),即 \(x=y\)。

---

\textbf{1.9}

在 \(\RR^2\) 上:

\[
\|x\|_1=|x_1|+|x_2|\le1
\Rightarrow B_1
\]
是一个菱形(顶点在 \((\pm1,0),(0,\pm1)\))。

\[
\|x\|_2=\sqrt{x_1^2+x_2^2}\le1
\Rightarrow B_2
\]
是以原点为圆心、半径 1 的圆盘。

\[
\|x\|_{\infty}=\max(|x_1|,|x_2|)\le1
\Rightarrow B_{\infty}
\]
是顶点在 \((\pm1,\pm1)\) 的正方形(轴对齐)。

对其他 \(1<p<\infty\),单位球
\[
B_p=\{(x_1,x_2):|x_1|^p+|x_2|^p\le1\}
\]
的边界是一条光滑的“圆角正方形”:  
当 \(p\to1\) 时趋近菱形,当 \(p\to\infty\) 时趋近方形;\(p=2\) 时恰为圆。


下面只写解答内容,不加额外环境。

\medskip

\textbf{2.1}

设 \(\xx=(x_1,x_2,x_3,x_4)^T\)。正交条件是
\[
\begin{cases}
x_1+x_2+x_3+x_4=0,\\
x_1+2x_2+3x_3+4x_4=0.
\end{cases}
\]
相减得
\[
x_2+2x_3+3x_4=0 \quad\Rightarrow\quad x_2=-2x_3-3x_4.
\]
再由第一式
\[
x_1+x_2+x_3+x_4=0
\Rightarrow x_1+x_3+x_4=2x_3+3x_4
\Rightarrow x_1=x_3+2x_4.
\]
令自由参数 \(s=x_3,\ t=x_4\),则
\[
\xx=(x_1,x_2,x_3,x_4)^T
=s(1,-2,1,0)^T+t(2,-3,0,1)^T.
\]
所以所求集合为
\[
\{\,s(1,-2,1,0)^T+t(2,-3,0,1)^T : s,t\in\RR\,\}.
\]

\medskip

\textbf{2.2}

\(\operatorname{Ran}A^T\subset\FF^n\)。对 \(x\in\FF^n\),
\[
x\perp \operatorname{Ran}A^T
\quad\Longleftrightarrow\quad
\forall y\in\FF^m:\ (x,A^Ty)=0
\quad\Longleftrightarrow\quad
\forall y:\ (Ax,y)=0
\quad\Longleftrightarrow\quad
Ax=0.
\]
因此
\[
(\operatorname{Ran}A^T)^\perp=\text{Ker } A.
\]

类似地,\(\operatorname{Ran}A\subset\FF^m\)。对 \(z\in\FF^m\),
\[
z\perp \operatorname{Ran}A
\quad\Longleftrightarrow\quad
\forall x\in\FF^n:\ (z,Ax)=0
\quad\Longleftrightarrow\quad
\forall x:\ (A^Tz,x)=0
\quad\Longleftrightarrow\quad
A^Tz=0.
\]
因此
\[
(\operatorname{Ran}A)^\perp=\text{Ker } A^T.
\]

\medskip

\textbf{2.3}

设 \(\{v_1,\dots,v_n\}\) 是标准正交基,故
\((v_j,v_k)=\delta_{jk}\).

(a) 令
\(\displaystyle x=\sum_{k=1}^n\alpha_k v_k,\quad y=\sum_{k=1}^n\beta_k v_k\)。则
\[
(x,y)=\Bigl(\sum_{k=1}^n\alpha_k v_k,\;\sum_{j=1}^n\beta_j v_j\Bigr)
=\sum_{k,j}\alpha_k\overline{\beta_j}(v_k,v_j)
=\sum_{k=1}^n\alpha_k\overline{\beta_k}.
\]

(b) 由 (a) 知
\[
\alpha_k=(x,v_k),\quad \beta_k=(y,v_k),
\]
因为
\[
(x,v_k)=\Bigl(\sum_j\alpha_j v_j,v_k\Bigr)
=\sum_j\alpha_j(v_j,v_k)=\alpha_k.
\]
代入 (a) 得
\[
(x,y)=\sum_{k=1}^n(x,v_k)\,\overline{(y,v_k)},
\]
这就是帕塞瓦尔恒等式。

(c) 若 \(\{v_k\}\) 只是正交基,且 \((v_k,v_k)=\|v_k\|^2\neq1\),写
\[
x=\sum_k\alpha_k v_k,\quad
\alpha_k=\frac{(x,v_k)}{\|v_k\|^2},\quad
y=\sum_k\beta_k v_k,\quad
\beta_k=\frac{(y,v_k)}{\|v_k\|^2}.
\]
由 (a) 的同样计算(只是 \((v_j,v_k)=0\) 对 \(j\neq k\))得到
\[
(x,y)=\sum_{k=1}^n\alpha_k\overline{\beta_k}\,\|v_k\|^2
=\sum_{k=1}^n
\frac{(x,v_k)\,\overline{(y,v_k)}}{\|v_k\|^2}.
\]
这就是正交基情形的帕塞瓦尔恒等式。

\medskip

\textbf{2.4}

给定基 \(\{v_1,\dots,v_n\}\),对
\[
x=\sum_{k=1}^n\alpha_k v_k,\quad
y=\sum_{k=1}^n\beta_k v_k,
\]
定义
\[
\langle x,y\rangle:=\sum_{k=1}^n\alpha_k\overline{\beta_k}.
\]

(1) 对第二变量的共轭线性:取标量 \(\lambda,\mu\),
\[
\lambda y_1+\mu y_2
=\sum_k(\lambda\beta_{1k}+\mu\beta_{2k})v_k,
\]
于是
\[
\langle x,\lambda y_1+\mu y_2\rangle
=\sum_k\alpha_k\overline{\lambda\beta_{1k}+\mu\beta_{2k}}
=\overline{\lambda}\sum_k\alpha_k\overline{\beta_{1k}}
+\overline{\mu}\sum_k\alpha_k\overline{\beta_{2k}}
=\overline{\lambda}\langle x,y_1\rangle
+\overline{\mu}\langle x,y_2\rangle.
\]

(2) 对第一个变量的线性完全类似:
\[
\langle \lambda x_1+\mu x_2,y\rangle
=\lambda\langle x_1,y\rangle+\mu\langle x_2,y\rangle.
\]

(3) 共轭对称性:
\[
\overline{\langle x,y\rangle}
=\overline{\sum_k\alpha_k\overline{\beta_k}}
=\sum_k\beta_k\overline{\alpha_k}
=\langle y,x\rangle.
\]

(4) 正定性:
\[
\langle x,x\rangle
=\sum_k|\alpha_k|^2\ge0,
\]
且 \(\langle x,x\rangle=0\) 当且仅当 \(\alpha_k=0\ \forall k\),即 \(x=0\)。

因此 \(\langle\cdot,\cdot\rangle\) 满足内积的全部公理,是 \(V\) 上的一个内积。

\medskip

\textbf{2.5}

\(\operatorname{Ran}A\subset\FF^m\)。向量 \(z\in\FF^m\) 与 \(\operatorname{Ran}A\) 中所有向量正交,当且仅当
\[
\forall x\in\FF^n:\ (z,Ax)=0
\quad\Longleftrightarrow\quad
\forall x:\ (A^T z,x)=0
\quad\Longleftrightarrow\quad
A^Tz=0.
\]
因此
\[
\{\,z\in\FF^m : z\perp \operatorname{Ran}A\,\}
=\text{Ker } A^T.
\]



下面只给各题的计算与结论,格式按你书里的风格来写,不用列表环境。

\medskip

\textbf{3.1}

记
\[
x_1=\begin{pmatrix}1\\2\\-2\end{pmatrix},\quad
x_2=\begin{pmatrix}1\\-1\\4\end{pmatrix},\quad
x_3=\begin{pmatrix}2\\1\\1\end{pmatrix}.
\]

第一步:\(v_1=x_1\).

第二步:
\[
(x_2,v_1)=1\cdot1+(-1)\cdot2+4\cdot(-2)=-11,\quad
\|v_1\|^2=1^2+2^2+(-2)^2=9,
\]
\[
v_2=x_2-\frac{(x_2,v_1)}{\|v_1\|^2}v_1
=x_2+\frac{11}{9}v_1
=\begin{pmatrix}20/9\\13/9\\2/9\end{pmatrix}.
\]

第三步:
\[
(x_3,v_1)=2\cdot1+1\cdot2+1\cdot(-2)=2,\quad
(x_3,v_2)=\frac{1}{9},
\]
\[
v_3=x_3-\frac{(x_3,v_1)}{\|v_1\|^2}v_1-\frac{(x_3,v_2)}{\|v_2\|^2}v_2
=\begin{pmatrix}5/3\\-4/3\\2/3\end{pmatrix}.
\]

因此得到一组正交向量
\[
v_1=\begin{pmatrix}1\\2\\-2\end{pmatrix},\quad
v_2=\begin{pmatrix}20/9\\13/9\\2/9\end{pmatrix},\quad
v_3=\begin{pmatrix}5/3\\-4/3\\2/3\end{pmatrix}.
\]
若需要正交\emph{标准}系,只需再分别除以它们的范数。

\medskip

\textbf{3.2}

设
\[
x_1=\begin{pmatrix}1\\2\\3\end{pmatrix},\quad
x_2=\begin{pmatrix}1\\3\\1\end{pmatrix}.
\]

第一步:\(v_1=x_1\),\(\|v_1\|^2=1^2+2^2+3^2=14\).

第二步:
\[
(x_2,v_1)=1\cdot1+3\cdot2+1\cdot3=10,
\]
\[
v_2=x_2-\frac{(x_2,v_1)}{\|v_1\|^2}v_1
=x_2-\frac{10}{14}v_1
=\begin{pmatrix}2/7\\8/7\\-4/7\end{pmatrix},
\quad
\|v_2\|^2=12/7.
\]

于是得到正交系 \(v_1,v_2\)。到它们张成的二维子空间的正交投影为
\[
P_E x
=\frac{(x,v_1)}{\|v_1\|^2}v_1+\frac{(x,v_2)}{\|v_2\|^2}v_2.
\]
写成矩阵形式:
\[
P_E=\frac{1}{\|v_1\|^2}v_1v_1^T+\frac{1}{\|v_2\|^2}v_2v_2^T
=\frac1{14}
\begin{pmatrix}
1\\2\\3
\end{pmatrix}
\begin{pmatrix}
1&2&3
\end{pmatrix}
+\frac7{12}
\begin{pmatrix}
2/7\\8/7\\-4/7
\end{pmatrix}
\begin{pmatrix}
2/7&8/7&-4/7
\end{pmatrix}.
\]
化简得
\[
P_E=\frac1{42}
\begin{pmatrix}
19 & 17 & 10\\
17 & 53 & -2\\
10 & -2 & 25
\end{pmatrix}.
\]

\medskip

\textbf{3.3}

上题得到两个线性无关、彼此正交的向量 \(v_1,v_2\),它们张成的子空间是二维的,因此在 \(\RR^3\) 中还需添加 \textbf{一个} 与二者都正交的向量,就得到一个正交基。

例如可以取
\[
v_3=v_1\times v_2
=\begin{pmatrix}1\\2\\3\end{pmatrix}\times\begin{pmatrix}2/7\\8/7\\-4/7\end{pmatrix}
=\begin{pmatrix}-4\\2\\1\end{pmatrix},
\]
容易验证 \(v_3\perp v_1,v_2\)。于是 \(\{v_1,v_2,v_3\}\) 是 \(\RR^3\) 的一个正交基。

一般地,在 \(\RR^n\) 或 \(\C^n\) 中,若已有一个由 \(k<n\) 个两两正交非零向量组成的正交系统,可以:
\[
\text{取线性无关组的一个补基,用格拉姆–施密特对补基依次减去到已知系统的投影,得到另外 }n-k\text{ 个与原系统正交的向量。}
\]
这样便把正交系统补全为整个空间的正交基。

\medskip

\textbf{3.4}

设
\[
x=\begin{pmatrix}2\\3\\1\end{pmatrix},\quad
v_1=\begin{pmatrix}1\\2\\3\end{pmatrix},\quad
v_2=\begin{pmatrix}1\\3\\1\end{pmatrix},
\]
使用上题中的正交系 \(v_1,v_2\)。

先算
\[
(x,v_1)=2\cdot1+3\cdot2+1\cdot3=11,\quad
(x,v_2)=2\cdot1+3\cdot3+1\cdot1=12.
\]
于是
\[
P_E x
=\frac{11}{\|v_1\|^2}v_1+\frac{12}{\|v_2\|^2}v_2
=\frac{11}{14}v_1+\frac{7}{4}v_2.
\]
计算得
\[
P_E x=\begin{pmatrix}5/2\\37/4\\17/2\end{pmatrix}.
\]
故
\[
x-P_E x
=\begin{pmatrix}2\\3\\1\end{pmatrix}-\begin{pmatrix}5/2\\37/4\\17/2\end{pmatrix}
=\begin{pmatrix}-1/2\\-25/4\\-15/2\end{pmatrix},
\]
\[
\|x-P_E x\|^2
=\Bigl(-\frac12\Bigr)^2+\Bigl(-\frac{25}{4}\Bigr)^2+\Bigl(-\frac{15}{2}\Bigr)^2
=\frac14+\frac{625}{16}+\frac{225}{4}
=\frac{1669}{16}.
\]
所以距离为
\[
\operatorname{dist}(x,E)
=\|x-P_E x\|
=\frac{\sqrt{1669}}{4}.
\]

\medskip

\textbf{3.5}

设
\[
x=\begin{pmatrix}1\\1\\1\\1\end{pmatrix},\quad
v_1=\begin{pmatrix}1\\3\\1\\1\end{pmatrix},\quad
v_2=\begin{pmatrix}2\\-1\\1\\0\end{pmatrix},\quad v_1\perp v_2.
\]
则到 \(E=\operatorname{span}\{v_1,v_2\}\) 的正交投影为
\[
P_E x
=\frac{(x,v_1)}{\|v_1\|^2}v_1+\frac{(x,v_2)}{\|v_2\|^2}v_2.
\]
计算:
\[
(x,v_1)=1+3+1+1=6,\quad
\|v_1\|^2=1+9+1+1=12,
\]
\[
(x,v_2)=2-1+1+0=2,\quad
\|v_2\|^2=4+1+1+0=6.
\]
于是
\[
P_E x=\frac{6}{12}v_1+\frac{2}{6}v_2
=\frac12 v_1+\frac13 v_2
=\begin{pmatrix}7/6\\7/6\\1\\1/2\end{pmatrix}.
\]

\medskip

\textbf{3.6}

设
\[
x=\begin{pmatrix}1\\2\\3\\4\end{pmatrix},\quad
v_1=\begin{pmatrix}1\\-1\\1\\0\end{pmatrix},\quad
v_2=\begin{pmatrix}1\\2\\1\\1\end{pmatrix},\quad v_1\perp v_2.
\]
仍然只需
\[
\operatorname{dist}(x,E)=\|x-P_E x\|,\quad
P_E x=\frac{(x,v_1)}{\|v_1\|^2}v_1+\frac{(x,v_2)}{\|v_2\|^2}v_2.
\]
这里直接用
\[
\|x\|^2=\|P_E x\|^2+\|x-P_E x\|^2,
\]
因此
\[
\|x-P_E x\|^2=\|x\|^2-\|P_E x\|^2
=\|x\|^2-\left(\frac{|(x,v_1)|^2}{\|v_1\|^2}+\frac{|(x,v_2)|^2}{\|v_2\|^2}\right),
\]
只需内积,不必先算出 \(P_E x\)。

计算:
\[
\|x\|^2=1^2+2^2+3^2+4^2=30,
\]
\[
(x,v_1)=1\cdot1+2\cdot(-1)+3\cdot1+4\cdot0=2,\quad
\|v_1\|^2=1+1+1+0=3,
\]
\[
(x,v_2)=1\cdot1+2\cdot2+3\cdot1+4\cdot1=12,\quad
\|v_2\|^2=1+4+1+1=7.
\]
故
\[
\|x-P_E x\|^2
=30-\left(\frac{2^2}{3}+\frac{12^2}{7}\right)
=30-\left(\frac{4}{3}+\frac{144}{7}\right)
=\frac{190}{21}.
\]
于是
\[
\operatorname{dist}(x,E)=\sqrt{\frac{190}{21}}.
\]
这说明确实可以在不写出投影向量本身的情况下计算距离。

\medskip

\textbf{3.7}

命题:若 \(E\) 为内积空间 \(V\) 的子空间,则
\[
\dim E+\dim E^\perp=\dim V.
\]

证明:上一节已经证明
\[
V=E\oplus E^\perp,
\]
即任意 \(v\in V\) 可唯一写成 \(v=v_1+v_2\) 且 \(v_1\in E\), \(v_2\in E^\perp\),并且 \(E\cap E^\perp=\{0\}\)。对有限维空间,直和分解蕴含
\[
\dim V=\dim E+\dim E^\perp.
\]

\medskip

\textbf{3.8}

设 \(P\) 是到子空间 \(E\) 的正交投影,\(\dim V=n,\ \dim E=r\)。

对任意 \(x\in E\) 有 \(Px=x\),故
\[
Px=x\quad\Longrightarrow\quad x\text{ 是特征值 }1\text{ 的特征向量}.
\]
因此特征值 \(1\) 的特征子空间是 \(E\),几何重数为 \(\dim E=r\)。

另一方面,\(x\in E^\perp\Rightarrow Px=0\),于是
\[
Px=0\quad\Longrightarrow\quad x\text{ 是特征值 }0\text{ 的特征向量},
\]
特征子空间为 \(E^\perp\),几何重数为 \(\dim E^\perp=n-r\)。

特征多项式因此为 \(\lambda^{{n-r}}(\lambda-1)^r\),故特征值和代数重数为:
\[
\lambda=1,\ \text{代数重数 }r;\quad
\lambda=0,\ \text{代数重数 }n-r.
\]

\medskip

\textbf{3.9}

(a) 设 \(u=(1,1,\dots,1)^T\in\RR^n\)。到 \(\operatorname{span}\{u\}\) 的正交投影为
\[
P=\frac{uu^T}{\|u\|^2}
=\frac{1}{n}
\begin{pmatrix}
1\\\vdots\\1
\end{pmatrix}
\begin{pmatrix}
1&\dots&1
\end{pmatrix},
\]
即所有元素都为 \(1/n\) 的 \(n\times n\) 矩阵。

(b) 题中的矩阵 \(A\) 是主对角线为 1,其余元素也为 1 的矩阵,因此
\[
A=I+(\text{全 1 矩阵})=I+nP.
\]
因为 \(P\) 的特征值是 1(一次)和 0(重数 \(n-1\)),故 \(A\) 的特征值为
\[
\lambda_1=1+n\cdot1=n+1,\quad\text{重数 }1;
\]
\[
\lambda_2=1+n\cdot0=1,\quad\text{重数 }n-1.
\]

(c) \(A-I=nP\),所以 \(A-I\) 的特征值为
\[
\mu_1=n\cdot1=n,\quad\text{重数 }1;
\quad
\mu_2=n\cdot0=0,\quad\text{重数 }n-1.
\]

(d) 因此
\[
\det(A-I)=n\cdot0^{\,n-1}=0.
\]

\medskip

\textbf{3.10}

在多项式空间上内积
\[
(f,g)=\int_{-1}^1 f(t)g(t)\,dt.
\]
对系统 \(\{1,t,t^2,t^3\}\) 做格拉姆–施密特。

第一步:\(p_0(t)=1\).

第二步:
\[
(t,p_0)=\int_{-1}^1 t\,dt=0,
\]
故 \(p_1(t)=t\).

第三步:
\[
(t^2,p_0)=\int_{-1}^1 t^2\,dt=\frac{2}{3},\quad
(t^2,p_1)=\int_{-1}^1 t^3\,dt=0,
\]
\[
p_2(t)=t^2-\frac{(t^2,p_0)}{\|p_0\|^2}p_0
=t^2-\frac{2/3}{\int_{-1}^11\,dt}\cdot1
=t^2-\frac13.
\]

第四步:
\[
(t^3,p_0)=\int_{-1}^1 t^3\,dt=0,\quad
(t^3,p_1)=\int_{-1}^1 t^4\,dt=\frac25,\quad
(t^3,p_2)=\int_{-1}^1 t^3(t^2-1/3)\,dt=0,
\]
\[
\|p_1\|^2=\int_{-1}^1 t^2\,dt=\frac23.
\]
于是
\[
p_3(t)=t^3-\frac{(t^3,p_1)}{\|p_1\|^2}p_1
=t^3-\frac{2/5}{2/3}t
=t^3-\frac35 t.
\]

所以得到一个正交系统
\[
p_0(t)=1,\quad
p_1(t)=t,\quad
p_2(t)=t^2-\frac13,\quad
p_3(t)=t^3-\frac35 t,
\]
这就是勒让德多项式 \(P_0,P_1,P_2,P_3\)(差一个规范化常数)。

\medskip

\textbf{3.11}

设 \(P\) 为到子空间 \(E\) 的正交投影。

(a) 证明 \(P^*=P\)。

对任意 \(x,y\in V\),有分解
\[
x=x_E+x_{E^\perp},\quad
y=y_E+y_{E^\perp},
\]
其中 \(x_E,y_E\in E\), \(x_{E^\perp},y_{E^\perp}\in E^\perp\)。由于 \(Px=x_E\),\(Py=y_E\),并且 \(E\perp E^\perp\),
\[
(Px,y)=(x_E,y_E+y_{E^\perp})=(x_E,y_E),
\]
\[
(x,Py)=(x_E+x_{E^\perp},y_E)=(x_E,y_E).
\]
于是 \((Px,y)=(x,Py)\) 对所有 \(x,y\) 成立,从而 \(P^*=P\)。

(b) 证明 \(P^2=P\)。

任取 \(x\in V\),\(Px=x_E\in E\),再次投影仍为自身:
\[
P^2x=P(Px)=P(x_E)=x_E=Px,
\]
故 \(P^2=P\)。

\medskip

\textbf{3.12}

已知对任意 \(v\in V\) 有唯一分解
\[
v=v_E+v_{E^\perp},\quad v_E\in E,\ v_{E^\perp}\in E^\perp.
\]

首先 \(E\subset (E^\perp)^\perp\):若 \(x\in E\),对任何 \(y\in E^\perp\) 有 \(x\perp y\),因此 \(x\) 与 \(E^\perp\) 内每个向量都正交,即 \(x\in (E^\perp)^\perp\)。

再证 \((E^\perp)^\perp\subset E\)。取 \(x\in (E^\perp)^\perp\),写
\[
x=x_E+x_{E^\perp}.
\]
因为 \(x_{E^\perp}\in E^\perp\),
\[
0=(x,x_{E^\perp})\quad\text{(因 }x\perp E^\perp\text{)}
=(x_E,x_{E^\perp})+\,(x_{E^\perp},x_{E^\perp})
=\|x_{E^\perp}\|^2.
\]
故 \(x_{E^\perp}=0\),于是 \(x=x_E\in E\)。

综上 \((E^\perp)^\perp=E\)。

\medskip

\textbf{3.13}

设 \(P\) 为到 \(E\) 的正交投影,\(Q\) 为到 \(E^\perp\) 的正交投影。

(a) 任意 \(x\in V\) 都可唯一写成
\[
x=x_E+x_{E^\perp},\quad x_E\in E,\ x_{E^\perp}\in E^\perp.
\]
于是
\[
Px=x_E,\quad Qx=x_{E^\perp},
\]
故
\[
(P+Q)x=Px+Qx=x_E+x_{E^\perp}=x,
\]
所以 \(P+Q=I\)。

另一方面,
\[
PQx=P(Qx)=P(x_{E^\perp})=0,
\]
\[
QPx=Q(Px)=Q(x_E)=0,
\]
故 \(PQ=QP=0\)。

(b) 证明 \(P-Q\) 是它自己的逆。

由 (a) 知
\[
(P-Q)^2=P^2-PQ-QP+Q^2=P-Q,
\]
又因为 \(P+Q=I\),所以
\[
P-Q=2P-(P+Q)=2P-I,
\]
于是
\[
(P-Q)^2=(2P-I)^2=4P^2-4P+I=4P-4P+I=I.
\]
因此
\[
(P-Q)^{-1}=P-Q.
\]


下面只写解答内容,可直接放进答案册。

\medskip

\textbf{4.1}

设
\[
A=\begin{pmatrix}1&0\\0&1\\1&1\end{pmatrix},\quad
\bb=\begin{pmatrix}1\\1\\0\end{pmatrix},\quad
\xx=\begin{pmatrix}x\\yy\end{pmatrix}.
\]
正规方程:
\[
A^TA\xx=A^T\bb.
\]
计算
\[
A^TA=\begin{pmatrix}2&1\\1&2\end{pmatrix},\quad
A^T\bb=\begin{pmatrix}1\\1\end{pmatrix}.
\]
所以
\[
\begin{pmatrix}2&1\\1&2\end{pmatrix}\begin{pmatrix}x\\yy\end{pmatrix}=
\begin{pmatrix}1\\1\end{pmatrix}
\Rightarrow
\begin{cases}
2x+y=1,\\
x+2y=1.
\end{cases}
\]
解得
\[
x=y=\dfrac13.
\]
最小二乘解为
\[
\xx_0=\begin{pmatrix}\dfrac13\\[2pt]\dfrac13\end{pmatrix}.
\]

\medskip

\textbf{4.2}

设
\[
A=\begin{pmatrix}1&1\\2&-1\\-2&4\end{pmatrix},\quad
a_1=\begin{pmatrix}1\\2\\-2\end{pmatrix},\ 
a_2=\begin{pmatrix}1\\-1\\4\end{pmatrix}.
\]

\emph{(1) 格拉姆–施密特法.}

\(v_1=a_1,\ \|v_1\|^2=1^2+2^2+(-2)^2=9.\)

\[
(a_2,v_1)=1\cdot1+(-1)\cdot2+4\cdot(-2)=-11,
\]
\[
v_2=a_2-\frac{(a_2,v_1)}{\|v_1\|^2}v_1
=a_2+\frac{11}{9}v_1
=\begin{pmatrix}20/9\\13/9\\2/9\end{pmatrix},\quad
\|v_2\|^2=\frac{549}{81}=\frac{61}{9}.
\]

到 \(\operatorname{Ran}A\) 的正交投影是
\[
P=\frac{v_1v_1^T}{\|v_1\|^2}+\frac{v_2v_2^T}{\|v_2\|^2}.
\]
计算得
\[
P
=\frac1{61}
\begin{pmatrix}
21&7&-14\\[2pt]
7&53&-34\\[2pt]
-14&-34&52
\end{pmatrix}.
\]

\emph{(2) 公式法.}

\[
P=A(A^TA)^{-1}A^T.
\]
先算
\[
A^TA=
\begin{pmatrix}
9&-11\\
-11&21
\end{pmatrix},\quad
\det(A^TA)=68,
\]
\[
(A^TA)^{-1}=\frac1{68}
\begin{pmatrix}
21&11\\
11&9
\end{pmatrix}.
\]
于是
\[
P=A(A^TA)^{-1}A^T
=\frac1{61}
\begin{pmatrix}
21&7&-14\\
7&53&-34\\
-14&-34&52
\end{pmatrix},
\]
与格拉姆–施密特所得一致。

\medskip

\textbf{4.3}

点:\((-2,4),(-1,3),(0,1),(2,0)\)。求最小二乘直线 \(y=a+bx\)。

构造
\[
A=\begin{pmatrix}
1&-2\\
1&-1\\
1&0\\
1&2
\end{pmatrix},\quad
\bb=\begin{pmatrix}4\\3\\1\\0\end{pmatrix},\quad
\xx=\begin{pmatrix}a\\b\end{pmatrix}.
\]
正规方程 \(A^TA\xx=A^T\bb\):
\[
A^TA=\begin{pmatrix}
4&-1\\
-1&9
\end{pmatrix},\quad
A^T\bb=\begin{pmatrix}8\\-17\end{pmatrix}.
\]
解
\[
\begin{pmatrix}
4&-1\\
-1&9
\end{pmatrix}
\begin{pmatrix}a\\b\end{pmatrix}
=
\begin{pmatrix}8\\-17\end{pmatrix}
\Rightarrow
\begin{cases}
4a-b=8,\\
-a+9b=-17.
\end{cases}
\]
由第一式 \(b=4a-8\),代入第二式得 \( -a+9(4a-8)=-17\),即
\(35a=55\),
\[
a=\dfrac{11}{7},\quad b=-\dfrac{4}{7}.
\]
最佳拟合直线为
\[
y=\frac{11}{7}-\frac{4}{7}x.
\]

\medskip

\textbf{4.4}

四点:\((1,1,3),(0,3,6),(2,1,5),(0,0,0)\)。拟合平面 \(z=a+bx+cy\)。

(a) 方程组:
\[
\begin{cases}
3=a+b+c,\\
6=a+3c,\\
5=a+2b+c,\\
0=a.
\end{cases}
\]
或矩阵形式
\[
\begin{pmatrix}
1&1&1\\
1&0&3\\
1&2&1\\
1&0&0
\end{pmatrix}
\begin{pmatrix}a\\b\\c\end{pmatrix}
=
\begin{pmatrix}3\\6\\5\\0\end{pmatrix}.
\]

(b) 记
\[
A=\begin{pmatrix}
1&1&1\\
1&0&3\\
1&2&1\\
1&0&0
\end{pmatrix},\quad
\bb=\begin{pmatrix}3\\6\\5\\0\end{pmatrix}.
\]
求最小二乘解 \(A^TA\xx=A^T\bb\),\(\xx=(a,b,c)^T\)。

计算
\[
A^TA=
\begin{pmatrix}
4&3&5\\
3&5&3\\
5&3&11
\end{pmatrix},\quad
A^T\bb=
\begin{pmatrix}
14\\13\\27
\end{pmatrix}.
\]
解
\[
\begin{pmatrix}
4&3&5\\
3&5&3\\
5&3&11
\end{pmatrix}
\begin{pmatrix}a\\b\\c\end{pmatrix}
=
\begin{pmatrix}14\\13\\27\end{pmatrix}.
\]
消元得
\[
a=0,\quad b=1,\quad c=2.
\]
因此最佳拟合平面为
\[
z=x+2y.
\]

\medskip

\textbf{4.5}

设 \(A\xx=\bb\) 有解,且 \(\text{Ker } A\neq\{0\}\)。

(a) 任取一个特解 \(\xx_p\) 与 \(\text{Ker } A\) 的一组基 \(\{k_1,\dots,k_r\}\),则所有解为
\[
\xx=\xx_p+\yy,\quad \yy\in\text{Ker } A.
\]
把 \(\yy\) 按正交分解写成
\[
\yy=\yy_0+\yy_1,\quad
\yy_0\in(\text{Ker } A)^\perp,\ \yy_1\in\text{Ker } A,
\]
但 \(\yy\in\text{Ker } A\) 且 \((\text{Ker } A)^\perp\cap\text{Ker } A=\{0\}\),故 \(\yy_0=0\),即任何解都形如
\(\xx=\xx_p+\yy_1\) 且 \(\yy_1\in\text{Ker } A\)。

把 \(\xx_p\) 再分解为
\[
\xx_p=\xx_0+z,\quad
\xx_0\in(\text{Ker } A)^\perp,\ z\in\text{Ker } A.
\]
于是一般解为
\[
\xx=\xx_0+(z+\yy_1),\quad z+\yy_1\in\text{Ker } A.
\]
因此
\[
\|\xx\|^2=\|\xx_0\|^2+\|z+\yy_1\|^2\ge\|\xx_0\|^2,
\]
等号当且仅当 \(z+\yy_1=0\),即 \(\xx=\xx_0\)。所以 \(\xx_0\) 在所有解中使范数最小,并且唯一。

(b) 任取一个解 \(\xx\)。按直和分解
\[
\xx=\xx_\perp+\xx_{\text{Ker }},\quad
\xx_\perp\in(\text{Ker } A)^\perp,\ \xx_{\text{Ker }}\in\text{Ker } A.
\]
因为 \(\xx,\xx_{\text{Ker }}\) 都是方程的解,
\[
A\xx=A(\xx_\perp+\xx_{\text{Ker }})=A\xx_\perp+A\xx_{\text{Ker }}=A\xx_\perp,
\]
即 \(\xx_\perp\) 也是解;而且 \(\xx_\perp\in(\text{Ker } A)^\perp\),由 (a) 的唯一性知 \(\xx_\perp=\xx_0\)。  
另一方面 \(\xx_\perp=P_{(\text{Ker } A)^\perp}\xx\),故
\[
\xx_0=P_{(\text{Ker } A)^\perp}\xx
\quad\text{对任一解 }\xx\text{ 成立}.
\]

\medskip

\textbf{4.6}

考虑正规方程
\[
A\xx=P_{\operatorname{Ran}A}\bb.
\]
设 \(\bb':=P_{\operatorname{Ran}A}\bb\in\operatorname{Ran}A\)。

(a) 对此系统,上一题适用:因为 \(\bb'\in\operatorname{Ran}A\),方程 \(A\xx=\bb'\) 有解;其所有解的集合为一个仿射子空间 \(\xx_p+\text{Ker } A\)。依 4.5,对这个方程存在唯一最小范数解 \(\xx_0\),即在所有解中 \(\|\xx_0\|\) 最小。

另一方面,\(A\xx=\bb'\) 的解恰好是 \(A\xx=\bb\) 的所有最小二乘解(因为 \(\bb-\bb'\perp\operatorname{Ran}A\))。因此 \(\xx_0\) 在所有最小二乘解中也具有最小范数,且唯一。这就是所求的最小范数最小二乘解。

(b) 设 \(\xx\) 是任意一个 \(A\xx=\bb\) 的最小二乘解,则 \(A\xx=P_{\operatorname{Ran}A}\bb=\bb'\),也就是 \(\xx\) 是 \(A\xx=\bb'\) 的一个解。由 4.5(b) 对方程 \(A\xx=\bb'\) 的结论,最小范数解 \(\xx_0\) 满足
\[
\xx_0=P_{(\text{Ker } A)^\perp}\xx.
\]
因此,对任何最小二乘解 \(\xx\),都有
\[
\xx_0=P_{(\text{Ker } A)^\perp}\xx.
\]


下面只给习题解答内容,按原书风格写,方便直接放进解答册。

\medskip

\textbf{5.1}

设 \(A\) 为 \(n\times n\) 矩阵。由行列式的伴随运算公式
\[
\det(A^T)=\det(A),\quad
\det(\overline{A})=\overline{\det(A)},
\]
以及 \(A^*=\overline{A}^T\),得
\[
\det(A^*)
=\det\bigl((\overline{A})^{T}\bigr)
=\det(\overline{A})
=\overline{\det(A)}.
\]

\medskip

\textbf{5.2}

对
\[
A=
\begin{pmatrix}
1&1&1\\
1&3&2\\
2&4&3
\end{pmatrix}
\]
要求四个基本子空间的正交投影矩阵。

先算
\[
A^T A=
\begin{pmatrix}
6&10&8\\
10&26&18\\
8&18&14
\end{pmatrix},\quad
\det(A^TA)=4\neq0,
\]
故 \(A^TA\) 可逆,\(\operatorname{rank}A=3\),于是
\[
\operatorname{Ran}A=\RR^3,\quad
\text{Ker } A=\{0\}.
\]
由于 \(\operatorname{Ran}A=\RR^3\),到 \(\operatorname{Ran}A\) 的正交投影就是恒等算子:
\[
P_{\operatorname{Ran}A}=I_3,\quad
P_{\text{Ker } A}=0_3.
\]

再看 \(A^*A=A^TA\) 可逆,故 \(\text{Ker } A^*=\{0\}\),同时
\[
\dim\operatorname{Ran}A^*=\operatorname{rank}A=3,
\]
而 \(\operatorname{Ran}A^*\subset\RR^3\),于是
\(\operatorname{Ran}A^*=\RR^3\)。因此
\[
P_{\operatorname{Ran}A^*}=I_3,\quad
P_{\text{Ker } A^*}=0_3.
\]

总结四个基本子空间的正交投影矩阵为
\[
P_{\operatorname{Ran}A}=I_3,\quad
P_{\text{Ker } A}=0_3,\quad
P_{\operatorname{Ran}A^*}=I_3,\quad
P_{\text{Ker } A^*}=0_3.
\]

\medskip

\textbf{5.3}

要证 \(\text{Ker } A=\text{Ker }(A^*A)\)。

一方面,若 \(\xx\in\text{Ker } A\),则 \(A\xx=0\),于是
\[
A^*A\xx=A^*(A\xx)=A^*0=0,
\]
故 \(\xx\in\text{Ker }(A^*A)\),即
\[
\text{Ker } A\subseteq\text{Ker }(A^*A).
\]

反之,若 \(\xx\in\text{Ker }(A^*A)\),则
\[
0=(A^*A\xx,\xx)=\|A\xx\|^2.
\]
范数为零当且仅当向量为零,因此 \(A\xx=0\),即 \(\xx\in\text{Ker } A\),故
\[
\text{Ker }(A^*A)\subseteq\text{Ker } A.
\]

两边合并得到
\[
\text{Ker } A=\text{Ker }(A^*A).
\]

\medskip

\textbf{5.4}

(a) 由上一题 \(\text{Ker } A=\text{Ker }(A^*A)\),因此
\[
\dim\text{Ker } A=\dim\text{Ker }(A^*A).
\]
对 \(A\) 是 \(m\times n\) 矩阵,应用秩–零度定理:
\[
\operatorname{rank}A
=n-\dim\text{Ker } A,\quad
\operatorname{rank}(A^*A)
=n-\dim\text{Ker }(A^*A),
\]
于是
\[
\operatorname{rank}A
=\operatorname{rank}(A^*A).
\]

(b) 若 \(A\xx=0\) 只有平凡解,则 \(\text{Ker } A=\{0\}\),从而 \(\dim\text{Ker } A=0\),由秩–零度定理得
\(\operatorname{rank}A=n\),也就是说 \(A\) 的列向量线性无关。于是 \(A^*A\) 为 \(n\times n\) 可逆矩阵(因其核为 \(\{0\}\),也可直接由 5.3 得出)。

定义
\[
L=(A^*A)^{-1}A^*.
\]
则
\[
LA=(A^*A)^{-1}A^*A=I_n.
\]
所以 \(L\) 是 \(A\) 的一个左逆,\(A\) 左可逆。

\medskip

\textbf{5.5}

假设 \(A^*A\) 可逆,则到 \(\operatorname{Ran}A\) 的正交投影为
\[
P_{\operatorname{Ran}A}=A(A^*A)^{-1}A^*.
\]

利用正交补关系
\[
\text{Ker } A=(\operatorname{Ran}A^*)^\perp,\quad
\text{Ker } A^*=(\operatorname{Ran}A)^\perp,
\]
以及对任意子空间 \(E\),
\[
P_{E^\perp}=I-P_E,
\]
得到另外三个基本子空间的投影为

\[
P_{\text{Ker } A}=I-P_{\operatorname{Ran}A^*},\quad
P_{\text{Ker } A^*}=I-P_{\operatorname{Ran}A}.
\]

其中
\[
\operatorname{Ran}A^*=\operatorname{Ran}(A^*),
\]
且 \((AA^*)\) 在 \(\operatorname{Ran}A\) 上可逆,因而
\[
P_{\operatorname{Ran}A^*}=A^*(AA^*)^{-1}A.
\]

综上:
\[
P_{\operatorname{Ran}A}=A(A^*A)^{-1}A^*,\quad
P_{\text{Ker } A^*}=I-A(A^*A)^{-1}A^*,
\]
\[
P_{\operatorname{Ran}A^*}=A^*(AA^*)^{-1}A,\quad
P_{\text{Ker } A}=I-A^*(AA^*)^{-1}A.
\]

(在 \(m\ge n\) 且 \(\operatorname{rank}A=n\) 的情况下,\(AA^*\) 也是可逆的,上式有意义。)

\medskip

\textbf{5.6}

已知 \(P^*=P\) 且 \(P^2=P\),要证 \(P\) 是某子空间 \(E\) 的正交投影矩阵。

令
\[
E=\operatorname{Ran}P.
\]

先证:若 \(\xx\in E\),则 \(P\xx=\xx\)。的确,\(\xx\in E\) 意味着存在 \(\yy\) 使 \(\xx=P\yy\)。于是
\[
P\xx=P(P\yy)=P^2\yy=P\yy=\xx,
\]
这里用到了 \(P^2=P\)。

再证:若 \(\xx\perp E\),则 \(P\xx=0\)。由 \(\xx\perp E=\operatorname{Ran}P\) 可得
\[
(P\yy,\xx)=0\quad\forall\,\yy.
\]
利用自伴随性
\[
(P\yy,\xx)=(\yy,P^*\xx)=(\yy,P\xx)\quad\forall\,\yy,
\]
于是
\[
(\yy,P\xx)=0\quad\forall\,\yy.
\]
这意味着 \(P\xx=0\)。因此对所有 \(\xx\perp E\) 都有 \(P\xx=0\)。

任意向量 \(\xx\) 可唯一写成
\[
\xx=\xx_1+\xx_2,\quad
\xx_1\in E,\ \xx_2\perp E.
\]
由上面两步,
\[
P\xx=P\xx_1+P\xx_2=\xx_1+0=\xx_1,
\]
即 \(P\) 把 \(\xx\) 正交投影到 \(E\)。故 \(P\) 是到 \(\operatorname{Ran}P\) 的正交投影矩阵。


下面只给习题解答内容,按你现有的排版风格写,不用列表环境。

\medskip

\textbf{6.1}

第一个矩阵
\[
A_1=\begin{pmatrix}1&2\\2&1\end{pmatrix}.
\]
特征多项式
\[
\det(A_1-\lambda I)=(1-\lambda)^2-4=\lambda^2-2\lambda-3=(\lambda-3)(\lambda+1),
\]
特征值为 \(\lambda_1=3,\ \lambda_2=-1\)。

对 \(\lambda_1=3\),解
\[
(A_1-3I)x=0,\quad
\begin{pmatrix}-2&2\\2&-2\end{pmatrix}x=0,
\]
得特征向量可以取 \((1,1)^T\),归一化
\[
u_1=\frac1{\sqrt2}\begin{pmatrix}1\\1\end{pmatrix}.
\]

对 \(\lambda_2=-1\),解
\[
(A_1+I)x=0,\quad
\begin{pmatrix}2&2\\2&2\end{pmatrix}x=0,
\]
得特征向量可取 \((1,-1)^T\),归一化
\[
u_2=\frac1{\sqrt2}\begin{pmatrix}1\\-1\end{pmatrix}.
\]

于是
\[
U_1=\frac1{\sqrt2}\begin{pmatrix}1&1\\1&-1\end{pmatrix},\quad
D_1=\begin{pmatrix}3&0\\0&-1\end{pmatrix},
\]
满足
\[
A_1=U_1D_1U_1^*.
\]

\medskip

第二个矩阵
\[
A_2=\begin{pmatrix}0&-1\\1&0\end{pmatrix}.
\]
它是实正交矩阵,但不是自伴随矩阵,因此不能被\emph{实正交}对角化;在复数域上它是酉的,并且是酉等价于对角矩阵的。

解特征值:
\[
\det(A_2-\lambda I)=
\det\begin{pmatrix}-\lambda&-1\\1&-\lambda\end{pmatrix}
=\lambda^2+1=0,
\]
故特征值为 \(\lambda_1=i,\ \lambda_2=-i\)。

对 \(\lambda_1=i\),解
\[
(A_2-iI)x=0,\quad
\begin{pmatrix}-i&-1\\1&-i\end{pmatrix}x=0.
\]
从第一行得 \(-ix_1-x_2=0\Rightarrow x_2=-ix_1\),取特征向量 \((1,-i)^T\),其模长为
\[
\|(1,-i)\|=\sqrt{1+|-i|^2}=\sqrt2,
\]
归一化
\[
u_1=\frac1{\sqrt2}\begin{pmatrix}1\\-i\end{pmatrix}.
\]

对 \(\lambda_2=-i\),解
\[
(A_2+iI)x=0,\quad
\begin{pmatrix}i&-1\\1&i\end{pmatrix}x=0,
\]
得 \(ix_1-x_2=0\Rightarrow x_2=ix_1\),特征向量可取 \((1,i)^T\),归一化
\[
u_2=\frac1{\sqrt2}\begin{pmatrix}1\\i\end{pmatrix}.
\]

因此
\[
U_2=\frac1{\sqrt2}\begin{pmatrix}1&1\\-i&i\end{pmatrix},\quad
D_2=\begin{pmatrix}i&0\\0&-i\end{pmatrix},
\]
满足
\[
A_2=U_2D_2U_2^*.
\]

\medskip

第三个矩阵
\[
A_3=\begin{pmatrix}
0&2&2\\
2&0&2\\
2&2&0
\end{pmatrix}.
\]
这是实对称矩阵,可以正交对角化。

设 \(\lambda\) 为特征值,解
\[
\det(A_3-\lambda I)=0.
\]
计算
\[
\det\begin{pmatrix}
-\lambda&2&2\\
2&-\lambda&2\\
2&2&-\lambda
\end{pmatrix}
=-(\lambda+2)^2(\lambda-4),
\]
因此特征值为
\[
\lambda_1=4,\quad \lambda_2=\lambda_3=-2.
\]

对 \(\lambda_1=4\),解
\[
(A_3-4I)x=0,\quad
\begin{pmatrix}
-4&2&2\\
2&-4&2\\
2&2&-4
\end{pmatrix}x=0.
\]
易见 \((1,1,1)^T\) 是特征向量,归一化
\[
u_1=\frac1{\sqrt3}\begin{pmatrix}1\\1\\1\end{pmatrix}.
\]

对 \(\lambda=-2\),解
\[
(A_3+2I)x=0,\quad
\begin{pmatrix}
2&2&2\\
2&2&2\\
2&2&2
\end{pmatrix}x=0,
\]
这等价于
\[
x_1+x_2+x_3=0.
\]
所以对应的特征子空间是
\[
E_{-2}=\{(x_1,x_2,x_3)^T\in\RR^3:\ x_1+x_2+x_3=0\},
\]
是二维的。

取两个线性无关且与 \(u_1\) 正交的向量。例如
\[
w_2=\begin{pmatrix}1\\-1\\0\end{pmatrix},\quad
w_3=\begin{pmatrix}1\\1\\-2\end{pmatrix}.
\]
它们都满足坐标和为 0。

归一化并正交化。先求模:
\[
\|w_2\|=\sqrt{1^2+(-1)^2+0^2}=\sqrt2,\quad
\|w_3\|=\sqrt{1^2+1^2+(-2)^2}=\sqrt6.
\]
先取
\[
u_2=\frac1{\sqrt2}\begin{pmatrix}1\\-1\\0\end{pmatrix}.
\]
再将 \(w_3\) 对 \(u_2\) 做正交化:
\[
(w_3,u_2)=\frac1{\sqrt2}(1\cdot1+1\cdot(-1)+(-2)\cdot0)=0,
\]
所以 \(w_3\) 已与 \(u_2\) 正交,只需归一化:
\[
u_3=\frac1{\sqrt6}\begin{pmatrix}1\\1\\-2\end{pmatrix}.
\]

得到一个标准正交基 \(\{u_1,u_2,u_3\}\),令
\[
U_3=\begin{pmatrix}
\frac1{\sqrt3}&\frac1{\sqrt2}&\frac1{\sqrt6}\\[1mm]
\frac1{\sqrt3}&-\frac1{\sqrt2}&\frac1{\sqrt6}\\[1mm]
\frac1{\sqrt3}&0&-\frac2{\sqrt6}
\end{pmatrix},\quad
D_3=\begin{pmatrix}
4&0&0\\
0&-2&0\\
0&0&-2
\end{pmatrix},
\]
则
\[
A_3=U_3D_3U_3^*.
\]

\medskip

\textbf{6.2}

命题:一个矩阵酉等价于一个对角矩阵,当且仅当它具有一个\textbf{正交}的特征向量基。

“若”方向:若 \(A\) 具有由特征向量组成的正交基 \(\{u_1,\dots,u_n\}\),将其归一化成标准正交基并作为列组成酉矩阵
\[
U=(u_1\ \dots\ u_n).
\]
则
\[
Au_j=\lambda_j u_j,\quad j=1,\dots,n,
\]
于是
\[
U^*AU=\operatorname{diag}(\lambda_1,\dots,\lambda_n)=D,
\]
即 \(A=UDU^*\),所以 \(A\) 酉等价于对角矩阵。

“只若”方向:若 \(A\) 酉等价于对角矩阵,即存在酉矩阵 \(U\) 使
\[
A=UDU^*,\quad D=\operatorname{diag}(\lambda_1,\dots,\lambda_n).
\]
记 \(u_j\) 为 \(U\) 的第 \(j\) 列,则
\[
AUe_j=UDU^*Ue_j=UD e_j=\lambda_j Ue_j
\]
即
\[
Au_j=\lambda_j u_j.
\]
因此 \(\{u_1,\dots,u_n\}\) 是 \(A\) 的一组特征向量。由于 \(U\) 是酉矩阵,其列向量构成标准正交基,所以这是一组\textbf{正交}的特征向量基。

两边合并,命题成立。

\medskip

\textbf{6.3}

先证实数情形,\(A=A^*\)。

计算
\[
(A(x+y),x+y)
=(Ax,x)+(Ax,y)+(Ay,x)+(Ay,y).
\]
利用 \(A=A^*\) 得
\[
(Ax,y)=(x,Ay)=(Ay,x),
\]
因此
\[
(A(x+y),x+y)
=(Ax,x)+2(Ax,y)+(Ay,y).
\]

同理
\[
(A(x-y),x-y)
=(Ax,x)-(Ax,y)-(Ay,x)+(Ay,y)
=(Ax,x)-2(Ax,y)+(Ay,y).
\]

于是
\[
(A(x+y),x+y)-(A(x-y),x-y)
=4(Ax,y),
\]
也就是
\[
(Ax,y)=\frac14\bigl[(A(x+y),x+y)-(A(x-y),x-y)\bigr].
\]

\medskip

复数情形,\(A\) 任意。对内积使用标准约定 \((\cdot,\cdot)\) 对第一变量线性,对第二变量共轭线性。令
\[
S:=\frac14\sum_{\alpha=\pm1,\pm i}\alpha\,(A(x+\alpha y),x+\alpha y).
\]

逐项展开。对一般复数 \(\alpha\):
\[
(A(x+\alpha y),x+\alpha y)
=(Ax,x)+\alpha(Ax,y)+\overline{\alpha}(Ay,x)+|\alpha|^2(Ay,y).
\]

因此
\[
\alpha(A(x+\alpha y),x+\alpha y)
=\alpha(Ax,x)+\alpha^2(Ax,y)+|\alpha|^2\alpha(Ay,x)+|\alpha|^2\alpha(Ay,y).
\]

对四个值 \(\alpha=1,-1,i,-i\) 求和。注意
\[
\sum_{\alpha=\pm1,\pm i}\alpha=0,\quad
\sum_{\alpha=\pm1,\pm i}\alpha^2=0,\quad
|\alpha|^2=1.
\]
于是
\[
\sum_{\alpha}\alpha(Ax,x)= (Ax,x)\sum_{\alpha}\alpha=0,
\]
\[
\sum_{\alpha}\alpha^2(Ax,y)=(Ax,y)\sum_{\alpha}\alpha^2=0,
\]
同样
\[
\sum_{\alpha}|\alpha|^2\alpha(Ay,y)=(Ay,y)\sum_{\alpha}\alpha=0.
\]

剩下的项是
\[
\sum_{\alpha}\alpha|\alpha|^2(Ay,x)
=\sum_{\alpha}\alpha(Ay,x)
=(Ay,x)\sum_{\alpha}\alpha=0
\]
若误用此展开;正确做法是采用第二变量共轭线性展开:

重新写:
\[
(A(x+\alpha y),x+\alpha y)
=(A x,x)
+\overline{\alpha}(A x,y)
+\alpha(Ay,x)
+|\alpha|^2(Ay,y).
\]

于是
\[
\alpha(A(x+\alpha y),x+\alpha y)
=\alpha(Ax,x)
+\alpha\overline{\alpha}(Ax,y)
+\alpha^2(Ay,x)
+\alpha|\alpha|^2(Ay,y).
\]

对四个 \(\alpha\) 求和。仍有 \(\sum\alpha=0\)、\(\sum\alpha^2=0\)、\(\sum\alpha|\alpha|^2=\sum\alpha=0\)。唯一留下的项是
\[
\sum_{\alpha}\alpha\overline{\alpha}(Ax,y)
=(Ax,y)\sum_{\alpha}|\alpha|^2=4(Ax,y).
\]

因此
\[
S=\frac14\cdot 4(Ax,y)=(Ax,y),
\]
即
\[
(Ax,y)=\frac14\sum_{\alpha=\pm1,\pm i}\alpha (A(x+\alpha y),x+\alpha y).
\]

\medskip

\textbf{6.4}

设 \(U,V\) 为酉矩阵,即
\[
U^*U=I,\quad V^*V=I.
\]
则
\[
(UV)^*(UV)=V^*U^*UV=V^*V=I,
\]
说明 \(UV\) 也是酉矩阵。

在实数情形,同理可证:若 \(U,V\) 正交,即 \(U^TU=I,\ V^TV=I\),则
\[
(UV)^T(UV)=V^TU^TUV=V^TV=I,
\]
故 \(UV\) 正交。

\medskip

\textbf{6.5}

(a) 对。

若 \(\|Ux\|=\|x\|\) 对所有 \(x\in X\) 成立,则由极化恒等式得
\[
(Ux,Uy)=(x,y)\quad\forall x,y\in X.
\]
因此
\[
(U^*Ux,y)=(Ux,Uy)=(x,y)\quad\forall x,y,
\]
从而
\[
U^*U=I.
\]
在有限维空间中,这意味着 \(U\) 可逆,并且 \(U^{-1}=U^*\),故 \(U\) 是酉算子。

(b) 错。

仅对某个标准正交基 \(\{e_k\}\) 有
\(\|Ue_k\|=\|e_k\|=1\),只说明 \(Ue_k\) 的长度是 1,但并未保证它们彼此正交。若 \(Ue_i\) 与 \(Ue_j\) 不正交,则 \(U\) 不保持内积。

构造一个 \(2\times2\) 反例。取标准正交基
\[
e_1=\begin{pmatrix}1\\0\end{pmatrix},\quad
e_2=\begin{pmatrix}0\\1\end{pmatrix},
\]
令
\[
U=\begin{pmatrix}
1&1\\
0&1
\end{pmatrix}.
\]
则
\[
Ue_1=\begin{pmatrix}1\\0\end{pmatrix},\quad
Ue_2=\begin{pmatrix}1\\1\end{pmatrix},
\]
都有范数
\(\|Ue_1\|=1,\ \|Ue_2\|=\sqrt2\neq1\)——为使条件满足,可改为
\[
U=\begin{pmatrix}
1&\tfrac12\\
0&\tfrac{\sqrt3}{2}
\end{pmatrix},
\]
则
\[
Ue_1=\begin{pmatrix}1\\0\end{pmatrix},\quad
Ue_2=\begin{pmatrix}\tfrac12\\\tfrac{\sqrt3}{2}\end{pmatrix},
\]
两者范数都为 1,但
\[
(Ue_1,Ue_2)=\frac12\ne0,
\]
说明它们不正交,所以 \(U\) 不是等距同构,也不是酉的。

因此 (b) 命题为假。

\medskip

\textbf{6.6}

设 \(A,B\) 酉等价,即存在酉矩阵 \(U\) 使
\[
B=UAU^*.
\]

(a) 计算
\[
B^*B=(UAU^*)^*(UAU^*)=UA^*U^*UAU^*=UA^*AU^*.
\]
于是
\[
\operatorname{trace}(B^*B)
=\operatorname{trace}(UA^*AU^*).
\]
利用迹的循环不变性:
\[
\operatorname{trace}(UA^*AU^*)
=\operatorname{trace}(A^AU^*U)
=\operatorname{trace}(A^*A).
\]
所以
\[
\operatorname{trace}(A^*A)=\operatorname{trace}(B^*B).
\]

(b) 对任意矩阵 \(C=(c_{jk})\),
\[
(C^*C)_{ii}=\sum_k\overline{c_{ki}}c_{ki}
=\sum_k|c_{ki}|^2.
\]
故
\[
\operatorname{trace}(C^*C)
=\sum_i(C^*C)_{ii}
=\sum_{i,k}|c_{ki}|^2
=\sum_{j,k}|c_{jk}|^2.
\]
将 \(C\) 分别取为 \(A\) 和 \(B\),并用 (a) 得
\[
\sum_{j,k}|A_{jk}|^2
=\operatorname{trace}(A^*A)
=\operatorname{trace}(B^*B)
=\sum_{j,k}|B_{jk}|^2.
\]

(c) 对给出的矩阵
\[
A=\begin{pmatrix}1&2\\2&i\end{pmatrix},\quad
B=\begin{pmatrix}i&4\\1&1\end{pmatrix},
\]
计算各自的元素平方绝对值之和:
\[
\sum|A_{jk}|^2
=|1|^2+|2|^2+|2|^2+|i|^2
=1+4+4+1=10.
\]
\[
\sum|B_{jk}|^2
=|i|^2+|4|^2+|1|^2+|1|^2
=1+16+1+1=19.
\]
二者不同,因此 \(A\) 与 \(B\) 不酉等价。

\medskip

\textbf{6.7}

利用酉等价保持特征值(含重数)、迹、行列式以及上一题 (b) 的不变量逐一判断。

(a)
\[
A=\begin{pmatrix}1&0\\0&1\end{pmatrix},\quad
B=\begin{pmatrix}0&1\\1&0\end{pmatrix}.
\]
两者特征值均为 \(\{1,1\}\),均可被实正交对角化为 \(I\),故它们酉等价。实际上
\[
B=SAS^{-1},\quad
S=\frac1{\sqrt2}\begin{pmatrix}1&1\\1&-1\end{pmatrix}
\]
是一个正交矩阵。

(b)
\[
A=\begin{pmatrix}0&1\\1&0\end{pmatrix},\quad
B=\begin{pmatrix}0&\tfrac12\\[1mm]\tfrac12&0\end{pmatrix}.
\]
特征值:
\[
\operatorname{spec}(A)=\{1,-1\},\quad
\operatorname{spec}(B)=\{\tfrac12,-\tfrac12\}.
\]
不相同,故不酉等价。

(c)
\[
A=\begin{pmatrix}
0&1&0\\
-1&0&0\\
0&0&1
\end{pmatrix},\quad
B=\begin{pmatrix}
2&0&0\\
0&-1&0\\
0&0&0
\end{pmatrix}.
\]
\(A\) 为旋转加恒等,行列式
\[
\det A=1,\quad
\det B=0.
\]
行列式不同,所以不酉等价。

(d)
\[
A=\begin{pmatrix}
0&1&0\\
-1&0&0\\
0&0&1
\end{pmatrix},\quad
B=\begin{pmatrix}
1&0&0\\
0&-i&0\\
0&0&i
\end{pmatrix}.
\]
两者特征值均为 \(\{1,i,-i\}\),且都可以被酉对角化为
\(\operatorname{diag}(1,i,-i)\),因此它们酉等价。

(e)
\[
A=\begin{pmatrix}
1&1&0\\
0&2&2\\
0&0&3
\end{pmatrix},\quad
B=\begin{pmatrix}
1&0&0\\
0&2&0\\
0&0&3
\end{pmatrix}.
\]
A 是上三角矩阵,其特征值也是 \(\{1,2,3\}\),与 \(B\) 相同;然而 \(A\) 不是正规矩阵:
\[
AA^*\ne A^*A.
\]
而 \(B\) 是对角矩阵,当然正规。酉等价保持正规性,所以 \(A,B\) 不酉等价。

\medskip

\textbf{6.8}

设 \(U\) 为 \(2\times2\) 正交矩阵,且 \(\det U=1\)。正交即
\[
U^TU=I.
\]
写
\[
U=\begin{pmatrix}a&b\\c&d\end{pmatrix}.
\]
条件 \(U^TU=I\) 给出
\[
a^2+c^2=1,\quad b^2+d^2=1,\quad ab+cd=0.
\]
\(\det U=1\) 给出
\[
ad-bc=1.
\]

先从第一式引入一个角 \(\alpha\),令
\[
a=\cos\alpha,\quad c=\sin\alpha.
\]
则向量 \((b,d)\) 必须与 \((a,c)\) 正交且单位长度,于是
\[
b=-\sin\alpha,\quad d=\cos\alpha
\]
或
\[
b=\sin\alpha,\quad d=-\cos\alpha.
\]

检查行列式:
\[
\det\begin{pmatrix}\cos\alpha&-\sin\alpha\\\sin\alpha&\cos\alpha\end{pmatrix}
=\cos^2\alpha+\sin^2\alpha=1,
\]
\[
\det\begin{pmatrix}\cos\alpha&\sin\alpha\\\sin\alpha&-\cos\alpha\end{pmatrix}
=-\cos^2\alpha-\sin^2\alpha=-1.
\]
由于给定 \(\det U=1\),只能取第一种情形。因此
\[
U=\begin{pmatrix}
\cos\alpha&-\sin\alpha\\
\sin\alpha&\cos\alpha
\end{pmatrix},
\]
即 \(U\) 是某个角度 \(\alpha\) 的旋转矩阵。

\medskip

\textbf{6.9}

设 \(U\) 是 \(3\times3\) 正交矩阵,\(\det U=1\)。

(a) 特征多项式记为
\[
p(\lambda)=\det(U-\lambda I).
\]
由特征值与行列式、迹的关系知,\(U\) 的三个特征值 \(\lambda_1,\lambda_2,\lambda_3\) 满足
\[
\lambda_1\lambda_2\lambda_3=\det U=1,
\quad
|\lambda_j|=1\quad\text{(正交矩阵特征值模为 1)}.
\]

由于系数是实数,若 \(\lambda\) 是非实特征值,则其共轭 \(\overline{\lambda}\) 也是特征值。于是要么全部特征值实,要么存在一对共轭复特征值。后一种情形中,设
\[
\lambda_2=e^{i\alpha},\quad \lambda_3=e^{-i\alpha},
\]
则
\[
\lambda_1\lambda_2\lambda_3=\lambda_1=1.
\]
因此 \(\lambda_1=1\) 是特征值。

若三特征值都实,而又模长为 1,只可能是 \(\pm1\)。积为 1,故奇数个 \(-1\) 不可能,只能是零个或两个 \(-1\)。无论哪种情况必有至少一个特征值 \(1\)。所以 1 总是 \(U\) 的特征值。

(b) 设 \(\{v_1,v_2,v_3\}\) 是一个标准正交基,且
\[
Uv_1=v_1,
\]
即 \(v_1\) 是特征值 1 的特征向量。要证明在此基下 \(U\) 的矩阵为
\[
\begin{pmatrix}
1&0&0\\
0&\cos\alpha&-\sin\alpha\\
0&\sin\alpha&\cos\alpha
\end{pmatrix}
\]
某个 \(\alpha\) 下。

在基 \(\{v_1,v_2,v_3\}\) 中,\(U\) 的矩阵记为
\[
[U]=\begin{pmatrix}
u_{11}&u_{12}&u_{13}\\
u_{21}&u_{22}&u_{23}\\
u_{31}&u_{32}&u_{33}
\end{pmatrix}.
\]
由 \(Uv_1=v_1\) 得
\[
[U]e_1=e_1,
\]
即第一列为
\[
\begin{pmatrix}1\\0\\0\end{pmatrix},
\]
故
\[
[U]=\begin{pmatrix}
1&*&*\\
0&*&*\\
0&*&*
\end{pmatrix}.
\]

因为 \(U\) 正交,\(U^T\) 也是正交,且 \(U^{-1}=U^T\) 的特征值也为 1。事实上,若 \(Uv_1=v_1\),则
\[
(U^Tv_1,v)=(v_1,Uv)=(v_1,v),
\]
对所有 \(v\) 成立,于是 \(U^Tv_1=v_1\)。因此 \(v_1\) 同时是 \(U^T\) 的特征向量,对应特征值 1。于是
\[
[U^T]e_1=e_1.
\]
而
\[
[U^T]=[U]^T=
\begin{pmatrix}
1&0&0\\
u_{12}&u_{22}&u_{32}\\
u_{13}&u_{23}&u_{33}
\end{pmatrix},
\]
第一列必须为 \((1,0,0)^T\),故
\[
u_{12}=u_{13}=0.
\]
所以
\[
[U]=\begin{pmatrix}
1&0&0\\
0&u_{22}&u_{23}\\
0&u_{32}&u_{33}
\end{pmatrix}
=
\begin{pmatrix}
1&0\\[1mm]
0&V
\end{pmatrix},
\]
其中
\[
V=\begin{pmatrix}
u_{22}&u_{23}\\
u_{32}&u_{33}
\end{pmatrix}
\]
是 \(2\times2\) 矩阵。

由于 \(U\) 正交,求
\[
[U]^T[U]=I_3
\]
可得
\[
V^TV=I_2,
\]
故 \(V\) 是 \(2\times2\) 正交矩阵。同样
\[
\det U=\det[U]=1=\det V.
\]
于是 \(V\) 是一个行列式为 1 的 \(2\times2\) 正交矩阵。由 6.8 知存在角 \(\alpha\) 使
\[
V=\begin{pmatrix}
\cos\alpha&-\sin\alpha\\
\sin\alpha&\cos\alpha
\end{pmatrix}.
\]

因此在基 \(\{v_1,v_2,v_3\}\) 下,
\[
[U]=\begin{pmatrix}
1&0&0\\
0&\cos\alpha&-\sin\alpha\\
0&\sin\alpha&\cos\alpha
\end{pmatrix},
\]
这就是结论。


\textbf{8.1}

对每个 \(k\),
\[
z_k\bar w_k=(x_k+i y_k)(u_k-i v_k)
=(x_ku_k+y_kv_k)+i(-x_kv_k+y_ku_k).
\]
因此
\[
\sum_{k=1}^n z_k\bar w_k
=\sum_{k=1}^n(x_ku_k+y_kv_k)+
i\sum_{k=1}^n(-x_kv_k+y_ku_k),
\]
取实部即得
\[
\Re\!\Bigl(\sum_{k=1}^n z_k\bar w_k\Bigr)
=\sum_{k=1}^n x_ku_k+\sum_{k=1}^n y_kv_k.
\]

\medskip

\textbf{8.2}

在复内积空间中,\((\cdot,\cdot)_\CC\) 满足:对第一变量线性、对第二变量共轭线性,且
\((x,x)_\CC\ge0\) 且 \( (x,x)_\CC=0\iff x=0\)。

令
\[
(x,y)_\RR:=\Re (x,y)_\CC.
\]
则:

(1) 双线性(实):  
对实数 \(\alpha,\beta\),
\[
(\alpha x+\beta y,z)_\RR
=\Re(\alpha x+\beta y,z)_\CC
=\Re\bigl(\alpha(x,z)_\CC+\beta(y,z)_\CC\bigr)
=\alpha(x,z)_\RR+\beta(y,z)_\RR,
\]
第二变量同理,所以是实双线性的。

(2) 对称性:  
由共轭对称 \((x,y)_\CC=\overline{(y,x)_\CC}\) 得
\[
(x,y)_\RR
=\Re(x,y)_\CC
=\Re\overline{(y,x)_\CC}
=\Re(y,x)_\CC
=(y,x)_\RR.
\]

(3) 正定性:  
\[
(x,x)_\RR=\Re(x,x)_\CC=(x,x)_\CC\ge0,
\]
且 \((x,x)_\RR=0\Rightarrow (x,x)_\CC=0\Rightarrow x=0\)。

故 \((\cdot,\cdot)_\RR\) 是实内积。

\medskip

\textbf{8.3}

\(U\) 正交说明 \(\|Ux\|=\|x\|\) 且
\[
(Ux,Uy)=(x,y)\quad\forall x,y.
\]

由 \(U^2=-I\) 得 \(U(Ux)=-x\)。取内积:
\[
(Ux, U(Ux))=(Ux,-x)=-(Ux,x).
\]
另一方面,由正交性,
\[
(Ux, U(Ux))=(x,Ux).
\]
因为内积在实空间对称,\((Ux,x)=(x,Ux)\)。于是
\[
(x,Ux)=-(Ux,x)=-(x,Ux),
\]
故
\[
(x,Ux)=0.
\]
即 \(Ux\perp x\)。

\medskip

\textbf{8.4}

正交性给出 \(U^*U=I\),而 \(U^2=-I\) 给出 \(U^{-1}=-U\)。在有限维内积空间中
\(U^{-1}=U^*\),于是
\[
U^*=-U.
\]

\medskip

\textbf{8.5}

先证 \(\dim X\) 为偶数。

\(U^2=-I\Rightarrow U\) 无非零不变向量:若 \(Ux=\lambda x\),则
\[
- x=U^2x=U(\lambda x)=\lambda^2x,
\]
故 \(\lambda^2=-1\) 不可能为实数。于是不存在特征值 \(1,-1\)。特别地,\(\text{Ker }(U-I)=\{0\}\)。

考虑复化 \(X_\CC\)。在 \(X_\CC\) 上 \(U\) 仍满足 \(U^2=-I\),因而特征多项式只含根 \(i,-i\)。又因多项式
\(\lambda^2+1\) 无重根,\(U\) 在 \(X_\CC\) 上可对角化,其所有特征值是 \(i\) 与 \(-i\),重数相等(共轭成对)。设
\(\dim_\RR X=2n\),则 \(\dim_\CC X_\CC=2n\),故
\[
\dim E_i=\dim E_{-i}=n.
\]

于是 \(\dim X\) 必为偶数,记 \(\dim X=2n\)。

现在在 \(X\) 中构造 \(E,U_0\)。

取 \(E_i\subset X_\CC\) 中的一组特征向量基
\(\{z_1,\dots,z_n\}\),其中
\[
Uz_k=i z_k.
\]
写成实虚部 \(z_k=x_k+i y_k\),\(x_k,y_k\in X\)。则
\[
U(x_k+i y_k)=i(x_k+i y_k)
\]
等价于
\[
Ux_k=-y_k,\quad Uy_k=x_k.
\]

(1) \(\{x_1,\dots,x_n,y_1,\dots,y_n\}\) 为 \(X\) 的实基:  
从 \(\{z_k,\overline{z_k}\}\) 生成了 \(X_\CC\) 可知它们在 \(X_\CC\) 中生成;取实虚部可得是一组实基,故
\(\dim X=2n\)。

(2) 令
\[
E:=\operatorname{span}\{x_1,\dots,x_n\},\quad
E^\perp=\operatorname{span}\{y_1,\dots,y_n\};
\]
由 8.3 已知 \(Ux\perp x\),并且上式关系表明 \(U\) 在 \(\{x_k,y_k\}\) 基下的作用是
\[
Ux_k=-y_k,\quad Uy_k=x_k,
\]
所以 \(U(E)=E^\perp,\ U(E^\perp)=E\),且 \(\dim E=\dim E^\perp=n\),因此
\[
X=E\oplus E^\perp.
\]

(3) 定义 \(U_0:E\to E^\perp\) 为
\[
U_0x:=Ux\quad(x\in E).
\]
由正交性,\(\|U_0x\|=\|x\|\),且
\[
(U_0x,U_0y)=(Ux,Uy)=(x,y),
\]
所以 \(U_0\) 是从 \(E\) 到 \(E^\perp\) 的正交同构。于是其伴随 \(U_0^*:E^\perp\to E\) 也正交,且
\[
U_0^*U_0=I_E,\quad U_0U_0^*=I_{E^\perp}.
\]

(4) 在分解 \(X=E\oplus E^\perp\) 下写 \(U\)。对 \(x\in E,y\in E^\perp\),有
\[
Ux=U_0x\in E^\perp,\quad
Uy=-U_0^*y\in E.
\]
因此矩阵形式为
\[
U=
\begin{pmatrix}
0 & -U_0^*\\
U_0 & 0
\end{pmatrix}.
\]

这就是所要证明的分解。


\end{exer}







\section{第六章答案}

\begin{exer}

\textbf{1.1}

设算子(或矩阵) \(A\) 在某个基下的矩阵是上三角矩阵
\[
T=\begin{pmatrix}
\lambda_1 & *      & \cdots & *\\
0         & \lambda_2 & \cdots & *\\
\vdots    & \vdots    & \ddots & \vdots\\
0         & 0         & \cdots & \lambda_n
\end{pmatrix},
\]
其对角元 \(\lambda_1,\dots,\lambda_n\) 正是 \(A\) 的特征值(按代数重数计)。

1. 行列式与特征值乘积的关系:

上三角矩阵的行列式等于对角线元素的乘积:
\[
\det T = \lambda_1\lambda_2\cdots\lambda_n.
\]
因为 \(A\) 与 \(T\) 相似(\(A=UTU^{-1}\) 或 \(A=UTU^*\)),相似矩阵行列式相等:
\[
\det A = \det T = \lambda_1\lambda_2\cdots\lambda_n.
\]

2. 迹与特征值和的关系:

上三角矩阵的迹是对角线元素之和:
\[
\operatorname{tr}T = \lambda_1+\lambda_2+\cdots+\lambda_n.
\]
又因为相似矩阵的迹相同(\(\operatorname{tr}(UTU^{-1})=\operatorname{tr}T\)),得到
\[
\operatorname{tr}A = \operatorname{tr}T
= \lambda_1+\lambda_2+\cdots+\lambda_n.
\]

因此利用算子的上三角表示,可以直接得到:
\[
\det A = \prod_{k=1}^n\lambda_k,\quad
\operatorname{tr}A = \sum_{k=1}^n\lambda_k,
\]
其中 \(\lambda_k\) 为 \(A\) 的特征值(按代数重数计)。


下面按题号依次给出结论和证明草稿,只用普通环境与数学环境,你可直接嵌到解答中。

\medskip

\textbf{2.1}

a) 对。酉算子满足 \(U^*U=I\),而正规要求 \(U^*U=UU^*\),显然成立。

b) 错。可逆只是 \(\det A\ne0\),酉还要求 \(A^*A=I\)。例如
\(\begin{pmatrix}2&0\\0&2\end{pmatrix}\) 可逆但不是酉。

c) 对。若 \(B=U^*AU\),则 \(B=U^{-1}AU\),这正是相似关系。

d) 对。若 \(A^*=A,B^*=B\),则
\((A+B)^*=A^*+B^*=A+B\)。

e) 对。若 \(U^*U=I\),则
\((U^*)^*U^* = UU^* = I\),所以 \(U^*\) 也酉。

f) 对。若 \(N^*N=NN^*\),取伴随得
\((N^*N)^*=(NN^*)^*\),即
\(N^*(N^*)^* = (N^*)^*N^*\),
所以 \(N^*\) 也满足正规条件。

g) 错。特征值全是 1 只保证特征多项式是 \((\lambda-1)^n\),但矩阵可非正规,例如
\(\begin{pmatrix}1&1\\0&1\end{pmatrix}\) 的唯一特征值是 1,却既不酉也不正交。

h) 对。正规算子可酉对角化:\(T=UDU^*\) 且 \(D\) 对角。若所有特征值都是 1,则 \(D=I\),故 \(T=UIU^*=I\)。

i) 对。举例:
\[
T=\begin{pmatrix}1&0\\0&-1\end{pmatrix}
\]
在 \(\RR^2\) 上保持欧氏范数,但
\(\langle T(1,0),(0,1)\rangle = \langle(1,0),(0,-1)\rangle=0\),
而
\(\langle(1,0),(0,1)\rangle=0\)——这个例子其实也保持内积;换一个:定义 \(T:\RR^2\to\RR^2,\ T(x,y)=(x,-y)\) 在标准内积下既保持范数也保持内积,是正交算子,不合要求。要“只保持范数不保持内积”的例子:在 \(\CC^2\) 上,取共轭算子
\(T(z_1,z_2)=(\bar z_1,\bar z_2)\)。对标准内积
\(\langle z,w\rangle = z_1\bar w_1+z_2\bar w_2\),
有 \(\|Tz\|=\|z\|\),但
\(\langle Tz,Tw\rangle=\overline{\langle z,w\rangle}\),
一般不等于 \(\langle z,w\rangle\)。故命题为真。

\medskip

\textbf{2.2}

命题:两个正规的和不一定正规。

反例:
\[
A=\begin{pmatrix}1&0\\0&-1\end{pmatrix},\quad
B=\begin{pmatrix}0&1\\0&0\end{pmatrix}.
\]
\(A\) 自伴随故正规;\(B\) 满足 \(B^2=0\) 且 \(B^*B=\begin{pmatrix}0&0\\0&1\end{pmatrix}\),
\(BB^*=\begin{pmatrix}1&0\\0&0\end{pmatrix}\),不相等?——这样 \(B\) 不是正规。换一个标准例子:取
\[
N_1=\begin{pmatrix}1&0\\0&-1\end{pmatrix},\quad
N_2=\begin{pmatrix}0&1\\1&0\end{pmatrix}.
\]
它们都是自伴随(从而正规),但
\[
N_1+N_2=\begin{pmatrix}1&1\\1&-1\end{pmatrix},
\quad
(N_1+N_2)^*(N_1+N_2)\ne (N_1+N_2)(N_1+N_2)^*
\]
其实这里仍然相等,因为实对称还是自伴随。需要非交换的正常算子反例,典型作法:取两个不同的酉矩阵,它们之和非正规。可用
\[
U=\begin{pmatrix}1&0\\0&-1\end{pmatrix},\quad
V=\begin{pmatrix}0&1\\1&0\end{pmatrix}.
\]
\(U,V\) 都是正交(酉)矩阵,故正规。它们的和
\[
U+V=
\begin{pmatrix}1&1\\1&-1\end{pmatrix}
\]
仍然是实对称,从而还是正规——所以这也不行。

更简单的标准反例(书上常用):
\[
N_1=\begin{pmatrix}1&0\\0&i\end{pmatrix},\quad
N_2=\begin{pmatrix}1&0\\0&-i\end{pmatrix}.
\]
它们都是对角酉矩阵(正规的)。但
\[
N_1+N_2=\begin{pmatrix}2&0\\0&0\end{pmatrix}
\]
还是正规……这个也不行。

一个行之有效的反例是:取非对易的正规矩阵:
\[
N_1=\begin{pmatrix}1&0\\0&-1\end{pmatrix},\quad
N_2=\begin{pmatrix}0&1\\-1&0\end{pmatrix}.
\]
\(N_2\) 是二维旋转 \(90^\circ\) 的矩阵,酉的,所以正规。计算
\[
N_1+N_2=
\begin{pmatrix}1&1\\-1&-1\end{pmatrix}.
\]
直接算
\[
(N_1+N_2)^*(N_1+N_2)=
\begin{pmatrix}2&2\\2&2\end{pmatrix},\quad
(N_1+N_2)(N_1+N_2)^*=
\begin{pmatrix}2&-2\\-2&2\end{pmatrix},
\]
不相等,因此和不正规。故命题为假。

\medskip

\textbf{2.3}

若 \(A\) 酉等价于对角矩阵,即
\[
A=UDU^*,\quad U\text{ 酉},\ D\text{ 对角},
\]
则
\[
A^*=(UDU^*)^*=UD^*U^*.
\]
因为 \(D\) 对角,\(D^*D=DD^*\),于是
\[
A^*A = (UD^*U^*)(UDU^*) = UD^*DU^*,
\]
\[
AA^* = (UDU^*)(UD^*U^*) = UDD^*U^*.
\]
中间部分相等,故 \(A^*A=AA^*\),即 \(A\) 正规。

\medskip

\textbf{2.4}

矩阵
\[
A=\begin{pmatrix}3&2\\2&3\end{pmatrix}.
\]

特征多项式:
\[
\det(A-\lambda I) =
\begin{vmatrix}3-\lambda&2\\2&3-\lambda\end{vmatrix}
=(3-\lambda)^2-4=\lambda^2-6\lambda+5.
\]
根为 \(\lambda_1=5,\lambda_2=1\)。

对应特征向量:
\[
\lambda=5:\ (3-5)x+2y=0\Rightarrow -2x+2y=0\Rightarrow y=x,
\]
可取 \(v_1=(1,1)^T\);归一化 \(u_1=\frac1{\sqrt2}(1,1)^T\).

\[
\lambda=1:\ (3-1)x+2y=0\Rightarrow 2x+2y=0\Rightarrow y=-x,
\]
可取 \(v_2=(1,-1)^T\),归一化 \(u_2=\frac1{\sqrt2}(1,-1)^T\).

于是
\[
U=\frac1{\sqrt2}\begin{pmatrix}1&1\\1&-1\end{pmatrix},\quad
D=\begin{pmatrix}5&0\\0&1\end{pmatrix},
\]
满足 \(A=UDU^*\)。

平方根:设 \(D^{1/2}=\begin{pmatrix}\mu_1&0\\0&\mu_2\end{pmatrix}\) 且 \(\mu_1^2=5,\mu_2^2=1\),于是
\(\mu_1=\pm\sqrt5,\ \mu_2=\pm1\)。任取符号组合,令
\[
B=UD^{1/2}U^*,
\]
则 \(B^2 = U D^{1/2} U^* U D^{1/2} U^* = U D U^* = A\),且 \(B\) 自伴随(因为 \(D^{1/2}\) 实对角)。这样得到 4 个自伴随平方根。

\medskip

\textbf{2.5}

命题:“任何自伴随矩阵都有一个自伴随平方根”。

这是假的。

反例:\(A=-I\)(实或复都可)。它是自伴随,但若 \(B^2=-I\) 且 \(B=B^*\),那么 \(B\) 的特征值实,而特征值的平方必须是 \(-1\),不可能。故不存在自伴随平方根。

另一方面,如果要求 \(A\) 正定(即所有特征值 \(>0\)),则确实存在唯一的自伴随正定平方根,这是极分解/谱定理的标准结论。

\medskip

\textbf{2.6}

\[
A=\begin{pmatrix}7&2\\2&4\end{pmatrix}.
\]

求特征多项式:
\[
\det(A-\lambda I)
=\begin{vmatrix}7-\lambda&2\\2&4-\lambda\end{vmatrix}
=(7-\lambda)(4-\lambda)-4
=\lambda^2-11\lambda+24.
\]
解得 \(\lambda_1=8,\ \lambda_2=3\)。

特征向量:

\(\lambda=8\): \((7-8)x+2y=0\Rightarrow -x+2y=0\Rightarrow y=\frac12 x\). 取 \(v_1=(2,1)^T\),
\(\|v_1\|=\sqrt5\),归一化 \(u_1=\frac1{\sqrt5}(2,1)^T\).

\(\lambda=3\): \((7-3)x+2y=0\Rightarrow 4x+2y=0\Rightarrow y=-2x\). 取 \(v_2=(1,-2)^T\),
\(\|v_2\|=\sqrt5\),归一化 \(u_2=\frac1{\sqrt5}(1,-2)^T\).

于是
\[
U=\frac1{\sqrt5}\begin{pmatrix}2&1\\1&-2\end{pmatrix},\quad
D=\begin{pmatrix}8&0\\0&3\end{pmatrix},
\]
满足 \(A=UDU^*\)。

具有正特征值的平方根:取
\[
D^{1/2}=
\begin{pmatrix}\sqrt8&0\\0&\sqrt3\end{pmatrix},
\]
则
\[
B = U D^{1/2} U^*
\]
就是所求;\(B\) 自伴随且其特征值为 \(\sqrt8,\sqrt3>0\),并且 \(B^2=A\)。可以保留为乘积形式。

\medskip

\textbf{2.7}

a) “两个自伴随矩阵的乘积是自伴随的”——一般为假。

反例:在 \(\RR^2\) 上
\[
A=\begin{pmatrix}1&0\\0&-1\end{pmatrix},\quad
B=\begin{pmatrix}0&1\\1&0\end{pmatrix}
\]
都自伴随。它们的乘积
\[
AB=\begin{pmatrix}0&1\\-1&0\end{pmatrix},\quad
(AB)^* = -AB\ne AB.
\]
因此 \(AB\) 不是自伴随。

b) 若 \(A\) 自伴随,则 \(A^k\) 自伴随(\(k\in\mathbb{N}\))。

证明:用归纳或伴随性质:
\[
(A^k)^* = (A^{k-1}A)^* = A^* (A^{k-1})^*.
\]
若已知 \(A^{k-1}\) 自伴随,则上式为 \(AA^{k-1}=A^k\)。基础情形 \(k=1\) 成立,因此对一切 \(k\) 成立。

\medskip

\textbf{2.8}

a) \((A^*A)^* = A^* (A^*)^* = A^*A\),故自伴随。

b) 设 \(A^*A x = \lambda x\)。取内积:
\[
\lambda\|x\|^2 = \langle A^*Ax,x\rangle
= \langle Ax,Ax\rangle = \|Ax\|^2 \ge 0.
\]
若 \(x\ne0\),则 \(\|x\|^2>0\),故 \(\lambda\ge0\)。即特征值非负。

c) \(A^*A+I\) 可逆。若 \((A^*A+I)x=0\),则
\[
\langle(A^*A+I)x,x\rangle = \|Ax\|^2 + \|x\|^2 = 0.
\]
这只可能在 \(x=0\) 时发生,因此内核为 \(\{0\}\),矩阵可逆。

\medskip

\textbf{2.9}

a) 真。若 \(A\) 自伴随,其谱实:\(\lambda\in\RR\)。若 \((A+iI)x=0\),则 \(Ax=-ix\),说明 \(-i\) 是特征值,矛盾。因此零不是 \(A+iI\) 的特征值,矩阵可逆。

b) 真。若 \(U\) 酉,则其特征值满足 \(|\lambda|=1\)。若 \((U+\tfrac34 I)x=0\),则
\(Ux=-\tfrac34 x\),说明 \(-\tfrac34\) 是特征值,但其模为 \(\tfrac34\ne1\),矛盾。故 \(U+\tfrac34I\) 可逆。

c) 假。取实矩阵 \(A=0\) 即可。则 \(A-iI=-iI\) 可逆?其实 \(-iI\) 行列式 \((-i)^n\ne0\),是可逆;因此反例不成立。要找反例,我们希望 \(\lambda=i\) 是特征值且 \(A\) 实。若 \(A\) 实且有复特征值 \(i\),必伴随有 \(-i\) 也是特征值。只要保证 \(i\) 是特征值即可,例:
\[
A=\begin{pmatrix}0&-1\\1&0\end{pmatrix}
\]
是实旋转 \(90^\circ\) 的矩阵,它的特征值为 \(i,-i\)。则 \(A-iI\) 有非平凡核,不可逆。于是命题为假。

\medskip

\textbf{2.10}

\[
R_\alpha=
\begin{pmatrix}
\cos\alpha & -\sin\alpha\\
\sin\alpha & \cos\alpha
\end{pmatrix},\quad \alpha\notin\pi\mathbb{Z}.
\]

特征多项式:
\[
\det(R_\alpha-\lambda I)
=
\begin{vmatrix}\cos\alpha-\lambda & -\sin\alpha\\ \sin\alpha & \cos\alpha-\lambda\end{vmatrix}
=(\cos\alpha-\lambda)^2+\sin^2\alpha
= \lambda^2-2\cos\alpha\,\lambda+1.
\]
根为
\[
\lambda_{1,2}=\cos\alpha\pm i\sin\alpha=e^{\pm i\alpha}.
\]

对应特征向量:

\(\lambda_1=e^{i\alpha}\): 解
\[
(\cos\alpha-e^{i\alpha})x - \sin\alpha\,y=0.
\]
利用 \(\cos\alpha-e^{i\alpha}=-i\sin\alpha\),得
\[
-i\sin\alpha\, x -\sin\alpha\,y=0\Rightarrow y=-ix.
\]
取 \(v_1=(1,-i)^T\),归一化
\(u_1=\frac1{\sqrt2}(1,-i)^T\).

\(\lambda_2=e^{-i\alpha}\): 类似得 \(v_2=(1,i)^T\),归一化
\(u_2=\frac1{\sqrt2}(1,i)^T\).

于是
\[
U=\frac1{\sqrt2}
\begin{pmatrix}
1&1\\
-i&i
\end{pmatrix},\quad
D=\begin{pmatrix}e^{i\alpha}&0\\0&e^{-i\alpha}\end{pmatrix},
\]
满足 \(R_\alpha = UDU^*\)。

\medskip

\textbf{2.11}

\[
A=
\begin{pmatrix}
\cos\alpha & \sin\alpha\\
\sin\alpha & -\cos\alpha
\end{pmatrix}.
\]

特征多项式:
\[
\det(A-\lambda I)
=
\begin{vmatrix}\cos\alpha-\lambda & \sin\alpha\\ \sin\alpha & -\cos\alpha-\lambda\end{vmatrix}
= -\cos^2\alpha+\lambda^2-\sin^2\alpha
=\lambda^2-1.
\]
故特征值为 \(\lambda_1=1,\ \lambda_2=-1\)。

求特征向量。

\(\lambda=1\): 解
\[
(\cos\alpha-1)x+\sin\alpha\,y=0.
\]
用三角恒等式
\(\cos\alpha-1=-2\sin^2\frac\alpha2\), \(\sin\alpha=2\sin\frac\alpha2\cos\frac\alpha2\),得
\[
-2\sin^2\frac\alpha2\,x
+2\sin\frac\alpha2\cos\frac\alpha2\,y=0
\Rightarrow -\sin\frac\alpha2\,x+\cos\frac\alpha2\,y=0.
\]
取
\[
v_1=
\begin{pmatrix}\cos\frac\alpha2\\ \sin\frac\alpha2\end{pmatrix},
\]
易检验满足上式且归一。

\(\lambda=-1\): 解
\[
(\cos\alpha+1)x+\sin\alpha\,y=0.
\]
用 \(\cos\alpha+1=2\cos^2\frac\alpha2\),\(\sin\alpha=2\sin\frac\alpha2\cos\frac\alpha2\),得
\[
\cos\frac\alpha2\,x+\sin\frac\alpha2\,y=0,
\]
取
\[
v_2=
\begin{pmatrix}-\sin\frac\alpha2\\ \cos\frac\alpha2\end{pmatrix}.
\]
两向量规范正交,因此
\[
U=\begin{pmatrix}
\cos\frac\alpha2 & -\sin\frac\alpha2\\
\sin\frac\alpha2 & \cos\frac\alpha2
\end{pmatrix},\quad
D=\begin{pmatrix}1&0\\0&-1\end{pmatrix},
\]
满足 \(A=UDU^*\)(在实情形 \(U^*=U^T\))。

\medskip

\textbf{2.12}

上一题中的 \(A\) 有特征值 \(1\) 与 \(-1\),对应特征向量分别是
\[
u_1=\begin{pmatrix}\cos\frac\alpha2\\\sin\frac\alpha2\end{pmatrix},\quad
u_2=\begin{pmatrix}-\sin\frac\alpha2\\\cos\frac\alpha2\end{pmatrix}.
\]
几何上,\(A\) 是沿着经过原点、方向为 \(u_1\) 的直线的\textbf{镜面对称}:  
在该直线方向(\(u_1\))上的分量保持不变(特征值 1),在与之垂直的方向(\(u_2\))上的分量取相反数(特征值 -1)。因此 \(A\) 是绕原点、相对于直线“旋转角 \(\alpha/2\)”后得到的一条直线的反射变换。

\medskip

\textbf{2.13}

正规算子 \(N\) 具有模为 1 的特征值,即对所有 \(k\) 有 \(|\lambda_k|=1\),要证 \(N\) 酉。

由谱定理:存在酉 \(U\) 使
\[
N=UDU^*,\quad D=\operatorname{diag}(\lambda_1,\dots,\lambda_n).
\]
则
\[
N^*N = UD^*U^*UDU^* = UD^*DU^*,
\quad
NN^* = UDD^*U^*.
\]
当 \(|\lambda_k|=1\) 时,\(|\lambda_k|^2=1\),故 \(D^*D = DD^* = I\),于是
\[
N^*N = U I U^* = I,\quad
NN^* = U I U^* = I.
\]
所以 \(N\) 酉。

\medskip

\textbf{2.14}

设正规算子 \(N\) 的特征值全是实数。按谱定理,
\[
N=UDU^*,\quad D=\operatorname{diag}(\lambda_1,\dots,\lambda_n),\ \lambda_k\in\RR.
\]
则
\[
N^* = (UDU^*)^* = U D^* U^*.
\]
因为 \(\lambda_k\in\RR\),有 \(D^*=D\),于是 \(N^*=UDU^*=N\)。因此 \(N\) 自伴随。

\medskip

\textbf{2.15}

a) 构造一个可对角化但不能正交对角化的 \(2\times2\) 复对称矩阵。

例:
\[
A=\begin{pmatrix}0&1\\1&i\end{pmatrix}.
\]
显然 \(A^T=A\),但
\[
A^*=
\begin{pmatrix}0&1\\1&-i\end{pmatrix}\ne A,
\]
故不是自伴随。计算特征多项式:
\[
\det(A-\lambda I)=
\begin{vmatrix}-\lambda&1\\1&i-\lambda\end{vmatrix}
=\lambda^2 - i\lambda - 1.
\]
判别式 \(i^2+4=-1+4=3\ne0\),所以有两个不同特征值,矩阵可对角化。但因为不是正规矩阵(\(A^*A\ne AA^*\)),无法被酉(正交)对角化,因此不存在一组正交特征向量基。

b) 构造一个不能对角化的 \(2\times2\) 复对称矩阵。

例:
\[
B=\begin{pmatrix}0&1\\1&0\end{pmatrix}
\]
是复对称且实自伴随,实际可对角化,不合要求。需要非正规、只有一个特征向量。可取
\[
B=\begin{pmatrix}i&1\\1&i\end{pmatrix}.
\]
它对称(\(B^T=B\)),特征多项式
\[
\det(B-\lambda I)=
\begin{vmatrix}i-\lambda&1\\1&i-\lambda\end{vmatrix}
=(i-\lambda)^2-1
=\lambda^2-2i\lambda-2.
\]
判别式 \((-2i)^2-4(-2)=-4(-1)+8=12\ne0\),所以这个例子还是可对角化。需要构造一个对称 Jordan 块。

注意若 \(B\) 为 \(2\times2\) 对称矩阵
\(\begin{pmatrix}a&b\\b&d\end{pmatrix}\),其特征多项式是
\(\lambda^2-(a+d)\lambda+ad-b^2\),判别式 \((a+d)^2-4(ad-b^2)=(a-d)^2+4b^2\),在复数域上永远非零除非 \(a=d\) 且 \(b=0\)。当 \(a=d,b=0\) 时矩阵是标量矩阵 \(aI\),虽只有一个特征值但整个空间都是特征子空间,仍然对角化。因此 \textbf{任何 \(2\times2\) 复对称矩阵实际上总是可对角化},所以严格来说 b) 要求的矩阵并不存在,这恰好说明定理 2.2 的结论“对复数对称矩阵不成立”的意思是:它们不是都\emph{正交}可对角化(a) 已给反例),而不是“存在不可对角化的复对称矩阵”——这在 \(2\times2\) 情形下是不可能的。若你按原英文版题意,b) 是在更高维度构造一个不可对角化的复对称矩阵,可以仿照书后给的标准例子;在 \(2\times2\) 维度事实上无解,需要在译注中略作说明。


下面只给习题解答内容,方便你直接嵌到解答稿中。

---

\textbf{3.1}\quad

设 \(A:\mathbb C^n\to\mathbb C^m\),其 SVD 为
\[
A=W\Sigma V^*,\quad
\Sigma=\operatorname{diag}(s_1,\dots,s_r,0,\dots).
\]
非零奇异值 \(s_k>0\) 的个数是 \(r\)。由 Schmidt 分解
\[
A=\sum_{k=1}^r s_k w_k v_k^*
\]
可见 \(\text{Ran } A=\operatorname{span}\{w_1,\dots,w_r\}\),维数为 \(r\)。因此
\[
\text{rank } A=\dim\text{Ran } A=r,
\]
即矩阵的秩等于其非零奇异值的个数(计重数)。

---

\textbf{3.2}\quad

(1)\(A=\begin{pmatrix}2&3\\0&2\end{pmatrix}\).

\[
A^*A=
\begin{pmatrix}2&0\\3&2\end{pmatrix}
\begin{pmatrix}2&3\\0&2\end{pmatrix}
=
\begin{pmatrix}4&6\\6&13\end{pmatrix}.
\]
其特征多项式
\(\lambda^2-17\lambda+16=0\),特征值
\[
\sigma_1^2=\frac{17+3\sqrt{17}}2,\quad
\sigma_2^2=\frac{17-3\sqrt{17}}2,
\]
奇异值
\[
s_1=\sqrt{\frac{17+3\sqrt{17}}2},\quad
s_2=\sqrt{\frac{17-3\sqrt{17}}2}.
\]

对 \(\sigma_1^2\) 的特征向量可取
\[
v_1=
\begin{pmatrix}
\displaystyle\frac{3}{\,1+\sqrt{17}\,}\\[4pt]
1
\end{pmatrix},\quad
\|v_1\|^2
=\frac{9}{(1+\sqrt{17})^2}+1
=\frac{34+2\sqrt{17}}{(1+\sqrt{17})^2},
\]
归一化
\(\displaystyle \vv_1=\frac1{\|v_1\|}v_1\).

再取
\[
\vv_2=
\frac1{\sqrt{1+\beta^2}}
\begin{pmatrix}-1\\ \beta\end{pmatrix},\quad
\beta=\frac{3}{1+\sqrt{17}},
\]
使 \(\{\vv_1,\vv_2\}\) 成为 \(\mathbb R^2\) 的标准正交基。

由
\[
\ww_k=\frac1{s_k}A\vv_k,\quad k=1,2,
\]
即可得单位向量 \(\ww_1,\ww_2\)。于是
\[
A=s_1\,\ww_1\vv_1^*+s_2\,\ww_2\vv_2^*
\]
就是施密特分解(矩阵形式即 SVD)。

(2)\(A=\begin{pmatrix} 7 & 1 & 0 \\ 0 & 0 & 5 \\ 5 & 0 & 5 \end{pmatrix}\).

计算
\[
A^*A=
\begin{pmatrix}
74&7&25\\
7&1&0\\
25&0&50
\end{pmatrix}.
\]
求其三个特征值 \(\sigma_1^2,\sigma_2^2,\sigma_3^2\),令
\[
(A^*A-\sigma_k^2 I)v_k=0
\]
得单位特征向量 \(\vv_k\);再令
\[
\ww_k=\frac1{\sigma_k}A\vv_k,\quad k=1,2,3,
\]
即可写出
\[
A=\sum_{k=1}^3\sigma_k \ww_k\vv_k^*.
\]
(具体数值根较复杂,通常保留为“解特征值—特征向量”这一计算过程即可。)

(3)\(A=\begin{pmatrix} 1 & 1 & 0 \\ 1 & 2 & 2 \\ 0 & -1 & 1 \end{pmatrix}\).

同样先算
\[
A^*A=
\begin{pmatrix}
2&3&2\\
3&6&3\\
2&3&5
\end{pmatrix},
\]
求其三个特征值 \(\sigma_k^2\) 与单位特征向量 \(\vv_k\),再令
\(\ww_k=\frac1{\sigma_k}A\vv_k\),得到
\[
A=\sum_{k=1}^3 \sigma_k \ww_k\vv_k^*.
\]

(本题主要是练习“通过 \(A^*A\) 的谱\(\Rightarrow\) 施密特分解”的套路,具体根式写出即可。)

---

\textbf{3.3}\quad

已知
\[
A=W\Sigma V^*.
\]

1)\(A^*\) 的 SVD:

\[
A^*=(W\Sigma V^*)^* = V\Sigma W^*.
\]
这已经是一个 SVD:奇异值仍是 \(\Sigma\) 的对角元,酉矩阵分别是 \(V\) 与 \(W\)。

2)\(A^{-1}\) 的 SVD(\(A\) 可逆,故 \(\Sigma\) 无零奇异值):

\[
A^{-1}=(W\Sigma V^*)^{-1}
=V\Sigma^{-1}W^*,
\]
其中
\(\Sigma^{-1}=\operatorname{diag}(1/s_1,\dots,1/s_n)\)。
因此 \(A^{-1}\) 的奇异值是 \(1/s_k\),对应的酉矩阵为 \(V,W\)。

---

\textbf{3.4}\quad

\emph{a)} \(A=\begin{pmatrix}-3&1\\6&-2\\6&-2\end{pmatrix}\).

\[
A^*A=
\begin{pmatrix}
81&-27\\
-27&9
\end{pmatrix}
=9
\begin{pmatrix}
9&-3\\-3&1
\end{pmatrix}.
\]
后面那矩阵秩为 1,故特征值为 \(0\) 与其迹 \(10\)。于是
\[
s_1=\sqrt{9\cdot10}=3\sqrt{10},\quad s_2=0.
\]

对特征值 \(10\) 的特征向量:
\[
\begin{pmatrix}9&-3\\-3&1\end{pmatrix}
\begin{pmatrix}x\\yy\end{pmatrix}=10\begin{pmatrix}x\\yy\end{pmatrix}
\Rightarrow -x-3y=0,
\]
可取 \(v_1=( -3,1)^T\),归一化
\(\displaystyle \vv_1=\frac1{\sqrt{10}}(-3,1)^T\)。

选取与之正交的
\(\displaystyle \vv_2=\frac1{\sqrt{10}}(1,3)^T\),
则
\[
V=\begin{pmatrix}
-3/\sqrt{10}&1/\sqrt{10}\\
1/\sqrt{10}&3/\sqrt{10}
\end{pmatrix}.
\]

再令
\[
\ww_1=\frac1{s_1}A\vv_1
=\frac1{3\sqrt{10}}A\frac1{\sqrt{10}}
\begin{pmatrix}-3\\1\end{pmatrix}
=
\frac1{30}
\begin{pmatrix}
10\\-20\\-20
\end{pmatrix}
=
\begin{pmatrix}
1/3\\-2/3\\-2/3
\end{pmatrix}.
\]
补全 \(\{\ww_1\}\) 为 \(\mathbb R^3\) 的正交基得到酉矩阵 \(W\),例如可取两个与 \(\ww_1\) 正交的单位向量 \(\ww_2,\ww_3\)(任选)。

于是
\[
\Sigma=
\begin{pmatrix}
3\sqrt{10}&0\\
0&0\\
0&0
\end{pmatrix},\quad
A=W\Sigma V^*.
\]

\emph{b)} \(A=\begin{pmatrix}3&2&2\\2&3&-2\end{pmatrix}\).

\[
A^*A=
\begin{pmatrix}
13&12&2\\
12&13&-2\\
2&-2&8
\end{pmatrix}.
\]
求其特征值 \(\sigma_1^2,\sigma_2^2,\sigma_3^2\) 与相应单位特征向量 \(\vv_k\),构成酉矩阵
\(V=[\vv_1\ \vv_2\ \vv_3]\)。再令
\[
\ww_k=\frac1{\sigma_k}A\vv_k, \quad k=1,2,
\]
以及把 \(\{\ww_1,\ww_2\}\) 补全成 \(\mathbb R^2\) 的正交基得到 \(W\)。则
\[
\Sigma=
\begin{pmatrix}
\sigma_1&0&0\\
0&\sigma_2&0
\end{pmatrix},\quad
A=W\Sigma V^*.
\]

(同样,这题重在过程:通过 \(A^*A\) 的谱构造 \(V,\Sigma,W\)。)

---

\textbf{3.5}\quad

上一题 3.2(1) 已经得到矩阵
\[
A=\begin{pmatrix}2&3\\0&2\end{pmatrix}
\]
的 SVD:奇异值为
\(s_1\ge s_2>0\),右奇异向量为 \(\vv_1,\vv_2\),左奇异向量为 \(\ww_1,\ww_2\)。

SVD 形式:
\[
A=W\Sigma V^*,\quad
\Sigma=\operatorname{diag}(s_1,s_2).
\]

a) \(\displaystyle \max_{\|\xx\|\le1}\|A\xx\|=s_1\),最大值在
\[
\xx=\pm \vv_1
\]
处取得。

b) \(\displaystyle \min_{\|\xx\|=1}\|A\xx\|=s_2\),最小值在
\[
\xx=\pm \vv_2
\]
处取得。

c) 单位球 \(B\) 在 \(V\) 基下仍是单位球;\(\Sigma\) 把单位圆拉伸成椭圆
\(\{(s_1 y_1,s_2 y_2):y_1^2+y_2^2\le1\}\),再由酉矩阵 \(W\) 旋转(或正交变换)到标准坐标系。  
几何上:\(A(B)\) 是以原点为中心、长短轴分别为 \(s_1,s_2\),且主轴方向分别是 \(\ww_1,\ww_2\) 的椭圆。

---

\textbf{3.6}\quad

设 \(A\) 方阵,SVD 为
\[
A=W\Sigma V^*,\quad
\Sigma=\operatorname{diag}(s_1,\dots,s_n),\ s_k\ge0.
\]
则
\[
|\det A|
=|\det W|\cdot\det\Sigma\cdot|\det V^*|
=\prod_{k=1}^n s_k,
\]
因为 \(W,V\) 酉,行列式的模为 1。

又
\[
|A|=(A^*A)^{1/2}
=V\Sigma^2 V^*{}^{1/2}
=V\Sigma V^*.
\]
因此
\[
\det|A|
=\det(V\Sigma V^*)
=\det\Sigma
=\prod_{k=1}^n s_k.
\]
于是 \(|\det A|=\det|A|\)。

---

\textbf{3.7}\quad

a) 错。奇异值是 \(|A|\)(或 \(A^*A\))的特征值的平方根,一般不是 \(A\) 自身的特征值。

b) 错。奇异值的\textbf{平方}是 \(A^*A\) 的特征值;奇异值本身是这些特征值的非负平方根。

c) 对。若 \(s\) 是 \(A\) 的奇异值,则 \(s^2\) 是 \(A^*A\) 的特征值。对 \(cA\),
\[
(cA)^*(cA)=|c|^2 A^*A,
\]
故其特征值为 \(|c|^2s^2\),对应奇异值为 \(|c|s\)。

d) 对。定义上,奇异值是自伴随半正定算子 \(|A|\) 的特征值,故非负。

e) 对。若 \(A=A^*\),其奇异值是 \(|A|\) 的特征值。谱分解给出 \(|A|\) 与 \(A\) 具有同一组特征向量,而 \(|A|\) 的特征值是 \(A\) 特征值的绝对值;但对自伴随算子,我们通常称“奇异值”就是 \(|\lambda_k|\)。如果按题意采用该约定,则“自伴随矩阵的奇异值与其特征值相等”应理解为“模相等且仅有符号差别”。在本书中,一般表述为:自伴随矩阵的奇异值等于其特征值的绝对值。

---

\textbf{3.8}\quad

设 \(A\in M_{m\times n}\) 的 SVD:
\[
A=W\Sigma V^*,\quad
\Sigma=
\begin{pmatrix}
\sigma_1\\ &\ddots\\ &&\sigma_r\\ &&&0
\end{pmatrix},
\ \sigma_k>0.
\]

则
\[
A^*A=V\Sigma^2V^*,\quad
AA^*=W\Sigma^2W^*.
\]
因此 \(\sigma_1^2,\dots,\sigma_r^2\) 是 \(A^*A\) 与 \(AA^*\) 的共同非零特征值(计重数)。

零特征值的重数:  
\(\dim\text{Ker }(A^*A)=\dim\text{Ker } A\),  
\(\dim\text{Ker }(AA^*)=\dim\text{Ker } A^*\)。  
由秩–零度定理,
\[
\dim\text{Ker } A = n-\text{rank } A,\quad
\dim\text{Ker } A^* = m-\text{rank } A.
\]
所以当且仅当 \(m=n\)(即 \(A\) 方阵)时,二者的零特征值重数相同。

---

\textbf{3.9}\quad

设 \(A\) 的最大奇异值为 \(s\),即 \(\|A\|=s\)。设 \(\lambda\) 是 \(A\) 的一个特征值,\(|\lambda|\) 在所有特征值中最大。取对应特征向量 \(x\ne0\),则
\[
Ax=\lambda x,\quad
\|Ax\|=|\lambda|\|x\|.
\]
而算子范数定义给出
\[
\|Ax\|\le\|A\|\|x\|=s\|x\|.
\]
故
\[
|\lambda|\le s.
\]

---

\textbf{3.10}\quad

这题与 3.1 相同。由 SVD 表示
\[
A=\sum_{k=1}^r s_k\ww_k\vv_k^*
\]
可知 \(\text{Ran } A=\operatorname{span}\{\ww_1,\dots,\ww_r\}\),其维数为 \(r\),即 \(\text{rank } A=r\)。而 \(r\) 正是非零奇异值的数目(计重数)。

---

\textbf{3.11}\quad

设 \(A\) 的非零奇异值为 \(\sigma_1\ge\sigma_2\ge\cdots\ge\sigma_r>0\)。

算子范数:
\[
\|A\|=\sigma_1.
\]
Frobenius 范数:
\[
\|A\|_2^2=\sum_{i,j}|a_{ij}|^2
=\sum_{k=1}^r\sigma_k^2.
\]

若 \(\text{rank } A=1\),则只有一个非零奇异值 \(\sigma_1\),于是
\[
\|A\|_2=\sqrt{\sigma_1^2}=\sigma_1=\|A\|.
\]

反之,若 \(\|A\|=\|A\|_2\),则
\[
\sigma_1^2
=\|A\|^2
=\|A\|_2^2
=\sum_{k=1}^r\sigma_k^2
\ge\sigma_1^2+\sigma_2^2,
\]
从而 \(\sigma_2=0\)。同理所有 \(\sigma_k(k\ge2)\) 必为 0,因此只有一个非零奇异值,即 \(\text{rank } A=1\)。

---

\textbf{3.12}\quad

\(A=\begin{pmatrix}2&-3\\0&2\end{pmatrix}\).

设 \(A=W\Sigma V^*\) 是其 SVD,\(\Sigma=\operatorname{diag}(s_1,s_2)\),\(s_1\ge s_2>0\)。  
单位球的逆像集合:
\[
\{\xx\in\mathbb R^2:\|A\xx\|\le1\}
=\{\xx:\|W\Sigma V^*\xx\|\le1\}.
\]
因 \(W\) 酉,
\[
\|W\Sigma V^*\xx\|
=\|\Sigma V^*\xx\|.
\]
记 \(\yy=V^*\xx\)(酉变换保持范数、体积),则条件等价于
\[
\|\Sigma\yy\|^2
=s_1^2 y_1^2+s_2^2 y_2^2\le1.
\]
因此 \(\yy\) 组成以原点为中心、轴向为坐标轴的闭椭圆
\[
E=\Bigl\{(y_1,y_2):s_1^2y_1^2+s_2^2y_2^2\le1\Bigr\}.
\]
于是
\[
\{\xx:\|A\xx\|\le1\}
=V E.
\]

几何描述:这是一个以原点为中心的椭圆,它是先对坐标轴方向按比例 \(1/s_1,1/s_2\) 缩放得到的椭圆,再经正交变换 \(V\) 旋转而成;主轴方向为右奇异向量 \(\vv_1,\vv_2\),轴长分别为 \(1/s_1,1/s_2\)。


只写解答内容,便于直接嵌入习题册。

---

\textbf{4.1}

a) \(\displaystyle A=\begin{pmatrix}4&0\\1&3\end{pmatrix}\).

\[
A^TA=\begin{pmatrix}4&1\\0&3\end{pmatrix}\begin{pmatrix}4&0\\1&3\end{pmatrix}
=\begin{pmatrix}17&3\\3&9\end{pmatrix}.
\]
特征多项式
\[
\lambda^2-26\lambda+144=0,\quad
\lambda_{1,2}=13\pm\sqrt{25}=18,8.
\]
故奇异值
\[
s_1=\sqrt{18}=3\sqrt2,\quad s_2=\sqrt8=2\sqrt2.
\]
算子范数
\[
\|A\|=s_1=3\sqrt2,\quad
\|A^{-1}\|=\frac1{s_2}=\frac1{2\sqrt2},
\]
条件数
\[
\kappa(A)=\|A\|\,\|A^{-1}\|=\frac{3\sqrt2}{2\sqrt2}=\frac32.
\]

一个达到估计等号的例子(按书上的构造):

取 \(A=W\Sigma V^*\) 的 SVD。令  
\[
\bb=\vv_1,\quad \Delta\bb=\alpha\,\vv_2,
\]
其中 \(\vv_1,\vv_2\) 是右奇异向量,\(\alpha\neq0\) 为任意标量。设
\[
\xx=A^{-1}\bb,\quad \xx+\Delta\xx=A^{-1}(\bb+\Delta\bb),
\]
则
\[
\Delta\xx=A^{-1}\Delta\bb.
\]
利用 \(A\vv_k=s_k\ww_k,\ A^{-1}\ww_k=\frac1{s_k}\vv_k\) 得
\[
\xx=A^{-1}\bb=\frac1{s_1}\vv_1,\quad
\Delta\xx=A^{-1}\Delta\bb=\frac{\alpha}{s_2}\vv_2.
\]
于是
\[
\frac{\|\Delta\xx\|}{\|\xx\|}
=\frac{|\alpha|/s_2}{1/s_1}
=\frac{s_1}{s_2}|\alpha|
=\kappa(A)\,\frac{\|\Delta\bb\|}{\|\bb\|},
\]
因为 \(\|\bb\|=\|\vv_1\|=1,\ \|\Delta\bb\|=|\alpha|\)。  
对本题矩阵,只要给出满足上述关系的一组 \(\bb,\Delta\bb\)(例如直接在正文中说“取 \(\bb\) 为最大奇异值方向的右奇异向量 \(\vv_1\),\(\Delta\bb\) 为 \(\vv_2\) 方向的向量”即可,不必把 \(V\) 显式算出)。

b) \(\displaystyle A=\begin{pmatrix}5&3\\-3&3\end{pmatrix}\).

\[
A^TA=\begin{pmatrix}25+9&15-9\\15-9&9+9\end{pmatrix}
=\begin{pmatrix}34&6\\6&18\end{pmatrix}.
\]
特征多项式
\[
\lambda^2-52\lambda+576=0,\quad
\lambda_{1,2}=26\pm10\sqrt5.
\]
奇异值
\[
s_1=\sqrt{26+10\sqrt5},\quad
s_2=\sqrt{26-10\sqrt5}.
\]
因此
\[
\|A\|=s_1,\quad
\|A^{-1}\|=\frac1{s_2},\quad
\kappa(A)=\frac{s_1}{s_2}.
\]

---

\textbf{4.2}

设 \(A\) 正常,谱分解
\[
A=\sum_{k=1}^n \lambda_k P_k,
\]
其中 \(P_k\) 为正交投影,互相正交,\(\sum P_k=I\)。则
\[
A^*A=\sum_{k}|\lambda_k|^2 P_k,
\]
因而
\[
|A|=(A^*A)^{1/2}=\sum_k |\lambda_k| P_k.
\]
所以 \(|A|\) 的特征值是 \(|\lambda_1|,\dots,|\lambda_n|\)(计重数),这正是 \(A\) 的奇异值。

---

\textbf{4.3}

\[
A=\begin{pmatrix}2&1&1\\1&2&1\\1&1&2\end{pmatrix}.
\]

注意
\[
A=I+J,\quad
J=\begin{pmatrix}1&1&1\\1&1&1\\1&1&1\end{pmatrix}.
\]
矩阵 \(J\) 的特征值:3(沿 \((1,1,1)^T\)),以及 0(在其正交补上)。  
故 \(A\) 的特征值:
\[
3+1=4\quad(\text{一次}),
\quad 0+1=1\quad(\text{重数 }2).
\]
\(A\) 是实对称矩阵,自伴随,因此奇异值等于特征值的绝对值:
\[
s_1=4,\quad s_2=s_3=1.
\]
算子范数
\[
\|A\|=4,\quad
\|A^{-1}\|=\frac1{1}=1,
\]
条件数
\[
\kappa(A)=4.
\]

与提示问题的对应:

a) 正交投影 \(P_E\) 的特征值只有 1(在 \(E\) 上)和 0(在 \(E^\perp\) 上),因而奇异值也只有 1 和 0。  

b) 跨越 \((1,1,1)^T\) 的子空间的零空间就是它的正交补;该一维子空间的正交投影矩阵正是 \(\frac13J\),其正交补的投影为 \(I-\frac13J\)。  

c) 若 \(T\) 的特征值为 \(\mu_k\),则 \(aT+bI\) 的特征值为 \(a\mu_k+b\)。  
这里 \(A=I+J\),即 \(a=1,b=1\),由 \(J\) 的谱立即得到 \(A\) 的谱。

---

\textbf{4.4}

约简 SVD:
\[
A=\tilde W\tilde\Sigma \tilde V^*,
\]
其中 \(\tilde\Sigma\) 为 \(r\times r\) 对角矩阵,\(r=\text{rank } A\),\(\tilde W\) 和 \(\tilde V\) 的列正交,列数为 \(r\)。

对任意 \(\xx\),
\[
A\xx=\tilde W\tilde\Sigma\tilde V^*\xx.
\]
右侧显然属于 \(\text{Ran }\tilde W\),故 \(\text{Ran } A\subset\text{Ran }\tilde W\)。  
另一方面,\(\tilde\Sigma\) 可逆(对 \(r\times r\) 部分),故任给 \(\tilde W\) 的列向量 \(\ww_k\),有
\[
\ww_k=\tilde W e_k
=A\bigl(\tilde V\tilde\Sigma^{-1}e_k\bigr)\in\text{Ran } A.
\]
于是 \(\text{Ran }\tilde W\subset\text{Ran } A\)。两边相等:
\[
\text{Ran } A=\text{Ran }\tilde W.
\]

取伴随:
\[
A^*=\tilde V\tilde\Sigma\tilde W^*.
\]
同理可得
\[
\text{Ran } A^*=\text{Ran }\tilde V.
\]

---

\textbf{4.5}

若 \(A=W\Sigma V^*\) 为(完整)SVD,\(\Sigma=\operatorname{diag}(s_1,\dots,s_r,0,\dots)\),则
\[
A^+=V\Sigma^+W^*,
\]
其中
\[
\Sigma^+=\operatorname{diag}\Bigl(\frac1{s_1},\dots,\frac1{s_r},0,\dots\Bigr).
\]

若用约简 SVD \(A=\tilde W\tilde\Sigma\tilde V^*\),则
\[
A^+=\tilde V\tilde\Sigma^{-1}\tilde W^*.
\]

---

\textbf{4.6}

用 SVD 证明即可。

设
\[
A=W\Sigma V^*
\]
为(可能约简的)SVD,\(\Sigma=\operatorname{diag}(s_1,\dots,s_r,0,\dots)\)。则
\[
A^*A=V\Sigma^2V^*.
\]
于是
\[
A^*A+\varepsilon I
=V(\Sigma^2+\varepsilon I)V^*,
\]
其中
\(\Sigma^2+\varepsilon I=\operatorname{diag}(s_1^2+\varepsilon,\dots,s_r^2+\varepsilon,\varepsilon,\dots)\),
可逆,其逆为
\[
(\Sigma^2+\varepsilon I)^{-1}
=\operatorname{diag}\Bigl(\frac1{s_1^2+\varepsilon},\dots,
\frac1{s_r^2+\varepsilon},\frac1{\varepsilon},\dots\Bigr).
\]

于是
\[
(A^*A+\varepsilon I)^{-1}A^*
=V(\Sigma^2+\varepsilon I)^{-1}V^*V\Sigma W^*
=V(\Sigma^2+\varepsilon I)^{-1}\Sigma W^*,
\]
而
\[
(\Sigma^2+\varepsilon I)^{-1}\Sigma
=\operatorname{diag}\Bigl(\frac{s_1}{s_1^2+\varepsilon},\dots,
\frac{s_r}{s_r^2+\varepsilon},0,\dots\Bigr).
\]

当 \(\varepsilon\to0^+\) 时,
\[
\frac{s_k}{s_k^2+\varepsilon}\to\frac1{s_k}\quad(k\le r),\quad
0\text{ 元素保持为 }0.
\]
故
\[
\lim_{\varepsilon\to0^+}(A^*A+\varepsilon I)^{-1}A^*
=V\Sigma^+W^*=A^+.
\]

同理,
\[
AA^*=W\Sigma^2W^*,\quad
(AA^*+\varepsilon I)^{-1}A
=W(\Sigma^2+\varepsilon I)^{-1}\Sigma V^*,
\]
同样的逐对角元极限给出
\[
\lim_{\varepsilon\to0^+}A^*(AA^*+\varepsilon I)^{-1}
=V\Sigma^+W^*=A^+.
\]

于是
\[
A^+ = \lim_{\varepsilon\to 0^+} (A^*A + \varepsilon I)^{-1} A^*
= \lim_{\varepsilon\to 0^+} A^*(AA^* + \varepsilon I)^{-1}.
\]


只写解答内容,方便直接放进解答稿。

---

\textbf{6.1}

新基为 \(\vv_1=\ee_2,\ \vv_2=\ee_1\)。  
设从新基到标准基的过渡矩阵为
\[
C=\begin{pmatrix} | & | \\ \vv_1 & \vv_2 \\ | & | \end{pmatrix}
=\begin{pmatrix} 0&1\\1&0 \end{pmatrix},\quad C^{-1}=C.
\]
则 \(R_\alpha\) 在基 \((\vv_1,\vv_2)\) 下的矩阵为
\[
[R_\alpha]_{\{\vv\}}=C^{-1}R_\alpha C
=\begin{pmatrix}0&1\\1&0\end{pmatrix}
\begin{pmatrix}\cos\alpha&-\sin\alpha\\\sin\alpha&\cos\alpha\end{pmatrix}
\begin{pmatrix}0&1\\1&0\end{pmatrix}
=
\begin{pmatrix}\cos\alpha&\sin\alpha\\ -\sin\alpha&\cos\alpha\end{pmatrix}.
\]

---

\textbf{6.2}

取
\[
V(t)=R_{t\alpha},\quad t\in[0,1].
\]
则每个 \(V(t)\) 都是可逆(正交)矩阵,且
\[
V(0)=R_0=I_2,\quad V(1)=R_\alpha.
\]
因此 \(I_2\) 可以通过一族可逆矩阵 \(V(t)\) 连续变换为 \(R_\alpha\)。

---

\textbf{6.3}

由定理 5.1 可知:若 \(U\) 是 \(\RR^n\) 中的正交算子且 \(\det U=1\),则存在一个正交基,使得 \(U\) 在该基下的矩阵分块对角为若干个 \(2\times2\) 旋转块 \(R_{\phi_k}\) 与一个单位块 \(I_{n-2k}\) 的直和;即
\[
U\sim \operatorname{diag}(R_{\phi_1},\dots,R_{\phi_k},I_{n-2k}).
\]

对每个 \(R_{\phi_k}\),由 6.2 存在一条可逆连续路径(在对应的二维子空间内)
\[
t\mapsto R_{t\phi_k},\quad t\in[0,1],
\]
把 \(I_2\) 变为 \(R_{\phi_k}\)。在 \(\RR^n\) 中,可把这条路径扩展为
\[
t\mapsto \operatorname{diag}(I,\dots,I,R_{t\phi_k},I,\dots,I),
\]
它始终是可逆矩阵。依次对各个二维不变子空间进行这样的变形,就得到一条从 \(I_n\) 到 \(U\) 的连续可逆矩阵路径 \(V(t)\)。

形式上:将 \(U\) 写为有限个平面旋转(初等等旋转)的乘积
\[
U=R_1R_2\cdots R_m.
\]
对每个 \(R_j\) 取一条从 \(I_n\) 到 \(R_j\) 的连续可逆路径 \(R_j(t)\)(如上构造),再按时间分段拼接这些路径,得到从 \(I_n\) 到 \(U\) 的连续可逆路径。

---

\textbf{6.4}

\(A\) 为正定埃尔米特矩阵,\(A=A^*>0\)。由谱定理,
\[
A=U D U^*,
\]
其中 \(U\) 酉,\(D=\operatorname{diag}(\lambda_1,\dots,\lambda_n)\) 为实对角矩阵,且 \(\lambda_k>0\)。

先在对角矩阵空间里把 \(I_n\) 连续变为 \(D\):令
\[
D(t)=\operatorname{diag}\bigl((1-t)+t\lambda_1,\dots,(1-t)+t\lambda_n\bigr),\quad t\in[0,1].
\]
因每个 \((1-t)+t\lambda_k>0\),所以 \(D(t)\) 始终可逆;且
\[
D(0)=I_n,\quad D(1)=D.
\]

再用同一个酉矩阵夹逼,定义
\[
V(t)=U D(t) U^*,\quad t\in[0,1].
\]
则 \(V(t)\) 连续且可逆(为正定埃尔米特矩阵),并且
\[
V(0)=U I U^*=I_n,\quad V(1)=UDU^*=A.
\]
于是 \(I_n\) 可以通过可逆矩阵连续变换为 \(A\)。

---

\textbf{6.5}

极分解给出:任一可逆矩阵 \(T\) 可以唯一写成
\[
T=UP,
\]
其中 \(U\) 是正交(或酉)矩阵,\(P\) 是正定埃尔米特矩阵。

若 \(T\) 可逆,则 \(\det U=\pm1\)。定理 6.3(这里的“仅当”方向)要证明的是:

> 一个基 \(B\) 与标准正交基 \((\ee_1,\dots,\ee_n)\) 具有相同方向,当且仅当存在一条由可逆矩阵组成的连续路径把 \(I_n\) 变为把 \(\ee_k\) 送到 \(B\) 的那个变换矩阵。

“如果”方向已经在书中证明。  
在“仅当”方向里,我们从给定的可逆矩阵 \(T\) 出发,需要构造这样的路径。  

由极分解 \(T=UP\):

- 由 6.3,存在一条连续的可逆路径 \(V_1(t)\) 从 \(I_n\) 到 \(U\);
- 由 6.4,存在一条连续的可逆路径 \(V_2(t)\) 从 \(I_n\) 到 \(P\)。

于是可定义一条从 \(I_n\) 到 \(T\) 的连续可逆路径,例如
\[
W(t)=
\begin{cases}
V_2(2t), & 0\le t\le \tfrac12,\\[2mm]
V_1(2t-1)\,P, & \tfrac12\le t\le 1,
\end{cases}
\]
则
\[
W(0)=I_n,\quad W(1)=UP=T.
\]

因此,任何可逆矩阵 \(T\) 都可以通过可逆矩阵连续变换得到;结合第 6.2 节关于基与方向的形式定义,这完成了定理 6.3 “仅当” 部分的证明:  
当且仅当存在这样的可逆连续路径时,这个基与标准基具有相同的方向。





\end{exer}








\section{第七章答案}

\begin{exer}


下面只写解答内容,方便你直接嵌入习题解答。

---

\textbf{1.1}

将
\[
L(\xx,\yy)=x_1y_1+2x_1y_2+14x_1y_3-5x_2y_1+2x_2y_2-3x_2y_3
+8x_3y_1+19x_3y_2-2x_3y_3
\]
与一般形式
\[
L(\xx,\yy)=\sum_{i,j=1}^3 a_{ij}x_i y_j
\]
比较,可得
\[
A=(a_{ij})=
\begin{pmatrix}
1 & 2 & 14\\
-5 & 2 & -3\\
8 & 19 & -2
\end{pmatrix}.
\]

---

\textbf{1.2}

在 \(\RR^2\) 上,令
\[
L(\xx,\yy)=\det[\xx,\yy],\quad
\xx=(x_1,x_2)^T,\ \yy=(y_1,y_2)^T.
\]
则
\[
L(\xx,\yy)=
\begin{vmatrix}
x_1 & y_1\\
x_2 & y_2
\end{vmatrix}
=x_1y_2-x_2y_1
=0\cdot x_1y_1+1\cdot x_1y_2+(-1)\cdot x_2y_1+0\cdot x_2y_2.
\]
因此
\[
A=
\begin{pmatrix}
0 & 1\\
-1 & 0
\end{pmatrix},
\quad
L(\xx,\yy)=(A\xx,\yy)=\xx^T A^T\yy.
\]

---

\textbf{1.3}

有
\[
Q[\xx]=x_1^2+2x_1x_2-3x_1x_3-9x_2^2+6x_2x_3+13x_3^2.
\]
记 \(A=(a_{ij})\) 为对称矩阵,使
\[
Q[\xx]=\xx^T A\xx
=\sum_{i=1}^3 a_{ii}x_i^2
+2\sum_{1\le i<j\le 3} a_{ij}x_i x_j.
\]
与系数比较:

- \(x_1^2\) 的系数为 \(a_{11}=1\);
- \(x_2^2\) 的系数为 \(a_{22}=-9\);
- \(x_3^2\) 的系数为 \(a_{33}=13\);
- \(x_1x_2\) 的系数为 \(2a_{12}=2\),故 \(a_{12}=a_{21}=1\);
- \(x_1x_3\) 的系数为 \(2a_{13}=-3\),故 \(a_{13}=a_{31}=-\tfrac32\);
- \(x_2x_3\) 的系数为 \(2a_{23}=6\),故 \(a_{23}=a_{32}=3\)。

于是
\[
A=
\begin{pmatrix}
1 & 1 & -\tfrac32\\
1 & -9 & 3\\
-\tfrac32 & 3 & 13
\end{pmatrix}.
\]

---

\textbf{1.4}

要证:若在 \(\C^n\) 上的内积 \((\cdot,\cdot)\) 中,对任意 \(\xx\in\C^n\) 有
\((A\xx,\xx)\in\RR\),则 \(A=A^*\);等价地要证
\[
(A\xx,\yy)=(\xx,A^*\yy),\quad \forall\,\xx,\yy\in\C^n.
\]

考虑
\[
f(z)=(A(\xx+z\yy),\xx+z\yy),\quad z\in\C.
\]
按线性与共轭线性展开(约定内积对第一变量线性,对第二变量共轭线性):
\[
\begin{aligned}
f(z)
&=(A\xx,\xx)+z(A\yy,\xx)+\bar z(A\xx,\yy)+|z|^2(A\yy,\yy).
\end{aligned}
\]
其中 \((A\xx,\xx),(A\yy,\yy)\) 已知为实数。题设条件是:对所有 \(z\in\C\),\(f(z)\) 都是实数。

令 \(z=t\in\RR\)。则 \(|z|^2=t^2\),且
\[
f(t)=(A\xx,\xx)+t\bigl((A\yy,\xx)+(A\xx,\yy)\bigr)+t^2(A\yy,\yy)\in\RR,\ \forall t\in\RR.
\]
因为两端其他项都实,故一次项系数
\((A\yy,\xx)+(A\xx,\yy)\) 必为实数。  

再令 \(z=it\)(\(t\in\RR\)),得
\[
f(it)=(A\xx,\xx)+it(A\yy,\xx)-it(A\xx,\yy)+t^2(A\yy,\yy)\in\RR.
\]
同理,一次项系数 \(i\bigl((A\yy,\xx)-(A\xx,\yy)\bigr)\) 也必须为实,这等价于
\((A\yy,\xx)-(A\xx,\yy)\) 为纯虚数。与上一结论合并,可推出
\[
(A\yy,\xx)=\overline{(A\xx,\yy)}.
\]
利用内积的共轭对称性 \((u,v)=\overline{(v,u)}\),
\[
(A\yy,\xx)=\overline{(\xx,A\yy)}.
\]
于是
\[
\overline{(\xx,A\yy)}=\overline{(A\xx,\yy)}
\;\Longrightarrow\;
(\xx,A\yy)=(A\xx,\yy).
\]
另一方面
\[
(\xx,A\yy)=(A^*\xx,\yy)
\]
按伴随的定义应成立;将两式比较,可得
\[
(A\xx,\yy)=(A^*\xx,\yy),\quad\forall\,\xx,\yy.
\]
由内积的非退化性(对所有 \(\yy\) 都相等,意味着向量相等),从而
\[
A\xx=A^*\xx,\quad\forall\,\xx\in\C^n,
\]
即 \(A=A^*\)。这就证明了引理 1.1。


只写解答内容,方便直接放进习题解答里。

---

\textbf{2.1}

对应二次型
\[
Q[\xx]=\xx^TA\xx,\quad
A=\begin{pmatrix}1&2&1\\2&3&2\\1&2&1\end{pmatrix}.
\]

\textbf{(1) 配方法}

记 \(\xx=(x_1,x_2,x_3)^T\)。则
\[
\begin{aligned}
Q[\xx]
&=x_1^2+3x_2^2+x_3^2+4x_1x_2+2x_1x_3+4x_2x_3\\
&=(x_1+2x_2+x_3)^2.
\end{aligned}
\]
令
\[
y_1=x_1+2x_2+x_3,\quad y_2=x_2,\quad y_3=x_3,
\]
则
\[
Q[\xx]=y_1^2,\quad
[y]=\begin{pmatrix}y_1\\y_2\\y_3\end{pmatrix},
\]
在 \((y_1,y_2,y_3)\) 坐标下的矩阵为
\[
D=\begin{pmatrix}1&0&0\\0&0&0\\0&0&0\end{pmatrix}.
\]

\textbf{(2) 行运算法}

对增广矩阵作行\emph{列}运算:
\[
(A\mid I)
=\left(
\begin{array}{ccc|ccc}
1&2&1&1&0&0\\
2&3&2&0&1&0\\
1&2&1&0&0&1
\end{array}
\right).
\]
只对左块做行运算、右块做同样列运算,得到对角形:
\[
\left(
\begin{array}{ccc|ccc}
1&0&0&1&-2&-1\\
0&0&0&0&1&0\\
0&0&0&0&0&1
\end{array}
\right)
=
\left(
\begin{array}{ccc|ccc}
1&0&0\\
0&0&0\\
0&0&0
\end{array}
\ \Bigg|\ 
S
\right).
\]
于是
\[
D=S^TAS=\text{diag }(1,0,0),
\]
与配方法结果一致。

\textbf{(3) 正定性}

特征值为对角化后矩阵的对角元 \(1,0,0\),因此并非全部正,\(A\) 不是正定矩阵(它是半正定的)。

---

\textbf{2.2}

\[
A=\begin{pmatrix}
2&1&1\\
1&2&1\\
1&1&2
\end{pmatrix}
\]
是实对称矩阵,可正交对角化。

\textbf{特征值与特征向量}

注意到
\[
A\begin{pmatrix}1\\1\\1\end{pmatrix}
=\begin{pmatrix}4\\4\\4\end{pmatrix}
=4\begin{pmatrix}1\\1\\1\end{pmatrix},
\]
故 \(\lambda_1=4\),特征向量 \(\vv_1=(1,1,1)^T\)。

求其它特征值,可在与 \((1,1,1)\) 正交的平面(和为零的平面)上工作。取
\[
\vv_2=(1,-1,0)^T,\quad
\vv_3=(1,1,-2)^T,
\]
则 \(1-1+0=0,\ 1+1-2=0\),二者与 \(\vv_1\) 正交。计算
\[
A\vv_2=\begin{pmatrix}1\\-1\\0\end{pmatrix}=\vv_2,\quad
A\vv_3=\begin{pmatrix}0\\0\\-3\end{pmatrix}
=1\cdot\vv_3.
\]
因此 \(\lambda_2=\lambda_3=1\)。于是特征值为 \(4,1,1\)。

\textbf{正交归一化}

\[
\|\vv_1\|=\sqrt3,\quad
\|\vv_2\|=\sqrt2,\quad
\|\vv_3\|=\sqrt6.
\]
归一化:
\[
\uu_1=\frac1{\sqrt3}\begin{pmatrix}1\\1\\1\end{pmatrix},\quad
\uu_2=\frac1{\sqrt2}\begin{pmatrix}1\\-1\\0\end{pmatrix},\quad
\uu_3=\frac1{\sqrt6}\begin{pmatrix}1\\1\\-2\end{pmatrix}.
\]
这些向量两两正交且模为 1,故构成一个酉(实正交)矩阵的列。

取
\[
U=\begin{pmatrix}
\frac1{\sqrt3} & \frac1{\sqrt2} & \frac1{\sqrt6}\\[1mm]
\frac1{\sqrt3} & -\frac1{\sqrt2} & \frac1{\sqrt6}\\[1mm]
\frac1{\sqrt3} & 0 & -\frac2{\sqrt6}
\end{pmatrix},
\quad
D=\begin{pmatrix}
4&0&0\\
0&1&0\\
0&0&1
\end{pmatrix},
\]
则
\[
D=U^TAU=U^*AU.
\]

这就是相应二次型的正交对角化。


下面只给习题解答内容,方便你直接嵌入答案册。

---

\textbf{4.1}

先用塞尔维斯特正定性判据判断 \(A,B\) 是否正定。记左上 \(k\times k\) 子式为 \(A_k,B_k\)。

\[
A=\begin{pmatrix}
4&2&1\\[0.2em]
2&3&-1\\[0.2em]
1&-1&2
\end{pmatrix}.
\]

\[
\det A_1=4>0,\quad
\det A_2=\det\begin{pmatrix}4&2\\2&3\end{pmatrix}=12-4=8>0.
\]
\[
\det A=
4\det\!\begin{pmatrix}3&-1\\-1&2\end{pmatrix}
 -2\det\!\begin{pmatrix}2&-1\\1&2\end{pmatrix}
 +1\det\!\begin{pmatrix}2&3\\1&-1\end{pmatrix}
=4\cdot5-2\cdot5+(-5)=5>0.
\]
所有主子式正,故 \(A\) 正定。

---

\[
B=\begin{pmatrix}
3&-1&2\\[0.2em]
-1&4&-2\\[0.2em]
2&-2&1
\end{pmatrix}.
\]

\[
\det B_1=3>0,\quad 
\det B_2=\det\begin{pmatrix}3&-1\\-1&4\end{pmatrix}=12-1=11>0,
\]
\[
\det B=
3\det\!\begin{pmatrix}4&-2\\-2&1\end{pmatrix}
 +1\det\!\begin{pmatrix}-1&-2\\2&1\end{pmatrix}
 +2\det\!\begin{pmatrix}-1&4\\2&-2\end{pmatrix}
=3\cdot0+1\cdot3+2\cdot(-6)=-9<0.
\]
故 \(B\) 不是正定(实际上是不定)。

---

各矩阵的正定性:

- \(-A\):所有特征值取相反数,因此全部负,故 \(-A\) 负定,不是正定。
- \(A^3\):\(A\) 正定 \(\Rightarrow\) 所有特征值 \(\lambda_i>0\),故 \(A^3\) 特征值为 \(\lambda_i^3>0\),仍正定。
- \(A^{-1}\):正定矩阵可逆,特征值变为 \(1/\lambda_i>0\),故 \(A^{-1}\) 正定。
- \(A+B^{-1}\):\(B\) 不正定,不保证可逆,甚至 \(B^{-1}\) 不一定存在;即便存在,\(B^{-1}\) 也不必半正定,因此 \(A+B^{-1}\) 不必正定,一般\emph{不能}断言正定。
- \(A+B\):因为 \(B\) 有正、负特征值,\(A+B\) 可能正定也可能不定,不能一概而论。  
  直接算此例:
  \[
  A+B=
  \begin{pmatrix}
  7&1&3\\
  1&7&-3\\
  3&-3&3
  \end{pmatrix},
  \]
  其行列式可算出为正,主子式亦都为正(你可单独列算),所以\emph{在这一具体例子}中 \(A+B\) 恰好正定,但这不是一般结论。
- \(A-B\):
  \[
  A-B=
  \begin{pmatrix}
  1&3&-1\\
  3&-1&1\\
  -1&1&1
  \end{pmatrix}.
  \]
  \(\det(A-B)=1\cdot\det\begin{pmatrix}-1&1\\1&1\end{pmatrix}
  -3\cdot\det\begin{pmatrix}3&1\\-1&1\end{pmatrix}
  -1\cdot\det\begin{pmatrix}3&-1\\-1&1\end{pmatrix}
  =(-2)-3\cdot4-4=-18<0\),故 \(A-B\) 不正定(不定)。

---

\textbf{4.2}

a) 对。  
若 \(A\) 正定,则所有特征值 \(\lambda_i>0\),\(A^5\) 的特征值为 \(\lambda_i^5>0\),故 \(A^5\) 正定。

b) 错。  
若 \(A\) 负定,则特征值 \(\lambda_i<0\),\(A^8\) 的特征值为 \(\lambda_i^8>0\),所以 \(A^8\) 反而正定,不是负定。

c) 对。  
同理,\(\lambda_i<0\Rightarrow \lambda_i^{12}>0\),故 \(A^{12}\) 正定。

d) 对。  
 \(A\) 正定,\(B\) 半负定,故对任意 \(\xx\neq0\),
\[
\langle(A-B)\xx,\xx\rangle
=(A\xx,\xx)-(B\xx,\xx)\ge (A\xx,\xx)>0,
\]
因此 \(A-B\) 正定。

e) 错。  
\(A\) 不定、\(B\) 正定时,\(A+B\) 可能变成正定。简单例子:
\[
A=\begin{pmatrix}1&0\\0&-2\end{pmatrix}\ \text{(不定)},\quad
B=\begin{pmatrix}2&0\\0&3\end{pmatrix}\ \text{(正定)},
\]
则
\[
A+B=\begin{pmatrix}3&0\\0&1\end{pmatrix}
\]
正定,并非不定。

---

\textbf{4.3}

设
\[
A=\begin{pmatrix}
a_{11}&a_{12}\\
\overline{a_{12}}&a_{22}
\end{pmatrix},\quad
a_{11}\ge0,\ \det A\ge0.
\]
特征多项式为
\[
\lambda^2-(a_{11}+a_{22})\lambda+\det A=0.
\]
解得特征值
\[
\lambda_{1,2}
=\frac{a_{11}+a_{22}\pm\sqrt{(a_{11}+a_{22})^2-4\det A}}{2}.
\]
由 \(\det A\ge0\) 知 \(\lambda_1\lambda_2\ge0\)。  
又
\[
\lambda_1+\lambda_2=a_{11}+a_{22}=\operatorname{tr}A\in\RR.
\]
若存在某个特征值 \(<0\),则两者同号,故两者都 \(<0\),从而
\(\lambda_1+\lambda_2<0\Rightarrow a_{22}< -a_{11}\le0\)。  
但这时
\[
\det A=a_{11}a_{22}-|a_{12}|^2\le a_{11}a_{22}\le0,
\]
与 \(\det A\ge0\) 矛盾,故不可能有负特征值;于是 \(\lambda_1,\lambda_2\ge0\),\(A\) 半正定。

等价地,可直接使用 2×2 版本的塞尔维斯特判据:  
对埃尔米特矩阵,\(a_{11}\ge0,\det A\ge0\) 等价于所有特征值非负。

---

\textbf{4.4}

要一个 \(n\times n\)(\(n\ge3\))实对称矩阵,使所有主顺序子式 \(\det A_k\ge0\),但 \(A\) 不是半正定。  
取 \(n=3\),例如
\[
A=
\begin{pmatrix}
1&0&0\\
0&1&0\\
0&0&-1
\end{pmatrix}.
\]
这是实对称且不半正定(因为 \((A(0,0,1)^T,(0,0,1)^T)=-1<0\))。  
其主顺序子式
\[
A_1=(1),\quad \det A_1=1>0,\quad
A_2=\begin{pmatrix}1&0\\0&1\end{pmatrix},\ \det A_2=1>0,\quad
A_3=A,\ \det A=-1<0.
\]
这\emph{不}满足题目要求,因为 \(\det A_3<0\)。需要的是所有 \(\det A_k\ge0\),而矩阵仍不半正定——这必须利用 \(n\ge3\) 时塞尔维斯特判据对“半正定”失效这一点:  
可以让所有\emph{顺序主子式}非负,但存在其它方向上二次型取负。

一种标准构造是(教材常见例):
\[
A=
\begin{pmatrix}
1&1&0\\
1&1&1\\
0&1&1
\end{pmatrix}.
\]
它是对称的。主顺序子式:
\[
\det A_1=1>0,\quad
\det A_2=\det\begin{pmatrix}1&1\\1&1\end{pmatrix}=0,\quad
\det A=
\det\begin{pmatrix}
1&1&0\\
1&1&1\\
0&1&1
\end{pmatrix}=1>0.
\]
所以 \(\det A_k\ge0\) 对 \(k=1,2,3\) 都成立。  
但
\[
Q(x_1,x_2,x_3)=(Ax,x)
=(x_1+x_2)^2+(x_2+x_3)^2-x_2^2.
\]
取 \((x_1,x_2,x_3)=(1,-2,1)\),可算得
\[
(Ax,x)=-2<0,
\]
故 \(A\) 不半正定。  
(你可根据个人口味改用更对称的例子,只要顺序主子式全非负而二次型在某方向为负即可。)

---

\textbf{4.5}

设 \(A\) 为 \(n\times n\) 埃尔米特矩阵,\(\det A_k>0\) 对 \(k=1,\dots,n-1\),且 \(\det A\ge0\)。  
由前面的定理 4.2(塞尔维斯特正定性判据)知道:  
若 \(\det A_k>0\) 对所有 \(k=1,\dots,n\),则 \(A\) 正定。  
这里仅知道前 \(n-1\) 个严格正,最后一个非负。

把 \(A\) 对角化:存在酉矩阵 \(U\) 使
\[
U^*AU=D=\text{diag }(\lambda_1,\dots,\lambda_n),
\]
\(\lambda_j\in\RR\) 为特征值。由于酉变换不改变主顺序子式的符号,可转而研究 \(D\)。  
定理 4.2 的证明(书中已给)表明:
\[
\det A_k>0\ \forall k=1,\dots,n-1
\quad\Longrightarrow\quad
\lambda_1,\dots,\lambda_{n-1}>0.
\]
而
\[
\det A=\lambda_1\cdots\lambda_n\ge0,
\]
结合前 \(n-1\) 个特征值都 \(>0\),可得
\(\lambda_n\ge0\)。于是所有特征值均非负,故 \(A\) 半正定。

(若 \(\det A>0\) 进一步严格为正,则 \(\lambda_n>0\),得到 \(A\) 正定。)

---

\textbf{4.6}

需要一个 \(3\times3\) 实对称矩阵 \(A\),满足
\[
a_{11}>0,\quad
\det A_2\ge0,\quad
\det A_3\ge0,
\]
但 \(A\) 不是半正定。

可取
\[
A=
\begin{pmatrix}
1&1&0\\
1&1&1\\
0&1&1
\end{pmatrix},
\]
即上题 4.4 中给出的具体例子;它是实对称。  
显然 \(a_{11}=1>0\),
\[
\det A_2=\det\begin{pmatrix}1&1\\1&1\end{pmatrix}=0\ge0,\quad
\det A=\det A_3=1>0.
\]
与上一题分析一样,存在向量 \(\xx\)(例如 \(\xx=(1,-2,1)^T\))使得
\[
(A\xx,\xx)<0,
\]
故 \(A\) 不是半正定。





\end{exer}








\section{第八章答案}

\begin{exer}


只写习题解答内容,方便你直接嵌到答案册。

---

\textbf{1.1}

a) 设
\[
\sum_{j=1}^r \alpha_j \vv_j = 0.
\]
对上式两边施加线性泛函 \(\vv'_k\),利用双正交关系:
\[
0=\vv'_k\Bigl(\sum_{j=1}^r \alpha_j \vv_j\Bigr)
   =\sum_{j=1}^r \alpha_j\,\vv'_k(\vv_j)
   =\sum_{j=1}^r \alpha_j \delta_{kj}=\alpha_k.
\]
对每个 \(k=1,\dots,r\) 都得到 \(\alpha_k=0\),故 \(\{\vv_1,\dots,\vv_r\}\) 线性无关。

b) 假设 \(\{\vv_1,\dots,\vv_r\}\) 不是生成集,则 \(\dim X = n>r\)。  
把 \(\{\vv_1,\dots,\vv_r\}\) 扩展为 \(X\) 的一组基
\[
\{\vv_1,\dots,\vv_r,\vv_{r+1},\dots,\vv_n\}.
\]
在这组基下,第二章命题 5.4 告诉我们:存在唯一的一组对偶基
\[
\{\phi_1,\dots,\phi_n\}\subset X',
\]
满足
\[
\phi_k(\vv_j)=\delta_{kj},\quad k,j=1,\dots,n.
\]
显然 \(\{\phi_1,\dots,\phi_r\}\) 就是满足题中条件的一组“双正交”系统。

现在说明这样的系统不是唯一的。  
取任意不全为零的线性泛函 \(f\in X'\),满足
\[
f(\vv_j)=0,\quad j=1,\dots,r.
\]
这样的 \(f\) 必然存在,因为 \(\{\vv_1,\dots,\vv_r\}\) 不是生成集,所以它张成的子空间 \(V=\operatorname{span}\{\vv_1,\dots,\vv_r\}\) 真包含于 \(X\),对偶空间中存在非零泛函在 \(V\) 上恒为零(例如,先在商空间 \(X/V\) 上取一个非零线性泛函,再拉回到 \(X\))。

给定这样的 \(f\),对任意标量 \(\lambda_1,\dots,\lambda_r\),定义新的泛函
\[
\psi_k = \phi_k + \lambda_k f,\quad k=1,\dots,r.
\]
则对所有 \(j\le r\) 有
\[
\psi_k(\vv_j)
= \phi_k(\vv_j) + \lambda_k f(\vv_j)
= \delta_{kj} + \lambda_k\cdot 0
= \delta_{kj},
\]
因此 \(\{\psi_1,\dots,\psi_r\}\) 也是一组满足题目条件的“双正交”系统。  
只要选取某个 \(\lambda_k\ne0\) 且 \(f\ne0\),就有 \(\psi_k\ne\phi_k\),所以它与 \(\{\phi_1,\dots,\phi_r\}\) 不同,从而双正交系统不唯一。

---

\textbf{1.2}

设 \(P_n\) 为所有次数不超过 \(n\) 的多项式空间,\(\dim P_n = n+1\)。  
对每个给定的点 \(a_k\) 定义线性泛函
\[
f_k : P_n\to\FF,\quad f_k(p)=p(a_k),\quad k=1,\dots,n+1.
\]
这是 \(P_n'\) 中的 \(n+1\) 个线性泛函。

要证满足
\[
p(a_k)=y_k,\quad k=1,\dots,n+1
\]
且 \(\deg p\le n\) 的多项式 \(p\) 唯一,只需证明:若 \(\deg q\le n\) 且
\[
q(a_k)=0,\quad k=1,\dots,n+1,
\]
则 \(q\equiv0\)。  
线性代数语言下,就是要证明:如果
\[
f_k(q)=0,\quad k=1,\dots,n+1,
\]
则 \(q=0\)。

设有两个满足条件的多项式 \(p_1,p_2\),即
\[
p_1(a_k)=p_2(a_k)=y_k,\quad k=1,\dots,n+1.
\]
令 \(q=p_1-p_2\),则 \(\deg q\le n\) 且
\[
q(a_k)=p_1(a_k)-p_2(a_k)=0,\quad k=1,\dots,n+1.
\]
于是 \(q\) 属于线性算子
\[
T:P_n\to \FF^{n+1},\quad T(p)=(p(a_1),\dots,p(a_{n+1}))
\]
的核:\(T(q)=0\)。

我们证明 \(\text{Ker } T=\{0\}\)。  
若 \(q\neq 0\),则 \(\deg q\le n\) 的非零多项式在域 \(\FF\)(实或复)上最多有 \(n+1\) 个根,其中\emph{严格意义上线性代数不应使用这条分析结果},所以换一种线性代数表述:

注意到
\[
q(a_k)=0,\ k=1,\dots,n+1
\]
等价于
\[
f_k(q)=0,\ k=1,\dots,n+1.
\]
也就是说,\(\{f_1,\dots,f_{n+1}\}\subset P_n'\) 这一组泛函如果是线性无关的,那么由引理 1.3(书中本节的唯一性定理)可知
\[
f_k(q)=0\ \forall k \Longrightarrow q=0.
\]
因此关键是证明 \(\{f_k\}\) 线性无关。

设
\[
\sum_{k=1}^{n+1} \alpha_k f_k = 0
\quad\text{(作为 \(P_n'\) 中的零泛函)}.
\]
这意味着对任意 \(p\in P_n\),
\[
0=\Bigl(\sum_{k=1}^{n+1} \alpha_k f_k\Bigr)(p)
=\sum_{k=1}^{n+1} \alpha_k p(a_k).
\]
特别地,取题中已经构造好的那组多项式 \(p_j\)(书中在公式 (1.5) 前后定义的拉格朗日基):
\[
p_j(a_k)=\delta_{jk},\quad j,k=1,\dots,n+1.
\]
代入上式,得
\[
0=\sum_{k=1}^{n+1} \alpha_k p_j(a_k)
  =\sum_{k=1}^{n+1} \alpha_k \delta_{jk}
  =\alpha_j,\quad j=1,\dots,n+1.
\]
于是所有 \(\alpha_j=0\),\(\{f_1,\dots,f_{n+1}\}\) 线性无关。

因此 \(\text{Ker } T=\{0\}\),从而 \(T\) 单射。  
若 \(p_1,p_2\) 都满足插值条件,则
\[
T(p_1)=T(p_2)=(y_1,\dots,y_{n+1}),
\]
由单射性得 \(p_1=p_2\)。  

所以满足 (1.5) 且 \(\deg p\le n\) 的多项式 \(p\) 唯一。


只写习题解答内容,方便你直接嵌入。

---

\textbf{3.1}

对任意 \(\xx\in X\) 及任意 \(\yy'\in Y'\),
\[
\langle T\xx,\yy'\rangle=\langle T_1\xx,\yy'\rangle
\quad\Longrightarrow\quad
\langle T\xx-T_1\xx,\yy'\rangle=0.
\]
记 \(\mathbb{Z}:=T\xx-T_1\xx\in Y\)。上式对所有 \(\yy'\in Y'\) 成立,即
\[
\langle \mathbb{Z},\yy'\rangle=0,\quad \forall\,\yy'\in Y'.
\]
由引理 1.3(对偶空间与向量唯一对应)知,只能有 \(\mathbb{Z}=0\)。于是
\[
T\xx-T_1\xx=0,\quad\forall\,\xx\in X,
\]
即 \(T=T_1\)。

---

\textbf{3.2}

由里斯表示定理:对内积空间 \(H\) 上的任意连续线性泛函 \(\ell\in H'\),存在唯一 \(\yy\in H\) 使
\[
\ell(\xx)=(\xx,\yy),\quad\forall\,\xx\in H.
\]

给定线性算子 \(A:H\to H\)。  
对每个固定的 \(\yy\in H\),考虑泛函
\[
\ell_\yy:H\to\FF,\quad
\ell_\yy(\xx):=(A\xx,\yy).
\]
\(\ell_\yy\) 显然是线性的且连续。由里斯定理,存在唯一向量 \(\eta\in H\) 使得
\[
(A\xx,\yy)=(\xx,\eta),\quad\forall\,\xx\in H.
\]
把这唯一的 \(\eta\) 记为 \(A^*\yy\)。这样就得到一个映射
\[
A^*:H\to H,\quad
\yy\mapsto A^*\yy,
\]
满足
\[
(A\xx,\yy)=(\xx,A^*\yy),\quad\forall\,\xx,\yy\in H.
\]
利用 3.1 题(对所有 \(\xx,\yy\) 成立的这种“配对相等”唯一确定算子),可知这样的 \(A^*\) 存在且唯一,这就是内积空间中算子埃尔米特伴随的无坐标定义。

---

\textbf{3.3}

设 \(\{\vv_1,\dots,\vv_n\}\) 是 \(X\) 的一组基,\(\{\vv_1',\dots,\vv_n'\}\subset X'\) 是它的对偶基:
\[
\vv_i'(\vv_j)=\delta_{ij}.
\]
设 \(E=\operatorname{span}\{\vv_1,\dots,\vv_r\}\),\(r<n\)。命题 3.6 说明可把对偶向量当作 \(X\) 中的一组向量,并且
\[
E^\perp=(E^\perp)^\perp{}^\perp=\text{span}\{\vv_{r+1},\dots,\vv_n\}.
\]
再对空间 \(E^\perp\) 应用命题 3.6 可知
\[
(E^\perp)^\perp=\operatorname{span}\{\vv_1',\dots,\vv_r'\},\quad
E=\operatorname{span}\{\vv_1,\dots,\vv_r\}.
\]
另一方面,由零化子与正交补之间的同构关系(命题 3.6 的一般形式)
\[
E^\perp \cong E^0=\{\varphi\in X' : \varphi|_E=0\},
\]
而
\[
\varphi|_E=0\quad\Longleftrightarrow\quad
\varphi\in\operatorname{span}\{\vv_{r+1}',\dots,\vv_n'\}.
\]
故
\[
E^\perp=\operatorname{span}\{\vv_{r+1}',\dots,\vv_n'\}.
\]

(按书中的叙述:在把 \(X\) 与 \(X'\) 通过里斯定理规约同构后,对偶基 \(\vv_j'\) 正好是一组与 \(\vv_j\) 双正交的基,从而上式成立。)

---

\textbf{3.4}

设 \(E\subset X\) 是任一子空间。命题 3.6 给出
\[
(E^\perp)^\perp=E.
\]
从题 3.3 的结果,对某组基及其对偶基,可写
\[
E=\operatorname{span}\{\vv_1,\dots,\vv_r\},\quad
E^\perp=\operatorname{span}\{\vv_{r+1}',\dots,\vv_n'\}.
\]
于是
\[
\dim E = r,\quad \dim E^\perp = n-r,
\]
其中 \(n=\dim X\)。因此
\[
\dim E+\dim E^\perp = r+(n-r)=n=\dim X.
\]


只写解答内容,方便你直接嵌入习题解答。

---

\textbf{4.1}

设原坐标为 \((x_1,\dots,x_n)\),新坐标为 \((\tilde x_1,\dots,\tilde x_n)\),二者之间的关系为
\[
x_k = x_k(\tilde x_1,\dots,\tilde x_n),\quad k=1,\dots,n,
\]
且这个坐标变换可微并且可逆。

给定微分算子
\[
D=\sum_{k=1}^n v_k(x)\,\frac{\partial}{\partial x_k}.
\]
我们要在新坐标下把同一个算子写成
\[
D=\sum_{j=1}^n \tilde v_j(\tilde x)\,\frac{\partial}{\partial \tilde x_j},
\]
并证明 \(\tilde v\) 是按向量坐标变换规律从 \(v\) 得到的。

对任意光滑函数 \(\Phi\),利用链式法则有
\[
\frac{\partial\Phi}{\partial x_k}
=
\sum_{j=1}^n
\frac{\partial\Phi}{\partial \tilde x_j}
\frac{\partial \tilde x_j}{\partial x_k}.
\]
因此
\[
D\Phi
=
\sum_{k=1}^n v_k\,\frac{\partial\Phi}{\partial x_k}
=
\sum_{k=1}^n v_k
\sum_{j=1}^n
\frac{\partial \tilde x_j}{\partial x_k}
\frac{\partial\Phi}{\partial \tilde x_j}
=
\sum_{j=1}^n
\Bigl(\sum_{k=1}^n v_k\,\frac{\partial \tilde x_j}{\partial x_k}\Bigr)
\frac{\partial\Phi}{\partial \tilde x_j}.
\]
与
\[
D\Phi
=
\sum_{j=1}^n \tilde v_j\,\frac{\partial\Phi}{\partial \tilde x_j}
\]
比较,可得
\[
\tilde v_j(\tilde x)
=
\sum_{k=1}^n
v_k(x)\,\frac{\partial \tilde x_j}{\partial x_k},
\quad j=1,\dots,n.
\]

记
\[
v=(v_1,\dots,v_n)^T,\quad
\tilde v=(\tilde v_1,\dots,\tilde v_n)^T,
\]
以及雅可比矩阵
\[
J=\Bigl(\frac{\partial \tilde x_j}{\partial x_k}\Bigr)_{j,k}.
\]
上式就是
\[
\tilde v = J\,v.
\]

但 \(J\) 正是从旧坐标到新坐标的线性近似——也就是同一个几何向量在坐标变化下的坐标变换矩阵。因此
\[
\tilde v = J v
\]
表明:微分算子 \(D\) 的系数 \((v_k)\) 在坐标变换下,正是按向量坐标的变换规则变化的。也即,\(D\) 的系数在每一点构成一个切向量,其坐标变换与向量坐标一致。



只写习题解答内容,方便你直接嵌入。

---

**5.1**

按定义,\(\vv_1\otimes\cdots\otimes\vv_p\) 是一个多线性函数
\[
F(\varphi_1,\dots,\varphi_p)
=(\vv_1\otimes\cdots\otimes\vv_p)(\varphi_1,\dots,\varphi_p)
:=\varphi_1(\vv_1)\cdots\varphi_p(\vv_p),
\quad \varphi_k\in V_k'.
\]
对第 \(k\) 个参数验证线性即可。例如对 \(\vv_k\):
\[
(\vv_1\otimes\cdots\otimes(\alpha\vv_k+\beta\ww_k)\otimes\cdots\otimes\vv_p)
(\varphi_1,\dots,\varphi_p)
=
\varphi_1(\vv_1)\cdots \varphi_k(\alpha\vv_k+\beta\ww_k)\cdots\varphi_p(\vv_p)
\]
\[
=\alpha\,\varphi_1(\vv_1)\cdots\varphi_k(\vv_k)\cdots\varphi_p(\vv_p)
+\beta\,\varphi_1(\vv_1)\cdots\varphi_k(\ww_k)\cdots\varphi_p(\vv_p)
\]
\[
=\alpha\,(\vv_1\otimes\cdots\otimes\vv_k\otimes\cdots\otimes\vv_p)
+\beta\,(\vv_1\otimes\cdots\otimes\ww_k\otimes\cdots\otimes\vv_p)
\]
在多线性函子意义下相等,因此在第 \(k\) 个变量上线性;其它变量类似。故在每个参数 \(\vv_k\) 上均线性。

---

**5.2**

设 \(\dim V_k=n_k>0\)。则
\[
\dim\bigl(V_1\otimes\cdots\otimes V_p\bigr)=n_1\cdots n_p.
\]

向量张量积集合
\(\{\vv_1\otimes\cdots\otimes\vv_p:\vv_k\in V_k\}\) 中的每个元素称为「可分张量」或「纯张量」。  
如果这个集合等于整个张量积空间,那么张量积空间中任何元素都必须是一个纯张量。

取最简单情形 \(p=2\)、\(\dim V_1=\dim V_2=2\) 即可给出反例。设
\[
V_1=V_2=\RR^2,\quad
e_1,e_2\text{ 为标准基}.
\]
则
\[
e_1\otimes e_1,\ e_1\otimes e_2,\ e_2\otimes e_1,\ e_2\otimes e_2
\]
构成 \(V_1\otimes V_2\) 的一组基。考虑向量
\[
w:= e_1\otimes e_1 + e_2\otimes e_2.
\]
若 \(w\) 是一个纯张量,则存在
\[
(a_1e_1+a_2e_2)\otimes(b_1e_1+b_2e_2)=w.
\]
展开得
\[
a_1b_1\,e_1\otimes e_1
+a_1b_2\,e_1\otimes e_2
+a_2b_1\,e_2\otimes e_1
+a_2b_2\,e_2\otimes e_2
= e_1\otimes e_1+e_2\otimes e_2.
\]
比较基系数,得到方程组
\[
a_1b_1=1,\quad
a_1b_2=0,\quad
a_2b_1=0,\quad
a_2b_2=1.
\]
由 \(a_1b_1=1\) 知 \(a_1\ne0,b_1\ne0\),从而 \(a_2b_1=0\) 推出 \(a_2=0\)。  
再由 \(a_2b_2=1\) 得 \(0\cdot b_2=1\),矛盾。因此不存在这样的 \(a_i,b_j\),故 \(w\) 不是纯张量。

于是张量积空间中至少存在一个向量不是任何向量张量积的形式,说明
\[
\{\vv_1\otimes\cdots\otimes\vv_p:\vv_k\in V_k\}
\subsetneq
V_1\otimes\cdots\otimes V_p.
\]

---

**5.3**

命题 5.6:给定张量
\[
F\in L(V_1,\dots,V_p;V'),
\]
存在唯一的线性变换
\[
\tilde T:V_1\otimes\cdots\otimes V_p\to V'
\]
满足
\[
\tilde T(\vv_1\otimes\cdots\otimes\vv_p)=F(\vv_1,\dots,\vv_p),
\quad \forall\,\vv_k\in V_k.
\]

唯一性证明如下。设 \(\tilde T_1,\tilde T_2:V_1\otimes\cdots\otimes V_p\to V'\) 都是线性变换,且对所有 \(\vv_k\in V_k\) 满足
\[
\tilde T_1(\vv_1\otimes\cdots\otimes\vv_p)
=\tilde T_2(\vv_1\otimes\cdots\otimes\vv_p)
=F(\vv_1,\dots,\vv_p).
\]
则对于任意纯张量 \(w=\vv_1\otimes\cdots\otimes\vv_p\),有
\[
\tilde T_1(w)=\tilde T_2(w).
\]

另一方面,按定义,纯张量的线性张成空间等于整个张量积空间 \(V_1\otimes\cdots\otimes V_p\)。因此任意
\[
w=\sum_{s} \alpha_s\,
\vv^{(s)}_1\otimes\cdots\otimes\vv^{(s)}_p
\]
有
\[
\tilde T_1(w)
=\sum_s \alpha_s \tilde T_1(\vv^{(s)}_1\otimes\cdots\otimes\vv^{(s)}_p)
=\sum_s \alpha_s \tilde T_2(\vv^{(s)}_1\otimes\cdots\otimes\vv^{(s)}_p)
=\tilde T_2(w).
\]
故 \(\tilde T_1=\tilde T_2\)。这就证明了命题 5.6 中 \(\tilde T\) 的唯一性。





\end{exer}








\section{第九章答案}

\begin{exer}


只写解答内容,便于直接放进习题解答里。

---

\textbf{1.1}

设
\[
A=SDS^{-1},\quad 
D=\operatorname{diag}(\lambda_1,\dots,\lambda_n),
\]
其中 \(\lambda_1,\dots,\lambda_n\) 是 \(A\) 的全部特征值(按代数重数计)。则 \(A\) 和 \(D\) 相似,故有相同的特征多项式:
\[
p(\lambda)=\det(A-\lambda I)=\det(D-\lambda I)
=\prod_{j=1}^n(\lambda_j-\lambda).
\]

1)先计算 \(p(D)\)。  
因为 \(D\) 是对角矩阵,而多项式在对角矩阵上的值仍是对角矩阵,且对角元是把该多项式作用在每个特征值上:
\[
p(D)=\sum_{k=0}^n c_k D^k
=\operatorname{diag}\bigl(p(\lambda_1),\dots,p(\lambda_n)\bigr).
\]
但由上式 \(p(\lambda)=\prod_{j=1}^n(\lambda_j-\lambda)\),对每个 \(i\) 有
\[
p(\lambda_i)=\prod_{j=1}^n(\lambda_j-\lambda_i)=0,
\]
因为当 \(j=i\) 时因子为 \(0\)。于是
\[
p(D)=\operatorname{diag}\bigl(p(\lambda_1),\dots,p(\lambda_n)\bigr)
=\operatorname{diag}(0,\dots,0)=0.
\]

2)再计算 \(p(A)\)。  
利用 \(A=SDS^{-1}\) 以及多项式与相似变换的相容性:
\[
A^k=(SDS^{-1})^k
=SD^kS^{-1},\quad k\ge0,
\]
因此
\[
p(A)=\sum_{k=0}^n c_k A^k
=\sum_{k=0}^n c_k SD^kS^{-1}
=S\Bigl(\sum_{k=0}^n c_k D^k\Bigr)S^{-1}
=Sp(D)S^{-1}.
\]
由上一步 \(p(D)=0\),于是
\[
p(A)=S\cdot 0\cdot S^{-1}=0.
\]

这就证明了当 \(A\) 与对角矩阵相似(即可对角化)时,凯莱–哈密顿定理成立。


只写解答内容,便于直接嵌入。

---

\textbf{2.1}

设 \(A\) 幂零,即存在某个正整数 \(k\) 使 \(A^k=0\)。

设 \(\lambda\) 是 \(A\) 的一个特征值,对应特征向量 \(\vv\neq0\),即
\[
A\vv=\lambda\vv.
\]
连续作用 \(A\) 得
\[
A^2\vv = A(A\vv)=A(\lambda\vv)=\lambda^2\vv,\quad\ldots,\quad
A^m\vv=\lambda^m\vv,\ \forall m\ge1.
\]
取 \(m=k\),由 \(A^k=0\) 得
\[
0=A^k\vv=\lambda^k\vv.
\]
由于 \(\vv\neq0\),只能有 \(\lambda^k=0\),因此 \(\lambda=0\)。

这说明 \(A\) 的任何特征值都必须为 \(0\),于是
\[
\sigma(A)=\{0\}.
\]

整个证明只用到了特征值的定义,没有用到谱映射定理。





\end{exer}







