\chapter{内积空间中算子的结构}

在本章中,我们再次假设所有空间都是有限维的。同样,我们只处理复数或实数空间,内积空间的理论不适用于任意域上的空间。当没有提及我们所处的空间时,所有结果都适用于复数和实数空间。为了避免重复书写基本相同的公式,我们将使用复数情况的记号:在实数情况下,它给出正确但有时稍显复杂的公式。

\section{算子的上三角(舒尔)表示}

\textbf{定理 1.1。} 设 $A: X \to X$ 是作用在复内积空间中的算子。存在一个标准正交基 $\{u_1, u_2, \dots, u_n\}$ 在 $X$ 中,使得 $A$ 在该基下的矩阵是上三角矩阵。换句话说,任何 $n \times n$ 矩阵 $A$ 都可以表示为 $A = UTU^*$,其中 $U$ 是酉矩阵,而 $T$ 是上三角矩阵。

\textbf{证明}~ 我们使用 $\dim X$ 的数学归纳法来证明定理。如果 $\dim X = 1$,则定理是平凡的,因为任何 $1 \times 1$ 矩阵都是上三角矩阵。假设我们已经证明了当 $\dim X = n-1$ 时定理成立,并且我们想证明对于 $\dim X = n$ 时定理成立。
设 $\lambda_1$ 是 $A$ 的一个特征值,设 $u_1$, $\|u_1\| = 1$ 是相应的特征向量,$Au_1 = \lambda_1 u_1$。记 $E = u_1^\perp$,并设 $\{v_2, \dots, v_n\}$ 是 $E$ 中的某个标准正交基(显然 $\dim E = \dim X - 1 = n-1$),那么 $\{u_1, v_2, \dots, v_n\}$ 是 $X$ 中的一个标准正交基。在这一基下,$A$ 的矩阵具有形式 (1.1)
$\begin{pmatrix} \lambda_1 & * & \dots & * \\ 0 & & & \\ \vdots & & A_1 & \\ 0 & & & \end{pmatrix}$;
这里 $\lambda_1$ 下方的所有元素都是零,而 $*$ 表示我们不关心 $\lambda_1$ 右侧的元素。我们足够关心右下角的 $(n-1) \times (n-1)$ 块,以便给它命名:我们将其记为 $A_1$。注意,$A_1$ 定义了 $E$ 中的一个线性变换,并且由于 $\dim E = n-1$,归纳假设表明存在一个标准正交基(我们将其记为 $\{u_2, \dots, u_n\}$),使得 $A_1$ 在该基下的矩阵是上三角矩阵。因此,$A$ 在该标准正交基 $\{u_1, u_2, \dots, u_n\}$ 下的矩阵也是上三角矩阵。

\textbf{注释} ~注意,在证明中引入的子空间 $E = u_1^\perp$ 对于 $A$ 不是不变的,即 $AE \subseteq E$ 不一定成立。这意味着 $A_1$ 不是 $A$ 的一部分,它是从 $A$ 构建的某个算子。还需注意,$AE \subseteq E$ 当且仅当所有标记为 $*$ 的元素(即除了 $\lambda_1$ 之外的第一行的所有元素)都为零。

\textbf{注释} ~注意,即使我们从一个实数矩阵 $A$ 开始,矩阵 $U$ 和 $T$ 也可以是复数的。旋转矩阵 $\begin{pmatrix} \cos \alpha & -\sin \alpha \\ \sin \alpha & \cos \alpha \end{pmatrix}$, $\alpha \neq k\pi, k \in \mathbb{Z}$ 与实数上三角矩阵不是酉等价的(甚至不是相似的)。因为该矩阵的特征值是复数,而上三角矩阵的特征值是对角线元素。

\textbf{注释} ~1.1 定理的一个类似版本可以陈述并证明用于任意向量空间,而不要求它具有内积。在这种情况下,定理声称在某个基下任何算子都有上三角形式。可以通过模仿定理 1.1 的证明来完成。另一种方法是为 $V$ 装备内积,方法是固定一个基并声明它是标准正交基,见第 5 章第 2.4 节。


注意,内积空间版本(定理 1.1)比向量空间版本更强大,因为它说明我们总能找到一个标准正交基,而不仅仅是一个基。下面的定理是定理 1.1 的实数版本:

\textbf{定理 1.2。} 设 $A: X \to X$ 是作用在\textbf{实数}内积空间中的算子。假设 $A$ 的所有特征值都是实数(意味着 $A$ 恰好有 $n = \dim X$ 个实数特征值,计入重数)。那么存在 $X$ 中的一个标准正交基 $\{u_1, u_2, \dots, u_n\}$,使得 $A$ 在该基下的矩阵是上三角矩阵。换句话说,任何具有所有实数特征值的实数 $n \times n$ 矩阵 $A$ 都可以表示为 $A = UTU^* = U T U^T$,其中 $U$ 是正交矩阵,而 $T$ 是实数上三角矩阵。

\textbf{证明}~ 为了证明定理,我们只需要分析定理 1.1 的证明。让我们假设(我们可以无损于一般性地这样做)算子(矩阵)$A$ 作用在 $\mathbb{R}^n$ 上。假设定理对 $(n-1) \times (n-1)$ 矩阵成立。在定理 1.1 的证明中,设 $\lambda_1$ 是 $A$ 的一个实特征值,$u_1 \in \mathbb{R}^n$, $\|u_1\| = 1$ 是相应的特征向量,设 $\{v_2, \dots, v_n\}$ 是 $\mathbb{R}^n$ 中的一个标准正交系统,使得 $\{u_1, v_2, \dots, v_n\}$ 是 $\mathbb{R}^n$ 中的一个标准正交基。在该基下 $A$ 的矩阵具有形式 (1.1),其中 $A_1$ 是某个实数矩阵。如果我们能证明矩阵 $A_1$ 只有实数特征值,那么我们就完成了。确实,根据归纳假设,存在 $E = u_1^\perp$ 中的一个标准正交基 $\{u_2, \dots, u_n\}$,使得 $A_1$ 在该基下的矩阵是上三角矩阵,因此 $A$ 在 $\{u_1, u_2, \dots, u_n\}$ 基下的矩阵也是上三角矩阵。为了证明 $A_1$ 只有实数特征值,让我们注意到 $\det(A - \lambda I) = ( \lambda_1 - \lambda ) \det( A_1 - \lambda )$(例如,通过第一行的代数余子式展开),所以 $A_1$ 的任何特征值也是 $A$ 的特征值。但是 $A$ 只有实数特征值!

\section{自伴随和正规算子的谱定理}
在本章中,我们处理的是酉等价于对角矩阵的矩阵(算子)。

让我们回忆一下,如果一个算子满足 $A = A^*$,则称其为\textbf{自伴随}的。在某个标准正交基下的自伴随算子(即满足 $A^* = A$ 的矩阵)称为\textbf{Hermitian}矩阵。术语“自伴随”和“Hermitian”基本上是同义的。通常人们在谈论算子(变换)时说自伴随,在谈论矩阵时说 Hermitian。我们将尝试遵循这个约定,但由于我们经常不区分算子和它们的矩阵,所以有时会混合使用这两个术语。

\textbf{定理 2.1。} 设 $A = A^*$ 是内积空间 $X$(空间可以是复数或实数)中的一个自伴随算子。那么 $A$ 的所有特征值都是实数,并且 $X$ 中存在 $A$ 的特征向量的标准正交基。

这个定理可以用矩阵形式重述如下:

\textbf{定理 2.2。} 设 $A = A^*$ 是一个自伴随(因此是方阵)矩阵。那么 $A$ 可以表示为 $A = UDU^*$,其中 $U$ 是酉矩阵,$D$ 是具有实数项的对角矩阵。而且,如果矩阵 $A$ 是实数的,矩阵 $U$ 可以选择为实数的(即正交的)。

\textbf{证明}~ 为了证明定理 2.1 和定理 2.2,我们首先对内积空间 $X$(或实数空间 $X$)应用定理 1.1(或定理 1.2)来找到一个标准正交基,使得 $A$ 在该基下的矩阵是上三角矩阵。现在让我们问自己一个问题:什么样的上三角矩阵是自伴随的?答案是显而易见的:上三角矩阵是自伴随的当且仅当它是具有实数项的对角矩阵。定理 2.1(以及因此定理 2.2)得证。

\textbf{注释} ~注意,在许多教科书中只考虑实数矩阵,并且定理 2.2 通常被称为“\textbf{对称矩阵的谱定理}”。然而,我们应该强调,定理 2.2 的结论对于\textbf{复数}对称矩阵是不成立的:该定理适用于 Hermitian 矩阵,特别是\textbf{实数}对称矩阵。

让我们给出一个 $A=A^*$ 的算子的特征值是实数的独立证明。设 $A=A^*$ 且 $Ax=\lambda x$, $x \neq 0$。那么 $(Ax, x) = (\lambda x, x) = \lambda(x, x) = \lambda\|x\|^2$。另一方面,$(Ax, x) = (x, A^*x) = (x, Ax) = (x, \lambda x) = \bar{\lambda}(x, x) = \bar{\lambda}\|x\|^2$(这里我们用了 $(x, \lambda y) = \bar{\lambda}(x, y)$)。所以 $\lambda\|x\|^2 = \bar{\lambda}\|x\|^2$。由于 $x \neq 0$,$\|x\|^2 \neq 0$,我们可以得出 $\lambda = \bar{\lambda}$,所以 $\lambda$ 是实数。

从定理 2.1 也可以得出,自伴随算子的特征子空间是相互正交的。让我们给出一个该结果的独立证明。

\textbf{命题 2.3。} 设 $A = A^*$ 是一个自伴随算子,设 $u, v$ 是它的特征向量,$Au = \lambda u$, $Av = \mu v$。那么,如果 $\lambda \neq \mu$,则特征向量 $u$ 和 $v$ 是正交的。

\textbf{证明}~ 这个命题虽然可以从谱定理(定理 1.1)得出,但我们在这里给出一个直接的证明。即,$(Au, v) = (\lambda u, v) = \lambda(u, v)$。另一方面,$(Au, v) = (u, A^*v) = (u, Av) = (u, \mu v) = \mu(u, v)$(最后一个等式成立是因为自伴随算子的特征值是实数),所以 $\lambda(u, v) = \mu(u, v)$。如果 $\lambda \neq \mu$,则这只有在 $(u, v) = 0$ 时才可能。

现在让我们尝试找到哪些矩阵是酉等价于一个对角矩阵。可以很容易地检验出,对于对角矩阵 $D$, $D^*D = DD^*$。因此,如果 $A$ 在某个标准正交基下的矩阵是对角矩阵,那么 $A^*A = AA^*$。

\textbf{定义}~ 称算子(矩阵)$N$ 是\textbf{正规}的,如果 $N^*N = NN^*$。显然,任何自伴随算子($A^*A = AA^*$)都是正规的。同样,任何酉算子 $U: X \to X$ 也是正规的,因为 $U^*U = UU^* = I$。注意,正规算子是作用在同一个空间上的算子,而不是从一个空间到另一个空间。所以,如果 $U$ 是作用在一个空间到另一个空间上的酉算子,我们就不能说 $U$ 是正规的。

\textbf{定理 2.4。} 任何复数向量空间中的正规算子 $N$ 都有一个标准正交的特征向量基。换句话说,任何满足 $N^*N = NN^*$ 的矩阵 $N$ 都可以表示为 $N = UDU^*$,其中 $U$ 是酉矩阵,$D$ 是对角矩阵。

\textbf{注释} ~注意,在上述定理中,即使 $N$ 是实数矩阵,我们也没有声称矩阵 $U$ 和 $D$ 是实数。而且,可以很容易地证明,如果 $D$ 是实数,那么 $N$ 必须是自伴随的。



\textbf{定理 2.4 的证明。} 为了证明定理 2.4,我们应用定理 1.1 来得到一个标准正交基,使得 $N$ 在该基下的矩阵是上三角矩阵。为了完成定理的证明,我们只需要证明一个上三角正规矩阵必须是对角矩阵。我们将使用矩阵维数的数学归纳法来证明这一点。$1 \times 1$ 矩阵的情况是平凡的,因为任何 $1 \times 1$ 矩阵都是对角矩阵。假设我们已经证明了任何 $(n-1) \times (n-1)$ 的上三角正规矩阵都是对角矩阵,并且我们想证明对于 $n \times n$ 矩阵也成立。设 $N$ 是一个 $n \times n$ 上三角正规矩阵。我们可以将其写成
$N = \begin{pmatrix} a_{1,1} & a_{1,2} & \dots & a_{1,n} \\ 0 & & & \\ \vdots & & N_1 & \\ 0 & & & \end{pmatrix}$
其中 $N_1$ 是一个 $(n-1) \times (n-1)$ 的上三角矩阵。让我们比较 $N^*N$ 和 $NN^*$ 的左上角元素(第一行第一列)。直接计算表明 $(N^*N)_{1,1} = \bar{a}_{1,1}a_{1,1} = |a_{1,1}|^2$,而 $(NN^*)_{1,1} = |a_{1,1}|^2 + |a_{1,2}|^2 + \dots + |a_{1,n}|^2$。所以,$(N^*N)_{1,1} = (NN^*)_{1,1}$ 当且仅当 $a_{1,2} = \dots = a_{1,n} = 0$。因此,矩阵 $N$ 具有形式
$N = \begin{pmatrix} a_{1,1} & 0 & \dots & 0 \\ 0 & & & \\ \vdots & & N_1 & \\ 0 & & & \end{pmatrix}$。
从上述表示可以得出 $N^*N = \begin{pmatrix} |a_{1,1}|^2 & & \\ & N_1^*N_1 & \\ & & \end{pmatrix}$,$NN^* = \begin{pmatrix} |a_{1,1}|^2 & & \\ & N_1N_1^* & \\ & & \end{pmatrix}$。所以 $N_1^*N_1 = N_1N_1^*$。这意味着矩阵 $N_1$ 也是正规的,并且根据归纳假设它是对角矩阵。所以矩阵 $N$ 也是对角矩阵。

以下命题给出了正规算子的一个非常有用的刻画。

\textbf{命题 2.5。} 算子 $N: X \to X$ 是正规的当且仅当 $\|Nx\| = \|N^*x\|$ $\forall x \in X$。

\textbf{证明}~ 设 $N$ 是正规的,$N^*N = NN^*$。那么 $\|Nx\|^2 = (Nx, Nx) = (N^*Nx, x) = (NN^*x, x) = (N^*x, N^*x) = \|N^*x\|^2$,所以 $\|Nx\| = \|N^*x\|$。现在设 $\|Nx\| = \|N^*x\|$ $\forall x \in X$。极化恒等式(第 5 章引理 1.9)暗示对于所有 $x, y \in X$,$(N^*Nx, y) = (Nx, Ny) = \frac{1}{4} \sum_{\alpha = \pm 1, \pm i} \alpha \|Nx + \alpha Ny\|^2 = \frac{1}{4} \sum_{\alpha = \pm 1, \pm i} \alpha \|N(x + \alpha y)\|^2 = \frac{1}{4} \sum_{\alpha = \pm 1, \pm i} \alpha \|N^*(x + \alpha y)\|^2 = (N^*x, N^*y) = (NN^*x, y)$。因此(见推论 1.6),$N^*N = NN^*$。

\textbf{练习}~

2.1. 真或假:
a) 任何酉算子 $U: X \to X$ 都是正规的。
b) 矩阵是酉的当且仅当它是可逆的。
c) 如果两个矩阵酉等价,那么它们也相似。
d) 两个自伴随算子之和是自伴随的。
e) 酉算子的伴随是酉的。
f) 正规算子的伴随是正规的。
g) 如果一个线性算子的所有特征值都是 1,那么该算子必须是酉的或正交的。
h) 如果一个正规算子的所有特征值都是 1,那么该算子是恒等算子。
i) 线性算子可能保持范数但不保持内积。

2.2. 真或假:两个正规算子之和是正规的?证明你的结论。

2.3. 证明一个酉等价于对角矩阵的矩阵是正规的。

2.4. \textbf{正交对角化}矩阵 $\begin{pmatrix} 3 & 2 \\ 2 & 3 \end{pmatrix}$。找出 $A$ 的所有平方根,即找出所有满足 $B^2 = A$ 的矩阵 $B$。
\textbf{注:} $A$ 的所有平方根都是自伴随的。




2.5. 真或假:任何自伴随矩阵都有一个自伴随的平方根。证明你的结论。

2.6. \textbf{正交对角化}矩阵 $A = \begin{pmatrix} 7 & 2 \\ 2 & 4 \end{pmatrix}$,即将其表示为 $A = UDU^*$,其中 $D$ 是对角矩阵,$U$ 是酉矩阵。在 $A$ 的所有平方根中,找出具有正特征值的平方根。你可以将 $B$ 表示为乘积形式。

2.7. 真或假:a) 两个自伴随矩阵的乘积是自伴随的。b) 如果 $A$ 是自伴随的,那么 $A^k$ 是自伴随的。证明你的结论。

2.8. 设 $A$ 是 $m \times n$ 矩阵。证明:
a) $A^*A$ 是自伴随的。
b) $A^*A$ 的所有特征值都是非负的。
c) $A^*A + I$ 是可逆的。

2.9. 如果陈述为真,则证明;如果陈述为假,则给出反例:
a) 如果 $A$ 是自伴随的,那么 $A + iI$ 是可逆的。
b) 如果 $U$ 是酉的,$U + \frac{3}{4}I$ 是可逆的。
c) 如果矩阵 $A$ 是实数的,那么 $A - iI$ 是可逆的。

2.10. \textbf{正交对角化}旋转矩阵 $R_\alpha = \begin{pmatrix} \cos \alpha & -\sin \alpha \\ \sin \alpha & \cos \alpha \end{pmatrix}$,其中 $\alpha$ 不是 $\pi$ 的整数倍。注意,在这种情况下你会得到复数特征值。

2.11. \textbf{正交对角化}矩阵 $A = \begin{pmatrix} \cos \alpha & \sin \alpha \\ \sin \alpha & -\cos \alpha \end{pmatrix}$。
\textbf{提示:} 你会得到实数特征值。此外,三角恒等式 $\sin^2 x = 2 \sin x \cos x$, $\sin^2 x = (1 - \cos 2x)/2$, $\cos^2 x = (1 + \cos 2x)/2$(应用于 $x = \alpha/2$)将有助于简化特征向量的表达式。

2.12. 你能从几何上描述上一问题中矩阵 $A$ 所代表的线性变换吗?它有一个非常简单的几何解释。

2.13. 证明一个具有模为 1 的特征值(即所有特征值满足 $|\lambda_k| = 1$)的正规算子是酉的。
\textbf{提示:} 考虑对角化。

2.14. 证明一个具有实数特征值的正规算子是自伴随的。




\section{正定算子~平方根}
\textbf{定义}~ 称自伴随算子 $A: X \to X$ 为\textbf{正定}的,如果 $(Ax, x) > 0$ $\forall x \neq 0$,称其为\textbf{半正定}的,如果 $(Ax, x) \geq 0$ $\forall x \in X$。我们将使用记号 $A > 0$ 表示正定算子,$A \geq 0$ 表示半正定算子。下面的定理描述了正定和半正定算子。

\textbf{定理 3.1。} 设 $A = A^*$。那么
1. $A > 0$ 当且仅当 $A$ 的所有特征值都是正的。
2. $A \geq 0$ 当且仅当 $A$ 的所有特征值都是非负的。
3. $A < 0$ 当且仅当 $A$ 的所有特征值都是负的。
4. $A \leq 0$ 当且仅当 $A$ 的所有特征值都是非正的。
5. $A$ 是不定的,当且仅当它既有正特征值也有负特征值。

\textbf{证明}~ 通过选取一个标准正交基,使得 $A$ 在该基下的矩阵是对角矩阵(见定理 2.1),我们可以无损于一般性地证明。要完成证明,只需注意到,对于对角矩阵,当且仅当其对角线元素都为正(非负)时,该矩阵才是正定(半正定)的。

\textbf{推论 3.2。} 设 $A = A^* \geq 0$ 是一个半正定算子。存在一个唯一的半正定算子 $B$,使得 $B^2 = A$。这样的 $B$ 被称为 $A$ 的(正)\textbf{平方根},并记作 $\sqrt{A}$ 或 $A^{1/2}$。

\textbf{证明}~ 我们来证明 $\sqrt{A}$ 的存在性。设 $\{v_1, v_2, \dots, v_n\}$ 是 $A$ 的特征向量的标准正交基,并设 $\lambda_1, \lambda_2, \dots, \lambda_n$ 是相应的特征值。注意,由于 $A \geq 0$,所有 $\lambda_k \geq 0$。在基 $\{v_1, v_2, \dots, v_n\}$ 下,$A$ 的矩阵是对角矩阵 $\text{diag}\{\lambda_1, \lambda_2, \dots, \lambda_n\}$,对角线上是 $\lambda_1, \lambda_2, \dots, \lambda_n$。定义 $B$ 在同一基下的矩阵为 $\text{diag}\{\sqrt{\lambda_1}, \sqrt{\lambda_2}, \dots, \sqrt{\lambda_n}\}$。显然,$B = B^* \geq 0$ 且 $B^2 = A$。

为了证明 $B$ 的唯一性,让我们假设存在一个算子 $C = C^* \geq 0$ 使得 $C^2 = A$。设 $\{u_1, u_2, \dots, u_n\}$ 是 $C$ 的特征向量的标准正交基,并设 $\mu_1, \mu_2, \dots, \mu_n$ 是相应的特征值(注意 $\mu_k \geq 0$ $\forall k$)。$C$ 在该基下的矩阵是对角矩阵 $\text{diag}\{\mu_1, \mu_2, \dots, \mu_n\}$,因此 $A = C^2$ 在同一基下的矩阵是 $\text{diag}\{\mu_1^2, \mu_2^2, \dots, \mu_n^2\}$。这暗示 $A$ 的任何特征值 $\lambda$ 都必须是 $\mu_k^2$ 的形式,并且,更重要的是,如果 $Ax = \lambda x$,那么 $Cx = \sqrt{\lambda} x$。因此,在上面的基 $\{v_1, v_2, \dots, v_n\}$ 下,$C$ 的矩阵是对角矩阵 $\text{diag}\{\sqrt{\lambda_1}, \sqrt{\lambda_2}, \dots, \sqrt{\lambda_n}\}$,即 $B = C$。

3.2. \textbf{算子的模。奇异值。} 考虑算子 $A: X \to Y$。它的\textbf{Hermitian平方} $A^*A$ 是作用在 $X$ 上的半正定算子。确实,$(A^*A)^* = A^*(A^*)^* = A^*A$ 并且 $(A^*Ax, x) = (Ax, Ax) = \|Ax\|^2 \geq 0$ $\forall x \in X$。因此,存在一个(唯一的)半正定算子 $R = \sqrt{A^*A}$。这个算子 $R$ 被称为算子 $A$ 的\textbf{模},通常记为 $|A|$。$A$ 的模显示了算子 $A$ 的“大小”:

\textbf{命题 3.3。} 对于线性算子 $A: X \to Y$,$\| |A| x \| = \|Ax\|$ $\forall x \in X$。

\textbf{证明}~ 对于任何 $x \in X$,$\| |A| x \|^2 = (|A|x, |A|x) = (|A|^*|A|x, x) = ( |A|^2 x, x ) = (A^*Ax, x) = (Ax, Ax) = \|Ax\|^2$。

\textbf{推论 3.4。} $\text{Ker } A = \text{Ker } |A| = (\text{Ran } |A|)^\perp$。

\textbf{证明}~ 第一个等式直接来自命题 3.3,第二个等式来自恒等式 $\text{Ker } T = (\text{Ran } T^*)^\perp$($|A|$ 是自伴随的)。

\textbf{定理 3.5(算子的极分解)。} 设 $A: X \to X$ 是一个算子(方阵)。那么 $A$ 可以表示为 $A = U|A|$,其中 $U$ 是酉算子。

\textbf{注释} ~酉算子 $U$ 通常不是唯一的。正如从定理的证明中可以看出,$U$ 仅在 $A$ 可逆时才唯一。

\textbf{注释} ~极分解 $A = U|A|$ 也适用于作用在一个空间到另一个空间上的算子 $A: X \to Y$。但在这种情况下,我们只能保证 $U$ 是从 $\text{Ran } |A| = (\text{Ker } A)^\perp$ 到 $Y$ 的一个等距同构。如果 $\dim X \leq \dim Y$,则此等距同构可以扩展为从整个 $X$ 到 $Y$ 的等距同构(如果 $\dim X = \dim Y$,则它将是一个酉算子)。

\textbf{定理 3.5 的证明。} 考虑向量 $x \in \text{Ran } |A|$。那么向量 $x$ 可以表示为 $x = |A|v$ 对于某个向量 $v \in X$。定义 $U_0 x := Av$。根据命题 3.3 $\|U_0 x\| = \|Ax\| = \||A|v\| = \|x\|$,所以看起来 $U$ 是从 $\text{Ran } |A|$ 到 $X$ 的一个等距同构。但首先我们需要证明 $U_0$ 是良好定义的。设 $v_1$ 是另一个使得 $x = |A|v_1$ 的向量。但是 $x = |A|v = |A|v_1$ 意味着 $v - v_1 \in \text{Ker } |A| = \text{Ker } A$(参见推论 3.4),所以 $Av = Av_1$,这意味着 $U_0 x$ 是良好定义的。根据构造,$A = U_0|A|$。我们将 $U_0$ 扩展为一个酉算子 $U$ 的证明留给读者,读者需要检查 $U_0$ 是一个线性变换。设 $U = U_0 + U_1$,其中 $U_1$ 是从 $(\text{Ran } |A|)^\perp = \text{Ker } A$ 到 $(\text{Ran } A)^\perp$ 的一个酉算子(注意,从秩定理可知 $\dim \text{Ker } A = \dim \text{Ker } A^* = \dim (\text{Ran } A)^\perp$)。容易看出 $U$ 是一个酉算子,并且 $A = U|A|$。

3.3. \textbf{奇异值。施密特分解。}
\textbf{定义}~ $|A|$ 的特征值被称为 $A$ 的\textbf{奇异值}。换句话说,如果 $\lambda_1, \lambda_2, \dots, \lambda_n$ 是 $A^*A$ 的特征值,那么 $\sqrt{\lambda_1}, \sqrt{\lambda_2}, \dots, \sqrt{\lambda_n}$ 就是 $A$ 的奇异值。
\textbf{注释} ~在许多文献中,奇异值被定义为 $A^*A$ 的特征值的非负平方根,而不提及算子 $|A|$。我认为算子 $|A|$ 的概念很重要,所以上面已经介绍了。然而,算子 $|A|$ 的概念对于后续内容(定义舒尔和奇异值分解)不是必需的。此外,正如下面将要显示的,算子 $|A|$ 可以很容易地从奇异值分解构造出来。

设 $A: X \to Y$ 是一个算子,并设 $\sigma_1, \sigma_2, \dots, \sigma_n$ 是 $A$ 的奇异值(计入重数)。假设 $\sigma_1, \sigma_2, \dots, \sigma_r$ 是 $A$ 的\textbf{非零}奇异值(计入重数)。这意味着,特别地,$\sigma_k = 0$ 对于 $k > r$。




根据奇异值的定义,数字 $\sigma_1^2, \sigma_2^2, \dots, \sigma_n^2$ 是 $A^*A$ 的特征值。设 $\{v_1, v_2, \dots, v_n\}$ 是 $A^*A$ 的特征向量的标准正交基,$A^*Av_k = \sigma_k^2 v_k$。

\textbf{命题 3.6。} 系统 $\{w_k := \frac{1}{\sigma_k} Av_k, k = 1, 2, \dots, r\}$ 是一个标准正交系统。

\textbf{证明}~ $(Av_j, Av_k) = (A^*Av_j, v_k) = (\sigma_j^2 v_j, v_k) = \sigma_j^2 (v_j, v_k) = \begin{cases} 0, & j \neq k \\ \sigma_j^2, & j = k \end{cases}$,因为 $\{v_1, v_2, \dots, v_r\}$ 是一个标准正交系统。

在上述命题的记号中,算子 $A$ 可以表示为
$A = \sum_{k=1}^r \sigma_k w_k v_k^*$, (3.1)
或者等价地
$Ax = \sum_{k=1}^r \sigma_k (x, v_k) w_k$。(3.2)
确实,我们知道 $\{v_1, v_2, \dots, v_n\}$ 是 $X$ 的一个标准正交基。那么将 $x = v_j$ 代入 (3.2) 的右侧,我们得到 $\sum_{k=1}^r \sigma_k (v_j, v_k) w_k = \sigma_j (v_j, v_j) w_j = \sigma_j w_j = Av_j$ 如果 $j=1, 2, \dots, r$,并且 $\sum_{k=1}^r \sigma_k (v_k^* v_j) w_k = 0 = Av_j$ 对于 $j > r$。所以,(3.1) 中左右两侧的算子在基 $\{v_1, v_2, \dots, v_n\}$ 上是相同的,因此它们是相等的。

\textbf{定义}~ 上述分解 (3.1)(或 (3.2))被称为算子 $A$ 的\textbf{施密特分解}。

\textbf{注释} ~施密特分解不是唯一的。为什么?



\textbf{引理 3.7。} 设 $A = \sum_{k=1}^r \sigma_k w_k v_k^*$,其中 $\sigma_k > 0$ 并且 $\{v_1, v_2, \dots, v_r\}$, $\{w_1, w_2, \dots, w_r\}$ 是某些标准正交系统。那么这个表示给出了 $A$ 的施密特分解。

\textbf{证明}~ 我们只需要证明 $v_1, v_2, \dots, v_r$ 是 $A^*A$ 的特征向量,$A^*Av_k = \sigma_k^2 v_k$。由于 $\{w_1, w_2, \dots, w_r\}$ 是标准正交系统,$w_k^* w_j = (w_j, w_k) = \delta_{k,j} := \begin{cases} 0, & j \neq k \\ 1, & j = k \end{cases}$,因此 $A^*A = \sum_{k=1}^r \sigma_k^2 v_k v_k^*$。由于 $\{v_1, v_2, \dots, v_r\}$ 是标准正交系统,$A^*Av_j = \sum_{k=1}^r \sigma_k^2 v_k v_k^* v_j = \sigma_j^2 v_j$,因此 $v_k$ 是 $A^*A$ 的特征向量。

\textbf{推论 3.8。} 设 $A = \sum_{k=1}^r \sigma_k w_k v_k^*$ 是 $A$ 的施密特分解。那么 $A^* = \sum_{k=1}^r \sigma_k v_k w_k^*$ 是 $A^*$ 的施密特分解。

3.4. \textbf{施密特分解的矩阵表示。奇异值分解。} 施密特分解可以写成一个很好的矩阵形式。即,假设 $A: F^n \to F^m$(这里 $F$ 总是 $\mathbb{C}$ 或 $\mathbb{R}$;我们可以通过选取 $X$ 和 $Y$ 中的标准正交基并处理这些基下的坐标来完成)。设 $\sigma_1, \sigma_2, \dots, \sigma_r$ 是 $A$ 的非零奇异值(计入重数),并设 $A = \sum_{k=1}^r \sigma_k w_k v_k^*$ 是 $A$ 的施密特分解。
如你所见,这个等式可以重写为
$A = \tilde{W} \tilde{\Sigma} \tilde{V}^*$, (3.3)
其中 $\tilde{\Sigma} = \text{diag}\{\sigma_1, \sigma_2, \dots, \sigma_r\}$ 并且 $\tilde{V}$ 和 $\tilde{W}$ 是分别以 $v_1, v_2, \dots, v_r$ 和 $w_1, w_2, \dots, w_r$ 为列的矩阵。(你能说出每个矩阵的大小吗?)注意,由于 $\{v_1, v_2, \dots, v_r\}$ 和 $\{w_1, w_2, \dots, w_r\}$ 是标准正交系统,矩阵 $\tilde{V}$ 和 $\tilde{W}$ 是等距同构。还需注意 $r = \text{rank } A$,见下面的练习 3.1。如果矩阵 $A$ 是可逆的,那么 $m=n=r$,矩阵 $\tilde{V}$, $\tilde{W}$ 是酉的,并且 $\tilde{\Sigma}$ 是一个可逆的对角矩阵。事实证明,总是可以写出一个类似的表示(3.3),用酉矩阵 $V$ 和 $W$ 来代替 $\tilde{V}$ 和 $\tilde{W}$,并且在许多情况下,处理这样的表示会更方便。
为了写出这个表示,我们首先需要将系统 $\{v_1, v_2, \dots, v_r\}$ 和 $\{w_1, w_2, \dots, w_r\}$ \textbf{补全}为 $F^n$ 和 $F^m$ 中的正交基。回想一下,要将 $\{v_1, v_2, \dots, v_r\}$ 补全为 $F^n$ 中的标准正交基,只需找到 $\text{Ker } V^*$ 的一个标准正交基 $\{v_{r+1}, \dots, v_n\}$;那么系统 $\{v_1, v_2, \dots, v_n\}$ 将是 $F^n$ 中的一个标准正交基。并且人们总是能通过格拉姆-施密特正交化从任意系统得到一个标准正交基。然后 $A$ 可以表示为
$A = W \Sigma V^*$, (3.4)
其中 $V \in M_F^{n \times n}$ 和 $W \in M_F^{m \times m}$ 是以 $v_1, v_2, \dots, v_n$ 和 $w_1, w_2, \dots, w_m$ 为列的酉矩阵,而 $\Sigma \in M_\mathbb{R}^{+, m \times n}$ 是一个“对角”矩阵(意思是 $\sigma_{k,k} \geq 0$ 对于所有 $k = 1, 2, \dots, \min\{m, n\}$,并且 $\sigma_{j,k} = 0$ 对于所有 $j \neq k$)。(3.5)
也就是说,为了得到矩阵 $\Sigma$,你需要取对角矩阵 $\text{diag}\{\sigma_1, \sigma_2, \dots, \sigma_r\}$ 并通过在“南方”和“东方”添加额外的零将其变成一个 $m \times n$ 矩阵。

\textbf{定义 3.9。} 对于矩阵 $A \in M_F^{m \times n}$(这里 $F$ 总是 $\mathbb{C}$ 或 $\mathbb{R}$),其\textbf{奇异值分解} (SVD) 是形如 (3.4) 的分解,即分解 $A = W \Sigma V^*$,其中 $W \in M_F^{n \times n}$ 和 $V \in M_F^{m \times m}$ 是酉矩阵,而 $\Sigma \in M_\mathbb{R}^{+, m \times n}$ 是“对角”矩阵(意思是 $\sigma_{k,k} \geq 0$ 对于所有 $k = 1, 2, \dots, \min\{m, n\}$,并且 $\sigma_{j,k} = 0$ 对于所有 $j \neq k$)。更精确地说,\textbf{约简 SVD} 是一个表示 $A = \tilde{W} \tilde{\Sigma} \tilde{V}^*$,其中 $\tilde{\Sigma} \in M_\mathbb{R}^{+, r \times r}$, $r \leq \min\{m, n\}$ 是一个对角矩阵,其对角线元素严格为正,而 $\tilde{W} \in M_F^{n \times r}$, $\tilde{V} \in M_F^{m \times r}$ 是等距同构;而且,我们要求 $\tilde{W}$ 和 $\tilde{V}$ 中至少有一个不是方阵。

\textbf{注 3.10。} 很容易看出,如果 $A = W \Sigma V^*$ 是 $A$ 的奇异值分解,那么 $\sigma_k := \sigma_{k,k}$ 是 $A$ 的奇异值,即 $\sigma_k^2$ 是 $A^*A$ 的特征值。而且,$V$ 的列 $v_k$ 是 $A^*A$ 的相应特征向量,$A^*Av_k = \sigma_k^2 v_k$。还要注意,如果 $\sigma_k \neq 0$,那么 $w_k = \frac{1}{\sigma_k} Av_k$。所有这些都意味着任何奇异值分解 $A = W \Sigma V^*$ 都可以通过本节上面描述的构造从施密特分解 (3.2) 得到。对于不可逆矩阵 $A$,约简奇异值分解可以解释为施密特分解 (3.2) 的矩阵形式。对于可逆矩阵 $A$,施密特分解的矩阵形式给出了奇异值分解。

\textbf{注 3.11。} $A = W \Sigma V^*$ 的奇异值分解的另一种解释是,$\Sigma$ 是 $A$ 在(标准正交)基 $\{v_1, v_2, \dots, v_n\}$ 和 $\{w_1, w_2, \dots, w_n\}$ 下的矩阵,即 $\Sigma = [A]_{B,A}$。我们将在后面使用这个解释。

3.4.1. \textbf{从奇异值分解到极坐标分解。} 注意,如果我们知道方阵 $A$ 的奇异值分解 $A = W \Sigma V^*$,我们可以写出 $A$ 的极坐标分解:
(3.6) $A = W \Sigma V^* = (WV^*) (V \Sigma V^*) = U|A|$
其中 $|A| = V \Sigma V^*$ 并且 $U = WV^*$。为了说明这确实是一个极坐标分解,让我们注意到 $V\Sigma V^*$ 是一个自伴随的、半正定的算子,并且 $A^*A = (W\Sigma V^*)^*(W\Sigma V^*) = V \Sigma^* W^* W \Sigma V^* = V \Sigma^*\Sigma V^* = V (\Sigma^* \Sigma) V^* = (V \Sigma V^*)(V \Sigma V^*) = (|A|)^2$。所以根据 $|A|$ 的定义(它是 $A^*A$ 的唯一半正定平方根),我们可以看出 $|A| = V \Sigma V^*$。变换 $WV^*$ 显然是一个酉变换,因为它是由两个酉变换相乘得到的,所以(3.6)确实给出了 $A$ 的一个极坐标分解。请注意,此推理仅适用于方阵,因为如果 $A$ 不是方阵,则乘积 $V\Sigma$ 是未定义的(维度不匹配,你能看出为什么吗?)。

\textbf{练习}~

3.1. 证明矩阵 $A$ 的非零奇异值的数量(计入重数)与其秩相等。





3.2. 为以下矩阵 $A$ 找出施密特分解 $A = \sum_{k=1}^r s_k w_k v_k^*$:
$\begin{pmatrix} 2 & 3 \\ 0 & 2 \end{pmatrix}$, $\begin{pmatrix} 7 & 1 & 0 \\ 0 & 0 & 5 \\ 5 & 0 & 5 \end{pmatrix}$, $\begin{pmatrix} 1 & 1 & 0 \\ 1 & 2 & 2 \\ 0 & -1 & 1 \end{pmatrix}$。

3.3. 设 $A$ 是一个可逆矩阵,设 $A = W \Sigma V^*$ 是它的奇异值分解。求 $A^*$ 和 $A^{-1}$ 的奇异值分解。

3.4. 为以下矩阵 $A$ 找出奇异值分解 $A = W \Sigma V^*$,其中 $V$ 和 $W$ 是酉矩阵:
a) $A = \begin{pmatrix} -3 & 1 \\ 6 & -2 \\ 6 & -2 \end{pmatrix}$;
b) $A = \begin{pmatrix} 3 & 2 & 2 \\ 2 & 3 & -2 \end{pmatrix}$。

3.5. 找出矩阵 $A = \begin{pmatrix} 2 & 3 \\ 0 & 2 \end{pmatrix}$ 的奇异值分解。使用它来找出:
a) $\max_{\|x\| \leq 1} \|Ax\|$ 以及最大值达到的向量;
b) $\min_{\|x\|=1} \|Ax\|$ 以及最小值达到的向量;
c) $A$ 对 $\mathbb{R}^2$ 中的闭单位球 $B = \{x \in \mathbb{R}^2 : \|x\| \leq 1\}$ 的像 $A(B)$。几何上描述 $A(B)$。

3.6. 证明对于方阵 $A$,$|\det A| = \det |A|$。

3.7. 真或假:
a) 矩阵的奇异值也是该矩阵的特征值。
b) 矩阵 $A$ 的奇异值是 $A^*A$ 的特征值。
c) 如果 $s$ 是矩阵 $A$ 的一个奇异值,而 $c$ 是一个标量,那么 $|c|s$ 是 $cA$ 的奇异值。
d) 任何线性算子的奇异值都是非负的。
e) 自伴随矩阵的奇异值与其特征值相等。

3.8. 设 $A$ 是一个 $m \times n$ 矩阵。证明 $A^*A$ 和 $AA^*$ 的\textbf{非零}特征值(计入重数)是相同的。你能说出 $A^*A$ 的零特征值和 $AA^*$ 的零特征值何时具有相同的重数吗?

3.9. 设 $s$ 是算子 $A$ 的最大奇异值,设 $\lambda$ 是 $A$ 具有最大绝对值的特征值。证明 $|\lambda| \leq s$。

3.10. 证明矩阵的秩等于其非零奇异值的数量(计入重数)。





3.11. 证明算子范数 $\|A\|$ 与 Frobenius 范数 $\|A\|_2$ 相等当且仅当该矩阵秩为 1。
\textbf{提示:} 上一个问题可能有所帮助。

3.12. 对于矩阵 $A = \begin{pmatrix} 2 & -3 \\ 0 & 2 \end{pmatrix}$,描述单位球的逆像,即所有 $x \in \mathbb{R}^2$ 使得 $\|Ax\| \leq 1$ 的集合。使用奇异值分解。

\section{奇异值分解的应用}

正如上面讨论的,奇异值分解仅仅是在两个不同标准正交基下的对角化。由于这里有两个不同的基,我们无法从其奇异值分解中得知关于算子谱性质的太多信息。例如,奇异值分解中的 $\Sigma$ 的对角线元素不是 $A$ 的特征值。注意,对于 $A = W \Sigma V^*$,其中 $A$ 不一定是方阵,并且(或)奇异值不全非零。考虑“对角”矩阵 $\Sigma$ 的形式 (3.5)。很容易看出单位球 $B$ 的像 $\Sigma(B)$ 是一个椭球体(不是在整个空间中,而是在 $\text{Ran } \Sigma$ 中)其半轴为 $\sigma_1, \sigma_2, \dots, \sigma_r$。考虑一般情况, $A = W \Sigma V^*$,其中 $W$ 和 $V$ 是酉算子。酉变换不改变单位球(因为它们保持范数),所以 $V^*(B) = B$。我们知道 $\Sigma(B)$ 是 $\text{Ran } \Sigma$ 中的一个椭球体,其半轴为 $\sigma_1, \sigma_2, \dots, \sigma_r$。酉变换不改变物体的几何形状,所以 $W(\Sigma(B))$ 也是一个具有相同半轴的椭球体。很容易从分解 $A = W \Sigma V^*$(利用 $W$ 和 $V^*$ 都是可逆的事实)看出 $W$ 将 $\text{Ran } \Sigma$ 映射到 $\text{Ran } A$,所以我们可以得出结论:闭单位球 $B$ 的像 $A(B)$ 是 $\text{Ran } A$ 中的一个椭球体,其半轴为 $\sigma_1, \sigma_2, \dots, \sigma_r$。这里 $r$ 是非零奇异值的数量,即 $A$ 的秩。

4.2. \textbf{线性变换的算子范数。} 给定一个线性变换 $A: X \to Y$,让我们考虑以下优化问题:找到在闭单位球 $B = \{x \in X : \|x\| \leq 1\}$ 上 $\|Ax\|$ 的最大值。同样,奇异值分解允许我们解决这个问题。对于具有非负项的对角矩阵 $A$,最大值恰好是最大的对角项。确实,设 $s_1, s_2, \dots, s_n$ 是 $A$ 的非零对角项,设 $s_1$ 是最大的。由于对于 $x = (x_1, x_2, \dots, x_n)^T$,
$Ax = \sum_{k=1}^n s_k x_k e_k$, (4.1)
我们可以得出 $\|Ax\|^2 = \sum_{k=1}^n s_k^2 |x_k|^2 \leq s_1^2 \sum_{k=1}^n |x_k|^2 = s_1^2 \|x\|^2$,所以 $\|Ax\| \leq s_1 \|x\|$。另一方面,$\|Ae_1\| = \|s_1 e_1\| = s_1 \|e_1\|$,所以确实 $s_1$ 是闭单位球 $B$ 上 $\|Ax\|$ 的最大值。注意,在上述推理中,我们没有假设矩阵 $A$ 是方阵,我们只假设主对角线以外的所有项都为零,所以公式 (4.1) 成立。为了处理一般情况,让我们考虑奇异值分解 (3.5), $A = W \Sigma V^*$,其中 $W$ 和 $V$ 是酉算子,而 $\Sigma$ 是具有非负项的对角矩阵。由于酉变换不改变范数,我们可以得出 $B$ 上 $\|Ax\|$ 的最大值等于 $\Sigma$ 的最大对角项,即 $A$ 的算子范数等于其最大奇异值,即 $\|A\| = s_1$。所以我们可以得出 $\|A\| \leq \|A\|_2$,即矩阵的算子范数不能大于其 Frobenius 范数。这个陈述也可以通过使用柯西-施瓦茨不等式直接证明,并且这样的证明在一些教科书中都有介绍。我们这里提出的证明的美妙之处在于,它不需要任何计算,并且阐明了不等式背后的原因。

\textbf{定义}~ $A: X \to Y$ 的线性变换的\textbf{算子范数}被定义为 $\max\{\|Ax\| : x \in X, \|x\| \leq 1\}$,记作 $\|A\|$。很容易看出 $\|A\|$ 满足范数的所有性质:
1. $\|\alpha A\| = |\alpha| \|A\|$ 对于所有 $A$ 和所有标量 $\alpha$。
2. $\|A + B\| \leq \|A\| + \|B\|$。
3. $\|A\| \geq 0$ 对于所有 $A$;
4. $\|A\| = 0$ 当且仅当 $A = 0$,所以它确实是 $X$ 到 $Y$ 的线性变换空间上的一个范数。算子范数的一个主要性质是不等式 $\|Ax\| \leq \|A\|\|x\|$,这很容易从范数 $\|x\|$ 的齐次性得出。事实上,可以证明算子范数 $\|A\|$ 是最小的数 $C \geq 0$,使得 $\|Ax\| \leq C\|x\|$ $\forall x \in X$。这通常作为算子范数的定义。在线性变换空间中,我们已经有一个范数,即 Frobenius 范数,或 Hilbert-Schmidt 范数 $\|A\|_2 = \sqrt{\text{trace}(A^*A)}$。所以,让我们研究一下这两个范数如何比较。设 $s_1, s_2, \dots, s_n$ 是 $A$ 的非零奇异值(计入重数),设 $s_1$ 是最大的奇异值。那么 $s_1^2, s_2^2, \dots, s_n^2$ 是 $A^*A$ 的非零特征值(同样计入重数)。回想起迹等于特征值之和,我们得出 $\|A\|_2^2 = \text{trace}(A^*A) = \sum_{k=1}^n s_k^2$。另一方面,我们知道算子范数 $\|A\|$ 等于其最大奇异值,即 $\|A\| = s_1$。所以我们可以得出 $\|A\| \leq \|A\|_2$,即矩阵的算子范数不能大于其 Frobenius 范数。这个陈述也接受一个直接证明,使用柯西-施瓦茨不等式,并且这样的证明在一些教科书中被提出。我们这里提出的证明的美妙之处在于,它不需要任何计算,并且阐明了不等式背后的原因。

4.3. \textbf{矩阵的条件数。} 假设我们有一个可逆矩阵 $A$ 并且我们想求解方程 $Ax = b$。当然,解由 $x = A^{-1}b$ 给出,但我们想研究如果我们只近似知道数据会发生什么。这种情况在现实生活中会发生,当数据例如通过某些实验获得时。但即使我们有精确的数据,计算机计算过程中的舍入误差也可能产生相同的影响,即扭曲数据。让我们考虑最简单的模型,假设右侧方程有一个小的误差。这意味着,我们求解的是 $Ax = b + \Delta b$,而不是 $Ax = b$,其中 $\Delta b$ 是右侧 $b$ 的微小扰动。所以,我们得到的是 $A(x + \Delta x) = b + \Delta b$ 的近似解,而不是精确解 $x$。我们假设 $A$ 是可逆的,所以 $\Delta x = A^{-1}\Delta b$。我们想知道解中的相对误差 $\| \Delta x \| / \| x \|$ 与右侧的相对误差 $\| \Delta b \| / \| b \|$ 相比有多大。很容易看出 $\| \Delta x \| / \| x \| = \| A^{-1} \Delta b \| / \| x \| = \| A^{-1} \Delta b \| / \| b \| \cdot \| b \| / \| x \| = \| A^{-1} \Delta b \| / \| b \| \cdot \| b \| / \| x \|$。由于 $\| A^{-1} \Delta b \| \leq \| A^{-1} \| \cdot \| \Delta b \|$ 且 $\| Ax \| \leq \| A \| \cdot \| x \|$,我们可以得出 $\| \Delta x \| / \| x \| \leq \| A^{-1} \| \cdot \| A \| \cdot \| \Delta b \| / \| b \|$。数量 $\|A\| \cdot \|A^{-1}\|$ 被称为矩阵的\textbf{条件数}。它估计了解 $x$ 中的相对误差如何依赖于右侧 $b$ 中的相对误差。让我们看看这个数量与奇异值有何关系。设 $\sigma_1, \sigma_2, \dots, \sigma_n$ 是 $A$ 的奇异值,并假设 $\sigma_1$ 是最大奇异值,$\sigma_n$ 是最小奇异值。我们知道算子范数 $\|A\|$ 等于其最大奇异值,所以 $\|A\| = \sigma_1$, $\|A^{-1}\| = 1/\sigma_n$,所以 $\|A\| \cdot \|A^{-1}\| = \sigma_1/\sigma_n$。换句话说,矩阵的条件数等于最大和最小奇异值之比。





我们上面推导出 $\| \Delta x \| / \| x \| \leq \| A^{-1} \| \cdot \| A \| \cdot \| \Delta b \| / \| b \|$。不难看出这个估计是尖锐的,即可以选择右侧 $b$ 和误差 $\Delta b$ 使得我们得到等式 $\| \Delta x \| / \| x \| = \| A^{-1} \| \cdot \| A \| \cdot \| \Delta b \| / \| b \|$。我们只需选择 $b = w_1$ 和 $\Delta b = \alpha w_n$,其中 $w_1$ 和 $w_n$ 分别是奇异值分解 $A = W \Sigma V^*$ 中 $W$ 的第一个和最后一个列,$\alpha \neq 0$ 是任意标量。这里,正如通常所做的那样,奇异值假定为非递增顺序 $\sigma_1 \geq \sigma_2 \geq \dots \geq \sigma_n$,所以 $\sigma_1$ 是最大的,$\sigma_n$ 是最小的。我们将细节留给读者。如果一个矩阵的条件数不是太大的话,它被称为\textbf{良适定}的。如果条件数很大,则该矩阵被称为\textbf{病态}的。这里“大”取决于具体问题:你能在什么精度下找到你的右侧,$x$ 的解需要什么精度等等。

4.4. \textbf{矩阵的有效秩。} 理论上,计算矩阵的秩很容易:只需对矩阵进行行变换并计数主元。然而,在实际应用中,并非一切都那么容易。主要原因是,我们通常不知道精确的矩阵,只知道其近似值,精确到某个精度。此外,即使我们知道精确矩阵,大多数计算机程序在计算过程中也会引入舍入误差,所以我们实际上无法区分零主元和非常小的非零主元。一个简单的朴素想法是处理舍入误差:计算秩(以及与它相关的其他对象,如列空间、核等),只需设置一个容差(某个小的数),如果主元小于容差,则将其视为零。这种方法的优点是其简单性,因为它非常容易编程。然而,主要缺点是无法看出容差的作用。例如,如果我们设置容差为 $10^{-6}$,我们会丢失什么?如果设置为 $10^{-8}$ 会好多少?虽然上述方法对于良适定的矩阵效果很好,但在一般情况下并不可靠。更好的方法是使用奇异值。这需要更多的计算,但结果要好得多,也更容易解释。在这种方法中,我们也设置一个小的数字作为容差,然后执行奇异值分解。然后,我们只需将小于容差的奇异值视为零。这种方法的优点是我们能够看到我们正在做什么。奇异值是椭球体 $A(B)$($B$ 是闭单位球)的半轴,所以通过设置容差,我们只是决定椭球体应该“多薄”才能被认为是“扁平”的。

4.5. \textbf{摩尔-彭罗斯(伪)逆。} 正如我们在第 5 章第 4 节中讨论的,最小二乘解在方程 $Ax = b$ 没有解的情况下,为我们提供了“最好的替代方案”(并且在方程存在解时,为我们提供了 $Ax = b$ 的解)。注意,最小二乘解并没有解决唯一性的问题:正规方程 $A^*Ax = A^*b$ 的解不一定唯一。一个自然的、特定的解是具有最小范数的解;这样的解确实是唯一的,并且可以通过取任意一个解,然后将其投影到 $(\text{Ker } A^*)^\perp = (\text{Ker } A)^\perp$ 来得到(见第 5 章问题 4.5 和 4.6)。不难看出,如果 $A = \tilde{W} \tilde{\Sigma} \tilde{V}^*$ 是 $A$ 的\textbf{约简}奇异值分解,那么最小范数最小二乘解 $x_0$ 由下式给出:
$x_0 = \tilde{V} \tilde{\Sigma}^{-1} \tilde{W}^* b$。(4.2)
确实,$x_0$ 是 $Ax = b$ 的最小二乘解(即 $Ax = P_{\text{Ran } A} b$ 的解):$Ax_0 = \tilde{W} \tilde{\Sigma} \tilde{V}^* \tilde{V} \tilde{\Sigma}^{-1} \tilde{W}^* b = \tilde{W} \tilde{\Sigma} \tilde{\Sigma}^{-1} \tilde{W}^* b = \tilde{W} \tilde{W}^* b = P_{\text{Ran } A} b$(在最后一个等式链中,我们使用了 $\tilde{W} \tilde{W}^* = P_{\text{Ran } \tilde{W}}$($P_{\text{Ran } \tilde{W}} = \tilde{W}(\tilde{W}^*\tilde{W})^{-1}\tilde{W}^* = \tilde{W}\tilde{W}^*$)并且 $\text{Ran } \tilde{W} = \text{Ran } A$(见下面的问题 4.4)。$Ax = P_{\text{Ran } A} b$ 的一般解由 $x = x_0 + y$, $y \in \text{Ker } A$ 给出,所以 $x_0$ 确实是 $Ax = b$ 的最小范数最小二乘解。

\textbf{定义 4.1。} 算子 $A^+ := \tilde{V} \tilde{\Sigma}^{-1} \tilde{W}^*$,其中 $A = \tilde{W} \tilde{\Sigma} \tilde{V}^*$ 是 $A$ 的\textbf{约简}奇异值分解,被称为 $A$ 的\textbf{摩尔-彭罗斯逆}(或\textbf{摩尔-彭罗斯伪逆})。换句话说,\textbf{摩尔-彭罗斯逆}是给出 $Ax = b$ 的唯一最小范数最小二乘解的算子。

\textbf{注 4.2。} 在文献中,摩尔-彭罗斯逆通常定义为一个矩阵 $A^+$,满足:
1. $AA^+A = A$;
2. $A^+AA^+ = A^+$;
3. $(AA^+)^* = AA^+$;
4. $(A^+A)^* = A^+A$。
很容易验证算子 $A^+ := \tilde{V} \tilde{\Sigma}^{-1} \tilde{W}^*$ 满足上述性质 1-4。也可以(虽然稍微困难一些)证明满足性质 1-4 的算子 $A^+$ 是唯一的。确实,通过将恒等式 1 分别用 $A^+$ 从左边和右边乘以,我们得到 $(A^+A)^2 = A^+A$ 和 $(AA^+)^2 = AA^+$;结合性质 3 和 4,这意味着 $A^+A$ 和 $AA^+$ 是正交投影(见第 5 章问题 5.6)。显然,$\text{Ker } A \subseteq \text{Ker } A^+A$。另一方面,恒等式 1 暗示 $\text{Ker } A^+A \subseteq \text{Ker } A$(为什么?),所以 $\text{Ker } A^+A = \text{Ker } A$。但这表示 $A^+A$ 是到 $(\text{Ker } A)^\perp = \text{Ran } A^*$ 的正交投影,$A^+A = P_{\text{Ran } A^*}$。性质 1 也暗示 $AA^+y = y$ 对于所有 $y \in \text{Ran } A$。由于 $AA^+$ 是正交投影,我们得出 $\text{Ran } A \subseteq \text{Ran } AA^+$。相反的包含关系 $\text{Ran } AA^+ \subseteq \text{Ran } A$ 是平凡的,所以 $AA^+$ 是到 $\text{Ran } A$ 的正交投影,$AA^+ = P_{\text{Ran } A}$。知道 $A^+A$ 和 $AA^+$,我们可以重写性质 2 为 $P_{\text{Ran } A^*} A^+ = A^+$ 或 $A^+ P_{\text{Ran } A} = A^+$。结合上述恒等式,我们得到 $P_{\text{Ran } A^*} A^+ P_{\text{Ran } A} = A^+$。最后,对于目标空间 $A$ 的任何 $b$, $x_0 := A^+b = P_{\text{Ran } A^*} A^+ b \in \text{Ran } A^*$ 并且 $Ax_0 = AA^+ b = P_{\text{Ran } A} b$,这意味着 $x_0$ 是 $Ax = b$ 的最小二乘解。由于 $x_0 \in \text{Ran } A^* = (\text{Ker } A)^\perp$, $x_0$ 是最小范数最小二乘解,如我们上面所讨论的。但是,正如我们前面所显示的,这样的最小范数解由(4.2)给出,所以 $A^+ = \tilde{V} \tilde{\Sigma}^{-1} \tilde{W}^*$。

\textbf{练习}~

4.1. 找出以下矩阵的范数和条件数:
a) $A = \begin{pmatrix} 4 & 0 \\ 1 & 3 \end{pmatrix}$。
b) $A = \begin{pmatrix} 5 & 3 \\ -3 & 3 \end{pmatrix}$。
对于 a) 部分的矩阵 $A$,给出一个右侧 $b$ 和误差 $\Delta b$ 的例子,使得 $\|\Delta x\|/\|x\| = \|A\| \cdot \|A^{-1}\| \cdot \|\Delta b\|/\|b\|$;这里 $Ax = b$ 并且 $A \Delta x = \Delta b$。

4.2. 设 $A$ 是一个正规算子,$\lambda_1, \lambda_2, \dots, \lambda_n$ 是它的特征值(计入重数)。证明 $A$ 的奇异值是 $|\lambda_1|, |\lambda_2|, \dots, |\lambda_n|$。

4.3. 找出矩阵 $A = \begin{pmatrix} 2 & 1 & 1 \\ 1 & 2 & 1 \\ 1 & 1 & 2 \end{pmatrix}$ 的奇异值、范数和条件数。你可以在几乎不进行任何计算的情况下完成这个问题,如果你能回答以下问题:
a) 某个子空间 $E$ 上的\textbf{正交投影} $P_E$ 的奇异值是什么?
b) 由向量 $(1, 1, 1)^T$ 张成子空间的\textbf{正交投影}矩阵是什么?
c) $T$ 和 $aT + bI$ 的特征值(其中 $a$ 和 $b$ 是标量)之间有什么关系?当然,你也可以老老实实地进行计算。

4.4. 设 $A = \tilde{W} \tilde{\Sigma} \tilde{V}^*$ 是 $A$ 的\textbf{约简}奇异值分解。证明 $\text{Ran } A = \text{Ran } \tilde{W}$,然后通过取伴随得到 $\text{Ran } A^* = \text{Ran } \tilde{V}$。

4.5. 用奇异值分解 $A = W \Sigma V^*$ 写出摩尔-彭罗斯逆 $A^+$ 的公式。

4.6. \textbf{泰洪诺夫正则化:} 证明摩尔-彭罗斯逆 $A^+$ 可以计算为极限 $A^+ = \lim_{\epsilon \to 0^+} (A^*A + \epsilon I)^{-1} A^* = \lim_{\epsilon \to 0^+} A^*(AA^* + \epsilon I)^{-1}$。





\section{正交矩阵的结构}
行列式为 1 的正交矩阵 $U$ 通常被称为\textbf{旋转}。下面的定理解释了这个名称。

\textbf{定理 5.1。} 设 $U$ 是 $\mathbb{R}^n$ 中的一个正交算子,且 $\det U = 1$。那么存在一个标准正交基 $\{v_1, v_2, \dots, v_n\}$,使得 $U$ 在该基下的矩阵具有块对角形式
$\begin{pmatrix} R_{\phi_1} & & & & \\ & R_{\phi_2} & & & \\ & & \ddots & & \\ & & & R_{\phi_k} & \\ & & & & I_{n-2k} \end{pmatrix}$,
其中 $R_{\phi_k}$ 是 $2$ 维旋转矩阵,$R_{\phi_k} = \begin{pmatrix} \cos \phi_k & -\sin \phi_k \\ \sin \phi_k & \cos \phi_k \end{pmatrix}$,而 $I_{n-2k}$ 表示 $(n-2k) \times (n-2k)$ 的单位矩阵。

\textbf{证明}~ 我们利用数学归纳法来证明定理。如果 $p$ 是一个具有实数系数的多项式,$\lambda$ 是它的复数根,$p(\lambda) = 0$,那么 $\bar{\lambda}$ 也是 $p$ 的一个根,$p(\bar{\lambda}) = 0$(这可以通过将 $\lambda$ 代入 $p(z) = \sum_{k=0}^n a_k z^k$ 来轻易地验证)。因此,$A$ 的所有复数特征值可以配对为 $\lambda_k$ 和 $\bar{\lambda}_k$。我们知道酉矩阵的特征值的模为 1,所以 $A$ 的所有复数特征值都可以写成 $\lambda_k = \cos \alpha_k + i \sin \alpha_k$, $\bar{\lambda}_k = \cos \alpha_k - i \sin \alpha_k$。固定一对复数特征值 $\lambda$ 和 $\bar{\lambda}$,设 $u \in \mathbb{C}^n$ 是 $U$ 的特征向量,$Uu = \lambda u$。那么 $U\bar{u} = \bar{\lambda}\bar{u}$。现在,将 $u$ 分为实部和虚部,即定义 $x := \text{Re } u = (u + \bar{u})/2$, $y = \text{Im } u = (u - \bar{u})/(2i)$,所以 $u = x + iy$(注意,$x, y$ 是实向量,即具有实数项的向量)。那么 $Ux = U \frac{1}{2}(u + \bar{u}) = \frac{1}{2}(\lambda u + \bar{\lambda} \bar{u}) = \text{Re}(\lambda u)$。类似地,$Uy = \frac{1}{2i} U(u - \bar{u}) = \frac{1}{2i}(\lambda u - \bar{\lambda} \bar{u}) = \text{Im}(\lambda u)$。由于 $\lambda = \cos \alpha + i \sin \alpha$,我们有 $\lambda u = (\cos \alpha + i \sin \alpha)(x + iy) = ((\cos \alpha)x - (\sin \alpha)y) + i((\cos \alpha)y + (\sin \alpha)x)$,所以 $Ux = \text{Re}(\lambda u) = (\cos \alpha)x - (\sin \alpha)y$, $Uy = \text{Im}(\lambda u) = (\cos \alpha)y + (\sin \alpha)x$。换句话说,$U$ 不动向量 $x, y$ 张成的二维子空间 $E_\lambda$(注意 $x \perp y$),并且 $U$ 在该子空间上的限制矩阵是旋转矩阵 $R_{-\alpha} = \begin{pmatrix} \cos \alpha & \sin \alpha \\ -\sin \alpha & \cos \alpha \end{pmatrix}$。注意,由于向量 $u$ 和 $\bar{u}$(不同特征值的特征向量)是正交的,根据勾股定理 $\|x\| = \|y\| = \frac{1}{\sqrt{2}}\|u\|$。很容易看出 $x \perp y$,所以 $\{x, y\}$ 是 $E_\lambda$ 的一个正交基。如果我们乘以每个基向量相同的非零数,我们不会改变线性变换的矩阵,所以我们可以无损于一般性地假设 $\|x\| = \|y\| = 1$,即 $\{x, y\}$ 是 $E_\lambda$ 的一个正交基。将标准正交系统 $\{v_1 = x, v_2 = y\}$ 补全为 $\mathbb{R}^n$ 中的一个标准正交基。由于 $UE_\lambda \subset E_\lambda$,即 $E_\lambda$ 是 $U$ 的不变子空间,所以 $U$ 在该基下的矩阵具有块三角形式 $\begin{pmatrix} R_{-\alpha} & * \\ 0 & U_1 \end{pmatrix}$(其中 $0$ 是 $(n-2) \times 2$ 的零块)。由于旋转矩阵 $R_{-\alpha}$ 是可逆的,我们有 $U E_\lambda = E_\lambda$。因此 $U^*E_\lambda = U^{-1}E_\lambda = E_\lambda$,所以 $U$ 在我们构造的基下的矩阵实际上是块对角矩阵:$\begin{pmatrix} R_{-\alpha} & 0 \\ 0 & U_1 \end{pmatrix}$。由于 $U$ 是酉的, $I = U^*U = \begin{pmatrix} I_2 & 0 \\ 0 & U_1^*U_1 \end{pmatrix}$,所以(由于 $U_1$ 是方阵)$U_1^*U_1 = I$,即 $U_1$ 也是酉的。如果 $U_1$ 有复数特征值,我们可以应用相同的过程来减小其尺寸(通过 2),直到我们只剩下只有实数特征值的块。实数特征值只能是 $+1$ 或 $-1$,所以在一个标准正交基下,$U$ 的矩阵具有形式 $\begin{pmatrix} R_{-\alpha_1} & & & & \\ & R_{-\alpha_2} & & & \\ & & \ddots & & \\ & & & R_{-\alpha_d} & \\ & & & & -I_r \end{pmatrix}$;这里 $I_r$ 和 $I_l$ 是 $r \times r$ 和 $l \times l$ 的单位矩阵。由于 $\det U = 1$,$-1$ 的特征值的重数(即 $r$)必须是偶数。注意,$2 \times 2$ 矩阵 $-I_2$ 可以解释为通过角度 $\pi$ 的旋转。因此,上述矩阵具有定理结论中给出的形式,其中 $\phi_k = -\alpha_k$ 或 $\phi_k = \pi$。

我们把证明留给读者。将定理 5.1 的证明修改一下,以适应实数矩阵的情况,这并不难。注意,对于实数矩阵,$U$ 也可以是实数(正交矩阵)。




\textbf{定理 5.2。} 设 $U$ 是 $\mathbb{R}^n$ 中的一个正交算子,且 $\det U = -1$。那么存在一个标准正交基 $\{v_1, v_2, \dots, v_n\}$,使得 $U$ 在该基下的矩阵具有块对角形式
$\begin{pmatrix} R_{\phi_1} & & & & & \\ & R_{\phi_2} & & & & \\ & & \ddots & & & \\ & & & R_{\phi_k} & & \\ & & & & I_r & \\ & & & & & -1 \end{pmatrix}$,
其中 $r = n - 2k - 1$ 并且 $R_{\phi_k}$ 是 $2$ 维旋转矩阵,$R_{\phi_k} = \begin{pmatrix} \cos \phi_k & -\sin \phi_k \\ \sin \phi_k & \cos \phi_k \end{pmatrix}$,而 $I_r$ 是 $(n-2k-1) \times (n-2k-1)$ 的单位矩阵。
我们将证明留给读者。对定理 5.1 的证明所需要做的修改是显而易见的。注意,对于正交矩阵 $U$,$\det U = -1$,它总是\textbf{反射}。设我们现在固定一个标准正交基,例如 $\mathbb{R}^n$ 中的标准基。我们将\textbf{基本旋转}$^3$ 定义为在 $x_j - x_k$ 平面上的旋转,即一个只改变坐标 $x_j$ 和 $x_k$ 的线性变换,并且它在这两个坐标上表现为一个平面旋转。

\textbf{定理 5.3。} 任何旋转 $U$(即 $\det U = 1$ 的正交变换)都可以表示为至多 $n(n-1)/2$ 个基本旋转的乘积。

为了证明定理,我们将需要以下简单引理。

\textbf{引理 5.4。} 设 $x = (x_1, x_2)^T \in \mathbb{R}^2$。存在一个 $\mathbb{R}^2$ 的旋转 $R_\alpha$,它将向量 $x$ 移动到向量 $(a, 0)^T$,其中 $a = \sqrt{x_1^2 + x_2^2}$。
证明是基础的,我们留给读者。人们可以画一张图或者写出 $R_\alpha$ 的公式。

\textbf{引理 5.5。} 设 $x = (x_1, x_2, \dots, x_n)^T \in \mathbb{R}^n$。存在 $n-1$ 个基本旋转 $R_1, R_2, \dots, R_{n-1}$ 使得 $R_{n-1} \dots R_2 R_1 x = (a, 0, \dots, 0)^T \in \mathbb{R}^n$,其中 $a = \sqrt{x_1^2 + x_2^2 + \dots + x_n^2}$。

\textbf{证明}~ 引理证明的思路非常简单。我们使用一个基本旋转 $R_1$ 在 $x_{n-1} - x_n$ 平面上“消去” $x$ 的最后一个坐标(引理 5.4 保证了这种旋转的存在)。然后使用一个基本旋转 $R_2$ 在 $x_{n-2} - x_{n-1}$ 平面上“消去” $R_1 x$ 的第 $n-1$ 个坐标(旋转 $R_2$ 不改变最后一个坐标,所以 $R_2 R_1 x$ 的最后一个坐标保持为零),依此类推... 对于形式证明,我们将使用数学归纳法处理 $n$。$n=1$ 的情况是平凡的,因为 $\mathbb{R}^1$ 中的任何向量都具有所需的格式。$n=2$ 的情况由引理 5.4 处理。现在假设引理对 $n-1$ 成立,我们想为 $n$ 证明它。根据引理 5.4,存在一个 $2 \times 2$ 旋转矩阵 $R_\alpha$,使得 $R_\alpha \begin{pmatrix} x_{n-1} \\ x_n \end{pmatrix} = \begin{pmatrix} a_{n-1} \\ 0 \end{pmatrix}$,其中 $a_{n-1} = \sqrt{x_{n-1}^2 + x_n^2}$。所以如果我们定义 $n \times n$ 的基本旋转 $R_1$ 为 $R_1 = \begin{pmatrix} I_{n-2} & 0 \\ 0 & R_\alpha \end{pmatrix}$ ($I_{n-2}$ 是 $(n-2) \times (n-2)$ 的单位矩阵),那么 $R_1 x = (x_1, x_2, \dots, x_{n-2}, a_{n-1}, 0)^T$。我们假设引理的结论对 $n-1$ 成立,所以存在 $n-2$ 个基本旋转(我们称它们为 $R_2, R_3, \dots, R_{n-1}$)在 $\mathbb{R}^{n-1}$ 中将向量 $(x_1, x_2, \dots, x_{n-1}, a_{n-1})^T$ 变换为向量 $(a, 0, \dots, 0)^T \in \mathbb{R}^{n-1}$。换句话说,$R_{n-1} \dots R_3 R_2 (x_1, x_2, \dots, x_{n-1}, a_{n-1})^T = (a, 0, \dots, 0)^T$。我们可以总是假设基本旋转 $R_2, R_3, \dots, R_{n-1}$ 作用在整个 $\mathbb{R}^n$ 上,只需假设它们不改变最后一个坐标。那么 $R_{n-1} \dots R_3 R_2 R_1 x = (a, 0, \dots, 0)^T \in \mathbb{R}^n$。现在我们来证明 $a = \sqrt{x_1^2 + x_2^2 + \dots + x_n^2}$。可以直接计算证明这一点,但我们采用间接推理。我们知道正交变换保持范数,并且我们知道 $a \geq 0$。但是,那么我们就没有选择,对于 $a$ 唯一的可能性就是 $a = \sqrt{x_1^2 + x_2^2 + \dots + x_n^2}$。

\textbf{定理 5.3 的证明。} 根据引理 5.5,存在基本旋转 $R_1, R_2, \dots, R_N$, $N \leq n(n-1)/2$,使得矩阵 $U_1 = R_N \dots R_2 R_1 U$ 是上三角的,并且除了最后一个对角元素 $B_{n,n}$ 之外,所有对角元素的非负值。注意,矩阵 $U_1$ 是正交的。任何正交矩阵都是正规的,并且我们知道只有对角矩阵才是上三角正规矩阵。因此,$U_1$ 是一个对角矩阵。我们知道正交矩阵的特征值只能是 $1$ 或 $-1$,所以 $U_1$ 的对角线上的值只能是 $1$ 或 $-1$。但是,我们知道 $U_1$ 的所有对角元素,除了最后的可能值之外,都是非负的,所以 $U_1$ 的所有对角元素,除了最后的可能值之外,都是 $1$。最后一个对角元素可以是 $\pm 1$。由于基本旋转的行列式是 1,我们可以得出 $\det U_1 = \det U = 1$,所以最后一个对角元素也必须是 1。所以 $U_1 = I$,因此 $U$ 可以表示为基本旋转的乘积 $U = R_1^{-1} R_2^{-1} \dots R_N^{-1}$。这里我们使用了基本旋转的逆也是基本旋转的事实。

\section{方向}

6.1. \textbf{动机。} 在图 1 和图 2 中,我们看到 $\mathbb{R}^2$ 和 $\mathbb{R}^3$ 中的三个标准正交基。在每个图中,基 b) 可以通过旋转从标准基 a) 得到,而无法通过旋转将标准基旋转得到基 c)(即 $e_k$ 映射到 $v_k$ $\forall k$)。你可能以前听过“方向”这个词,你可能知道基 a) 和 b) 具有正方向,而基 c) 的方向是负的。你也可能知道一些确定方向的规则,比如物理学中的右手定则。所以,如果你能\textbf{看到}一个基,比如在 $\mathbb{R}^3$ 中,你可能就能说出它的方向是什么。但是如果只给你向量 $v_1, v_2, v_3$ 的坐标呢?当然,你可以尝试画一张图来可视化向量,然后确定方向。但这并不总是容易的。此外,你该如何“向计算机解释”这一点呢?事实证明,有一个更简单的方法。让我们来解释一下。我们需要检查是否可能通过旋转得到 $\mathbb{R}^3$ 中的基 $\{v_1, v_2, v_3\}$。存在唯一的线性变换 $U$,使得 $Ue_k = v_k$, $k=1, 2, 3$;它的矩阵(在标准基下)是 $v_1, v_2, v_3$ 作为列的矩阵。它是一个正交矩阵(因为它将标准正交基变换为标准正交基),所以我们需要看看何时它是旋转。定理 5.1 和 5.2 给了我们答案:矩阵 $U$ 是旋转当且仅当 $\det U = 1$。注意,(对于 $3 \times 3$ 矩阵)如果 $\det U = -1$,那么 $U$ 是关于某个轴的旋转与在旋转平面(即与该轴正交的平面)上的反射的组合。这为下面的形式化定义提供了动机。

e1 e2 v1 v2 v1 v2 v3 v1 v2 v3 a) b) c)
图 1. $\mathbb{R}^2$ 中的方向。

e1 e3 e2 v1 v2 v3 v1 v2 v3 a) b) c)
图 2. $\mathbb{R}^3$ 中的方向。

6.2. \textbf{形式定义。} 设 $A = \{a_1, a_2, \dots, a_n\}$ 和 $B = \{b_1, b_2, \dots, b_n\}$ 是 $X$ 中的两个基。我们说基 $A$ 和 $B$ 具有\textbf{相同方向},如果坐标变换矩阵 $[I]_{B,A}$ 的行列式为正,并且说它们具有\textbf{不同方向},如果 $[I]_{B,A}$ 的行列式为负。注意,由于 $[I]_{A,B} = [I]_{B,A}^{-1}$,可以使用矩阵 $[I]_{A,B}$ 来定义。我们通常假设 $\mathbb{R}^n$ 中的标准基 $\{e_1, e_2, \dots, e_n\}$ 具有正方向。在一个抽象空间中,只需固定一个基并声明其方向为正。

如果 $\mathbb{R}^n$ 中的一个标准正交基 $\{v_1, v_2, \dots, v_n\}$ 具有正方向(即与标准基具有相同方向),那么定理 5.1 和 5.2 说明基 $\{v_1, v_2, \dots, v_n\}$ 是通过旋转从标准基得到的。

6.3. \textbf{基的连续变换和方向。}
\textbf{定义}~ 我们说基 $A = \{a_1, a_2, \dots, a_n\}$ 可以\textbf{连续地}变换为基 $B = \{b_1, b_2, \dots, b_n\}$,如果存在一个连续的基族 $\{v_1(t), v_2(t), \dots, v_n(t)\}$, $t \in [a, b]$ 使得 $v_k(a) = a_k$, $v_k(b) = b_k$, $k=1, 2, \dots, n$。“连续的基族”意味着向量函数 $v_k(t)$ 是连续的(它们在某个基下的坐标是连续函数),并且,至关重要的是,系统 $\{v_1(t), v_2(t), \dots, v_n(t)\}$ 是所有 $t \in [a, b]$ 的基。注意,通过进行变量替换,如果需要,我们可以总是假设 $[a, b] = [0, 1]$。

\textbf{定理 6.1。} 两个基 $A = \{a_1, a_2, \dots, a_n\}$ 和 $B = \{b_1, b_2, \dots, b_n\}$ 具有相同方向,当且仅当一个基可以连续地变换为另一个基。

\textbf{证明}~ 假设基 $A$ 可以连续地变换为基 $B$,并设 $\{v_1(t), v_2(t), \dots, v_n(t)\}$, $t \in [a, b]$ 是执行此变换的连续基族。考虑矩阵函数 $V(t)$,其列是 $v_k(t)$ 在基 $A$ 下的坐标向量 $[v_k(t)]_A$。显然,$V(t)$ 的元素是连续函数,并且 $V(a) = I$, $V(b) = [I]_{A,B}$。注意,由于 $V(t)$ 始终是一个基,$\det V(t)$ 永远不为零。那么,根据介值定理,$\det V(a)$ 和 $\det V(b)$ 具有相同的符号。由于 $\det V(a) = \det I = 1$,我们可以得出 $[I]_{A,B}$ 的行列式 $\det[I]_{A,B} = \det V(b) > 0$,所以基 $A$ 和 $B$ 具有相同方向。为了证明反向蕴含,即定理的“仅当”部分,需要证明单位矩阵 $I$ 可以通过可逆矩阵连续地变换为任何满足 $\det B > 0$ 的矩阵 $B$。换句话说,存在一个在 $[a, b]$ 上的连续矩阵函数 $V(t)$,使得对于所有 $t \in [a, b]$,$V(t)$ 是可逆的,并且 $V(a) = I$, $V(b) = B$。我们将此事实的证明留给读者。有几种方法可以证明这一点,其中一种在下面的问题 6.2-6.5 中进行了概述。

\textbf{练习}~

6.1. 设 $R_\alpha$ 是通过 $\alpha$ 的旋转,所以它在标准基下的矩阵是 $\begin{pmatrix} \cos \alpha & -\sin \alpha \\ \sin \alpha & \cos \alpha \end{pmatrix}$。找出 $R_\alpha$ 在基 $\{v_1, v_2\}$, 其中 $v_1 = e_2$, $v_2 = e_1$ 下的矩阵。

6.2. 设 $R_\alpha$ 是旋转矩阵 $R_\alpha = \begin{pmatrix} \cos \alpha & -\sin \alpha \\ \sin \alpha & \cos \alpha \end{pmatrix}$。证明 $2 \times 2$ 单位矩阵 $I_2$ 可以通过可逆矩阵连续地变换为 $R_\alpha$。

6.3. 设 $U$ 是一个 $n \times n$ 正交矩阵,且 $\det U > 0$。证明单位矩阵 $I_n$ 可以通过可逆矩阵连续地变换为 $U$。
\textbf{提示:} 使用上一问题和 $n$ 维旋转表示为平面旋转的乘积。

6.4. 设 $A$ 是一个 $n \times n$ 正定 Hermite 矩阵,$A = A^* > 0$。证明单位矩阵 $I_n$ 可以通过可逆矩阵连续地变换为 $A$。
\textbf{提示:} 对角矩阵呢?

6.5. 使用极坐标分解和上面的问题 6.3, 6.4 完成定理 6.3 的“仅当”部分的证明。
