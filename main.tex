% !TEX program = xelatex
% 使用 texlive 完整编译:
% xelatex -> bibtex -> xelatex -> xelatex
% zhbook 中文书籍 LaTeX 模板

%--- 正文前后都没有空白页 ---
\documentclass[openany,twoside,zihao=-4,,fontset=windows]{zhbook}
% print 用于打印, 封面等生成空白页

%--- 正文前后都有空白页, 正文一章结束可空白页使新的一章是在奇数页开始 ---
% \documentclass[twoside,openright,zihao=-4]{zhbook}


% 书籍信息设置
% \vspace{5ex}


\title{「倒叙」的线性代数}  % 标题
\subtitle{——\textbf{{\tinro
Linear Algebra
Done Wrong}} 翻译版}  % 副标题


\author{Sergei Treil~(谢尔盖·特雷伊)\\ 布朗大学数学系 }  % 作者姓名
\bioinfo{译者}{董耀择\\南京大学匡亚明学院2025级本科生} 

\date{\today}  % 日期
\version{1.1}  % 版本
 % 其他信息



\extrainfo{\plogo \\[8pt] 出版社}

% 通用虚拟出版商标志
\newcommand{\plogo}{\fbox{$\mathfrak{NaN}$}}


%----- 设置英文字体 -----
%\usepackage{newtxtext}  %New TX font for text
%\setmainfont{TeX Gyre Termes}  %Times New Roman 的开源复刻版本
%\setsansfont{TeX Gyre Heros}  %Helvetica 的开源复刻版本
%\setmonofont{TeX Gyre Cursor}  %Courier New 的开源复刻版本
\setmainfont{Times New Roman}
\setsansfont{Arial}
%\setmonofont{Courier New}
\newfontfamily{\tinro}{Times New Roman}
%----- 设置数学字体 -----
%\usepackage{newtxmath}
%\usepackage{mathptmx}

%----- 添加其他宏包 -----
%\usepackage[notcite,notref]{showkeys}
\usepackage{listings}
\usepackage{subfig}

%----- 取消链接颜色和方框 -----
%\hypersetup{hidelinks}

%----- 参考文献格式 -----
%\bibliographystyle{plain} % abbrv, unsrt, siam
\bibliographystyle{thuthesis-numeric}
%\bibliographystyle{thuthesis-author-year}

%----- 参考文献引用格式 -----
\usepackage[numbers,sort&compress]{natbib}
%\usepackage[numbers,super,square,sort&compress]{natbib}
%\usepackage[authoryear,sort&compress]{natbib}
\def\bibfont{\small}  % 修改参考文献字体
\setlength{\bibsep}{7pt plus 3pt minus 3pt}  % 调整参考文献间距

%----- 制作索引 -----
\usepackage{imakeidx}
\makeindex[columns=2,intoc=true,title={索~~引}]
%\indexsetup{level=\chapter*}

% 使用 zhmakeindex 按照中文拼音排序
%\usepackage[noautomatic]{imakeidx}
%\makeindex[columns=2,intoc=true,title={索~~引}]


%----- 调整列表项的间距 -----
%\setlength{\itemsep}{3pt}

%----- 调整页面避免出现过大空白 -----
%\raggedbottom

%----- 定义符号描述命令  -----
\newcommand{\nameditem}[3][]{
\noindent\hspace{2em}\makebox[0.2\textwidth][l]{#2}{{#3}
\hfill\makebox[0.2\textwidth][l]{#1}\hspace*{2em}}\par}

%----- 插入 PDF 文件命令 -----
%\includepdf[pages=-]{pdfname.pdf}

%----- 微分算子 -----
\newcommand*{\dif}{\mathop{}\!\mathrm{d}}

%----- 自定义命令 -----
\newcommand{\CC}{\ensuremath{\mathbb{C}}}
\newcommand{\RR}{\ensuremath{\mathbb{R}}}
\newcommand{\A}{\mathcal{A}}
\newcommand{\bA}{\boldsymbol{A}}
\newcommand{\ii}{\mathrm{i}\mkern1mu}
\newcommand{\abs}[1]{\lvert#1\rvert}
\newcommand{\norm}[1]{\left\lVert#1\right\rVert}
\newcommand{\dx}[1][x]{\mathop{}\!\mathrm{d}#1}
\newcommand{\red}[1]{\textcolor{red}{#1}}


\begin{document}


% 生成封面
\maketitle

% 插入封面 PDF文件
%\thispagestyle{empty}
%\includepdf{cover.pdf}
%\cleardoublepage


%%%%%%%%%%%%%%%%%%%%%%%%%%%%%%%%%%%%%%%%%%%%%%%%%%%

\frontmatter
%\pagenumbering{Roman} % 摘要页码为大写罗马数字

%%%%%%%%%%%%%%%%%%%%%% 前言 %%%%%%%%%%%%%%%%%%%%%%%%


%%%%%%%%%%%%%%%%%%%% 前言 %%%%%%%%%%%%%%%%%%%%%

\begin{preface}
本书的标题听起来有些神秘。为什么有人会读这本书,如果它以一种错误的方式呈现了这个主题呢?这本书里到底哪里是“错误”的呢?在回答这些问题之前,请允许我先描述一下本书的目标读者。

本书源于“荣誉线性代数”(Honors Linear Algebra)课程的讲义。它旨在成为一门面向数学上训练有素的学生的入门线性代数课程。它适用于那些虽然还没有很熟悉抽象推理,但愿意比“菜谱式”(cookbook style)的微积分课程学习更严谨数学的学生。除了作为线性代数的第一门课程,它还旨在成为介绍严谨证明、形式化定义——简而言之,介绍现代理论(抽象)数学风格的第一门课程。本书的目标读者解释了它为何如此特别地融合了初级概念和具体示例(通常在入门线性代数教材中呈现),以及更抽象的定义和构造(通常在高级书籍中典型出现)。本书的另一个特点是它不是由代数专家所写,也不是为代数专家所写。因此,我试图强调那些对分析、几何、概率等重要的主题,而没有包含一些传统的主题。例如,我只考虑实数或复数域上的向量空间。完全不考虑其他域上的线性空间,因为我认为花费时间介绍和解释抽象域的知识,不如花在一些在其他学科中更必需的经典主题上。

后来,当学生在抽象代数课程中学习一般域时,他们会明白本书研究的许多构造也适用于一般域。

本书仅考虑有限维空间,并且基总是指有限基。原因是,要对无限维空间说出一些有意义的话,就必须引入收敛、范数、完备性等概念,即泛函分析的基础。而这绝对是一个单独的课程(教材)的主题。

因此,我在此不考虑无限维 Hamel 基:它们在大多数分析和几何应用中是不需要的,而且我认为它们属于抽象代数课程。

\textbf{给教师的说明}

本书的某些细节使其与标准的进阶线性代数教材有所不同。

首先是关于基、线性无关和生成集的定义。在书中,我首先将基定义为这样的系统:任何向量都能唯一地表示为线性组合。然后,线性无关和生成系统的性质自然地成为基的性质的组成部分,一个代表唯一性,另一个代表表示的存在性。这样处理的原因是,我认为基的概念比线性无关的概念更重要:在大多数应用中,我们并不真正关心线性无关,我们需要的是一个系统作为基。例如,在求解齐次方程组时,我们不仅仅是在寻找线性无关的解,而是在寻找解空间中的一组基。而且,向学生解释基的重要性很容易:它们允许我们引入坐标,并使用 $\mathbb{R}^n$(或 $\mathbb{C}^n$)来代替处理抽象向量空间。此外,我们需要坐标来进行计算机计算,而计算机非常擅长处理矩阵。而且,我真的不知道线性无关概念有什么简单的动机。

另一个细节是,我先介绍线性变换,然后再教授如何求解线性方程组。

一个缺点是我们直到第二章才证明只有方阵才能是可逆的,以及其他一些重要事实。然而,已经定义的线性变换允许更系统的行约简的呈现。此外,我花了很多时间(两节)来阐述矩阵乘法的动机。我希望我能很好地解释了为什么这种看起来很奇怪的乘法规则实际上非常自然,以至于我们别无选择。

许多关于基、线性变换等重要事实,例如在任何向量空间中,两个基具有相同数量的向量,都是通过行约简中的主元计数来证明的。虽然这些事实中的大多数都具有“无坐标”的证明,形式上不涉及高斯消元,但仔细分析这些证明会发现,高斯消元和主元计数并没有消失,它们只是在大多数证明中隐藏起来了。因此,与其呈现非常优雅(但对于初学者来说很难理解)的“无坐标”证明,这些证明通常在进阶线性代数书籍中呈现,我们则使用了“行约简”证明,这在“微积分类”教材中更为常见。这样做的优点是,可以轻松地看到所有证明背后的共同思想,并且对于不那么数学化的读者来说,这些证明更容易理解和记住。我还将在第二章的第 8 节中介绍一个简单且易于记忆的公式,用于计算基变换。

第三章处理行列式。我花了很多时间来阐述行列式的动机,然后才给出正式定义。行列式被介绍为计算体积的一种方式。书中表明,如果我们允许有符号体积,使得行列式在每列上都是线性的(此时学生应该很清楚线性关系非常有帮助,而允许负体积是为此付出的很小的代价),并且假设一些非常自然的性质,那么我们就别无选择,只能得到行列式的经典定义。我想强调的是,我一开始并没有假定行列式的反对称性,而是从体积的其他非常自然的性质中推导出来的。请注意,虽然在形式上,第一至第三章主要处理实数空间,但其中所有内容都适用于复数空间,甚至适用于任意域上的空间。

第四章是谱理论的介绍,这就是复数空间 $\mathbb{C}^n$ 自然出现的地方。它在本书开头被形式化定义,复数向量空间的定义也给出了,但在第四章之前,主要对象是实数空间 $\mathbb{R}^n$。现在,复数特征值的出现表明,对于谱理论,最自然的不是实数空间 $\mathbb{R}^n$,而是复数空间 $\mathbb{C}^n$,即使我们最初处理的是实数矩阵(实数空间中的算子)。这里的重点是特征值分解,并且特征值空间的基的概念也被引入。

第五章是内积空间,它出现在谱理论之后,因为我希望同时处理复数和实数两种情况,而谱理论为复数空间提供了强有力的动机。除了动机之外,第四章和第五章不互相依赖,教师可以先讲第五章。

尽管我在第九章中介绍了 Jordan 标准型,但我通常没有时间在一学期的课程中讲授它。我更喜欢花更多时间讲授第六章和第七章中的主题,例如法线算子和自伴随算子的特征值分解、极坐标和奇异值分解、正交矩阵的结构和方向,以及二次型理论。我认为这些主题比 Jordan 标准型对于应用更重要,尽管后者确实很优美。但是,我包含了第九章,以便教师可以跳过第六章和第七章中的某些主题,转而讲授 Jordan 分解定理。我还包括了(2009年新增)第八章,讨论对偶空间和张量。我认为其中的材料,特别是关于张量的部分,对于一年级的线性代数课程来说有点太难了,但有些主题(例如,对偶空间中的坐标变换)可以很容易地包含在教学大纲中。它还可以作为进阶课程中张量理论的介绍。请注意,本章中介绍的结果适用于任意域。

我试图以比较非正式的方式呈现本书的材料,偏爱直观的几何推理而不是形式化的代数运算,因此对于一些纯粹主义者来说,本书可能不够严谨。在本书通篇,我通常(当不引起混淆时)将线性变换与其矩阵等同起来。这允许使用更简单的符号,而且我认为对于没有经验的学生来说,过度强调变换与矩阵之间的区别可能会造成混淆。只有当区别至关重要时,例如在分析一个变换的矩阵如何在基变换下变化时,我才会使用特殊的符号来区分变换和它的矩阵。
\vspace{5ex}
\begin{flushright}
Sergei Treil~~~~~~~~~
\end{flushright}





\end{preface}

\begin{preface}[译者的话]

\textbf{我为什么翻译这本书?}

在南京大学匡亚明学院,时间是每位学生最宝贵的资源。我们的学习强度很高,课业繁重,几乎没有空闲。在这种情况下,再额外开启一个耗时巨大的翻译项目,似乎是一个不理智的选择。

但我还是决定这么做。

因为《Linear Algebra Done Wrong》是我们线代课程至关重要的参考书,而语言的障碍确实困扰着不少同学。我相信,一份高质量的中文译本能够实实在在地帮助到大家。

南京大学的校训精神中,“诚”字所蕴含的力量给了我巨大的鼓舞。“大哉一诚天下动”,它告诉我,一个真诚的、以服务之心出发的行动,即便微小,也能产生积极的影响。与其独自感叹学习之艰辛,不如动手为大家做一点实事。

因此,我启动了这个项目。我希望这份译稿不仅仅是我个人学习的沉淀,更能成为一份小小的礼物,送给每一位在知识海洋中奋力前行的同学。希望它能为你节省一些时间,扫清一些迷茫,让你更专注于领略线性代数的核心思想。

\vspace{2ex} 

本书中文版全文采用\LaTeX{}精心排版,力求高仿复刻原书中的所有公式与数学符号,以保证其准确性和专业性。本书的排版模板来源于\href{http://haixing-hu.github.io/xelatex-zh-book/}{zhbook}项目,在此对原作者的贡献表示由衷的感谢。

本书的GitHub开源项目地址为:\url{http://github.com/DongYaoZe/Translate-LADW}
诚挚欢迎各位同学加入翻译、贡献代码、参与纠错,共同完善此项目。

若您对本翻译项目有任何疑问、建议或想法,欢迎通过我的电子邮箱 

251840159@smail.nju.edu.cn 与我联系。

\vspace{5ex}
\begin{flushright}
董耀择~~~~~~~~~

2025年10月~~~~~
\end{flushright}

\end{preface}

%%%%%%%%%%%%%%%%% 中文摘要内容和关键字  %%%%%%%%%%%%%%

%
%%%%%%%%%%%%% 中文摘要内容和关键字  %%%%%%%%%%%%%

\begin{cnabstract}

摘要内容摘要内容摘要内容摘要内容摘要内容摘要内容摘要内容摘要内容摘要内容摘要内容摘要内容摘要内容摘要内容摘要内容摘要内容摘要内容摘要内容摘要内容摘要内容摘要内容摘要内容摘要内容摘要内容摘要内容摘要内容.

摘要内容摘要内容摘要内容摘要内容摘要内容摘要内容摘要内容摘要内容摘要内容摘要内容摘要内容摘要内容摘要内容摘要内容摘要内容摘要内容摘要内容摘要内容摘要内容摘要内容摘要内容摘要内容摘要内容摘要内容摘要内容.

摘要内容摘要内容摘要内容摘要内容摘要内容摘要内容摘要内容摘要内容摘要内容摘要内容摘要内容摘要内容摘要内容摘要内容摘要内容摘要内容摘要内容摘要内容摘要内容摘要内容摘要内容摘要内容摘要内容摘要内容摘要内容.

\cnkeywords{关键词 1;关键词 2;关键词 3.}

\end{cnabstract}



%%%%%%%%%%%%%%%%% 英文摘要内容和关键字 %%%%%%%%%%%%%%

%
%%%%%%%%%%%% 英文摘要内容和关键字 %%%%%%%%%%%%%%

\begin{enabstract}

This is abstract. This is abstract. This is abstract. This is abstract. This is abstract. This is abstract. This is abstract. This is abstract. This is abstract. This is abstract. This is abstract. This is abstract.

The quick brown fox jumps over the lazy dog. The quick brown fox jumps over the lazy dog. The quick brown fox jumps over the lazy dog. The quick brown fox jumps over the lazy dog. The quick brown fox jumps over the lazy dog.

The quick brown fox jumps over the lazy dog. The quick brown fox jumps over the lazy dog. The quick brown fox jumps over the lazy dog. The quick brown fox jumps over the lazy dog. The quick brown fox jumps over the lazy dog.

\enkeywords{Keyword 1;~ Keyword 2;~ Keyword 3.}

\end{enabstract}



%%%%%%%%%%%%%%%%%%%%%%% 目录 %%%%%%%%%%%%%%%%%%%%%%%

% 生成目录
\maketoc

% 生成插图清单, 如不需要可以注释
% \makelof

% 生成表格清单, 如不需要可以注释
% \makelot

%%%%%%%%%%%%%%%%%%%% 主要符号表 %%%%%%%%%%%%%%%%%%%%%

% 如不需要可以注释
% 
% 主要符号表

\begin{denotation}

如不加特殊说明, 本论文采用如下符号和记号

自定义命令 \verb|\nameditem[]{}{}|, 比如 \verb|\nameditem[单位]{符号}{描述}|

\nameditem[\textbf{单位}]{\textbf{符号}}{\textbf{描述}}
\nameditem[$\mathrm{m^{2} \cdot s^{-2}}$]{$E$}{系统的能量}
\nameditem[$\mathrm{m\cdot s^{-1}}$]{$c$}{真空中光速}
\nameditem[$\mathrm{m^{3}\cdot kg^{-1} \cdot s^{-2}}$]{$g$}{引力常数~(Gravitational constant)}

%\nameditem{\textbf{符号}}{\textbf{描述}}
\nameditem{$\mathbb{R}$}{实数集}
\nameditem{$\mathbb{R}^{n}$}{$n$ 维实向量空间}
\nameditem{$\mathbb{Z}$}{整数集}
\nameditem{$\mathbb{Z}_{+}$}{正整数集}
\nameditem{$\|\cdot\|_2$}{$2$-范数}
\nameditem{$\|\cdot\|_{\infty}$}{$\infty$-范数}
\nameditem{$L$}{\parbox[t]{17em}{这是一个很长的符号描述, 通过 parbox 命令实现自动换行并悬挂对齐}}
\vspace{8pt}
\nameditem{$A^{-1}$}{矩阵 $A$ 的逆}
\nameditem{$A^{*}$}{矩阵 $A$ 的共轭转置}

\vspace{2em}
\nameditem{\textbf{缩写}}{\textbf{全称}}
\nameditem{ODE}{Ordinary differential equation}
\nameditem{PDE}{Partial differential equation}
\nameditem{CFD}{Computational fluid dynamics}
\nameditem{FDM}{Finite difference method}
\nameditem{FEM}{Finite element method}

%\vspace{1em}
%使用 table 和 tabular 环境排版符号和描述.
%\begin{table}[htp!]
%\centering
%\renewcommand\arraystretch{1.2} %定义表格高度
%\setlength\tabcolsep{10pt} %调节列间距
%\begin{tabular}{ll}
%  \textbf{符号} & \textbf{描述} \\
%  $\mathbb{R}$  & 实数集   \\
%  $\mathbb{R}^{n}$  & $n$ 维实向量空间   \\
%  $\mathbb{Z}$  & 整数集  \\
%  $\mathbb{Z}_{+}$  & 正整数集  \\
%  $\|\cdot\|_2$  & $2$-范数   \\
%  $\|\cdot\|_{\infty}$  & $\infty$-范数   \\
%  $A^{-1}$  & 矩阵 $A$ 的逆  \\
%  $A^{*}$   & 矩阵 $A$ 的共轭转置
%\end{tabular}
%\end{table}

\end{denotation}




%%%%%%%%%%%%%%%%%%%%%%%%%%%%%%%%%%%%%%%%%%%%%%%%%%%%

\mainmatter

%%%%%%%%%%%%%%%%%%%% 正文内容 %%%%%%%%%%%%%%%%%%%%%%%




%%%%%%%%%%%%%%%%%%%% 引言 %%%%%%%%%%%%%%%%%%%%%

\chapter{基本概念}\label{chap:Intro}

\section{向量空间}\label{sec:background}


向量空间 $V$ 是由称为向量(本书中用小写粗体字母表示,如 $\mathbf{v}$)的对象组成,以及两个运算:向量加法和数(标量)乘法
\footnote{为了在向量和其他对象之间做出视觉区分,本书使用\textbf{加粗的小写字母}来表示向量,而使用\textbf{普通小写字母}来表示数字(标量)。在一些(更高级的)书中,拉丁字母保留给向量,而希腊字母用于标量;在更高级的文本中,任何字母都可以用于任何目的,读者必须根据上下文理解每个符号的含义。我认为,尤其对于初学者来说,在不同对象之间有一定的视觉区分是有帮助的,所以加粗的小写字母将始终表示一个向量。而在黑板上,通常会使用箭头(如 $\vec{v}$)来标识一个向量。
}
,使得以下 8 个性质(所谓的向量空间\textbf{公理})成立:

前 4 个性质涉及加法:
\footnote{
这时会引出一个问题:“我们该如何记住上述性质呢?” 而答案是,根本不需要记住,请看下文!
}

1. 交换律:$\mathbf{v} + \mathbf{w} = \mathbf{w} + \mathbf{v}$ 对所有 $\mathbf{v}, \mathbf{w} \in V$;

2. 结合律:$(\mathbf{u} + \mathbf{v}) + \mathbf{w} = \mathbf{u} + (\mathbf{v} + \mathbf{w})$ 对所有 $\mathbf{u}, \mathbf{v}, \mathbf{w} \in V$;

3. 零向量:存在一个特殊的向量,记作 $\mathbf{0}$,使得 $\mathbf{v} + \mathbf{0} = \mathbf{v}$ 对所有 $\mathbf{v} \in V$;

4. 加法逆元:对于每个向量 $\mathbf{v} \in V$ 都存在一个向量 $\mathbf{w} \in V$ 使得 $\mathbf{v} + \mathbf{w} = \mathbf{0}$。 这样的加法逆元通常记作 $-\mathbf{v}$;

接下来的两个性质涉及乘法:

5. 乘法单位元:$1 \mathbf{v} = \mathbf{v}$ 对所有 $\mathbf{v} \in V$;

6. 乘法结合律:$(\alpha\beta) \mathbf{v} = \alpha(\beta \mathbf{v})$ 对所有 $\mathbf{v} \in V$ 和所有标量 $\alpha, \beta$;

最后是两个分配律,它们连接了乘法和加法:

7. $\alpha (\mathbf{u} + \mathbf{v}) = \alpha \mathbf{u} + \alpha \mathbf{v}$ 对所有 $\mathbf{u}, \mathbf{v} \in V$ 和所有标量 $\alpha$;

8. $(\alpha + \beta) \mathbf{v} = \alpha \mathbf{v} + \beta \mathbf{v}$ 对所有 $\mathbf{v} \in V$ 和所有标量 $\alpha, \beta$。

 \textbf{注释}~上述性质似乎很难记忆,但没有必要。它们只是我们从高中学到的关于数字的代数运算的熟悉规则。这里唯一的陌生之处在于,你必须理解你可以在什么对象上应用什么运算。你可以相加向量,也可以用数字(标量)乘以向量。当然,你可以对数字进行所有你以前学过的运算。但是,你不能将两个向量相乘,也不能将一个数字加到一个向量上。

\textbf{注释} ~可以很容易地证明零向量 $\mathbf{0}$ 是唯一的,并且给定 $\mathbf{v} \in V$ 其加法逆元 $-\mathbf{v}$ 也是唯一的。

通过利用向量空间的性质5,6和8,也不难证明 $\mathbf{0} = 0 \mathbf{v}$ 对任何 $\mathbf{v} \in V$,并且 $-\mathbf{v} = (-1)\mathbf{v}$。注意,要做到这一点,仍然需要使用向量空间的其它性质的证明,特别是性质 3 和 4。


如果标量是通常的实数,我们称空间 $V$ 为\textbf{实}(real)向量空间。如果标量是复数,即如果我们能用复数乘以向量,我们称空间 $V$ 为\textbf{复}(complex)向量空间。

注意,任何复向量空间也都是实向量空间(如果我们能用复数乘以向量,那么我们也能用实数乘以向量),但反之则不然。

也有可能考虑标量是任意域 $\mathbb{F}$ 的元素的情况。在这种情况下,我们说 $V$ 是域 $\mathbb{F}$ 上的向量空间。虽然本书中的许多构造(特别是第一至第三章中的所有内容)适用于一般域,但本书仅考虑实数和复数向量空间。

如果我们不指定标量集,或者使用字母 $\mathbb{F}$ 来表示它,那么结果对实数和复数空间都成立。如果我们想区分实数和复数情况,我们会明确说明我们正在考虑哪种情况。

请注意,在定义域 $\mathbb{F}$ 上的向量空间定义中,我们\textbf{要求}标量集是一个域,因此我们可以始终进行除法(无余数),尽管不能进行整数除法。因此,可以考虑有理数域上的向量空间,但不能考虑整数环上的向量空间。





\textbf{1.1. 例子}

\textbf{示例}~ 空间 $\mathbb{R}^n$ 由所有大小为 $n$ 的列向量组成:
$$
\mathbf{v} = \begin{pmatrix} v_1 \\ v_2 \\ \vdots \\ v_n \end{pmatrix}
$$
其元素是实数。加法和乘法是逐个元素定义的,即
$$
\alpha \begin{pmatrix} v_1 \\ v_2 \\ \vdots \\ v_n \end{pmatrix} = \begin{pmatrix} \alpha v_1 \\ \alpha v_2 \\ \vdots \\ \alpha v_n \end{pmatrix}, \quad \begin{pmatrix} v_1 \\ v_2 \\ \vdots \\ v_n \end{pmatrix} + \begin{pmatrix} w_1 \\ w_2 \\ \vdots \\ w_n \end{pmatrix} = \begin{pmatrix} v_1 + w_1 \\ v_2 + w_2 \\ \vdots \\ v_n + w_n \end{pmatrix}
$$

\textbf{示例}~ 空间 $\mathbb{C}^n$ 也由大小为 $n$ 的列向量组成,只是元素现在是复数。加法和乘法与 $\mathbb{R}^n$ 中的定义完全相同,唯一的区别是我们现在可以乘以\textbf{复}数,即 $\mathbb{C}^n$ 是一个\textbf{复}向量空间。

本书中的许多结果对于 $\mathbb{R}^n$ 和 $\mathbb{C}^n$ 都成立。在这种情况我们使用符号 $\mathbb{F}^n$。

\textbf{示例}~ 空间 $M_{m \times n}$(也记作 $M_{m,n}$)是 $m \times n$ 矩阵的集合:加法和标量乘法是逐个元素定义的。如果我们只允许实数项(因此只允许实数乘法),那么我们得到一个实向量空间;如果我们允许复数项和复数乘法,那么我们得到一个复向量空间。

形式上,我们必须区分实数情况和复数情况,即写成 $M^{\mathbb{R}}_{m,n}$ 或 $M^{\mathbb{C}}_{m,n}$。然而,在大多数情况下,实数和复数情况之间没有区别,也无需指明我们正在考虑哪种情况。如果有区别,我们会明确说明正在考虑哪种情况。

\textbf{注释}~ 正如我们上面提到的,向量空间的公理仅仅是(实数或复数)数字的代数运算的熟悉规则,所以如果我们把标量(数字)当作向量,所有公理都会被满足!因此,实数集 $\mathbb{R}$ 是一个实向量空间,复数集 $\mathbb{C}$ 是一个复向量空间。

更重要的是,由于在上面的例子中,所有向量运算(加法和标量乘法)都是逐个元素执行的,因此对于这些例子,向量空间的公理自动满足,因为它们对于标量是满足的(你能看出为什么吗?)。所以,我们不必检查公理,而是自动获得了这些例子确实是向量空间的事实!

同样的情况也适用于下一个例子,即多项式,其中多项式的系数起着条目的作用。

\textbf{示例}~ 空间 $\mathbb{P}_n$ 是最多 $n$ 次的多项式,包含所有形式为
$$p(t) = a_0 + a_1 t + a_2 t^2 + \dots + a_n t^n$$
的多项式,其中 $t$ 是自变量。注意,一些或甚至所有系数 $a_k$ 可以是 0。

在实系数 $a_k$ 的情况下,我们得到一个实向量空间,复数系数则构成一个复向量空间。同样,我们只在情况至关重要时才明确说明我们正在处理实数或复数情况;否则,一切都适用于这两种情况。

\textbf{问题}~ 在以上每个例子中,零向量是什么?

\textbf{1.2. 矩阵表示}

一个 $m \times n$ 矩阵是具有 $m$ 行和 $n$ 列的矩形数组。数组的元素称为矩阵的\textbf{项}(entry)。

通常我们可以方便地用带下标的字母来表示矩阵的项:第一个下标表示项所在的行号,第二个下标表示列号。例如,
\begin{equation}
A = (a_{j,k})_{m \times n, j=1, k=1}^n = \begin{pmatrix}
a_{1,1} & a_{1,2} & \dots & a_{1,n} \\
a_{2,1} & a_{2,2} & \dots & a_{2,n} \\
\vdots & \vdots & \ddots & \vdots \\
a_{m,1} & a_{m,2} & \dots & a_{m,n}
\end{pmatrix}
\end{equation}
是一种写 $m \times n$ 矩阵的一般方式。

非常频繁地,对于矩阵 $A$,位于第 $j$ 行和第 $k$ 列的项表示为 $A_{j,k}$ 或 $(A)_{j,k}$,有时像上面 (1.1) 那个例子一样,用小写字母表示相同的字母,也用于表示矩阵的项。

给定矩阵 $A$,它的\textbf{转置}(transpose)(或转置矩阵)$A^T$ 是通过将 $A$ 的行变为列来定义的。例如
$$
\begin{pmatrix} 1 & 2 & 3 \\ 4 & 5 & 6 \end{pmatrix}^T = \begin{pmatrix} 1 & 4 \\ 2 & 5 \\ 3 & 6 \end{pmatrix}.
$$
所以,$A^T$ 的列是 $A$ 的行,反之亦然,$A^T$ 的行是 $A$ 的列。

正式定义如下:$(A^T)_{j,k} = (A)_{k,j}$ 意思是 $A^T$ 中第 $j$ 行第 $k$ 列的项等于 $A$ 中第 $k$ 行第 $j$ 列的项。

转置矩阵在线性变换方面有一个很好的解释,即它给出了所谓的\textbf{伴随}(adjoint)变换。我们将在后面详细讨论这一点,但现在转置只是一个有用的形式运算。

转置的一个早期用途是我们可以将列向量 $\mathbf{x} \in \mathbb{F}^n$(回想一下 $\mathbb{F}$ 是 $\mathbb{R}$ 或 $\mathbb{C}$)写成 $\mathbf{x} = (x_1, x_2, \dots, x_n)^T$。如果我们将列向量垂直放置,它将占用更多的空间。

\textbf{练习}~
\footnote{
按照问卷调查结果,译者不会为本书制作答案。所有练习请读者自行完成。
}

1.1. 令 $\mathbf{x} = (1, 2, 3)^T$, $\mathbf{y} = (y_1, y_2, y_3)^T$, $\mathbf{z} = (4, 2, 1)^T$。计算 $2\mathbf{x}$, $3\mathbf{y}$, $\mathbf{x} + 2\mathbf{y} - 3\mathbf{z}$。

1.2. 下列集合(在自然的加法和标量乘法下)哪些是向量空间?请给出你的理由。

a) $[0, 1]$ 区间上所有连续函数的集合;

b) $[0, 1]$ 区间上所有非负函数的集合;

c) \textbf{恰好} $n$ 次多项式的集合;

d) 所有对称 $n \times n$ 矩阵的集合,即满足 $A^T = A$ 的矩阵 $A = \{a_{j,k}\}_{j,k=1}^n $。


1.3. 对错题:

a) 每个向量空间都包含一个零向量;

b) 一个向量空间可以有多个零向量;

c) 一个 $m \times n$ 矩阵有 $m$ 行和 $n$ 列;

d) 如果 $f$ 和 $g$ 是 $n$ 次多项式,那么 $f+g$ 也是 $n$ 次多项式;

e) 如果 $f$ 和 $g$ 是最高为 $n$ 次的多项式,那么 $f+g$ 也是最高为 $n$ 次的多项式。

1.4. 证明向量空间 $V$ 的零向量 $\mathbf{0}$ 是唯一的。

1.5. 空间 $M_{2 \times 3}$ 的零向量是什么矩阵?

1.6. 证明向量空间公理 4 中定义的加法逆元是唯一的。

1.7. 证明 $0 \mathbf{v} = \mathbf{0}$ 对任何向量 $\mathbf{v} \in V$。

1.8. 证明对于任何向量 $\mathbf{v}$,其加法逆元 $-\mathbf{v}$ 由 $(-1)\mathbf{v}$ 给出。



\section{线性组合,基}

设 $V$ 为向量空间,又设 $\mathbf{v}_1, \mathbf{v}_2, \dots, \mathbf{v}_p \in V$ 为一组向量。向量 $\mathbf{v}_1, \mathbf{v}_2, \dots, \mathbf{v}_p$ 的\textbf{线性组合}(linear combination)是形式为
$$
\alpha_1 \mathbf{v}_1 + \alpha_2 \mathbf{v}_2 + \dots + \alpha_p \mathbf{v}_p = \sum_{k=1}^p \alpha_k \mathbf{v}_k
$$
的和。

\textbf{定义}~ 向量系统 $\mathbf{v}_1, \mathbf{v}_2, \dots, \mathbf{v}_n \in V$ 称为 $V$ 的\textbf{基}(或\textbf{基底}),如果任何向量 $\mathbf{v} \in V$ 都可以唯一地表示为线性组合
$$
\mathbf{v} = \alpha_1 \mathbf{v}_1 + \alpha_2 \mathbf{v}_2 + \dots + \alpha_n \mathbf{v}_n = \sum_{k=1}^n \alpha_k \mathbf{v}_k.
$$
系数 $\alpha_1, \alpha_2, \dots, \alpha_n$ 称为向量 $\mathbf{v}$ 的\textbf{坐标}(coordinates)(在基 $\mathbf{v}_1, \mathbf{v}_2, \dots, \mathbf{v}_n$ 下,或相对于基 $\mathbf{v}_1, \mathbf{v}_2, \dots, \mathbf{v}_n$)。

另一种说 $\mathbf{v}_1, \mathbf{v}_2, \dots, \mathbf{v}_n$ 是基的方式是说,对于任何可能的右侧 $\mathbf{v}$ 的选择,方程 $x_1 \mathbf{v}_1 + x_2 \mathbf{v}_2 + \dots + x_m \mathbf{v}_n = \mathbf{v}$(未知数为 $x_k$)有唯一解。

在讨论基的任何性质之前
\footnote{
"basis" 的复数是 "bases",与 "base" 的复数相同。
}
,让我们给出几个例子,说明这些对象确实存在,并且研究它们是有意义的。


\textbf{示例2.2.}~ 
在第一个例子中,空间 ${V}$ 是 $\mathbb{F}^n$,其中 $\mathbb{F}$ 是实数 $\mathbb{R}$ 或复数 $\mathbb{C}$。考虑向量
$$ \mathbf{e}_1 = \begin{pmatrix} 1 \\ 0 \\ 0 \\ \vdots \\ 0 \end{pmatrix}, \quad \mathbf{e}_2 = \begin{pmatrix} 0 \\ 1 \\ 0 \\ \vdots \\ 0 \end{pmatrix}, \quad \mathbf{e}_3 = \begin{pmatrix} 0 \\ 0 \\ 1 \\ \vdots \\ 0 \end{pmatrix}, \quad \dots, \quad \mathbf{e}_n = \begin{pmatrix} 0 \\ 0 \\ 0 \\ \vdots \\ 1 \end{pmatrix} ,$$
(向量 $\mathbf{e}_k$ 除第 $k$ 个分量为 1 外,其余分量均为 0)。向量组 $\mathbf{e}_1, \mathbf{e}_2, \dots, \mathbf{e}_n$ 是 $\mathbb{F}^n$ 的一个基。事实上,任意向量 $\mathbf{v} = \begin{pmatrix} x_1 \\ x_2 \\ \vdots \\ x_n \end{pmatrix} \in \mathbb{F}^n$ 都可以表示为线性组合
$$ \mathbf{v} = x_1 \mathbf{e}_1 + x_2 \mathbf{e}_2 + \dots + x_n \mathbf{e}_n = \sum_{k=1}^n x_k \mathbf{e}_k $$
并且这种表示是唯一的。向量组 $\mathbf{e}_1, \mathbf{e}_2, \dots, \mathbf{e}_n \in \mathbb{F}^n$ 被称为 $\mathbb{F}^n$ 中的\textbf{标准基}(standard basis)。


\textbf{示例2.3.}~ 
在这个例子中,空间是至多 $n$ 次多项式构成的空间 $\mathbb{P}_n$。考虑向量(多项式) $\mathbf{e}_0, \mathbf{e}_1, \mathbf{e}_2, \dots, \mathbf{e}_n \in \mathbb{P}_n$ 定义为
$$ \mathbf{e}_0 := 1, \quad \mathbf{e}_1 := t, \quad \mathbf{e}_2 := t^2, \quad \mathbf{e}_3 := t^3, \quad \dots, \quad \mathbf{e}_n := t^n $$
显然,任意多项式$p$, $p(t) = a_0 + a_1 t + a_2 t^2 + \dots + a_n t^n$ 都存在唯一的表示
$$ p = a_0 \mathbf{e}_0 + a_1 \mathbf{e}_1 + \dots + a_n \mathbf{e}_n $$
因此,向量组 $\mathbf{e}_0, \mathbf{e}_1, \mathbf{e}_2, \dots, \mathbf{e}_n \in \mathbb{P}_n$ 是 $\mathbb{P}_n$ 中的一个基。我们将它称为 $\mathbb{P}_n$ 中的标准基。

\textbf{注释}~
如果一个向量空间 $V$ 拥有基 $\mathbf{v}_1, \mathbf{v}_2, \dots, \mathbf{v}_n$,那么任何向量 $\mathbf{v}$ 都可以由其在分解 $\mathbf{v} = \sum_{k=1}^n \alpha_k \mathbf{v}_k$ 中的系数唯一确定
\footnote{这是一个非常重要的注记,将在本书中贯穿使用。它允许我们将任何关于标准列空间 $\mathbb{F}^n$ 的陈述转化为关于具有基 $\mathbf{v}_1, \mathbf{v}_2, \dots, \mathbf{v}_n$ 的向量空间 $V$ 的陈述。}。
因此,如果我们把系数 $\alpha_k$ 堆叠成一个列向量,我们可以像处理列向量一样处理它们,即像处理 $\mathbb{F}^n$ 的元素一样(同样,这里的 $\mathbb{F}$ 是 $\mathbb{R}$ 或 $\mathbb{C}$,但所有内容也都适用于抽象域 $\mathbb{F}$)。

具体来说,如果 $\mathbf{v} = \sum_{k=1}^n \alpha_k \mathbf{v}_k$ 且 $\mathbf{w} = \sum_{k=1}^n \beta_k \mathbf{v}_k$,那么
$$ \mathbf{v} + \mathbf{w} = \sum_{k=1}^n \alpha_k \mathbf{v}_k + \sum_{k=1}^n \beta_k \mathbf{v}_k = \sum_{k=1}^n (\alpha_k + \beta_k) \mathbf{v}_k $$
也就是说,要得到和的坐标列,只需将各个向量的坐标列相加。类似地,要得到 $\alpha \mathbf{v}$ 的坐标,只需将 $\mathbf{v}$ 的坐标列乘以 $\alpha$。



\textbf{2.1. 生成系统与线性无关系统}

基的定义是任何向量都可以表示为线性组合。这句话实际上是两个陈述,即表示存在和表示唯一。让我们分别分析这两个陈述。

如果我们只考虑存在性,我们就得到以下概念:

\textbf{定义}~  向量系统 $\mathbf{v}_1, \mathbf{v}_2, \dots, \mathbf{v}_p \in V$ 称为 $V$ 中的\textbf{生成系统}(generating system)(也称为\textbf{张成系统}(spanning system)或\textbf{完备系统}(complete system)),如果任何向量 $\mathbf{v} \in V$ 都可以表示为线性组合
$$
\mathbf{v} = \alpha_1 \mathbf{v}_1 + \alpha_2 \mathbf{v}_2 + \dots + \alpha_p \mathbf{v}_p = \sum_{k=1}^p \alpha_k \mathbf{v}_k
$$
与基的定义不同之处在于,我们不假定上面的表示是唯一的。


这里,“生成”、“张成”和“完备”是同义词。我个人更喜欢“完备”这个词,因为我的算子理论背景。生成和张成在更常见的线性代数教材中使用。

显然,任何基都是生成(完备)系统。此外,如果我们有一个基,例如 $\mathbf{v}_1, \mathbf{v}_2, \dots, \mathbf{v}_n$,并且我们向其中添加几个向量,例如 $\mathbf{v}_{n+1}, \dots, \mathbf{v}_p$,那么新的系统将是生成(完备)系统。实际上,我们可以将任何向量表示为向量 $\mathbf{v}_1, \mathbf{v}_2, \dots, \mathbf{v}_n$ 的线性组合,并将新向量(通过将相应的系数 $\alpha_k = 0$)忽略掉。

现在,让我们关注唯一性。我们不想担心存在性,所以让我们考虑零向量 $\mathbf{0}$,它总是可以表示为线性组合。

\textbf{定义}~  线性组合 $\alpha_1 \mathbf{v}_1 + \alpha_2 \mathbf{v}_2 + \dots + \alpha_p \mathbf{v}_p$ 称为\textbf{平凡}的(trivial),如果 $\alpha_k = 0 \ \forall k$。

平凡线性组合总是(对于所有选择的向量 $\mathbf{v}_1, \mathbf{v}_2, \dots, \mathbf{v}_p$)等于 $\mathbf{0}$,这也许就是这个名字的原因。


\textbf{定义}~ 向量系统 $\mathbf{v}_1, \mathbf{v}_2, \dots, \mathbf{v}_p \in V$ 称为\textbf{线性无关}(linearly independent)的,如果只有平凡线性组合($\sum_{k=1}^p \alpha_k \mathbf{v}_k$ 其中 $\alpha_k = 0 \ \forall k$)等于 $\mathbf{0}$。

换句话说,系统 $\mathbf{v}_1, \mathbf{v}_2, \dots, \mathbf{v}_p$ 是线性无关的,当且仅当方程 $x_1 \mathbf{v}_1 + x_2 \mathbf{v}_2 + \dots + x_p \mathbf{v}_p = \mathbf{0}$(未知数为 $x_k$)只有一个平凡解 $x_1 = x_2 = \dots = x_p = 0$。


如果系统不是线性无关的,则称为\textbf{线性相关}(linearly dependent)。通过否定线性无关的定义,我们得到以下定义:



\textbf{定义}~ 向量系统 $\mathbf{v}_1, \mathbf{v}_2, \dots, \mathbf{v}_p$ 称为\textbf{线性相关}的,如果 $\mathbf{0}$ 可以表示为非平凡的线性组合,即 $0 = \sum_{k=1}^p \alpha_k \mathbf{v}_k$。非平凡意味着至少有一个系数 $\alpha_k$ 非零。这可以(并且通常)写成 $\sum_{k=1}^p |\alpha_k| \neq 0$。


因此,重申定义,我们可以说,一个系统是线性相关的,当且仅当存在不全为零的标量 $\alpha_1, \alpha_2, \dots, \alpha_p$,使得 
$$\sum_{k=1}^p \alpha_k \mathbf{v}_k = \mathbf{0}.$$

另一个定义(关于方程)是,系统 $\mathbf{v}_1, \mathbf{v}_2, \dots, \mathbf{v}_p$ 是线性相关的,当且仅当方程 
$$x_1 \mathbf{v}_1 + x_2 \mathbf{v}_2 + \dots + x_p \mathbf{v}_p = \mathbf{0}$$
(未知数为 $x_k$)有一个非平凡解。非平凡,再次意味着至少有一个 $x_k$ 不为零,并且可以写成 $\sum_{k=1}^p |x_k| \neq 0$。

以下命题提供了线性相关系统的一个替代描述。

\textbf{命题 2.6.}~ 向量系统 $\mathbf{v}_1, \mathbf{v}_2, \dots, \mathbf{v}_p \in V$ 是线性相关的,当且仅当其中一个向量 $\mathbf{v}_k$ 可以表示为其他向量的线性组合,
\begin{equation}\nonumber
\mathbf{v}_k = \sum_{j=1, j \neq k}^p \beta_j \mathbf{v}_j ~~~~~~~~\quad (2.1)
\end{equation}


\textbf{证明}~ 
 假设系统 $\mathbf{v}_1, \mathbf{v}_2, \dots, \mathbf{v}_p$ 是线性相关的。那么存在不全为零的标量 $\alpha_k$($\sum_{k=1}^p |\alpha_k| \neq 0$),使得 
$$\alpha_1 \mathbf{v}_1 + \alpha_2 \mathbf{v}_2 + \dots + \alpha_p \mathbf{v}_p = \mathbf{0}.$$
设 $k$ 是 $\alpha_k \neq 0$ 的索引。那么,将除 $\alpha_k \mathbf{v}_k$ 之外的所有项移到右侧,我们得到 $$\alpha_k \mathbf{v}_k = -\sum_{j=1, j \neq k}^p \alpha_j \mathbf{v}_j.$$
将两边除以 $\alpha_k$,我们得到 (2.1) 式,其中 $\beta_j = -\alpha_j / \alpha_k$。

另一方面,如果 (2.1) 式成立,则 $\mathbf{0}$ 可以表示为非平凡线性组合 
$$\mathbf{v}_k - \sum_{j=1, j \neq k}^p \beta_j \mathbf{v}_j = \mathbf{0}.$$


显然,任何基都是线性无关的系统。实际上,如果一个系统 $\mathbf{v}_1, \mathbf{v}_2, \dots, \mathbf{v}_n$ 是基,则 $\mathbf{0}$ 允许唯一表示 
$$\mathbf{0} = \alpha_1 \mathbf{v}_1 + \alpha_2 \mathbf{v}_2 + \dots + \alpha_n \mathbf{v}_n = \sum_{k=1}^n \alpha_k \mathbf{v}_k.$$
因为平凡线性组合总是给出 $\mathbf{0}$,所以平凡组合必须是给出 $\mathbf{0}$ 的\textbf{唯一}组合。

因此,正如我们已经讨论过的,如果一个系统是基,那么它就是完备(生成)的并且线性无关的系统。以下命题表明其反向蕴含也成立。

\textbf{命题 2.7.} 向量系统 $\mathbf{v}_1, \mathbf{v}_2, \dots, \mathbf{v}_n \in V$ 是基,当且仅当它线性无关且完备(生成)。

\textbf{证明}~ 
我们已经知道基总是线性无关且完备的,所以命题的一个方向已经证明。

让我们证明另一个方向。假设系统 $\mathbf{v}_1, \mathbf{v}_2, \dots, \mathbf{v}_n$ 是线性和完备的。取任意向量 $\mathbf{v} \in V$。由于系统 $\mathbf{v}_1, \mathbf{v}_2, \dots, \mathbf{v}_n$ 是线性完备的(生成的),$\mathbf{v}$ 可以表示为 
$$\mathbf{v} = \alpha_1 \mathbf{v}_1 + \alpha_2 \mathbf{v}_2 + \dots + \alpha_n \mathbf{v}_n = \sum_{k=1}^n \alpha_k \mathbf{v}_k.$$
我们只需要证明这个表示是唯一的。

假设 $\mathbf{v}$ 还有另一个表示 $$\mathbf{v} = \sum_{k=1}^n \tilde{\alpha}_k \mathbf{v}_k.$$
那么 
$$\sum_{k=1}^n (\alpha_k - \tilde{\alpha}_k) \mathbf{v}_k = \sum_{k=1}^n \alpha_k \mathbf{v}_k - \sum_{k=1}^n \tilde{\alpha}_k \mathbf{v}_k = \mathbf{v} - \mathbf{v} = \mathbf{0}.$$
由于系统是线性无关的,$\alpha_k - \tilde{\alpha}_k = 0 \ \forall k$,因此表示 $\mathbf{v} = \alpha_1 \mathbf{v}_1 + \alpha_2 \mathbf{v}_2 + \dots + \alpha_n \mathbf{v}_n$ 是唯一的。

\textbf{注释}~ 在许多教材中,基被定义为完备且线性无关的系统(根据命题 2.7,这个定义等价于我们的定义)。虽然这个定义比本书提出的定义更常见,但我更喜欢后者。它强调了基的主要性质,即任何向量都可以唯一地表示为一个线性组合。

\textbf{命题 2.8.} 任何(有限)生成系统都包含一个基。

\textbf{证明}~ 
假设 $\mathbf{v}_1, \mathbf{v}_2, \dots, \mathbf{v}_p \in V$ 是生成(完备)集。如果它是线性无关的,那么它就是基,我们就完成了证明。

假设它不是线性无关的,即它是线性相关的。那么存在一个向量 $\mathbf{v}_k$ 可以表示为向量 $\mathbf{v}_j$ ($j \neq k$) 的线性组合。

由于 $\mathbf{v}_k$ 可以表示为向量 $\mathbf{v}_j$ ($j \neq k$) 的线性组合,因此任何向量 $\mathbf{v}_1, \mathbf{v}_2, \dots, \mathbf{v}_p$ 的线性组合都可以表示为相同的向量(即 $\mathbf{v}_j$, $1 \le j \le p$, $j \neq k$)的线性组合(即删掉 $\mathbf{v}_k$ 之后的向量)。因此,如果我们删除向量 $\mathbf{v}_k$,新的系统仍然是完备的。

如果新的系统是线性无关的,我们就完成了证明。如果不是,我们重复这个过程。

重复这个过程有限次后,我们将得到一个线性无关且完备的系统,否则我们将删除所有向量,最后得到一个空集。

因此,任何有限的完备(生成)集都包含一个完备的线性无关子集,即一个基。



\textbf{练习}~

2.1. 在 $3 \times 2$ 矩阵空间 $M_{3 \times 2}$ 中找到一个基。

2.2. 对错题:

a) 包含零向量的任何集合都是线性相关的;

b) 基必须包含 $\mathbf{0}$;

c) 线性相关集的子集是线性相关的;

d) 线性无关集的子集是线性无关的;

e) 如果 $\alpha_1 \mathbf{v}_1 + \alpha_2 \mathbf{v}_2 + \dots + \alpha_n \mathbf{v}_n = \mathbf{0}$,那么所有标量 $\alpha_k$ 都为零。

2.3. 回忆一下,如果 $A^T = A$,则矩阵称为\textbf{对称}矩阵。写下一个 $2 \times 2$ 对称矩阵空间的基(有许多可能的答案)。基中有多少个元素?

2.4. 写出以下空间的基:

a) $3 \times 3$ 对称矩阵;

b) $n \times n$ 对称矩阵;

c) $n \times n$ 反对称矩阵 ($A^T = -A$)。

2.5. 设向量系统 $\mathbf{v}_1, \mathbf{v}_2, \dots, \mathbf{v}_r$ 是线性无关的,但不是生成的。证明可以找到向量 $\mathbf{v}_{r+1}$ 使得系统 $\mathbf{v}_1, \mathbf{v}_2, \dots, \mathbf{v}_r, \mathbf{v}_{r+1}$ 是线性无关的。

提示:选择任何不能表示为 $\sum_{k=1}^r \alpha_k \mathbf{v}_k$ 的向量作为 $\mathbf{v}_{r+1}$,并证明系统 $\mathbf{v}_1, \mathbf{v}_2, \dots, \mathbf{v}_r, \mathbf{v}_{r+1}$ 是线性无关的。

2.6. 向量 $\mathbf{v}_1, \mathbf{v}_2, \mathbf{v}_3$ 是否可能是线性相关的,而向量 $\mathbf{w}_1 = \mathbf{v}_1 + \mathbf{v}_2$, $\mathbf{w}_2 = \mathbf{v}_2 + \mathbf{v}_3$ 和 $\mathbf{w}_3 = \mathbf{v}_3 + \mathbf{v}_1$ 是线性、\textbf{独立}的?



\section{线性变换~矩阵-向量乘法}

从集合 $X$ 到集合 $Y$ 的\textbf{变换}
\footnote{
单词“变换”(transformation)、“映射”(map, mapping)、“算子”(operator)、“函数”(function)这些词都表示同一个概念。
} (transformation)$T$ 是一个规则,它为每个自变量(输入)$x \in X$ 分配一个值(输出)$y = T(x) \in Y$。

集合 $X$ 称为 $T$ 的\textbf{定义域}(domain),集合 $Y$ 称为 $T$ 的\textbf{目标空间}(target space)或\textbf{上域}(codomain)。

我们写 $T: X \to Y$ 来表示 $T$ 是一个定义域为 $X$、目标空间为 $Y$ 的变换。

\textbf{定义}~  设 $V$,$W$ 为向量空间(在同一域 $\mathbb{F}$ 上)。变换 $T: V \to W$ 称为\textbf{线性}的,如果

1. $T(\mathbf{u} + \mathbf{v}) = T(\mathbf{u}) + T(\mathbf{v})$ $\forall \mathbf{u}, \mathbf{v} \in V$;

2. $T(\alpha \mathbf{v}) = \alpha T(\mathbf{v})$ 对所有 $\mathbf{v} \in V$ 和所有标量 $\alpha \in \mathbb{F}$。


性质 1 和 2 一起等价于以下一个性质:
$T(\alpha \mathbf{u} + \beta \mathbf{v}) = \alpha T(\mathbf{u}) + \beta T(\mathbf{v})$ ~~对所有 $\mathbf{u}, \mathbf{v} \in V$ ~~和所有标量 $\alpha, \beta$。



\textbf{一些例子}~
您以前接触过线性变换,可能甚至没有意识到,如下例所示。

\textbf{示例}~ 
\textbf{微分}: 设 $V = \mathbb{P}_n$(至多 $n$ 次多项式的集合),$W = \mathbb{P}_{n-1}$,令 $T : \mathbb{P}_n \to \mathbb{P}_{n-1}$ 为微分算子,
$$ T(p) := p' \quad \forall p \in \mathbb{P}_n. $$
因为 $(f+g)' = f' + g'$ 且 $(\alpha f)' = \alpha f'$,这是一个线性变换。


\textbf{示例}~ 
\textbf{旋转}: 在这个例子中,$V = W = \mathbb{R}^2$(通常的坐标平面),一个变换 $T_\gamma : \mathbb{R}^2 \to \mathbb{R}^2$ 将 $\mathbb{R}^2$ 中的一个向量逆时针旋转 $\gamma$ 弧度。由于 $T_\gamma$ 整体将平面旋转,它也整体旋转了用于定义两个向量之和的平行四边形(平行四边形法则)。因此,线性变换的性质 1 成立。也很容易看出性质 2 也为真。



\red{(在此处应插入rotation figure)}
% \begin{figure}[h]
%     \centering
%     \includegraphics[width=0.5\textwidth]{rotation_figure.png} % 假设有一个旋转图形文件
%     \caption{旋转}
%     \label{fig:rotation}
% \end{figure}

\textbf{示例}~ 
\textbf{反射}: 在这个例子中,同样 $V = W = \mathbb{R}^2$,变换 $T : \mathbb{R}^2 \to \mathbb{R}^2$ 是关于第一坐标轴的反射,参见图 \ref{fig:reflection}。
也可以几何地证明这个变换是线性的,但我们将使用另一种方法来证明。

即,很容易写出 $T$ 的公式:
$$ T\left(\begin{pmatrix} x_1 \\ x_2 \end{pmatrix}\right) = \begin{pmatrix} x_1 \\ -x_2 \end{pmatrix} $$
并且从这个公式可以看出,这个变换是线性的。



\textbf{示例}~ 
让我们来研究线性变换 $T : \mathbb{R} \to \mathbb{R}$。任何这样的变换都由公式 $T(x) = ax$ 给出,其中 $a = T(1)$。
事实上,$$T(x) = T(x \times 1) = x T(1) = xa = ax.$$
因此,$\mathbb{R}$ 的任何线性变换仅仅是乘以一个常数。





\textbf{3.2. 线性变换 $\mathbb{F}^n \to \mathbb{F}^m$。矩阵-向量乘法}

事实证明,从 $\mathbb{F}^n$ 到 $\mathbb{F}^m$ 的线性变换 $T$ 也表示为乘法,不是乘标量,而是乘矩阵。我们来看看怎么做。

设 $T: \mathbb{F}^n \to \mathbb{F}^m$ 是一个线性变换。计算 $T(\mathbf{x})$ 对所有向量 $\mathbf{x} \in \mathbb{F}^n$ 需要哪些信息?我的主张是,知道 $T$ 如何作用于 $\mathbb{F}^n$ 的标准基 $\mathbf{e}_1, \mathbf{e}_2, \dots, \mathbf{e}_n$ 就足够了。也就是说,知道 $n$ 个 $\mathbb{F}^m$ 中的向量(即大小为 $m$ 的向量), $$\mathbf{a}_1 = T(\mathbf{e}_1), \mathbf{a}_2 := T(\mathbf{e}_2), \dots, \mathbf{a}_n := T(\mathbf{e}_n)$$ 就足够了。

实际上,设 $$\mathbf{x} = \begin{pmatrix} x_1 \\ x_2 \\ \vdots \\ x_n \end{pmatrix}.$$那么 $\mathbf{x} = x_1 \mathbf{e}_1 + x_2 \mathbf{e}_2 + \dots + x_n \mathbf{e}_n = \sum_{k=1}^n x_k \mathbf{e}_k$ 并且 $$T(\mathbf{x}) = T(\sum_{k=1}^n x_k \mathbf{e}_k) = \sum_{k=1}^n T(x_k \mathbf{e}_k) = \sum_{k=1}^n x_k T(\mathbf{e}_k) = \sum_{k=1}^n x_k \mathbf{a}_k.$$

因此,如果我们把向量(列)$\mathbf{a}_1, \mathbf{a}_2, \dots, \mathbf{a}_n$ 组合成一个矩阵 $A = [\mathbf{a}_1, \mathbf{a}_2, \dots, \mathbf{a}_n]$($\mathbf{a}_k$ 是 $A$ 的第 $k$ 列,$k = 1, 2, \dots, n$),这个矩阵就包含了关于 $T$ 的所有信息。让我们看看如何定义矩阵与向量(列)的乘积来将变换 $T$ 表示为乘积,$T(\mathbf{x}) = A \mathbf{x}$。设
$$
A = \begin{pmatrix}
a_{1,1} & a_{1,2} & \dots & a_{1,n} \\
a_{2,1} & a_{2,2} & \dots & a_{2,n} \\
\vdots & \vdots & \ddots & \vdots \\
a_{m,1} & a_{m,2} & \dots & a_{m,n}
\end{pmatrix}
$$
回忆一下,$A$ 的第 $k$ 列是向量 $\mathbf{a}_k$,即 $$\mathbf{a}_k = \begin{pmatrix} a_{1,k} \\ a_{2,k} \\ \vdots \\ a_{m,k} \end{pmatrix}.$$
那么,如果我们希望 $A \mathbf{x} = T(\mathbf{x})$,我们得到
$$
A \mathbf{x} = \sum_{k=1}^n x_k \mathbf{a}_k = x_1 \begin{pmatrix} a_{1,1} \\ a_{2,1} \\ \vdots \\ a_{m,1} \end{pmatrix} + x_2 \begin{pmatrix} a_{1,2} \\ a_{2,2} \\ \vdots \\ a_{m,2} \end{pmatrix} + \dots + x_n \begin{pmatrix} a_{1,n} \\ a_{2,n} \\ \vdots \\ a_{m,n} \end{pmatrix}
$$
所以,矩阵-向量乘法应该通过以下\textbf{按列坐标规则}(column by coordinating rule)执行:
$$\fbox{将矩阵的每一列乘以向量的相应坐标。}$$

\textbf{示例}~
$$
\begin{pmatrix} 1 & 2 & 3 \\ 3 & 2 & 1 \end{pmatrix} \begin{pmatrix} 1 \\ 2 \\ 3 \end{pmatrix} = 1 \begin{pmatrix} 1 \\ 3 \end{pmatrix} + 2 \begin{pmatrix} 2 \\ 2 \end{pmatrix} + 3 \begin{pmatrix} 3 \\ 1 \end{pmatrix} = \begin{pmatrix} 1+4+9 \\ 3+4+3 \end{pmatrix} = \begin{pmatrix} 14 \\ 10 \end{pmatrix}.
$$

“按列坐标规则”对于表示变换为乘积非常适用。它在后面不同的理论构造中也将会非常重要。

然而,在手动计算时,逐个条目计算结果更方便。这可以表示为以下\textbf{按行列规则}(row by column rule):

\fbox{要得到结果的第 $k$ 个条目,需要将矩阵的第 $k$ 行乘以向量,即,如果 $A \mathbf{x} = \mathbf{y}$,}

\fbox{那么
$y_k = \sum_{j=1}^n a_{k,j} x_j, \quad k = 1, 2, \dots, m;$}

这里 $x_j$ 和 $y_k$ 分别是向量 $\mathbf{x}$ 和 $\mathbf{y}$ 的坐标,而 $a_{j,k}$ 是矩阵 $A$ 的项。

\textbf{示例}~
$$
\begin{pmatrix} 1 & 2 & 3 \\ 4 & 5 & 6 \end{pmatrix} \begin{pmatrix} 1 \\ 2 \\ 3 \end{pmatrix} = \begin{pmatrix} 1 \cdot 1 + 2 \cdot 2 + 3 \cdot 3 \\ 4 \cdot 1 + 5 \cdot 2 + 6 \cdot 3 \end{pmatrix} = \begin{pmatrix} 1+4+9 \\ 4+10+18 \end{pmatrix} = \begin{pmatrix} 14 \\ 32 \end{pmatrix}
$$


\textbf{3.3. 线性变换与生成集}

正如我们在上面讨论的,作用于 $\mathbb{F}^n$ 到 $\mathbb{F}^m$ 的线性变换 $T$ 完全由其在 $\mathbb{F}^n$ 标准基上的值定义。

我们考虑标准基的事实并非关键,可以考虑任何基,甚至任何生成(张成)集。也就是说,

\fbox{线性变换 $T: V \to W$ 完全由其在生成集上的值定义(特别是由其在基上的值定义)。}
因此,如果 $\mathbf{v}_1, \mathbf{v}_2, \dots, \mathbf{v}_n$ 是 $V$ 中的生成集(特别地,如果它是基),并且 $T$ 和 $T_1$ 是(具有相同定义域和目标空间的)两个线性变换 ($T, T_1: V \to W$) 并且 $$T \mathbf{v}_k = T_1 \mathbf{v}_k, k = 1, 2, \dots, n,$$ 
则 $T = T_1$。

这个命题的证明是微不足道的,留作练习。

\textbf{3.4. 结论}
\begin{itemize}
\item 要获得 $T: \mathbb{F}^n \to \mathbb{F}^m$ 的线性变换的矩阵,只需将向量 $\mathbf{a}_k = T \mathbf{e}_k$(其中 $\mathbf{e}_1, \mathbf{e}_2, \dots, \mathbf{e}_n$ 是 $\mathbb{F}^n$ 的标准基)组合成一个矩阵:矩阵的第 $k$ 列是 $\mathbf{a}_k$, $k = 1, 2, \dots, n$。
\item 如果已知线性变换 $T$ 的矩阵 $A$,则 $T(\mathbf{x})$ 可以通过矩阵-向量乘法找到,$T(\mathbf{x}) = A \mathbf{x}$。要执行矩阵-向量乘法,可以使用“按列坐标规则”或“按行列规则”。
\end{itemize}
后者似乎更适合手动计算。前者非常适合并行计算,并且将在后面不同的理论构造中使用。

对于线性变换 $T: \mathbb{F}^n \to \mathbb{F}^m$,其矩阵通常记作 $[T]$。然而,人们常常不区分线性变换和它的矩阵,并使用相同的符号表示两者。当它不引起混淆时,我们也将使用相同的符号表示变换和它的矩阵。

由于线性变换本质上是乘法,因此 $T \mathbf{v}$ 这个表示通常会被采用,而不是 $T(\mathbf{v})$
\footnote{
$T \mathbf{v}$ 的表示比 $T(\mathbf{v})$更常用。
}
。我们将使用这种表示法。注意,通常的代数运算顺序适用,即 $T \mathbf{v} + \mathbf{u}$ 表示 $T(\mathbf{v}) + \mathbf{u}$,而不是 $T(\mathbf{v} + \mathbf{u})$。

\textbf{注释}~ 在矩阵-向量乘法 $A \mathbf{x}$ 中,矩阵 $A$ 的列数必须与向量 $\mathbf{x}$ 的大小一致
\footnote{
在使用“行乘以列”规则进行矩阵向量乘法时,请确保行中的元素(项)数量与列中的元素数量相同。行和列的元素应该同时结束:如果不满足,则乘法未定义。
}
,即 $\mathbb{F}^n$ 中的向量只能被 $m \times n$ 矩阵相乘。

这是有意义的,因为 $m \times n$ 矩阵定义了一个从 $\mathbb{F}^n$ 到 $\mathbb{F}^m$ 的线性变换,所以向量 $\mathbf{x}$ 必须属于 $\mathbb{F}^n$。

最简单的记住这个事实的方法是,如果在进行乘法时,只要你先用完了某个元素,这时还有一些元素未利用,那么乘法就是未定义的。

\textbf{注释}~ 不需要将自己局限于标准基的 $\mathbb{F}^n$ 的情况:当存在定义域和目标空间中的基时,本节中描述的所有内容都适用于任意向量空间。当然,如果改变了基,线性变换的矩阵也会不同。这将在后面的第 8 节中讨论。


\textbf{练习}~

3.1. 乘法:

a) $\begin{pmatrix} 1 & 2 & 3 \\ 4 & 5 & 6 \end{pmatrix} \begin{pmatrix} 1 \\ 3 \\ 2 \end{pmatrix}$;

b) $\begin{pmatrix} 1 & 2 \\ 0 & 1 \\ 2 & 0 \end{pmatrix} \begin{pmatrix} 1 \\ 3 \end{pmatrix}$;

c) $\begin{pmatrix} 1 & 2 & 0 & 0 \\ 0 & 1 & 2 & 0 \\ 0 & 0 & 1 & 2 \\ 0 & 0 & 0 & 1 \end{pmatrix} \begin{pmatrix} 1 \\ 2 \\ 3 \\ 4 \end{pmatrix}$;

d) $\begin{pmatrix} 1 & 2 & 0 \\ 0 & 1 & 2 \\ 0 & 0 & 1 \\ 0 & 0 & 0 \end{pmatrix} \begin{pmatrix} 1 \\ 2 \\ 3 \\ 4 \end{pmatrix}$。

3.2. 找到 $\mathbb{R}^2$ 中关于直线 $x_1 = 3x_2$ 的反射的线性变换的矩阵。

3.3. 对于以下每个线性变换,找到它的矩阵:

a) $T: \mathbb{R}^2 \to \mathbb{R}^3$ 定义为 $T(\begin{pmatrix} x \\ y \end{pmatrix}) = \begin{pmatrix} x + 2y \\ 2x - 5y \\ 7y \end{pmatrix}$;

b) $T: \mathbb{R}^4 \to \mathbb{R}^3$ 定义为 $T(x_1, x_2, x_3, x_4)^T = (x_1 + x_2 + x_3 + x_4, x_2 - x_4, x_1 + 3x_2 + 6x_4)^T$;

c) $T: \mathbb{P}_n \to \mathbb{P}_n$, $T f(t) = f'(t)$(在标准基 $1, t, t^2, \dots, t^n$ 下找到矩阵);

d) $T: \mathbb{P}_n \to \mathbb{P}_n$, $T f(t) = 2 f(t) + 3 f'(t) - 4 f''(t)$(同样在标准基 $1, t, t^2, \dots, t^n$ 下找到矩阵)。

3.4. 找到表示 $\mathbb{R}^3$ 中变换的 $3 \times 3$ 矩阵,这些变换:

a) 将每个向量投影到 $x-y$ 平面;

b) 将每个向量反射到 $x-y$ 平面;

c) 将 $x-y$ 平面绕 $z$ 轴旋转 $30^\circ$,同时保持 $z$ 轴不变。

3.5. 设 $A$ 是一个线性变换。如果 $\mathbf{z}$ 是线段 $[\mathbf{x}, \mathbf{y}]$ 的中点,证明 $A \mathbf{z}$ 是线段 $[A \mathbf{x}, A \mathbf{y}]$ 的中点。

\textbf{提示}:$\mathbf{z}$ 是线段 $[\mathbf{x}, \mathbf{y}]$ 的中点意味着什么?

3.6. 复数集 $\mathbb{C}$ 可以通过将 $z = x + {\rm i}  y \in \mathbb{C}$ 视为列向量 $(x, y)^T \in \mathbb{R}^2$ 来进行规范识别。

a) 将 $\mathbb{C}$ 视为复向量空间,证明通过 $\alpha = a + {\rm i} b \in \mathbb{C}$ 的乘法是在 $\mathbb{C}$ 中的线性变换。它的矩阵是什么?

b) 将 $\mathbb{C}$ 视为实向量空间 $\mathbb{R}^2$,证明通过 $\alpha = a + {\rm i} b \in \mathbb{C}$ 的乘法在那里定义了一个线性变换。它的矩阵是什么?

c) 定义 $T(x + {\rm i} y) = 2x - y + {\rm i} (x - 3y)$。证明这个变换不是 $\mathbb{C}$ 复向量空间中的线性变换,但如果我们把 $\mathbb{C}$ 视为实向量空间 $\mathbb{R}^2$,那么它在那里是一个线性变换(即 $T$ 是一个\textbf{实线性}但不是\textbf{复线性}变换)。找到这个实线性变换的矩阵。

3.7. 证明 $\mathbb{C}$ 中的任何线性变换(视为复向量空间)都是通过乘以 $\alpha \in \mathbb{C}$ 来实现的。


\section{线性变换,作为向量空间}

我们可以对线性变换进行哪些运算?我们总是可以在一个线性变换上乘上一个标量,也即,如果我们有一个线性变换 $T: V \to W$ 和一个标量 $\alpha$,我们可以定义一个新的变换 $\alpha T$ 为 $$(\alpha T) \mathbf{v} = \alpha (T \mathbf{v})~~\forall \mathbf{v} \in V.$$
可以很容易地检查出 $\alpha T$ 也是一个线性变换:
%fill



$(\alpha T )(\alpha_1 \mathbf{v}_1 + \alpha_2 \mathbf{v}_2)= \alpha( T (\alpha_1 \mathbf{v}_1 + \alpha_2 \mathbf{v}_2))$(根据$\alpha T$的定义)

$= \alpha (\alpha_1 T \mathbf{v}_1 + \alpha_2 T \mathbf{v}_2)$(根据$T$的线性)

$= \alpha_1 \alpha T \mathbf{v}_1 +\alpha_2  \alpha T \mathbf{v}_2  = \alpha_1( \alpha T) \mathbf{v}_1 +\alpha_2( \alpha T) \mathbf{v}_2   $

如果 $T_1$ 和 $T_2$ 是具有相同定义域和目标空间的线性变换 ($T_1: V \to W$ 和 $T_2: V \to W$,或简写为 $T_1, T_2: V \to W$),那么我们可以将这些变换相加,即定义一个新的变换 $T = (T_1 + T_2): V \to W$ 为 $$(T_1 + T_2) \mathbf{v} = T_1 \mathbf{v} + T_2 \mathbf{v}~~\forall \mathbf{v} \in V.$$
可以很容易地检查出变换 $T_1 + T_2$ 是线性的,只需重复上面关于 $\alpha T$ 线性的推理即可。

因此,如果我们固定向量空间 $V$ 和 $W$ 并考虑从 $V$ 到 $W$ 的所有线性变换的集合(我们将其表示为 $\mathcal{L}(V, W)$),我们可以定义 $\mathcal{L}(V, W)$ 上的 2 个运算:标量乘法和加法。可以很容易地证明这些运算满足向量空间公理,该公理在第 1 节中定义。

这里,读者不应该感到惊讶,因为向量空间的公理基本上意味着向量上的运算遵循代数运算的熟悉规则。而线性变换上的运算定义就是为了满足这些规则!

作为说明,让我们为向量空间公理的第一个分配律(公理 7)写下正式证明。我们想证明 $\alpha (T_1 + T_2) = \alpha T_1 + \alpha T_2$。对于 $V$ 中的任何 $\mathbf{v}$,

$\alpha (T_1 + T_2) \mathbf{v} = \alpha ((T_1 + T_2) \mathbf{v})$ (根据乘法的定义)

$= \alpha (T_1 \mathbf{v} + T_2 \mathbf{v})$ (根据和的定义)

$= \alpha T_1 \mathbf{v} + \alpha T_2 \mathbf{v}$ (根据 $W$ 的公理 7)

$= (\alpha T_1 + \alpha T_2) \mathbf{v}$ (根据和的定义)

因此,确实 $\alpha (T_1 + T_2) = \alpha T_1 + \alpha T_2$。

\textbf{注释}~ 线性运算(加法和标量乘法)在 $T: \mathbb{F}^n \to \mathbb{F}^m$ 的线性变换上对应于它们矩阵上的相应运算。由于我们知道 $m \times n$ 矩阵的集合是一个向量空间,这立即意味着 $\mathcal{L}(\mathbb{F}^n, \mathbb{F}^m)$ 是一个向量空间。

我们首先提出了抽象证明,首先是因为它适用于一般的空间,例如,适用于没有基的向量空间,在那里我们不能使用坐标。其次,类似于这里展示的抽象推理,在许多地方都会使用,所以读者将受益于理解它。

并且随着读者获得一些数学上的成熟,他/她将看到这种抽象推理确实是非常简单的,几乎可以自动完成。



\section{线性变换的复合与矩阵乘法}

\textbf{5.1. 矩阵乘法的定义}

知道了矩阵-向量乘法,人们很容易猜出两个矩阵乘积 $AB$ 的自然定义:让我们用 $A$ 乘以 $B$ 的每个列(矩阵-向量乘法),并将得到的列向量连接成一个矩阵。形式上,

\fbox{如果 $\mathbf{b}_1, \mathbf{b}_2, \dots, \mathbf{b}_r$ 是 $B$ 的列,那么 $A \mathbf{b}_1, A \mathbf{b}_2, \dots, A \mathbf{b}_r$ 就是矩阵 $AB$ 的列。}

回忆矩阵-向量乘法的“按行列规则”,我们得到矩阵的\textbf{按行列规则}:

\fbox{$AB$ 的项 $(AB)_{j,k}$(第 $j$ 行第 $k$ 列的项)定义为}

\fbox{~~~~~~~~~$(AB)_{j,k} = $ ( $A$ 的第 $j$ 行 ) $\cdot$ ( $B$ 的第 $k$ 列 ~~)}\\
形式上可以写成 
$$(AB)_{j,k} = \sum_{l} a_{j,l} b_{l,k},$$
如果 $a_{j,k}$ 和 $b_{j,k}$ 分别是矩阵 $A$ 和 $B$ 的项。

我特意没有提及矩阵 $A$ 和 $B$ 的大小,但如果我们回忆矩阵-向量乘法的按行列规则,我们可以看到,为了使乘法有定义,$A$ 的行的大小应等于 $B$ 的列的大小。

换句话说,乘积 $AB$ 有定义当且仅当 $A$ 是 $m \times n$ 的矩阵,$B$ 是 $n \times r$ 的矩阵。(这与从按行列规则获得的条件相同。)

\textbf{5.2. 动机:线性变换的复合}

这里,可以问问自己:为什么我们要使用如此复杂的乘法规则?为什么我们不直接逐个元素地相乘矩阵?

答案是,如上定义的乘法自然地源于线性变换的复合。

假设我们有两个线性变换,$T_1: \mathbb{F}^n \to \mathbb{F}^m$ 和 $T_2: \mathbb{F}^r \to \mathbb{F}^n$。定义变换的\textbf{复合} $T = T_1 \circ T_2$ 为
$$T(\mathbf{x}) = T_1(T_2(\mathbf{x})) ~~\ \forall \mathbf{x} \in \mathbb{F}^r.$$
请注意,$T_2(\mathbf{x}) \in \mathbb{F}^n$。由于 $T_1: \mathbb{F}^n \to \mathbb{F}^m$,表达式 $T_1(T_2(\mathbf{x}))$ 是有定义的,并且结果属于 $\mathbb{F}^m$。所以,$T: \mathbb{F}^r \to \mathbb{F}^m.$
\footnote{我们通常将线性变换与其矩阵等同,但在接下来的几个段落中,我们将区分它们。}

可以很容易地证明 $T$ 是一个线性变换(练习),所以它由一个 $m \times r$ 矩阵定义。已知 $T_1$ 和 $T_2$ 的矩阵,如何找到 $T$ 的矩阵?

令 $A$ 为 $T_1$ 的矩阵,令 $B$ 为 $T_2$ 的矩阵。正如我们在上一节中所讨论的,$T$ 的列是向量 $T(\mathbf{e}_1), T(\mathbf{e}_2), \dots, T(\mathbf{e}_r)$,其中 $\mathbf{e}_1, \mathbf{e}_2, \dots, \mathbf{e}_r$ 是 $\mathbb{F}^r$ 中的标准基。对于 $k = 1, 2, \dots, r$,我们有
$$T(\mathbf{e}_k) = T_1(T_2(\mathbf{e}_k)) = T_1(B \mathbf{e}_k) = T_1(\mathbf{b}_k) = A \mathbf{b}_k$$
(变换 $T_2$ 和 $T_1$ 分别就是乘 $B$ 和乘 $A$)。

所以,$T$ 的矩阵的列是 $A \mathbf{b}_1, A \mathbf{b}_2, \dots, A \mathbf{b}_r$,这正是矩阵 $AB$ 的定义方式!

让我们回到等同的观点。由于矩阵乘法与复合一致,我们可以(并且将)写作 $T_1 T_2$ 而不是 $T_1 \circ T_2$,以及 $T_1 T_2 \mathbf{x}$ 而不是 $T_1(T_2(\mathbf{x}))$。
\footnote{\textbf{注意}:变换的顺序! }

注意在组合$T_1 T_2$中,先作变换$T_2$!记住这种方法的方法是看 $T_1 T_2 \mathbf{x}$ 中,变换 $T_2$ 首先与 $\mathbf{x}$ 相遇。

\textbf{注释}~ 除了“按行列规则”的矩阵乘法之外,还有另一种检查矩阵乘积维数的方法:对于复合 $T_1 T_2$ 的定义,必须使得 $T_2 \mathbf{x}$ 属于 $T_1$ 的定义域。如果 $T_2$ 作用于某个空间,比如 $\mathbb{F}^r$ 到 $\mathbb{F}^n$,那么 $T_1$ 必须作用于 $\mathbb{F}^n$ 到某个空间,比如 $\mathbb{F}^m$。因此,为了使 $T_1 T_2$ 有定义,$T_1$ 和 $T_2$ 的矩阵应该分别是 $m \times n$ 和 $n \times r$ 的大小——这与从“按行列规则”获得的条件相同。

\textbf{示例}~ 设 $T: \mathbb{R}^2 \to \mathbb{R}^2$ 是关于直线 $x_1 = 3x_2$ 的反射。这是一个线性变换,所以让我们找到它的矩阵。为了找到矩阵,我们需要计算 $T \mathbf{e}_1$ 和 $T \mathbf{e}_2$。然而,直接计算 $T \mathbf{e}_1$ 和 $T \mathbf{e}_2$ 需要比一个理智的人愿意记住的三角学更多的知识。

找到 $T$ 的矩阵的一个更简单的方法是将其表示为简单线性变换的复合。也就是说,设 $\gamma$ 是 $x_1$ 轴和直线 $x_1 = 3x_2$ 之间的夹角,设 $T_0$ 是关于 $x_1$ 轴的反射。那么要得到反射 $T$,我们可以先将平面旋转 $-\gamma$ 角,将直线 $x_1 = 3x_2$ 移动到 $x_1$ 轴,然后将所有内容在 $x_1$ 轴上反射,然后将平面旋转 $\gamma$ 角,将所有东西移回原位。形式上可以写成 
$$T = R_\gamma T_0 R_{-\gamma}$$
(注意项的顺序!),其中 $R_\gamma$ 是绕 $\gamma$ 角的旋转矩阵。$T_0$ 的矩阵很容易计算,是 
$$T_0 = \begin{pmatrix} 1 & 0 \\ 0 & -1 \end{pmatrix},$$
旋转矩阵是已知的 
$$R_\gamma = \begin{pmatrix} \cos \gamma & -\sin \gamma \\ \sin \gamma & \cos \gamma \end{pmatrix}$$

$$R_{-\gamma} = \begin{pmatrix} \cos(-\gamma) & -\sin(-\gamma) \\ \sin(-\gamma) & \cos(-\gamma) \end{pmatrix} = \begin{pmatrix} \cos \gamma & \sin \gamma \\ -\sin \gamma & \cos \gamma \end{pmatrix}$$
为了计算 $\sin \gamma$ 和 $\cos \gamma$,取直线 $x_1 = 3x_2$ 上的一个向量,例如向量 $(3, 1)^T$。那么 
$$\cos \gamma = \frac{\text{第一个坐标}}{\text{长度}} = \frac{3}{\sqrt{3^2 + 1^2}} = \frac{3}{\sqrt{10}},$$
类似地 
$$\sin \gamma = \frac{\text{第二个坐标}}{\text{长度}} = \frac{1}{\sqrt{3^2 + 1^2}} = \frac{1}{\sqrt{10}}.$$

将所有内容收集起来,我们得到
$$T = R_\gamma T_0 R_{-\gamma} = \frac{1}{\sqrt{10}} \begin{pmatrix} 3 & -1 \\ 1 & 3 \end{pmatrix} \begin{pmatrix} 1 & 0 \\ 0 & -1 \end{pmatrix} \frac{1}{\sqrt{10}} \begin{pmatrix} 3 & 1 \\ -1 & 3 \end{pmatrix} $$
$$= \frac{1}{10} \begin{pmatrix} 3 & -1 \\ 1 & 3 \end{pmatrix} \begin{pmatrix} 1 & 0 \\ 0 & -1 \end{pmatrix} \begin{pmatrix} 3 & 1 \\ -1 & 3 \end{pmatrix}$$
最后一步是进行矩阵乘法得到最终结果。


\textbf{5.3. 矩阵乘法的性质}

矩阵乘法享有许多我们从高中代数中熟悉的性质:

1. 结合律:$A(BC) = (AB)C$,只要等式的一侧或另一侧有定义;因此我们可以(并且将)简单地写成 $ABC$。

2. 分配律:$A(B + C) = AB + AC$, $(A + B)C = AC + BC$,只要每个等式的任一侧有定义。

3. 可以提取标量乘数:$A(\alpha B) = (\alpha A) B = \alpha (AB) = \alpha AB$。

这些性质很容易证明。可以证明对应于线性变换的性质,然后它们几乎自然地从定义中得出。线性变换的性质然后蕴含了矩阵乘法的性质。

这里新的特点是交换律失败了:

~~~~~~~矩阵乘法通常是非交换的,即通常 $AB \neq BA$。

容易看出,期望矩阵乘法满足交换律是不合理的。确实,如果 $A$ 和 $B$ 是 $m \times n$ 和 $n \times r$ 大小的矩阵,那么乘积 $AB$ 有定义,但如果 $m \neq r$,则 $BA$ 未定义。

即使两个乘积都有定义,例如,当 $A$ 和 $B$ 是 $n \times n$(方阵)时,乘法仍然是非交换的。如果我们随机选择矩阵 $A$ 和 $B$,那么 $AB \neq BA$ 的概率很高:我们非常幸运时才能得到 $AB = BA$。

\textbf{5.4. 转置矩阵与乘法}

给定矩阵 $A$,它的\textbf{转置}(transpose)(或转置矩阵)$A^T$ 通过将 $A$ 的行变为列来定义。例如
$$
\begin{pmatrix} 1 & 2 & 3 \\ 4 & 5 & 6 \end{pmatrix}^T = \begin{pmatrix} 1 & 4 \\ 2 & 5 \\ 3 & 6 \end{pmatrix}.
$$
所以,$A^T$ 的列是 $A$ 的行,反之亦然,$A^T$ 的行是 $A$ 的列。

形式定义如下:$(A^T)_{j,k} = (A)_{k,j}$ 意思是 $A^T$ 中第 $j$ 行第 $k$ 列的项等于 $A$ 中第 $k$ 行第 $j$ 列的项。

转置矩阵在线性变换方面有一个很好的解释,即它给出了所谓的\textbf{伴随}(adjoint)变换。我们将在后面详细研究这一点,但现在转置只是一个有用的形式运算。

转置的一个早期用途是我们可以将列向量 $\mathbf{x} \in \mathbb{F}^n$ 写成 $\mathbf{x} = (x_1, x_2, \dots, x_n)^T$。如果我们将列向量垂直放置,它将占用更多的空间。

一个简单的分析表明 $$(AB)^T = B^T A^T,$$
即当你取乘积的转置时,你需要改变项的顺序。

\textbf{5.5. 迹与矩阵乘法}

对于一个方阵($n \times n$)$A = (a_{j,k})$,它的\textbf{迹}(trace)(记作 $\text{trace } A$)是其对角线项的和:
$$
\text{trace } A = \sum_{k=1}^n a_{k,k}
$$

\textbf{定理 5.1} ~设 $A$ 和 $B$ 是大小分别为 $m \times n$ 和 $n \times m$ 的矩阵(因此两个乘积 $AB$ 和 $BA$ 都有定义)。那么
$$
\text{trace}(AB) = \text{trace}(BA).
$$
我们将此定理的证明留作练习,见下文问题 5.6。证明此定理基本上有两种方法。一种方法是计算 $AB$ 和 $BA$ 的对角线项并比较它们的和。这种方法需要一些处理 $\sum$ 符号中求和的熟练技巧。

如果你不熟悉代数运算,还有另一种方法。我们可以考虑两个线性变换$T$和$T_1$,它们作用于 $M_{n \times m}$ 到 $\mathbb{F} = \mathbb{F}^1$,由 
$T(X) = \text{trace}(AX)$ 和 $T_1(X) = \text{trace}(XA)$
定义。
为了证明该定理,只需检查 $T=T_1$ 即可;当 $X=B$ 时,等式即给出该定理。
%fill

\textbf{练习}~

5.1. 设 $A = \begin{pmatrix} 1 & 2 \\ 3 & 1 \end{pmatrix}$, $B = \begin{pmatrix} 1 & 0 & 2 \\ 3 & 1 & -2 \end{pmatrix}$, $C = \begin{pmatrix} 1 & -2 & 3 \\ -2 & 1 & -1 \end{pmatrix}$, $D = \begin{pmatrix} -2 \\ 2 \\ 1 \end{pmatrix}.$

a) 标记所有有定义的乘积,并给出结果的维数:$AB, BA, ABC, ABD, BC, BC^T$, $B^T C, DC, D^T C^T$。

b) 计算 $AB$, $A(3B + C)$, $B^T A$, $A(BD)$, $(AB)D$。

5.2. 设 $T_\gamma$ 是 $\mathbb{R}^2$ 中绕 $\gamma$ 角旋转的矩阵。通过矩阵乘法验证 $T_\gamma T_{-\gamma} = T_{-\gamma} T_\gamma = I$。

5.3. 乘以两个旋转矩阵 $T_\alpha$ 和 $T_\beta$(这是乘法是交换的罕见情况,即 $T_\alpha T_\beta = T_\beta T_\alpha$,所以顺序不重要)。从中推导出 $\sin(\alpha + \beta)$ 和 $\cos(\alpha + \beta)$ 的公式。

5.4. 找到 $\mathbb{R}^2$ 中关于直线 $x_1 = -2x_2$ 的正交投影矩阵。

提示:$x_1$ 轴上的投影矩阵是什么?

5.5. 找到 $A, B: \mathbb{R}^2 \to \mathbb{R}^2$ 的线性变换,使得 $AB = 0$ 但 $BA \neq 0$。

5.6. 证明定理 5.1,即证明 $\text{trace}(AB) = \text{trace}(BA)$。

5.7. 构建一个非零矩阵 $A$ 使得 $A^2 = 0$。

5.8. 找到直线 $y = -2x/3$ 的反射矩阵。执行所有乘法。


\section{可逆变换与矩阵~同构}

\textbf{6.1. 恒等变换与单位矩阵}

在所有线性变换中,有一个特殊的变换,即\textbf{恒等变换}(算子)$I$, $I \mathbf{x} = \mathbf{x}$, $\forall \mathbf{x}$。

准确地说,存在无数个恒等变换:对于任何向量空间 $V$,存在恒等变换 $I = I_V: V \to V$, $I_V \mathbf{x} = \mathbf{x}$, $\forall \mathbf{x} \in V$。但是,当它不引起混淆时,我们也将使用相同的符号 $I$ 来表示所有恒等操作(变换)。只有当我们想强调变换在哪一个空间中作用时,我们才会使用符号 $I_V$。
\footnote{
符号$E$经常在线性代数教科书中用来表示单位矩阵,但我更喜欢$I$,因为它也用于算子理论。
}

显然,如果 $I: \mathbb{F}^n \to \mathbb{F}^n$ 是 $\mathbb{F}^n$ 中的恒等变换,它的矩阵是
$$
I = I_n = \begin{pmatrix}
1 & 0 & \dots & 0 \\
0 & 1 & \dots & 0 \\
\vdots & \vdots & \ddots & \vdots \\
0 & 0 & \dots & 1
\end{pmatrix}
$$
(主对角线上为 1,其他地方为 0。)当我们要强调矩阵的大小,我们使用符号 $I_n$;否则,我们只使用 $I$。

显然,对于任意线性变换 $A$,等式 $$AI = A,~~~~IA = A$$ 成立(只要乘积有定义)。

\textbf{6.2. 可逆变换}

\textbf{定义}~
设 $A: V \to W$ 是一个线性变换。我们说变换 $A$ 是\textbf{左可逆}的,如果存在一个线性变换 $B: W \to V$ 使得 

$BA = I$(此处 $I = I_V$)。
\\
变换 $A$ 称为\textbf{右可逆}的,如果存在一个线性变换 $C: W \to V$ 使得 

$AC = I$(此处 $I = I_W$)。
\\
变换 $B$ 和 $C$ 分别称为 $A$ 的\textbf{左逆}(left inverse)和\textbf{右逆}(right inverse)。注意,我们没有假定 $B$ 或 $C$ 的唯一性,并且通常左逆和右逆不是唯一的。

\textbf{定义}~线性变换 $A: V \to W$ 称为\textbf{可逆的}(invertible),如果它既是右可逆又是左可逆。

\textbf{定理 6.1} 如果线性变换 $A: V \to W$ 是可逆的,那么它的左逆和右逆 $B$ 和 $C$ 是唯一并且相等的。

\textbf{推论}
\footnote{
更为常见的是,这个性质被用于对可逆变换的定义。
}
~ 变换 $A: V \to W$ 是可逆的,当且仅当存在一个唯一的线性变换(记作 $A^{-1}$),$A^{-1}: W \to V$,使得 $$A^{-1} A = I_V,~~~~A A^{-1} = I_W.$$

变换 $A^{-1}$ 称为 $A$ 的\textbf{逆}(inverse)。

\textbf{定理 6.1 的证明} ~设 $BA = I$ 且 $AC = I$。那么 
$$BAC = B(AC) = BI = B.$$
另一方面,
$$BAC = (BA)C = IC = C,$$
因此 $B = C$。

假设对于某个变换 $B_1$ 有 $B_1 A = I$。重复上面的推理,用 $B_1$ 而不是 $B$,我们得到 $B_1 = C$。因此左逆 $B$ 是唯一的。 $C$ 的唯一性同理可证。

\textbf{推论}~矩阵被称为\textbf{可逆}(分别地,\textbf{左可逆},\textbf{右可逆}),如果相应的线性变换是可逆的(分别地,左可逆,右可逆)。

定理 6.1断言,如果存在唯一的矩阵 $A^{-1}$ 使得 $A^{-1} A = I$, $A A^{-1} = I$,那么矩阵 $A$ 是可逆的。这个矩阵 $A^{-1}$ 称为(惊喜!)$A$ 的\textbf{逆}。

\textbf{例子}~

1. 恒等变换(矩阵)是可逆的,$I^{-1} = I$;

2. 旋转 $R_\gamma$ 
$$R_\gamma = \begin{pmatrix} \cos \gamma & -\sin \gamma \\ \sin \gamma & \cos \gamma \end{pmatrix}$$
是可逆的,并且其逆由 $(R_\gamma)^{-1} = R_{-\gamma}$ 给出。这个等式从 $R_\gamma$ 的几何描述中就清楚了,也可以通过矩阵乘法来验证;

3. 列向量 $(1, 1)^T$ 是左可逆但不是右可逆的。可能的左逆之一是行向量 $(1/2, 1/2)$。要证明这个矩阵不是右可逆的,我们只需注意到它有不止一个左逆。\textbf{练习}:描述这个矩阵的所有左逆。

4. 行向量 $(1, 1)$ 是右可逆但不是左可逆的。列向量 $(1/2, 1/2)^T$ 是一个可能的右逆。

\textbf{注释 6.2} 可逆矩阵\textbf{必须}是方阵(待会证明)。而且,如果一个方阵 $A$ 既有左逆又有右逆,那么它就是可逆的。所以,只需检查 $A A^{-1} = I$ 或 $A^{-1} A = I$ 其中一个即可。

这个事实将在后面证明。在此之前,我们不会使用它。我在这里呈现它只是为了阻止学生尝试错误的证明方向。

\textbf{6.2.1. 逆变换的性质}

\textbf{定理 6.3(乘积的逆)}~ 如果线性变换 $A$ 和 $B$ 是可逆的(并且乘积 $AB$ 有定义),那么乘积 $AB$ 也是可逆的,并且 $$(AB)^{-1} = B^{-1} A^{-1}.$$
注意顺序的变化!

\textbf{证明}~ 直接计算表明:$$(AB)(B^{-1} A^{-1}) = A(BB^{-1})A^{-1} = AIA^{-1} = AA^{-1} = I$$
同理
$$(B^{-1} A^{-1})(AB) = B^{-1}(A^{-1} A)B = B^{-1}IB = B^{-1}B = I.$$

\textbf{注释 6.4} ~乘积 $AB$ 的可逆性并不意味着因子 $A$ 和 $B$ 的可逆性(你能想到一个例子吗?)。然而,如果其中一个因子(无论是 $A$ 还是 $B$)以及乘积 $AB$ 都是可逆的,那么第二个因子也是可逆的。

我们将此事实的证明留作练习。

\textbf{定理 6.5($A^T$ 的逆)}~ 如果矩阵 $A$ 是可逆的,那么 $A^T$ 也是可逆的,并且 $$(A^T)^{-1} = (A^{-1})^T.$$

\textbf{证明}~ 使用 $(AB)^T = B^T A^T$ 我们得到 $$(A^{-1})^T A^T = (AA^{-1})^T = I^T = I,$$
同理 
$$A^T (A^{-1})^T = (A^{-1} A)^T = I^T = I.$$

最后,如果 $A$ 是可逆的,那么 $A^{-1}$ 也是可逆的,$(A^{-1})^{-1} = A$。

所以,让我们总结一下逆的三个主要性质:

1. 如果 $A$ 可逆,那么 $A^{-1}$ 也可逆,$(A^{-1})^{-1} = A$;

2. 如果 $A$ 和 $B$ 可逆且乘积 $AB$ 有定义,那么 $AB$ 可逆且 $(AB)^{-1} = B^{-1} A^{-1}$。

3. 如果 $A$ 可逆,那么 $A^T$ 也可逆且 $(A^T)^{-1} = (A^{-1})^T$。

\textbf{6.3. 同构~同构空间}

可逆线性变换 $A: V \to W$ 称为\textbf{同构}(isomorphism)。我们这里没有引入任何新东西,这只是我们已经研究过的对象的另一个名称。

两个向量空间 $V$ 和 $W$ 称为\textbf{同构}(记作 $V \cong W$),如果存在一个同构 $A: V \to W$。

同构空间可以被视为同一个空间的不同的表示,这意味着所有涉及向量空间运算的性质和构造在同构下都被保留。

下面的定理说明了这一点。

\textbf{定理 6.6} ~设 $A: V \to W$ 是一个同构,设 $\mathbf{v}_1, \mathbf{v}_2, \dots, \mathbf{v}_n$ 是 $V$ 中的一个基。那么向量系统 $A \mathbf{v}_1, A \mathbf{v}_2, \dots, A \mathbf{v}_n$ 是 $W$ 中的一个基。

我们把这个定理的证明留作练习。

\textbf{注释}~ 在上面的定理中,我们可以用“线性无关”、“生成”或“线性相关”来替换“基”——所有这些性质在同构下都被保留。

\textbf{注释}~ 如果 $A$ 是同构,那么 $A^{-1}$ 也是同构。因此,在上面的定理中,我们可以说 $\mathbf{v}_1, \mathbf{v}_2, \dots, \mathbf{v}_n$ 是基当且仅当 $A \mathbf{v}_1, A \mathbf{v}_2, \dots, A \mathbf{v}_n$ 是基。

定理 6.6 的逆命题也成立。

\textbf{定理 6.7} 设 $A: V \to W$ 是线性映射,设 $\mathbf{v}_1, \mathbf{v}_2, \dots, \mathbf{v}_n$ 和 $\mathbf{w}_1, \mathbf{w}_2, \dots, \mathbf{w}_n$ 分别是 $V$ 和 $W$ 中的基。如果 $A \mathbf{v}_k = \mathbf{w}_k$, $k = 1, 2, \dots, n$,那么 $A$ 是一个同构。

\textbf{证明}~ 定义逆变换 $A^{-1}$ 为 $A^{-1} \mathbf{w}_k = \mathbf{v}_k$, $k = 1, 2, \dots, n$(正如我们所知,线性变换由其在基上的值定义)。

\textbf{例子}~

1. $A: \mathbb{F}^{n+1} \to P^\mathbb{F}_n$ ($P^\mathbb{F}_n$ 是 $\sum_{k=0}^n a_k t^k$, $\alpha_k \in \mathbb{F}$ 的形式的 $n$ 次多项式集)定义为 
$$A \mathbf{e}_1 = 1, A \mathbf{e}_2 = t, \dots, A \mathbf{e}_n = t^{n-1}, A \mathbf{e}_{n+1} = t^n.$$

根据定理 6.7,$A$ 是同构,所以 $\mathbb{P}^\mathbb{F}_n \cong \mathbb{F}^{n+1}$。

2. 设 $V$ 是一个向量空间(在 $\mathbb{F}$ 上)有一个基 $\mathbf{v}_1, \mathbf{v}_2, \dots, \mathbf{v}_n$。定义变换 $A: \mathbb{F}^n \to V$ 为 
$$A \mathbf{e}_k = \mathbf{v}_k,~~~~k = 1, 2, \dots, n,$$
其中 $\mathbf{e}_1, \mathbf{e}_2, \dots, \mathbf{e}_n$ 是 $\mathbb{F}^n$ 的标准基。根据定理 6.7,$A$ 是同构,所以 $V \cong \mathbb{F}^n$。

3. $M^\mathbb{F}_{2 \times 3}$ 空间($\mathbb{F}$ 中的 $2 \times 3$ 矩阵)同构于 $\mathbb{R}^6$。

4. 更一般地,$M^\mathbb{F}_{m \times n} \cong \mathbb{F}^{m \cdot n}$。

\textbf{6.4. 可逆性与方程}

\textbf{定理 6.8} ~设 $A: X \to Y$ 是一个线性变换。那么 $A$ 是可逆的,当且仅当对于任意右侧 $\mathbf{b} \in Y$,方程 $$A \mathbf{x} = \mathbf{b}$$ 有唯一解 $\mathbf{x} \in X$。

\textbf{证明}~ 假设 $A$ 是可逆的。那么 $\mathbf{x} = A^{-1} \mathbf{b}$ 解决了方程 $A \mathbf{x} = \mathbf{b}$。为了证明解是唯一的,假设对于另一个向量 $\mathbf{x}_1 \in X$, 
$$A \mathbf{x}_1 = \mathbf{b}.$$
将这个恒等式从左边乘以 $A^{-1}$,我们得到 $$A^{-1} A \mathbf{x}_1 = A^{-1} \mathbf{b},$$
因此 $\mathbf{x}_1 = A^{-1} \mathbf{b} = \mathbf{x}$。注意,这里使用了两个恒等式,$AA^{-1} = I$ 和 $A^{-1} A = I$。

现在假设方程 $A \mathbf{x} = \mathbf{b}$ 对于任意 $\mathbf{b} \in Y$ 都有唯一解 $\mathbf{x} \in X$。让我们用 $\mathbf{y}$ 代替 $\mathbf{b}$。我们知道,对于给定的 $\mathbf{y} \in Y$,方程 
$$A \mathbf{x} = \mathbf{y}$$
有唯一解 $\mathbf{x} \in X$。让我们称这个解为 $B(\mathbf{y})$。

注意,$B(\mathbf{y})$ 对所有 $\mathbf{y} \in Y$ 都有定义,因此我们定义了一个变换 $B: Y \to X$。

让我们检查 $B$ 是否是线性变换。我们需要证明 $B(\alpha \mathbf{y}_1 + \beta \mathbf{y}_2) = \alpha B(\mathbf{y}_1) + \beta B(\mathbf{y}_2)$。设 $\mathbf{x}_k := B(\mathbf{y}_k)$, $k = 1, 2$,即
$A \mathbf{x}_k = \mathbf{y}_k$, $k = 1, 2$。那么
$$A(\alpha \mathbf{x}_1 + \beta \mathbf{x}_2) = \alpha A \mathbf{x}_1 + \beta A \mathbf{x}_2 = \alpha \mathbf{y}_1 + \beta \mathbf{y}_2,$$
这意味着
$$B(\alpha \mathbf{y}_1 + \beta \mathbf{y}_2) = \alpha B(\mathbf{y}_1) + \beta B(\mathbf{y}_2).$$

最后,让我们证明 $B$ 确实是 $A$ 的逆。取 $\mathbf{x} \in X$,设 $\mathbf{y} = A \mathbf{x}$,所以根据 $B$ 的定义,我们有 $\mathbf{x} = B \mathbf{y}$。那么对于所有 $\mathbf{x} \in X$, 
$$BA \mathbf{x} = B \mathbf{y} = \mathbf{x},$$
所以 $BA = I$。类似地,对于任意 $\mathbf{y} \in Y$,设 $\mathbf{x} = B \mathbf{y}$,所以 $\mathbf{y} = A \mathbf{x}$。那么对于所有 $\mathbf{y} \in Y$,
$$AB \mathbf{y} = A \mathbf{x} = \mathbf{y},$$
所以 $AB = I$。

回忆基的定义,我们得到以下定理 6.6 和 6.7 的推论。

\textbf{推论 6.9} 一个 $m \times n$ 矩阵可逆,当且仅当它的列在 $\mathbb{F}^m$ 中构成一个基。

\textbf{练习}~

6.1. 证明,如果 $A: V \to W$ 是一个同构(即一个可逆线性变换),并且 $\mathbf{v}_1, \mathbf{v}_2, \dots, \mathbf{v}_n$ 是 $V$ 中的一个基,那么 $A \mathbf{v}_1, A \mathbf{v}_2, \dots, A \mathbf{v}_n$ 是 $W$ 中的一个基。

6.2. 找到行向量 $A = (1, 1)$ 的所有右逆。由此得出 $A$ 行向量不是左可逆的。

6.3. 找到列向量 $(1, 2, 3)^T$ 的所有左逆。

6.4. 列向量 $(1, 2, 3)^T$ 是右可逆的吗?请给出理由。

6.5. 找到两个矩阵 $A$ 和 $B$,使得 $AB$ 是可逆的,但 $A$ 和 $B$ 都不是可逆的。提示:$A$ 和 $B$ 的方阵将不起作用。注意:即使 $AB$ 是 $1 \times 1$ 矩阵(标量),也很容易构造这样的 $A$ 和 $B$。但是你能得到 $2 \times 2$ 矩阵 $AB$ 吗?$3 \times 3$?$n \times n$?

6.6. 假设乘积 $AB$ 是可逆的。证明 $A$ 是右可逆的,$B$ 是左可逆的。提示:你可以直接写出右逆和左逆的公式。

6.7. 假设 $A$ 和 $AB$ 都是可逆的(假设乘积 $AB$ 有定义)。证明 $B$ 是可逆的。

6.8. 设 $A$ 是一个 $n \times n$ 矩阵。证明如果 $A^2 = 0$ 则 $A$ 不可逆。

6.9. 假设 $AB = 0$ 对某个非零矩阵 $B$ 成立。 $A$ 能是可逆的吗?请给出理由。

6.10. 在 $\mathbb{F}^5$ 中找到代表如下变换的矩阵 $T_1$ 和 $T_2$:$T_1$ 交换向量 $x$ 的坐标 $x_2$ 和 $x_4$,而 $T_2$ 只是将 $x_4$ 的 $a$ 倍加到坐标 $x_2$ 上,而不改变其他坐标,即
$$T_1 \begin{pmatrix} x_1 \\ x_2 \\ x_3 \\ x_4 \\ x_5 \end{pmatrix} = \begin{pmatrix} x_1 \\ x_4 \\ x_3 \\ x_2 \\ x_5 \end{pmatrix},~~~~T_2 \begin{pmatrix} x_1 \\ x_2 \\ x_3 \\ x_4 \\ x_5 \end{pmatrix} = \begin{pmatrix} x_1 \\ x_2 + ax_4 \\ x_3 \\ x_4 \\ x_5 \end{pmatrix};$$
这里 $a$ 是某个固定数。

证明 $T_1$ 和 $T_2$ 是可逆变换,并写出它们的逆矩阵。\textbf{提示}:先描述逆变换,然后再求它的矩阵,可能比猜测(或计算)$T_1$, $T_2$ 的逆矩阵要简单。

6.11. 找到 $\mathbb{R}^3$ 中绕向量 $(1, 2, 3)^T$ 轴绕 $\alpha$ 角旋转的变换的矩阵。我们假设旋转是从向量的尖端看向原点时逆时针旋转的。

您可以将答案表示为几个矩阵的乘积:您不必执行乘法。

6.12. 给出 $2 \times 2$ 矩阵的例子,使得:

a) $A+B$ 不可逆,尽管 $A$ 和 $B$ 都可逆;

b) $A+B$ 可逆,尽管 $A$ 和 $B$ 都不可逆;

c) $A$, $B$ 和 $A+B$ 都可逆。

6.13. 设 $A$ 是一个可逆的对称矩阵 ($A^T = A$)。$A$ 的逆是否对称?请给出理由。


\section{子空间}

向量空间 $V$ 的一个\textbf{子空间}(subspace)是 $V$ 的一个非空子集 $V_0 \subset V$,它对向量加法和标量乘法是封闭的,即

1. 如果 $\mathbf{v} \in V_0$,则 $\alpha \mathbf{v} \in V_0$ 对所有标量 $\alpha$;

2. 对于任何 $\mathbf{u}, \mathbf{v} \in V_0$,它们的和 $\mathbf{u} + \mathbf{v} \in V_0$;

再次,条件 1 和 2 可以被以下一个条件替换:

$\alpha \mathbf{u} + \beta \mathbf{v} \in V_0$ 对所有 $\mathbf{u}, \mathbf{v} \in V_0$,以及所有标量 $\alpha, \beta$。

请注意,子空间 $V_0 \subset V$ 连同从 $V$ 继承的运算(向量加法和标量乘法),是一个向量空间。实际上,由于 $V$ 非空,它至少包含 1 个向量 $\mathbf{v}$,并且由于 $\mathbf{0} = 0\mathbf{v}$,所以上述条件 1.意味着零向量 $\mathbf{0}$ 在 $V$ 中。此外,对于任何 $\mathbf{v} \in V$,它的加法逆元 $-\mathbf{v}$ 由 $-\mathbf{v} = (-1)\mathbf{v}$ 给出,所以再次根据性质 1.$-\mathbf{v} \in V$。向量空间的其余公理之所以成立,是因为所有运算都源于向量空间 $V$。唯一可能出错的是某个运算的结果不属于 $V_0$。但子空间的定义禁止了这一点!

现在我们来看一些例子:

1. 空间 $V$ 的\textbf{平凡}子空间,即 $V$ 本身和 $\{\mathbf{0}\}$(仅包含零向量的子空间)。注意,空集 $\emptyset$ 不是向量空间,因为它不包含零向量,所以它不是子空间。

任何线性变换 $A: V \to W$ 都可以关联以下两个子空间:

2. $A$ 的\textbf{零空间}(null space)或\textbf{核}(kernel),记作 $\text{Null } A$ 或 $\text{Ker } A$,由所有满足 $A \mathbf{v} = \mathbf{0}$ 的向量 $\mathbf{v} \in V$ 组成。

3. \textbf{像空间}(range)$\text{Ran } A$ 定义为所有可以表示为 $\mathbf{w} = A \mathbf{v}$ 的向量 $\mathbf{w} \in W$ 的集合,其中某个 $\mathbf{v} \in V$。

如果 $A$ 是一个矩阵,即 $A: \mathbb{F}^m \to \mathbb{F}^n$,那么回忆矩阵-向量乘法的“按列坐标规则”,我们可以看到任何向量 $\mathbf{w} \in \text{Ran } A$ 都可以表示为 $A$ 的列的线性组合。这解释了为什么\textbf{列空间}(表示为 $\text{Col } A$)这个术语经常用来表示矩阵的像空间。因此,对于矩阵 $A$,符号 $\text{Col } A$ 通常用于代替 $\text{Ran } A$。

还有最后一个例子。

4. 给定向量系统 $\mathbf{v}_1, \mathbf{v}_2, \dots, \mathbf{v}_r \in V$,它的\textbf{线性张成}(linear span)(有时简单称为\textbf{张成})$\mathcal{L}\{\mathbf{v}_1, \mathbf{v}_2, \dots, \mathbf{v}_r\}$ 是 $V$ 中所有可以表示为向量 $\mathbf{v}_1, \mathbf{v}_2, \dots, \mathbf{v}_r$ 的线性组合 $\mathbf{v} = \alpha_1 \mathbf{v}_1 + \alpha_2 \mathbf{v}_2 + \dots + \alpha_r \mathbf{v}_r$ 的向量的集合。符号 $\text{span}\{\mathbf{v}_1, \mathbf{v}_2, \dots, \mathbf{v}_r\}$ 也用于代替 $\mathcal{L}\{\mathbf{v}_1, \mathbf{v}_2, \dots, \mathbf{v}_r\}$。

可以很容易地检查出在所有这些例子中我们确实得到了子空间。我们将此作为读者的练习。其中一些陈述将在本书后面证明。


\textbf{练习}~

7.1. 设 $X$ 和 $Y$ 是向量空间 $V$ 的子空间。证明 $X \cap Y$ 是 $V$ 的子空间。

7.2. 设 $V$ 是一个向量空间。对于 $X, Y \subset V$,和 $X+Y$ 是所有可以表示为 $\mathbf{v} = \mathbf{x} + \mathbf{y}$, $\mathbf{x} \in X$, $\mathbf{y} \in Y$ 的向量的集合。证明如果 $X$ 和 $Y$ 是 $V$ 的子空间,那么 $X+Y$ 也是子空间。

7.3. 设 $X$ 是向量空间 $V$ 的子空间,设 $\mathbf{v} \in V$, $\mathbf{v} \notin X$。证明如果 $\mathbf{x} \in X$,则 $\mathbf{x} + \mathbf{v} \notin X$。

7.4. 设 $X$ 和 $Y$ 是向量空间 $V$ 的子空间。利用上一道练习,证明 $X \cup Y$ 是子空间当且仅当 $X \subset Y$ 或 $Y \subset X$。

7.5. 包含所有上三角矩阵($a_{j,k} = 0$ $\forall j > k$)和所有对称矩阵($A = A^T$)的 $4 \times 4$ 矩阵空间中最小的子空间是什么?包含在这两个子空间中的最大子空间是什么?


\section{应用于计算机图形学}

在本节中,我们将介绍一些线性代数在计算机图形学中的应用。我们将不深入细节,只是解释一些思想。特别是我们将解释为什么对三维图像的操作会简化为 $4 \times 4$ 矩阵的乘法。

\textbf{8.1.二维操作}

$x-y$ 平面(更准确地说,平面上的一个矩形)是计算机显示器的一个很好的模型。显示器上的任何对象都表示为\textbf{像素}(pixel)的集合,每个像素被分配一个特定的颜色。每个像素的位置由其列和行确定,它们充当平面上的 $x$ 和 $y$ 坐标。所以,具有 $x-y$ 坐标的平面上的矩形是计算机屏幕的一个好模型:而图形对象只是点的集合。

\textbf{注释}~ 有两种类型的图形对象:位图对象,其中描述了对象的每个像素,以及矢量对象,其中我们只描述\textbf{关键点}(critical points),然后图形引擎将它们连接起来以重建对象。照片是位图对象的一个好例子:它的每个像素都被描述。位图对象可能包含很多点,所以处理位图需要大量的计算能力。任何使用过位图处理程序(如 Adobe Photoshop)的人都知道,你需要一台相当强大的计算机,即使是现代和强大的计算机,操作也可能需要一些时间。

这就是为什么出现在计算机屏幕上的大多数对象都是矢量对象的原因:计算机只需要记住关键点。例如,要描述一个多边形,你只需要给出其顶点的坐标,以及哪些顶点连接到哪个顶点。当然,并非屏幕上的所有对象都可以表示为多边形,有些对象,如字母,具有平滑弯曲的边界。但是,存在标准方法可以允许我们通过一组点绘制平滑曲线,例如 Bezier 样条,在 PostScript 和 Adobe PDF(以及许多其他格式)中使用。

无论如何,这是另一本书的主题,我们在这里不讨论它。对我们来说,一个图形对象将是一组点(无论是线框模型(wireframe model)还是位图),我们想展示如何对这些对象执行一些操作。

最简单的变换是\textbf{平移}(translation)(移动)(shift),其中每个点(向量)$\mathbf{v}$ 被平移 $\mathbf{a}$,即向量 $\mathbf{v}$ 被替换为 $\mathbf{v} + \mathbf{a}$(表示 $ \mathbf{v} \mapsto \mathbf{v} + \mathbf{a} $)。向量加法非常适合计算机,因此平移很容易实现。注意,平移不是线性变换(如果 $\mathbf{a} \neq 0$):虽然它保留了直线,但它不保留 $\mathbf{0}$。

计算机图形学中使用的所有其他变换都是线性的。第一个想到的就是旋转。绕原点 $\mathbf{0}$ 的 $\gamma$ 角旋转由我们上面讨论过的旋转矩阵 $R_\gamma$ 给出, $$R_\gamma = \begin{pmatrix} \cos \gamma & -\sin \gamma \\ \sin \gamma & \cos \gamma \end{pmatrix}.$$
如果我们想绕一个点 $\mathbf{a}$ 旋转,我们首先需要平移图像 $-\mathbf{a}$,将点 $\mathbf{a}$ 移动到 $\mathbf{0}$,然后绕 $\mathbf{0}$ 旋转(乘以 $R_\gamma$),然后将所有内容平移回 $\mathbf{a}$。

另一个非常有用的变换是\textbf{缩放}(scaling),由矩阵 
$$\begin{pmatrix} a & 0 \\ 0 & b \end{pmatrix}$$
给出,$a, b \ge 0$。如果 $a=b$ 它是\textbf{均匀缩放}(uniform scaling),它放大(缩小)对象,保持其形状。如果 $a \neq b$ 则 $x$ 和 $y$ 坐标缩放不同;对象变得“更高”或“更宽”。

另一个经常使用的变换是\textbf{反射}(reflection):例如矩阵 
$$\begin{pmatrix} 1 & 0 \\ 0 & -1 \end{pmatrix}$$
定义了关于 $x$ 轴的反射。

我们将在本书后面证明, $\mathbb{R}^2$ 中的任何线性变换都可以表示为缩放、旋转和反射的复合。然而,有时考虑一些不同的变换,如\textbf{剪切变换}(shear transformation),由矩阵 
$$\begin{pmatrix} 1 & \tan \phi \\ 0 & 1 \end{pmatrix}$$
给出。这个变换使得所有对象倾斜,水平线保持水平,但垂直线变成与水平线成 $\phi$ 角的倾斜线。

\textbf{8.2. 三维图形}

三维图形更复杂。首先我们需要能够操作三维物体,然后需要将其表示在二维平面(显示器)上。

三维物体的操作非常直接,我们有相同的基本变换:平移、平面反射、缩放、旋转。这些变换的矩阵与其二维对应物的矩阵非常相似。例如,矩阵
$$
\begin{pmatrix} 1 & 0 & 0 \\ 0 & 1 & 0 \\ 0 & 0 & -1 \end{pmatrix}, \quad \begin{pmatrix} a & 0 & 0 \\ 0 & b & 0 \\ 0 & 0 & c \end{pmatrix}, \quad \begin{pmatrix} \cos \gamma & -\sin \gamma & 0 \\ \sin \gamma & \cos \gamma & 0 \\ 0 & 0 & 1 \end{pmatrix}
$$
分别代表了关于 $x-y$ 平面的反射、缩放和绕 $z$ 轴的旋转。

请注意,上述旋转本质上是二维变换,它不改变 $z$ 坐标。类似地,可以为绕 $x$ 轴和绕 $y$ 轴的其他 2 个基本旋转写出矩阵。后面将表明,任意轴上的旋转可以表示为基本旋转的复合。因此,我们知道如何操作三维物体。

现在让我们讨论如何将三维物体表示在二维平面上。最简单的方法是将其投影到平面,比如 $x-y$ 平面。要执行这种投影,只需将 $z$ 坐标替换为 0,这个投影(projection)的矩阵是
$$
\begin{pmatrix} 1 & 0 & 0 \\ 0 & 1 & 0 \\ 0 & 0 & 0 \end{pmatrix}.
$$
这种方法通常用于技术插图。旋转一个物体并对其进行投影相当于从不同的点看它。然而,这种方法没有给出非常逼真的图像,因为它没有考虑透视,即远处的物体看起来更小的事实。

为了获得更逼真的图像,我们需要使用所谓的\textbf{透视投影}(perspective projection)。要定义透视投影,我们需要选择一个点(投影中心或焦点)和一个要投影到的平面。然后,$\mathbb{R}^3$ 中的每个点被投影到一个平面上的点,使得该点、它的像以及\textbf{投影中心}(center of the projection)位于同一条线上,见图 2。

\red{这里放图片}

图 2. 透视投影到 $x-y$ 平面:$F$ 是投影中心(焦点)

这正是相机的工作方式,也是我们眼睛工作方式的一个合理初步近似。

让我们得到投影的公式。假设焦点是 $(0, 0, d)^T$,并且我们投影到 $x-y$ 平面,见图 3 a)。考虑一个点 $\mathbf{v} = (x, y, z)^T$,让 $\mathbf{v}^* = (x^*, y^*, 0)^T$ 是它的投影。分析相似三角形,见图 3 b),我们得到

\red{Put the figure 3 here.}

图 3. 找到点 $(x, y, z)^T$ 的透视投影的坐标 $x^*, y^*$。

$$\frac{x^*}{d} = \frac{x}{d-z},$$
所以 
$$x^* = \frac{xd}{d-z} = \frac{x}{1 - z/d},$$
类似地 
$$y^* = \frac{y}{1 - z/d}.$$
注意,这个公式在 $z > d$ 和 $z < 0$ 时也有效:你可以画出相应的相似三角形来验证它。


因此,透视投影将点 $(x, y, z)^T$ 映射到点 $(\frac{x}{1-z/d}, \frac{y}{1-z/d}, 0)^T.$

这个变换肯定不是线性的(由于分母中的 $z$)。然而,通过引入所谓的\textbf{齐次坐标}(homogeneous coordinates),仍然可以将其表示为线性变换。

在齐次坐标中,$\mathbb{R}^3$ 中的每个点都由 4 个坐标表示,最后一个(第四个)坐标起到了缩放系数的作用。因此,要从 $\mathbf{v} = (x, y, z)^T$ 的齐次坐标$\mathbf{v} = (x_1, x_2, x_3, x_4) ^T$得到其通常的三维坐标,需要将所有项除以最后一个坐标 $x_4$,然后取前 3 个坐标
\footnote{
如果我们对齐次坐标下一个在$\mathbb{R}^2$中的点乘上一个非零标量,我们没有改变这个点。换而言之,在齐次坐标下,一个在$\mathbb{R}^3$中的点被表示为在$\mathbb{R}^4$中有一行为0的点。
}
(如果 $x_4 = 0$,则此方法不适用,因此我们假设 $x_4 = 0$ 的情况对应于无穷远点)。

因为在齐次坐标中,向量$\mathbf{v}^*$可以被表示为$x, y, 0, 1-z/d)^T$,因此,在齐次坐标中,透视投影是一个线性变换:
$$
\begin{pmatrix} x \\ y \\ 0 \\ 1 - z/d \end{pmatrix} = \begin{pmatrix} 1 & 0 & 0 & 0 \\ 0 & 1 & 0 & 0 \\ 0 & 0 & 0 & 0 \\ 0 & 0 & -1/d & 1 \end{pmatrix} \begin{pmatrix} x \\ y \\ z \\ 1 \end{pmatrix}.
$$
请注意,在齐次坐标中,平移也是一个线性变换:
$$
\begin{pmatrix} x + a_1 \\ y + a_2 \\ z + a_3 \\ 1 \end{pmatrix} = \begin{pmatrix} 1 & 0 & 0 & a_1 \\ 0 & 1 & 0 & a_2 \\ 0 & 0 & 1 & a_3 \\ 0 & 0 & 0 & 1 \end{pmatrix} \begin{pmatrix} x \\ y \\ z \\ 1 \end{pmatrix}.
$$

但是,如果投影中心不是 $(0, 0, d)^T$ 这样的点,而是任意点 $(d_1, d_2, d_3)^T$ 呢?
那么我们首先需要应用 $-( d_1 , d_2 , 0 )^T$ 的平移来将中心移到 $(0, 0, d_3)^T$,同时保持 $x-y$ 平面不变,应用投影,然后通过 $(d_1, d_2, 0)^T$ 的平移将所有内容移回。类似地,如果投影平面不是 $x-y$ 平面,我们通过使用旋转和平移将其移到 $x-y$ 平面,等等。

所有这些操作只是 $4 \times 4$ 矩阵的乘法。这就解释了为什么现代图形卡将 $4 \times 4$ 矩阵运算嵌入到处理器中。

当然,这里我们只触及了三维图形背后的数学,还有更多内容。例如,如何确定物体的哪些部分可见,哪些部分被隐藏,如何制作逼真的照明、阴影等等。


\textbf{练习}~

8.1. $\mathbb{R}^3$ 中齐次坐标为 $(10, 20, 30, 5)^T$ 的向量是什么?

8.2. 证明 $\gamma$ 角的旋转可以表示为两次剪切-缩放变换的复合 
$$T_1 = \begin{pmatrix} 1 & 0 \\ \sin \gamma & \cos \gamma \end{pmatrix} ,~~~~~~~~T_2 = \begin{pmatrix} \sec \gamma & -\tan \gamma \\ 0 & 1 \end{pmatrix}.$$
应该以什么顺序进行变换?

8.3. 一个2维向量乘以一个任意的$2\times 2$矩阵通常需要4次乘法。

假设$2 \times 1000$ 矩阵 $D$ 包含 $\mathbb{R}^2$ 中 1000 个点的坐标。使用两个任意 $2 \times 2$ 矩阵 $A$ 和 $B$ 来变换这些点需要多少次乘法?比较两种可能性,$A(BD)$ 和 $(AB)D$。

8.4. 写一个 $4 \times 4$ 矩阵,执行透视投影到 $x-y$ 平面,其中心为 $(d_1, d_2, d_3)^T$。

8.5. 变换 $T$ 在 $\mathbb{R}^3$ 中是 $x-y$ 平面中直线 $y = 2x+3$ 绕 $\gamma$ 角的旋转。写出与此变换对应的 $4 \times 4$ 矩阵。你可以将结果表示为矩阵的乘积。





% \chapter{线性方程组}

\section{线性方程组的不同表示}

关于线性方程组,或者简而言之\textbf{线性系统},存在几种观点。第一种,朴素的观点是,它仅仅是 $n$ 个未知数 $x_1, x_2, \dots, x_n$ 的 $m$ 个线性方程的集合:
$$
\begin{cases}
a_{1,1} x_1 + a_{1,2} x_2 + \dots + a_{1,n} x_n = b_1 \\
a_{2,1} x_1 + a_{2,2} x_2 + \dots + a_{2,n} x_n = b_2 \\
\vdots \\
a_{m,1} x_1 + a_{m,2} x_2 + \dots + a_{m,n} x_n = b_m
\end{cases}
$$
求解该系统是指找到所有满足所有 $m$ 个方程的 $n$ 元数组 $x_1, x_2, \dots, x_n$。

如果我们记 $x := (x_1, x_2, \dots, x_n)^T \in F^n$, $b = (b_1, b_2, \dots, b_m)^T \in F^m$,以及
$$
A = \begin{pmatrix}
a_{1,1} & a_{1,2} & \dots & a_{1,n} \\
a_{2,1} & a_{2,2} & \dots & a_{2,n} \\
\vdots & \vdots & \ddots & \vdots \\
a_{m,1} & a_{m,2} & \dots & a_{m,n}
\end{pmatrix}
$$
那么上述线性系统可以用\textbf{矩阵形式}(作为\textbf{矩阵-向量方程})写成 $A \mathbf{x} = \mathbf{b}$。求解上述方程是指找到所有满足 $A \mathbf{x} = \mathbf{b}$ 的向量 $\mathbf{x} \in F^n$。

最后,回忆矩阵-向量乘法的“按列坐标规则”,我们可以将系统写成一个\textbf{向量方程}:
$$
x_1 \mathbf{a}_1 + x_2 \mathbf{a}_2 + \dots + x_n \mathbf{a}_n = \mathbf{b}
$$
其中 $\mathbf{a}_k$ 是矩阵 $A$ 的第 $k$ 列,$\mathbf{a}_k = (a_{1,k}, a_{2,k}, \dots, a_{m,k})^T$, $k = 1, 2, \dots, n$。

注意,这三个例子本质上只是同一个数学对象的不同表示。

在解释如何求解线性系统之前,让我们注意到我们如何称呼未知数,例如 $x_k$, $y_k$ 或其他名称,这并不重要。因此,所有求解系统所需的信息都包含在矩阵 $A$ 中,该矩阵称为系统的\textbf{系数矩阵}(coefficient matrix),以及向量(右侧)$b$。因此,我们所需的所有信息都包含在以下矩阵中:
$$
\begin{pmatrix}
a_{1,1} & a_{1,2} & \dots & a_{1,n} & | & b_1 \\
a_{2,1} & a_{2,2} & \dots & a_{2,n} & | & b_2 \\
\vdots & \vdots & \ddots & \vdots & | & \vdots \\
a_{m,1} & a_{m,2} & \dots & a_{m,n} & | & b_m
\end{pmatrix}
$$
该矩阵是通过将列 $b$ 连接到矩阵 $A$ 上形成的。这个矩阵称为系统的\textbf{增广矩阵}(augmented matrix)。我们通常会放一条垂直线来分隔 $A$ 和 $b$,以区分增广矩阵和系数矩阵。


\section{线性方程组的求解~阶梯形与简化阶梯形}

线性系统是通过\textbf{高斯-若尔当消元法}(有时称为\textbf{行约简})求解的。通过对系统增广矩阵的行(即方程)执行运算,我们将它简化为一种简单的形式,即所谓的\textbf{阶梯形}。当系统处于\textbf{阶梯形}时,可以轻松地写出解。

\textbf{2.1. 行运算。}

我们使用的行运算有三种类型:
1. 行交换:交换矩阵的任意两行;
2. 缩放:用一个非零标量 $a$ 乘以某一行;
3. 行替换:用第 $j$ 行的常数倍加上第 $k$ 行来替换第 $k$ 行;其他所有行保持不变。

可以清楚地看出,运算 1 和 2 不会改变系统的解集;它们基本上不改变系统。

至于运算 3,可以很容易地看出它不会丢失解。也就是说,设一个“新”系统是通过类型 3 的行运算从“旧”系统中得到的。那么“旧”系统的任何解都是“新”系统的解。为了证明我们没有得到任何额外的东西,即“新”系统的任何解也是“旧”系统的解,我们只需注意到类型 3 的行运算是\textbf{可逆}的,也就是说,“旧”系统也可以通过应用类型 3 的行运算从“新”系统中获得(你能说出是哪一种吗?)

\textbf{2.1.1. 行运算与初等矩阵的乘法。}

还有另一种更“高级”的解释,说明为什么上述行运算是合法的。也就是说,每个行运算都相当于从左边乘以一个特殊的初等矩阵。也就是说,乘以矩阵
$$
\begin{pmatrix}
1 & \dots & 0 & \dots & 0 \\
\vdots & \ddots & \vdots & \dots & \vdots \\
0 & \dots & 1 & \dots & 0 \\
\vdots & \dots & \vdots & \ddots & \vdots \\
0 & \dots & 0 & \dots & 1
\end{pmatrix}
$$
其中第 $j$ 行和第 $k$ 行是单位矩阵 $I$ 的第 $j$ 行和第 $k$ 行的交换。
其中第 $k$ 行乘以 $a$。
其中第 $k$ 行加上第 $j$ 行的 $a$ 倍。

要看到这些初等矩阵的乘法是按预期工作的,你可以简单地看看它们如何作用于向量(列)。注意,所有这些矩阵都是可逆的(与行运算的可逆性进行比较)。第一个矩阵的逆是它本身。要得到第二个矩阵的逆,只需将 $a$ 替换为 $1/a$。最后,第三个矩阵的逆是通过将 $a$ 替换为 $-a$ 来获得的。要看到逆确实是这样获得的,人们可以简单地通过检查它们如何作用于列来检查。

因此,对系统 $A \mathbf{x} = \mathbf{b}$ 的增广矩阵执行行运算,相当于将系统(从左边)乘以一个特殊的初等矩阵 $E$。将等式 $A \mathbf{x} = \mathbf{b}$ 从左边乘以 $E$,我们得到 $A \mathbf{x} = \mathbf{b}$ 的任何解也是 $EA \mathbf{x} = E \mathbf{b}$ 的解。将这个方程从左边乘以 $E^{-1}$,我们得到它的任何解也是 $E^{-1} EA \mathbf{x} = E^{-1} E \mathbf{b}$ 的解,也就是原始方程 $A \mathbf{x} = \mathbf{b}$。所以,行运算不改变系统的解集。

\textbf{2.2. 行约简。}

行约简的主要步骤包括三个子步骤:
1. 找到矩阵中最左边的非零列;
2. 通过应用行运算(必要时进行行交换),确保该列的第一个(最上面的)项非零。这个项将被称为\textbf{主元项}(pivot entry)或简称为\textbf{主元}(pivot);
3. 通过从第 2、3、...、m 行减去第一行的适当倍数来“消去”(即使其为 0)主元下的所有非零项。

我们将主步骤应用于一个矩阵,然后将第一行单独处理,并对第 2、...、m 行应用主步骤,然后对第 3、...、m 行应用主步骤,等等。需要记住的一点是,在将一行的一倍数从该行减去所有下面的行(步骤 3)之后,我们将该行单独处理,并且不以任何方式更改它,甚至不与其他行交换。在应用主步骤有限次(最多 $m$ 次)之后,我们得到所谓的矩阵的\textbf{阶梯形}。

\textbf{2.2.1. 行约简的例子。}

让我们考虑以下线性系统:
$$
\begin{cases}
x_1 + 2x_2 + 3x_3 = 1 \\
3x_1 + 2x_2 + x_3 = 7 \\
2x_1 + x_2 + 2x_3 = 1
\end{cases}
$$
系统的增广矩阵是
$$
\begin{pmatrix} 1 & 2 & 3 & | & 1 \\ 3 & 2 & 1 & | & 7 \\ 2 & 1 & 2 & | & 1 \end{pmatrix}
$$
从第二行减去 3 倍第一行,并从第三行减去 2 倍第一行,我们得到:
$$
\begin{pmatrix} 1 & 2 & 3 & | & 1 \\ 3 & 2 & 1 & | & 7 \\ 2 & 1 & 2 & | & 1 \end{pmatrix} \xrightarrow[-2R_1]{-3R_1} \begin{pmatrix} 1 & 2 & 3 & | & 1 \\ 0 & -4 & -8 & | & 4 \\ 0 & -3 & -4 & | & -1 \end{pmatrix}
$$
将第二行乘以 $-1/4$ 得到:
$$
\begin{pmatrix} 1 & 2 & 3 & | & 1 \\ 0 & 1 & 2 & | & -1 \\ 0 & -3 & -4 & | & -1 \end{pmatrix}
$$
将第三行加上 3 倍第二行得到:
$$
\begin{pmatrix} 1 & 2 & 3 & | & 1 \\ 0 & 1 & 2 & | & -1 \\ 0 & -3 & -4 & | & -1 \end{pmatrix} \xrightarrow{+3R_2} \begin{pmatrix} 1 & 2 & 3 & | & 1 \\ 0 & 1 & 2 & | & -1 \\ 0 & 0 & 2 & | & -4 \end{pmatrix}
$$
现在我们可以使用所谓的\textbf{反代入}来求解系统。即,从最后一行(方程)我们得到 $x_3 = -2$。然后从第二个方程我们得到 $x_2 = -1 - 2x_3 = -1 - 2(-2) = 3$,最后,从第一行(方程)$x_1 = 1 - 2x_2 - 3x_3 = 1 - 6 + 6 = 1$。

所以,解是 $\begin{cases} x_1 = 1 \\ x_2 = 3 \\ x_3 = -2 \end{cases}$,或者向量形式 $\mathbf{x} = \begin{pmatrix} 1 \\ 3 \\ -2 \end{pmatrix}$ 或 $\mathbf{x} = (1, 3, -2)^T$。我们可以通过乘以系数矩阵 $A$ 来检查解。

与其使用反代入,不如进行行约简从下到上,消去系数矩阵主对角线上的所有项:我们从将最后一行乘以 $1/2$ 开始,其余的都很直观:
$$
\begin{pmatrix} 1 & 2 & 3 & | & 1 \\ 0 & 1 & 2 & | & -1 \\ 0 & 0 & 1 & | & -2 \end{pmatrix} \xrightarrow[-2R_3]{-3R_3} \begin{pmatrix} 1 & 2 & 0 & | & 7 \\ 0 & 1 & 0 & | & 3 \\ 0 & 0 & 1 & | & -2 \end{pmatrix} \xrightarrow{-2R_2} \begin{pmatrix} 1 & 0 & 0 & | & 1 \\ 0 & 1 & 0 & | & 3 \\ 0 & 0 & 1 & | & -2 \end{pmatrix}
$$
我们只需从简化阶梯形矩阵中读出解 $\mathbf{x} = (1, 3, -2)^T$。我们将向读者说明向后阶段的行约简算法。

\textbf{2.3. 阶梯形。}

一个矩阵被称为\textbf{阶梯形}(echelon form),如果它满足以下两个条件:
1. 所有零行(即所有项都等于 0 的行),如果存在的话,都位于所有非零项的下方。对于非零行,让最左边的非零项称为\textbf{前导项}(leading entry)。那么阶梯形的第二性质可以表述如下:
2. 对于任何非零行,其前导项严格位于前一行前导项的右侧。

阶梯形中的每一行的前导项也称为\textbf{主元项}(pivot entry),或简称为\textbf{主元}(pivot),因为这些项正是我们在行约简中使用的主元。

我们上面得到的例子中的一个特殊情况是所谓的\textbf{三角}形(triangular)形式。在该形式中,系数矩阵是方阵($n \times n$),其主对角线上的所有项都非零,并且主对角线下的所有项都为零。右侧,即增广矩阵的最右边一列,可以是任意的。在行约简的向后阶段之后,我们得到所谓的矩阵的\textbf{简化阶梯形}:系数矩阵等于 $I$,如上例所示,这是简化阶梯形的一个特例。一般定义如下:我们说一个矩阵处于\textbf{简化阶梯形},如果它处于阶梯形并且
3. 所有主元项都等于 1;
4. 主元上方的所有项都为 0。

注意,由于阶梯形的原因,主元下方的所有项也为 0。为了从阶梯形得到简化阶梯形,我们从下往上,从右往左工作,使用行替换来消去主元上方的所有项。简化阶梯形的一个例子是系数矩阵等于 $I$ 的系统。在这种情况下,只需从简化阶梯形中读出解。通常情况下,也可以轻松地从阶梯形读出解。例如,设系统(增广矩阵)的简化阶梯形是
$$
\begin{pmatrix}
1 & 2 & 0 & 0 & 0 & | & 1 \\
0 & 0 & 1 & 5 & 0 & | & 2 \\
0 & 0 & 0 & 0 & 1 & | & 3
\end{pmatrix}
$$
这里我们框出了主元。想法是将与没有主元的列对应的变量(所谓的\textbf{自由变量})移到右侧。然后我们就可以直接写出解。
$$
\begin{cases}
x_1 = 1 - 2x_2 \\
x_2 \text{ 是自由变量} \\
x_3 = 2 - 5x_4 \\
x_4 \text{ 是自由变量} \\
x_5 = 3
\end{cases}
$$
或者,在向量形式下:
$$
\mathbf{x} = \begin{pmatrix} 1 - 2x_2 \\ x_2 \\ 2 - 5x_4 \\ x_4 \\ 3 \end{pmatrix} = \begin{pmatrix} 1 \\ 0 \\ 2 \\ 0 \\ 3 \end{pmatrix} + x_2 \begin{pmatrix} -2 \\ 1 \\ 0 \\ 0 \\ 0 \end{pmatrix} + x_4 \begin{pmatrix} 0 \\ 0 \\ -5 \\ 1 \\ 0 \end{pmatrix}, \quad x_2, x_4 \in F
$$
也可以通过反代入从阶梯形得到解:其思想是从下往上工作,将所有自由变量移到右侧。


\textbf{练习}~

2.1. 将以下方程组写成矩阵形式和向量方程形式:
a) $\begin{cases} x_1 + 2x_2 - x_3 = -1 \\ 2x_1 + 2x_2 + x_3 = 1 \\ 3x_1 + 5x_2 - 2x_3 = -1 \end{cases}$
b) $\begin{cases} x_1 - 2x_2 - x_3 = 1 \\ 2x_1 - 3x_2 + x_3 = 6 \\ 3x_1 - 5x_2 = 7 \\ x_1 + 5x_3 = 9 \end{cases}$
c) $\begin{cases} x_1 + 2x_2 + 2x_4 = 6 \\ 3x_1 + 5x_2 - x_3 + 6x_4 = 17 \\ 2x_1 + 4x_2 + x_3 + 2x_4 = 12 \\ 2x_1 - 7x_3 + 11x_4 = 7 \end{cases}$
d) $\begin{cases} x_1 - 4x_2 - x_3 + x_4 = 3 \\ 2x_1 - 8x_2 + x_3 - 4x_4 = 9 \\ -x_1 + 4x_2 - 2x_3 + 5x_4 = -6 \end{cases}$
e) $\begin{cases} x_1 + 2x_2 - x_3 + 3x_4 = 2 \\ 2x_1 + 4x_2 - x_3 + 6x_4 = 5 \\ x_2 + 2x_4 = 3 \end{cases}$
f) $\begin{cases} 2x_1 - 2x_2 - x_3 + 6x_4 - 2x_5 = 1 \\ x_1 - x_2 + x_3 + 2x_4 - x_5 = 2 \\ 4x_1 - 4x_2 + 5x_3 + 7x_4 - x_5 = 6 \end{cases}$
g) $\begin{cases} 3x_1 - x_2 + x_3 - x_4 + 2x_5 = 5 \\ x_1 - x_2 - x_3 - 2x_4 - x_5 = 2 \\ 5x_1 - 2x_2 + x_3 - 3x_4 + 3x_5 = 10 \\ 2x_1 - x_2 - 2x_4 + x_5 = 5 \end{cases}$
求解这些系统。以向量形式写出答案。

2.2. 找到向量方程 $x_1 \mathbf{v}_1 + x_2 \mathbf{v}_2 + x_3 \mathbf{v}_3 = \mathbf{0}$ 的所有解,其中 $\mathbf{v}_1 = (1, 1, 0)^T$, $\mathbf{v}_2 = (0, 1, 1)^T$ 和 $\mathbf{v}_3 = (1, 0, 1)^T$。你能从这里得出关于向量系统 $\mathbf{v}_1, \mathbf{v}_2, \mathbf{v}_3$ 线性无关(依赖)的什么结论?


\section{主元的分析}

关于解的存在性和唯一性的所有问题都可以通过分析增广矩阵在阶梯形(简化阶梯形)中的主元来回答。

首先,让我们研究一下方程 $A \mathbf{x} = \mathbf{b}$ \textbf{不一致}(即它没有解)的情况。答案立即得出,如果我们仅仅思考一下:当增广矩阵的阶梯形中最后一个列有一个主元时,线性方程组才是不一致的(没有解),即增广矩阵的阶梯形包含一行 $(0 \ 0 \ \dots \ 0 \ | \ b)$,其中 $b \neq 0$。实际上,这样的行对应于方程 $0 x_1 + 0 x_2 + \dots + 0 x_n = b \neq 0$,它没有解。

如果我们没有这样的行,我们只需将其化为简化阶梯形,然后从中读出解。现在,还有三个陈述。所有这些陈述都只涉及\textbf{系数矩阵},而不是系统的增广矩阵。
1. 解(如果存在)是唯一的,当且仅当没有自由变量,也就是说,当且仅当系数矩阵的阶梯形在每一列都有一个主元;
2. 方程 $A \mathbf{x} = \mathbf{b}$ 对于所有右侧 $b$ 都是一致的,当且仅当系数矩阵的阶梯形在每一行都有一个主元。
3. 方程 $A \mathbf{x} = \mathbf{b}$ 对于任意右侧 $b$ \textbf{都有唯一解},当且仅当系数矩阵 $A$ 的阶梯形在每一列和每一行都有一个主元。

第一个陈述是微不足道的,因为自由变量负责所有的不唯一性。我应该只强调这个陈述\textbf{并不说明}关于存在性的任何信息。

第二个陈述稍微复杂一些。如果我们有一个系数矩阵 $A$ 的阶梯形中的主元在每一行,那么我们不可能在\textbf{增广}矩阵的最后一列中有一个主元,所以系统总是有一致的解,无论右侧 $b$ 是什么。让我们证明,如果我们有一个系数矩阵 $A$ 的阶梯形中的零行,那么我们可以选择一个右侧 $b$ 使得系统 $A \mathbf{x} = \mathbf{b}$ 不一致。设 $A_e$ 是系数矩阵 $A$ 的阶梯形。那么 $A_e = EA$,其中 $E$ 是对应于行运算的初等矩阵的乘积,$E = E_N \dots E_2 E_1$。如果 $A_e$ 有一个零行,那么最后一行也是零。因此,如果我们取 $b_e = (0, \dots, 0, 1)^T$(所有项都是 0,除了最后一个是 1),那么方程 $A_e \mathbf{x} = b_e$ 没有解。从左边乘以 $E^{-1}$ 并回忆 $E^{-1} A_e = A$,我们得到方程 $A \mathbf{x} = E^{-1} b_e$ 没有解。

最后,陈述 3 直接从陈述 1 和 2 得出。

上述对主元的分析带来了几个非常重要的推论。我们使用的主要观察是:在阶梯形中,每一行和每一列最多有一个主元(它可以有 0 个主元)。

\textbf{3.1. 关于线性无关和基的推论。维数。}

关于向量系统在 $F^n$ 中是否是基、线性无关或生成系统的问题,都可以通过行约简轻松回答。

\textbf{命题 3.1。} 假设我们有一个向量系统 $\mathbf{v}_1, \mathbf{v}_2, \dots, \mathbf{v}_m \in F^n$,并且令 $A = [\mathbf{v}_1, \mathbf{v}_2, \dots, \mathbf{v}_m]$ 是以 $\mathbf{v}_1, \mathbf{v}_2, \dots, \mathbf{v}_m$ 为列的 $n \times m$ 矩阵。那么
1. 系统 $\mathbf{v}_1, \mathbf{v}_2, \dots, \mathbf{v}_m$ 线性无关,当且仅当 $A$ 的阶梯形在每一列都有一个主元;
2. 系统 $\mathbf{v}_1, \mathbf{v}_2, \dots, \mathbf{v}_m$ 是 $F^n$ 中的完备(生成)集,当且仅当 $A$ 的阶梯形在每一行都有一个主元;
3. 系统 $\mathbf{v}_1, \mathbf{v}_2, \dots, \mathbf{v}_m$ 是 $F^n$ 中的基,当且仅当 $A$ 的阶梯形在每一列和每一行都有一个主元。

\textbf{证明}~ 向量系统 $\mathbf{v}_1, \mathbf{v}_2, \dots, \mathbf{v}_m \in F^n$ 线性无关,当且仅当方程 $\mathbf{x}_1 \mathbf{v}_1 + \mathbf{x}_2 \mathbf{v}_2 + \dots + \mathbf{x}_m \mathbf{v}_m = \mathbf{0}$ 只有唯一的(平凡)解 $\mathbf{x}_1 = \mathbf{x}_2 = \dots = \mathbf{x}_m = 0$,或者等价地说,方程 $A \mathbf{x} = \mathbf{0}$ 有唯一解 $\mathbf{x} = \mathbf{0}$。根据上面的陈述 1,这发生在当且仅当矩阵的主元在每一列时。类似地,系统 $\mathbf{v}_1, \mathbf{v}_2, \dots, \mathbf{v}_m \in F^n$ 是 $F^n$ 中的完备集,当且仅当方程 $\mathbf{x}_1 \mathbf{v}_1 + \mathbf{x}_2 \mathbf{v}_2 + \dots + \mathbf{x}_m \mathbf{v}_m = \mathbf{b}$ 对任何右侧 $\mathbf{b} \in F^n$ 都有解。根据上面的陈述 2,这发生在当且仅当 $A$ 的矩阵的阶梯形在每一行都有一个主元。最后,系统 $\mathbf{v}_1, \mathbf{v}_2, \dots, \mathbf{v}_m \in F^n$ 是 $F^n$ 中的基,当且仅当方程 $\mathbf{x}_1 \mathbf{v}_1 + \mathbf{x}_2 \mathbf{v}_2 + \dots + \mathbf{x}_m \mathbf{v}_m = \mathbf{b}$ 对任何右侧 $\mathbf{b} \in F^n$ 都有唯一解。根据陈述 3,这发生在当且仅当 $A$ 的阶梯形在每一列和每一行都有一个主元。

\textbf{命题 3.2。} $F^n$ 中的任何线性无关系统不能包含超过 $n$ 个向量。

\textbf{证明}~ 设系统 $\mathbf{v}_1, \mathbf{v}_2, \dots, \mathbf{v}_m \in F^n$ 是线性无关的,并且设 $A = [\mathbf{v}_1, \mathbf{v}_2, \dots, \mathbf{v}_m]$ 是以 $\mathbf{v}_1, \mathbf{v}_2, \dots, \mathbf{v}_m$ 为列的 $n \times m$ 矩阵。根据命题 3.1, $A$ 的阶梯形必须在每一列都有一个主元,这在 $m > n$ 时是不可能的(主元数量不能超过行数)。

\textbf{命题 3.3。} 向量空间 $V$ 中的任何两个基具有相同数量的向量。

\textbf{证明}~ 设 $\mathbf{v}_1, \mathbf{v}_2, \dots, \mathbf{v}_n$ 和 $\mathbf{w}_1, \mathbf{w}_2, \dots, \mathbf{w}_m$ 是 $V$ 中的两个不同的基。不失一般性,我们假设 $n \le m$。考虑一个同构 $A: F^n \to V$,定义为 $A e_k = \mathbf{v}_k$, $k = 1, 2, \dots, n$,其中 $e_1, e_2, \dots, e_n$ 是 $\mathbb{R}^n$ 的标准基。由于 $A^{-1}$ 也是一个同构,系统 $A^{-1} \mathbf{w}_1, A^{-1} \mathbf{w}_2, \dots, A^{-1} \mathbf{w}_m$ 是一个基(见第 1 章定理 6.6)。所以它是线性无关的,根据命题 3.2, $m \le n$。结合假设 $n \le m$,我们得到 $m=n$。

上述命题的一个特例是以下命题。

\textbf{命题 3.4。} $F^n$ 中的任何基必须恰好有 $n$ 个向量。

\textbf{证明}~ 这个事实直接源于前面的命题,但也有一个直接的证明。设 $\mathbf{v}_1, \mathbf{v}_2, \dots, \mathbf{v}_m$ 是 $F^n$ 中的一个基,令 $A$ 为以 $\mathbf{v}_1, \mathbf{v}_2, \dots, \mathbf{v}_m$ 为列的 $n \times m$ 矩阵。系统是基的事实意味着方程 $A \mathbf{x} = \mathbf{b}$ 对任何(所有可能的)右侧 $b$ 都有唯一解。存在性意味着在(简化)阶梯形矩阵的每一行中都有一个主元,因此主元数量恰好是 $n$。唯一性意味着在系数矩阵(的阶梯形)的每一列中都有一个主元,所以 $m =$ 列数 $=$ 主元数 $= n$。

\textbf{命题 3.5。} $F^n$ 中的任何生成集必须至少有 $n$ 个向量。

\textbf{证明}~ 设 $\mathbf{v}_1, \mathbf{v}_2, \dots, \mathbf{v}_m$ 是 $F^n$ 中的完备集,令 $A$ 为以 $\mathbf{v}_1, \mathbf{v}_2, \dots, \mathbf{v}_m$ 为列的 $n \times m$ 矩阵。命题 3.1 的陈述 2 暗示 $A$ 的阶梯形在每一行都有一个主元。由于主元数量不能超过列的数量,所以 $n \le m$。

\textbf{3.2. 可逆矩阵的推论。}

\textbf{命题 3.6。} 一个矩阵 $A$ 是可逆的,当且仅当它的阶梯形在每一列和每一行都有一个主元。

\textbf{证明}~ 正如我们在本节开头所讨论的,方程 $A \mathbf{x} = \mathbf{b}$ 对于任何右侧 $b$ 都有唯一解,当且仅当 $A$ 的阶梯形在每一行和每一列都有一个主元。但是,我们知道(见第 1 章第 6.4 节的推论 6.9),一个矩阵是可逆的当且仅当它的列在(见推论 6.9,第 1 章 6.4 节)$F^m$ 中构成一个基。命题 3.4 以上表明,这发生在当且仅当在每一行和每一列都有一个主元时。

上述命题立即蕴含了以下推论。

\textbf{推论 3.7。} 可逆矩阵\textbf{必须}是方阵 ($n \times n$)。

\textbf{命题 3.8。} 如果一个方阵 ($n \times n$) $A$ 是左可逆的,或者它是右可逆的,那么它就是可逆的。换句话说,要检查方阵 $A$ 的可逆性,只需检查 $A A^{-1} = I$ 或 $A^{-1} A = I$ 其中一个条件即可。注意,这个命题仅适用于方阵!

\textbf{证明}~ 我们知道,矩阵 $A$ 是可逆的当且仅当方程 $A \mathbf{x} = \mathbf{b}$ 对于任何右侧 $b$ 都有唯一解。这发生在当且仅当 $A$ 的阶梯形在每一行和每一列都有一个主元。

如果矩阵 $A$ 是左可逆的,那么方程 $A \mathbf{x} = \mathbf{0}$ 有唯一解 $\mathbf{x} = \mathbf{0}$。实际上,如果 $B$ 是 $A$ 的左逆(即 $BA = I$),并且 $\mathbf{x}$ 满足 $A \mathbf{x} = \mathbf{0}$,那么从左边将这个恒等式乘以 $B$,我们得到 $x = 0$,所以解是唯一的。因此,$A$ 的阶梯形在每一列都有一个主元(没有自由变量)。如果矩阵 $A$ 是方的 ($n \times n$),那么阶梯形也在每一行都有一个主元($n$ 个主元,而一行最多有一个主元),所以矩阵是可逆的。

如果矩阵 $A$ 是右可逆的,并且 $C$ 是它的右逆 ($AC = I$),那么对于 $x = C b$, $b \in F^n$, $A x = AC b = I b = b$。因此,对于任何右侧 $b$,方程 $A \mathbf{x} = \mathbf{b}$ 都有解 $x = C b$。因此,$A$ 的阶梯形在每一行都有一个主元。如果 $A$ 是方的,那么它也在每一列都有一个主元,所以 $A$ 是可逆的。

\textbf{练习}~

3.1. 对于 $b$ 的哪个值,系统 $\begin{pmatrix} 1 & 2 & 2 \\ 2 & 4 & 6 \\ 1 & 2 & 3 \end{pmatrix} \mathbf{x} = \begin{pmatrix} 1 \\ 4 \\ b \end{pmatrix}$ 有解。对于该值 $b$,找到系统的通解。

3.2. 确定向量 $\begin{pmatrix} 1 \\ 1 \\ 0 \\ 0 \end{pmatrix}$, $\begin{pmatrix} 1 \\ 0 \\ 1 \\ 0 \end{pmatrix}$, $\begin{pmatrix} 0 \\ 0 \\ 1 \\ 1 \end{pmatrix}$, $\begin{pmatrix} 0 \\ 1 \\ 0 \\ 1 \end{pmatrix}$ 是否线性无关或相关。这四个向量是否张成 $\mathbb{R}^4$?(换句话说,它们是生成系统吗?)对于 $\mathbb{C}^4$ 呢?

3.3. 确定以下向量系统是否是 $\mathbb{R}^3$ 的基:
a) $(1, 2, -1)^T$, $(1, 0, 2)^T$, $(2, 1, 1)^T$;
b) $(-1, 3, 2)^T$, $(-3, 1, 3)^T$, $(2, 10, 2)^T$;
c) $(67, 13, -47)^T$, $(\pi, -7.84, 0)^T$, $(3, 0, 0)^T$。
哪个系统是 $\mathbb{C}^3$ 的基?

3.4. 多项式 $t^3 + 2t$, $t^2 + t + 1$, $t^3 + 5$ 是否生成(张成)$P_3$?给出你的理由。

3.5. $F^4$ 中的 5 个向量可能线性无关吗?给出你的理由。

3.6. 证明或反驳:如果一个方阵($n \times n$)$A$ 的列是线性无关的,那么 $A^2 = AA$ 的列也是线性无关的。

3.7. 证明或反驳:如果一个方阵($n \times n$)$A$ 的列是线性无关的,那么 $A^3 = AAA$ 的行也是线性无关的。

3.8. 证明方程 $A \mathbf{x} = \mathbf{0}$ 只有唯一解(即,如果 $A$ 的阶梯形在每一列都有一个主元),那么 $A$ 是左可逆的。提示:初等矩阵可能会有帮助。

\textbf{注释:} 这可能是一个非常难的问题,因为它需要对主题有深入的理解。但是,当你理解了该做什么之后,问题就变得几乎微不足道了。

3.9. 一个矩阵的简化阶梯形是唯一的吗?给出你的结论的理由。也就是说,假设通过执行某些行运算(不一定遵循任何算法)我们得到一个简化阶梯形矩阵。我们总是得到相同的矩阵,还是可能得到不同的矩阵?请注意,我们只允许执行行运算,“列运算”是被禁止的。提示:如果以可逆矩阵开始,会发生什么?此外,主元是否总是在相同的列中,还是取决于你执行的行运算?如果你能在不诉诸行运算的情况下告诉主元列是什么,那么主元列的位置就不依赖于它们。


\section{通过行约简求 $A^{-1}$}

正如我们在上面讨论的,可逆矩阵必须是方阵,并且其阶梯形在每一行和每一列都必须有主元。因此,可逆矩阵的简化阶梯形是单位矩阵 $I$。因此,任何可逆矩阵都可通过行约简(即通过行运算)化为单位矩阵。现在让我们陈述一个简单的算法来找到一个 $n \times n$ 矩阵的逆:
1. 通过在 $A$ 的右侧写入 $n \times n$ 单位矩阵来形成一个 $n \times 2n$ 的\textbf{增广}矩阵 $(A | I)$;
2. 对增广矩阵执行行运算将 $A$ 转化为单位矩阵 $I$;
3. $I$ 的那个将被自动转化为 $A^{-1}$;
4. 如果通过行运算无法将 $A$ 转化为单位矩阵,则 $A$ 是不可逆的。

以上算法有几个可能的解释。第一个,一个朴素的解释是这样的:我们知道(对于可逆矩阵 $A$),向量 $A^{-1} b$ 是方程 $A \mathbf{x} = \mathbf{b}$ 的解。所以,要找到 $A^{-1}$ 的第 $k$ 列,我们需要找到 $A \mathbf{x} = e_k$ 的解,其中 $e_1, e_2, \dots, e_n$ 是 $\mathbb{R}^n$ 的标准基。上述算法只是同时求解方程 $A \mathbf{x} = e_k$, $k = 1, 2, \dots, n$!

我们也提供另一种,更“高级”的解释。正如我们上面讨论的,每个行运算都可以通过左乘一个初等矩阵来实现。设 $E_1, E_2, \dots, E_N$ 是对应于我们执行的行运算的初等矩阵,令 $E = E_N \dots E_2 E_1$ 为它们的乘积。$^1$ 我们知道行运算将 $A$ 转化为单位矩阵,即 $EA = I$,所以 $E = A^{-1}$。但是,相同的行运算将增广矩阵 $(A | I)$ 转化为 $(EA | E) = (I | A^{-1})$。

$^1$ 虽然在这里并不重要,但请注意,如果行运算 $E_1$ 先执行,那么 $E_1$ 必须是乘积中最右边的项。

这个“高级”解释利用初等矩阵蕴含了一个重要的命题,该命题将在以后经常使用。

\textbf{定理 4.1。} 任何可逆矩阵都可以表示为初等矩阵的乘积。

\textbf{证明}~ 正如我们在上一段中所讨论的,$A^{-1} = E_N \dots E_2 E_1$,所以 $A = (A^{-1})^{-1} = (E_N \dots E_2 E_1)^{-1} = E_1^{-1} E_2^{-1} \dots E_N^{-1} I = E_1^{-1} E_2^{-1} \dots E_N^{-1}$(初等矩阵的逆也是初等矩阵)。

\textbf{一个例子。} 假设我们想找到矩阵
$$
\begin{pmatrix} 1 & 4 & -2 \\ -2 & -7 & 7 \\ 3 & 11 & -6 \end{pmatrix}
$$
的逆。将其与单位矩阵增广,并进行行约简,我们得到
$$
\begin{pmatrix} 1 & 4 & -2 & | & 1 & 0 & 0 \\ -2 & -7 & 7 & | & 0 & 1 & 0 \\ 3 & 11 & -6 & | & 0 & 0 & 1 \end{pmatrix} \xrightarrow[{-3R_1}]{+2R_1} \begin{pmatrix} 1 & 4 & -2 & | & 1 & 0 & 0 \\ 0 & 1 & 3 & | & 2 & 1 & 0 \\ 0 & -1 & 0 & | & -3 & 0 & 1 \end{pmatrix} \xrightarrow{+R_2} \begin{pmatrix} 1 & 4 & -2 & | & 1 & 0 & 0 \\ 0 & 1 & 3 & | & 2 & 1 & 0 \\ 0 & 0 & 3 & | & -1 & 1 & 1 \end{pmatrix}
$$
在最后一步行运算中,我们将第一行乘以 3 以避免在向后阶段的行约简中出现分数。
$$
\xrightarrow{\times 3} \begin{pmatrix} 3 & 12 & -6 & | & 3 & 0 & 0 \\ 0 & 1 & 3 & | & 2 & 1 & 0 \\ 0 & 0 & 3 & | & -1 & 1 & 1 \end{pmatrix} \xrightarrow[-2R_3]{+2R_3} \begin{pmatrix} 3 & 12 & 0 & | & 1 & 2 & 2 \\ 0 & 1 & 0 & | & 3 & 0 & -1 \\ 0 & 0 & 3 & | & -1 & 1 & 1 \end{pmatrix} \xrightarrow{-12R_2} \begin{pmatrix} 3 & 0 & 0 & | & -35 & 2 & 14 \\ 0 & 1 & 0 & | & 3 & 0 & -1 \\ 0 & 0 & 3 & | & -1 & 1 & 1 \end{pmatrix}
$$
将第一行和最后一行除以 3,我们得到逆矩阵
$$
\begin{pmatrix} -35/3 & 2/3 & 14/3 \\ 3 & 0 & -1 \\ -1/3 & 1/3 & 1/3 \end{pmatrix}
$$

\textbf{练习}~

4.1. 找到以下矩阵的逆:$\begin{pmatrix} 1 & 2 & 1 \\ 3 & 7 & 3 \\ 2 & 3 & 4 \end{pmatrix}$, $\begin{pmatrix} 1 & -1 & 2 \\ 1 & 1 & -2 \\ 1 & 1 & 4 \end{pmatrix}$。显示所有步骤。


\section{维数~有限维空间}

\begin{definition} 向量空间 $V$ 的\textbf{维数} $\dim V$ 是基中向量的数量。对于仅由零向量 $\mathbf{0}$ 组成的向量空间,我们设 $\dim V = 0$。如果 $V$ 没有(有限)基,我们设 $\dim V = \infty$。如果 $\dim V$ 是有限的,我们称空间 $V$ 为\textbf{有限维};否则,我们称其为\textbf{无限维}。
\end{definition}

命题 3.3 表明维数是良好定义的,即它不依赖于基的选择。

第 1 章第 2.8 节的命题表明,有限维向量空间中的任何有限生成集都包含一个基。这直接蕴含了以下命题。

\textbf{命题 5.1。} 向量空间 $V$ 是有限维的,当且仅当它有一个有限生成集。

假设我们有一个有限维向量空间中的向量系统,并且我们想检查它是否是基(或者它是否线性无关,或者是否完备)?可能是最简单的方法是使用同构 $A: V \to \mathbb{R}^n$, $n = \dim E$ 将问题转移到 $\mathbb{R}^n$,在 $\mathbb{R}^n$ 中,所有这些问题都可以通过行约简(研究主元)来回答。请注意,如果 $\dim V = n$,那么总存在一个同构 $A: V \to \mathbb{R}^n$。实际上,如果 $\dim V = n$,则存在一个基 $\mathbf{v}_1, \mathbf{v}_2, \dots, \mathbf{v}_n \in V$,并且可以定义一个同构 $A: V \to \mathbb{R}^n$ 为 $A e_k = \mathbf{v}_k$, $k = 1, 2, \dots, n$。

例如,让我们给出以下两个命题 3.2 和 3.5 的推论:

\textbf{命题 5.2。} 有限维向量空间 $V$ 中的任何线性无关系统不能包含超过 $\dim V$ 个向量。

\textbf{证明}~ 设 $\mathbf{v}_1, \mathbf{v}_2, \dots, \mathbf{v}_m \in V$ 是线性无关系统,令 $A$ 为以 $\mathbf{v}_1, \mathbf{v}_2, \dots, \mathbf{v}_m$ 为列的 $n \times m$ 矩阵。那么 $A \mathbf{v}_1, A \mathbf{v}_2, \dots, A \mathbf{v}_m$ 是 $\mathbb{R}^n$ 中的线性无关系统,根据命题 3.2, $m \le n$。

\textbf{命题 5.3。} $V$ 中的任何生成系统必须至少有 $\dim V$ 个向量。

\textbf{证明}~ 设 $\mathbf{v}_1, \mathbf{v}_2, \dots, \mathbf{v}_m \in V$ 是 $V$ 中的完备系统,令 $A$ 为以 $\mathbf{v}_1, \mathbf{v}_2, \dots, \mathbf{v}_m$ 为列的 $n \times m$ 矩阵。命题 3.1 的陈述 2 暗示 $A$ 的阶梯形在每一行都有一个主元。由于主元数量不能超过列的数量,所以 $m \ge n$。

\textbf{5.1. 将线性无关集补全为基。}

以下陈述将在后面扮演重要角色。

\textbf{命题 5.4(补全为基)。} 有限维空间中的线性无关向量系统可以补全为基,即,给定有限维向量空间 $V$ 中的线性无关向量 $\mathbf{v}_1, \mathbf{v}_2, \dots, \mathbf{v}_r$,可以找到向量 $\mathbf{v}_{r+1}, \mathbf{v}_{r+2}, \dots, \mathbf{v}_n$ 使得向量系统 $\mathbf{v}_1, \mathbf{v}_2, \dots, \mathbf{v}_n$ 是 $V$ 中的一个基。

\textbf{证明}~ 设 $\dim V = n$。选择一个不属于 $\text{span}\{\mathbf{v}_1, \mathbf{v}_2, \dots, \mathbf{v}_r\}$ 的向量并称之为 $\mathbf{v}_{r+1}$(由于系统 $\mathbf{v}_1, \mathbf{v}_2, \dots, \mathbf{v}_r$ 不是生成的,总能做到这一点)。根据第 1 章练习 2.5,系统 $\mathbf{v}_1, \dots, \mathbf{v}_r, \mathbf{v}_{r+1}$ 是线性无关的(注意在这种情况下 $r < n$,根据命题 5.2)。用新向量 $\mathbf{v}_{r+2}$ 重复这个过程,依此类推。我们将停止这个过程,直到得到一个生成系统。注意,这个过程不能无限进行,因为向量空间 $V$ 中的线性无关向量系统不能包含超过 $n = \dim V$ 个向量。

\textbf{5.2. 有限维空间的子空间。}

\textbf{定理 5.5。} 设 $V$ 是 $W$ 的一个子空间,$\dim W < \infty$。那么 $V$ 是有限维的,并且 $\dim V \le \dim W$。此外,如果 $\dim V = \dim W$,则 $V = W$(这里我们仍然假设 $V$ 是 $W$ 的子空间)。

\textbf{注释}~ 这个定理看起来像是一个平凡的论断,就像是命题 5.2 的一个容易的推论。但是,我们只能在已经有了 $V$ 的基的情况下才能应用命题 5.2。我们只有 $W$ 的基,而无法知道这个基中的多少向量属于 $V$;实际上,很容易构造一个例子,其中 $W$ 的基向量中没有一个属于 $V$。

\textbf{定理 5.5 的证明。} 如果 $V = \{\mathbf{0}\}$,那么定理是微不足道的,所以我们假设否则。我们想在 $V$ 中找到一个基。取一个非零向量 $\mathbf{v}_1 \in V$。如果 $V = \text{span}\{\mathbf{v}_1\}$,我们就找到了基(由单个向量 $\mathbf{v}_1$ 组成)。如果不是,我们继续归纳。假设我们已经构造了 $r$ 个线性无关向量 $\mathbf{v}_1, \dots, \mathbf{v}_r \in V$。如果 $V = \text{span}\{\mathbf{v}_k : 1 \le k \le r\}$,那么我们已经找到了 $V$ 中的一个基。如果不是,则存在一个向量 $\mathbf{v}_{r+1} \in V$, $\mathbf{v}_{r+1} \notin \text{span}\{\mathbf{v}_k : 1 \le k \le r\}$。根据第 1 章练习 2.5,系统 $\mathbf{v}_1, \dots, \mathbf{v}_r, \mathbf{v}_{r+1}$ 是线性无关的(注意在这种情况下 $r < n$,根据命题 5.2)。重复这个过程,用新向量 $\mathbf{v}_{r+2}$,依此类推。我们将停止这个过程,直到得到一个生成系统。注意,这个过程不能无限进行,因为向量空间 $V$ 中的线性无关向量系统不能包含超过 $n = \dim V$ 个向量。


\textbf{练习}~

5.1. 对错题:
a) 任何由有限集生成的向量空间都有基;
b) 任何向量空间都有(有限)基;
c) 一个向量空间不能有多个基;
d) 如果一个向量空间有有限基,那么所有基中的向量数量是相同的;
e) $P_n$ 的维数是 $n$;
f) $M_{m \times n}$ 的维数是 $m+n$;
g) 如果向量 $\mathbf{v}_1, \mathbf{v}_2, \dots, \mathbf{v}_n$ 生成(张成)向量空间 $V$,那么 $V$ 中的每个向量都可以唯一地表示为向量 $\mathbf{v}_1, \mathbf{v}_2, \dots, \mathbf{v}_n$ 的线性组合;
h) 任何有限维空间的子空间都是有限维的;
i) 如果向量空间 $V$ 的维数是 $n$,那么 $V$ 只有一个零维子空间和一个 $n$ 维子空间。

5.2. 证明如果 $V$ 是一个 $n$ 维向量空间,那么 $V$ 中的向量系统 $\mathbf{v}_1, \mathbf{v}_2, \dots, \mathbf{v}_n$ 是线性独立的当且仅当它张成 $V$。

5.3. 证明 $V$ 中的线性无关向量系统 $\mathbf{v}_1, \mathbf{v}_2, \dots, \mathbf{v}_n$ 是基当且仅当 $n = \dim V$。

5.4. (重温一个旧问题:现在这个问题应该很容易了)向量 $\mathbf{v}_1, \mathbf{v}_2, \mathbf{v}_3$ 是否可能是线性相关的,而向量 $\mathbf{w}_1 = \mathbf{v}_1 + \mathbf{v}_2$, $\mathbf{w}_2 = \mathbf{v}_2 + \mathbf{v}_3$ 和 $\mathbf{w}_3 = \mathbf{v}_3 + \mathbf{v}_1$ 是线性独立的?提示:向量空间 $\text{span}(\mathbf{v}_1, \mathbf{v}_2, \mathbf{v}_3)$ 可以是什么维数?

5.5. 设 $\mathbf{u}, \mathbf{v}, \mathbf{w}$ 是 $V$ 中的一个基。证明 $\mathbf{u}+\mathbf{v}+\mathbf{w}$, $\mathbf{v}+\mathbf{w}$, $\mathbf{w}$ 也是 $V$ 中的一个基。

5.6. 在 $\mathbb{R}^5$ 空间中考虑向量 $\mathbf{v}_1 = (2, -1, 1, 5, -3)^T$, $\mathbf{v}_2 = (3, -2, 0, 0, 0)^T$, $\mathbf{v}_3 = (1, 1, 50, -921, 0)^T$。
a) 证明这些向量是线性无关的。
b) 将此向量系统补全为基。如果你先做了 b) 部分,你可以完全不进行计算。


\section{线性方程组的通解}

在本节中,我们将讨论线性系统(所有解,即解集)的通解的结构。

我们称一个系统 $A \mathbf{x} = \mathbf{b}$ 为\textbf{齐次}的,如果右侧 $b = 0$,即齐次系统是 $A \mathbf{x} = \mathbf{0}$ 的形式。对于每个系统 $A \mathbf{x} = \mathbf{b}$,我们可以关联一个齐次系统,只需将 $b$ 设置为 $0$。

\textbf{定理 6.1(线性方程的通解)。} 设向量 $\mathbf{x}_1$ 满足方程 $A \mathbf{x} = \mathbf{b}$,设 $H$ 是相关齐次系统 $A \mathbf{x} = \mathbf{0}$ 的所有解的集合。那么集合 $\{\mathbf{x} = \mathbf{x}_1 + \mathbf{x}_h : \mathbf{x}_h \in H\}$ 是方程 $A \mathbf{x} = \mathbf{b}$ 的所有解的集合。

换句话说,这个定理可以陈述为:
$A \mathbf{x} = \mathbf{b}$ 的通解 = $A \mathbf{x} = \mathbf{b}$ 的一个特解 + $A \mathbf{x} = \mathbf{0}$ 的通解。

\textbf{定理 6.1 的证明。} 固定一个满足 $A \mathbf{x}_1 = \mathbf{b}$ 的向量 $\mathbf{x}_1$。设向量 $\mathbf{x}_h$ 满足 $A \mathbf{x}_h = \mathbf{0}$。那么对于 $\mathbf{x} = \mathbf{x}_1 + \mathbf{x}_h$,我们有 $A \mathbf{x} = A(\mathbf{x}_1 + \mathbf{x}_h) = A \mathbf{x}_1 + A \mathbf{x}_h = \mathbf{b} + \mathbf{0} = \mathbf{b}$,所以任何形式为 $\mathbf{x} = \mathbf{x}_1 + \mathbf{x}_h$, $\mathbf{x}_h \in H$ 的 $\mathbf{x}$ 都是 $A \mathbf{x} = \mathbf{b}$ 的解。现在设 $\mathbf{x}$ 满足 $A \mathbf{x} = \mathbf{b}$。那么对于 $\mathbf{x}_h := \mathbf{x} - \mathbf{x}_1$,我们得到 $A \mathbf{x}_h = A(\mathbf{x} - \mathbf{x}_1) = A \mathbf{x} - A \mathbf{x}_1 = \mathbf{b} - \mathbf{b} = \mathbf{0}$,所以 $\mathbf{x}_h \in H$。因此,$A \mathbf{x} = \mathbf{b}$ 的任何解都可以表示为 $\mathbf{x} = \mathbf{x}_1 + \mathbf{x}_h$,其中某个 $\mathbf{x}_h \in H$。

这个定理的威力在于它的普遍性。它适用于所有线性方程,我们不需要在这里假设向量空间是有限维的。你将在微分方程、积分方程、偏微分方程等领域遇到这个定理。

除了展示解集结构之外,这个定理还允许我们将唯一性与存在性的研究分开。也就是说,为了研究唯一性,我们只需要分析齐次方程 $A \mathbf{x} = \mathbf{0}$ 的唯一性,它总是有一个解。

在本书的这个例子中,我们考虑了系统
$$
\begin{pmatrix} 2 & 3 & 1 & 4 & -9 \\ 1 & 1 & 1 & 1 & -3 \\ 1 & 1 & 1 & 2 & -5 \\ 2 & 2 & 2 & 3 & -8 \end{pmatrix} \mathbf{x} = \begin{pmatrix} 17 \\ 6 \\ 8 \\ 14 \end{pmatrix} \quad (6.1)
$$
通过行约简,可以找到这个系统的解 $(6.1)$
$$
\mathbf{x} = \begin{pmatrix} 3 \\ 1 \\ 0 \\ 2 \\ 0 \end{pmatrix} + x_3 \begin{pmatrix} -2 \\ 1 \\ 1 \\ 0 \\ 0 \end{pmatrix} + x_5 \begin{pmatrix} 2 \\ -1 \\ 0 \\ 2 \\ 1 \end{pmatrix}, \quad x_3, x_5 \in F
$$
参数 $x_3, x_5$ 在这里可以记作其他字母,例如 $t$ 和 $s$;我们在这里保留符号 $x_3$ 和 $x_5$ 仅仅是为了提醒我们参数来自相应的自由变量。

现在,假设我们只是得到了这个解,并且我们想检查它是否正确。当然,我们可以重复行运算,但这太耗时了。而且,如果解是通过某种非标准方法得到的,它可能看起来与我们从行约简得到的结果不同。例如,公式 $(6.2)$
$$
\mathbf{x} = \begin{pmatrix} 3 \\ 1 \\ 0 \\ 2 \\ 0 \end{pmatrix} + s \begin{pmatrix} -2 \\ 1 \\ 1 \\ 0 \\ 0 \end{pmatrix} + t \begin{pmatrix} 0 \\ 0 \\ 1 \\ 2 \\ 1 \end{pmatrix}, \quad s, t \in F
$$
给出与 $(6.1)$ 相同的集合(你能说为什么吗?);这里我们只是将 $(6.1)$ 中的最后一个向量替换为它的和加上第二个向量。所以,这个公式与我们从行约简得到的解不同,但它仍然是正确的。检查 $(6.1)$ 和 $(6.2)$ 是否给出正确的解的最简单方法是检查第一个向量(3, 1, 0, 2, 0)$^T$ 是否满足方程 $A \mathbf{x} = \mathbf{b}$,而其他两个向量(带参数的向量,$x_3$ 或 $s$ 和 $t$ 在前面)应该满足相关的齐次方程 $A \mathbf{x} = \mathbf{0}$。如果这检查出来了,我们就能确保由 $(6.1)$ 或 $(6.2)$ 定义的任何向量 $\mathbf{x}$ 确实是一个解。

请注意,这种检查解的方法并不能保证 $(6.1)$(或 $(6.2)$)给出所有解。例如,如果我们只是通过某种方式遗漏了 $x_3$ 相关的项,上述方法仍然能正常工作。

因此,我们如何保证我们没有遗漏任何自由变量,并且不需要额外的项 $(6.1)$?我们想到的就是再次计算主元的数量。在这个例子中,如果进行行运算,主元的数量是 3。所以确实应该有 2 个自由变量,而且看起来我们没有遗漏 $(6.1)$ 中的任何内容。为了能够\textbf{证明}这一点,我们将需要基本子空间和矩阵秩的新概念。

我还需要提到,在上面的例子中,人们不必进行所有行运算就能检查出只有 2 个自由变量,并且公式 $(6.1)$ 和 $(6.2)$ 都给出了正确的通解。

\textbf{练习}~

6.1. 对错题:
a) 任何线性方程组都有至少一个解;
b) 任何线性方程组最多有一个解;
c) 任何齐次线性方程组至少有一个解;
d) $n$ 个未知数的 $n$ 个线性方程组至少有一个解;
e) $n$ 个未知数的 $n$ 个线性方程组最多有一个解;
f) 如果与给定线性方程组相对应的齐次方程组有解,则给定方程组有解;
g) 如果 $n$ 个未知数的 $n$ 个齐次线性方程组的系数矩阵是可逆的,那么该系统没有非零解;
h) $m$ 个方程 $n$ 个未知数的任何线性方程组的解集是 $\mathbb{R}^n$ 中的一个子空间;
i) 任何齐次线性方程组的解集 $m$ 个方程 $n$ 个未知数是 $\mathbb{R}^n$ 中的一个子空间。

6.2. 找到一个 $2 \times 3$ 系统(3 个未知数的 2 个方程),使得其通解具有形式 $\begin{pmatrix} 1 \\ 1 \\ 0 \end{pmatrix} + s \begin{pmatrix} 1 \\ 2 \\ 1 \end{pmatrix}$, $s \in \mathbb{R}$。


\section{矩阵的基本子空间~秩}

正如我们在第 1 章第 7 节中所讨论的,任何线性变换 $A: V \to W$ 都可以关联两个子空间,即它的核,或零空间 $\text{Ker } A = \text{Null } A := \{ \mathbf{v} \in V : A \mathbf{v} = \mathbf{0} \} \subset V$,以及它的像空间 $\text{Ran } A = \{ \mathbf{w} \in W : \mathbf{w} = A \mathbf{v} \text{ 对于某个 } \mathbf{v} \in V \}$ $\subset W$。

换句话说,核 $\text{Ker } A$ 是齐次方程 $A \mathbf{x} = \mathbf{0}$ 的解集,而像空间 $\text{Ran } A$ 正是方程 $A \mathbf{x} = \mathbf{b}$ 有解的所有右侧 $b \in W$ 的集合。如果 $A$ 是一个 $m \times n$ 矩阵,即从 $F^n$ 到 $F^m$ 的一个线性变换,那么回忆矩阵乘法的“按列坐标规则”,我们可以看到任何向量 $w \in \text{Ran } A$ 都可以表示为 $A$ 的列的线性组合。这解释了为什么\textbf{列空间}(表示为 $\text{Col } A$)这个术语经常用来表示矩阵的像空间。因此,对于矩阵 $A$,符号 $\text{Col } A$ 通常用于代替 $\text{Ran } A$。

如果 $A$ 是一个矩阵,那么除了 $\text{Ran } A$ 和 $\text{Ker } A$ 之外,我们还可以考虑转置矩阵 $A^T$ 的像空间和核。通常\textbf{行空间}用于表示 $\text{Ran } A^T$,而\textbf{左零空间}用于表示 $\text{Ker } A^T$(但通常没有特殊的符号)。四个子空间 $\text{Ran } A$, $\text{Ker } A$, $\text{Ran } A^T$, $\text{Ker } A^T$ 称为矩阵 $A$ 的\textbf{基本子空间}。

在本节中,我们将研究它们的维度之间重要的关系。我们需要以下定义,这是线性代数的基本概念之一。

\begin{definition} 给定一个线性变换(矩阵)$A$,它的\textbf{秩},$\text{rank } A$,是它的像空间的维数:
$$
\text{rank } A := \dim \text{Ran } A
$$
\end{definition}

\textbf{7.1. 计算基本子空间和秩。}

要计算矩阵的基本子空间和秩,需要进行行约简。也就是说,设 $A$ 是矩阵,设 $A_e$ 是其阶梯形:
1. $A$ 的\textbf{主元列}(即行约简后将有主元的列)给出了 $\text{Ran } A$ 的一个基(它是众多可能的基之一)。
2. 阶梯形 $A_e$ 的\textbf{主元行}给出了行空间的基。当然,也可以简单地转置矩阵,然后进行行约简。但是,如果我们已经有了 $A$ 的阶梯形,例如计算 $\text{Ran } A$ 时,那么我们就能免费得到 $\text{Ran } A^T$。
3. 要找到零空间 $\text{Ker } A$ 的基,需要求解齐次方程 $A \mathbf{x} = \mathbf{0}$:具体细节将从下面的例子中看出。

\textbf{示例}~ 考虑矩阵
$$
\begin{pmatrix}
1 & 1 & 2 & 2 & 1 \\
2 & 2 & 1 & 1 & 1 \\
3 & 3 & 3 & 3 & 2 \\
1 & 1 & -1 & -1 & 0
\end{pmatrix}
$$
进行行运算我们得到阶梯形
$$
\begin{pmatrix}
1 & 1 & 2 & 2 & 1 \\
0 & 0 & -3 & -3 & -1 \\
0 & 0 & 0 & 0 & 0 \\
0 & 0 & 0 & 0 & 0
\end{pmatrix}
$$
(这里主元已加框)。因此,\textbf{原始矩阵}的第 1 列和第 3 列,即列向量
$$
\begin{pmatrix} 1 \\ 2 \\ 3 \\ 1 \end{pmatrix}, \quad \begin{pmatrix} 2 \\ 1 \\ 3 \\ -1 \end{pmatrix}
$$
给出了 $\text{Ran } A$ 的一个基。我们也免费得到了行空间 $\text{Ran } A^T$ 的基:$A$ 的\textbf{阶梯形}的第一行和第二行,即向量
$$
\begin{pmatrix} 1 \\ 1 \\ 2 \\ 2 \\ 1 \end{pmatrix}, \quad \begin{pmatrix} 0 \\ 0 \\ -3 \\ -3 \\ -1 \end{pmatrix}
$$
(这里我们将向量垂直放置。向量是放在这里作为列,还是作为行,这真的只是一个约定问题。我们将其垂直放置的原因是,尽管我们称 $\text{Ran } A^T$ 为\textbf{行空间},但我们将其定义为 $A^T$ 的列空间)。

为了计算零空间 $\text{Ker } A$ 的基,我们需要求解方程 $A \mathbf{x} = \mathbf{0}$。计算 $A$ 的\textbf{简化}阶梯形,在这个例子中是
$$
\begin{pmatrix}
1 & 1 & 0 & 0 & 1/3 \\
0 & 0 & 1 & 1 & 1/3 \\
0 & 0 & 0 & 0 & 0 \\
0 & 0 & 0 & 0 & 0
\end{pmatrix}
$$
注意,在求解齐次方程 $A \mathbf{x} = \mathbf{0}$ 时,不必写出整个增广矩阵,只处理系数矩阵就足够了。实际上,在这种情况下,增广矩阵的最后一列是零列,它在行运算下不会改变。所以,我们可以在不实际写出它的情况下,只在心中记住这一列。在心中保留这个最后的零列,我们可以从上面的简化阶梯形中读出解:
$$
\begin{cases}
x_1 = -x_2 - \frac{1}{3} x_5, & x_2 \text{ 是自由变量} \\
x_3 = -x_4 - \frac{1}{3} x_5 & x_4 \text{ 是自由变量} \\
x_5 & \text{是自由变量}
\end{cases}
$$
或者,在向量形式下:
$$
\mathbf{x} = \begin{pmatrix} -x_2 - \frac{1}{3} x_5 \\ x_2 \\ -x_4 - \frac{1}{3} x_5 \\ x_4 \\ x_5 \end{pmatrix} = x_2 \begin{pmatrix} -1 \\ 1 \\ 0 \\ 0 \\ 0 \end{pmatrix} + x_4 \begin{pmatrix} 0 \\ 0 \\ -1 \\ 1 \\ 0 \end{pmatrix} + x_5 \begin{pmatrix} -1/3 \\ 0 \\ -1/3 \\ 0 \\ 1 \end{pmatrix}, \quad x_2, x_4, x_5 \in F
$$
向量在每个自由变量处,即在我们的例子中,向量
$$
\begin{pmatrix} -1 \\ 1 \\ 0 \\ 0 \\ 0 \end{pmatrix}, \quad \begin{pmatrix} 0 \\ 0 \\ -1 \\ 1 \\ 0 \end{pmatrix}, \quad \begin{pmatrix} -1/3 \\ 0 \\ -1/3 \\ 0 \\ 1 \end{pmatrix}
$$
构成 $\text{Ker } A$ 的一个基。不幸的是,没有捷径可以找到 $\text{Ker } A^T$ 的基,必须求解方程 $A^T \mathbf{x} = \mathbf{0}$。知道 $A$ 的阶梯形在此处无济于事。

\textbf{7.2. 解释计算基本子空间和秩的计算方法。}

那么,为什么上述方法确实给出了基本子空间的基呢?
\textbf{7.2.1. 零空间 $\text{Ker } A$。}
零空间 $\text{Ker } A$ 的情况可能是最简单的:由于我们求解了方程 $A \mathbf{x} = \mathbf{0}$,即找到了所有解,那么 $\text{Ker } A$ 中的任何向量都是我们得到的向量的线性组合。因此,我们得到的向量构成了 $\text{Ker } A$ 中的一个生成系统。要看到系统是线性无关的,让我们将每个向量乘以相应的自由变量并将所有向量相加,见 (7.1)。那么对于每个自由变量 $x_k$,结果向量的第 $k$ 个分量恰好是 $x_k$,再次见 (7.1),所以这个向量(线性组合)是 $\mathbf{0}$ 的唯一方式是当所有自由变量都为 0 时。

\textbf{7.2.2. 列空间 $\text{Ran } A$。}
让我们现在解释为什么用于找到列空间 $\text{Ran } A$ 的基的方法有效。首先,注意到 $A_{re}$ 的\textbf{主元列}($A$ 的简化阶梯形)构成了 $\text{Ran } A_{re}$ 的基(不是原始矩阵的列空间,而是其简化阶梯形列空间的基!)。由于行运算只是可逆矩阵的左乘,它们不改变线性无关性。因此,\textbf{原始矩阵} $A$ 的主元列是线性无关的。让我们现在证明 $A$ 的主元列张成 $A$ 的列空间。设 $v_1, v_2, \dots, v_r$ 是 $A$ 的主元列,设 $v$ 是 $A$ 的任意一列。我们想证明 $v$ 可以表示为主元列 $v_1, v_2, \dots, v_r$ 的线性组合, $v = \alpha_1 v_1 + \alpha_2 v_2 + \dots + \alpha_r v_r$。

$A_{re}$ 是通过左乘 $A_{re} = EA$ 从 $A$ 得到的,其中 $E$ 是初等矩阵的乘积,所以 $E$ 是可逆矩阵。向量 $E v_1, E v_2, \dots, E v_r$ 是 $A_{re}$ 的主元列,而 $A$ 的列 $v$ 被变换为 $A_{re}$ 的列 $E v$。由于 $A_{re}$ 的主元列构成了 $\text{Ran } A_{re}$ 的基,向量 $E v$ 可以表示为线性组合 $E v = \alpha_1 E v_1 + \alpha_2 E v_2 + \dots + \alpha_r E v_r$。将这个等式从左边乘以 $E^{-1}$,我们得到表示 $v = \alpha_1 v_1 + \alpha_2 v_2 + \dots + \alpha_r v_r$,因此 $A$ 的主元列确实张成了 $\text{Ran } A$。

\textbf{7.2.3. 行空间 $\text{Ran } A^T$。}
可以很容易地看出,$A_e$ 的\textbf{主元行}是线性无关的。实际上,设 $w_1, w_2, \dots, w_r$ 是 $A_e$ 的转置(因为我们同意总是将向量垂直放置)的主元行。假设 $\alpha_1 w_1 + \alpha_2 w_2 + \dots + \alpha_r w_r = 0$。考虑 $w_1$ 的第一个非零项。由于对于所有其他向量 $w_2, w_3, \dots, w_r$,相应的项等于 0(根据阶梯形的定义),我们可以得出 $\alpha_1 = 0$。所以我们可以简单地忽略和中的第一项。现在考虑 $w_2$ 的第一个非零项。向量 $w_3, \dots, w_r$ 的相应项为 0,所以 $\alpha_2 = 0$。重复这个过程,我们得到 $\alpha_k = 0 \ \forall k = 1, 2, \dots, r$。

要证明向量 $w_1, w_2, \dots, w_r$ 张成了行空间,人们可以注意到“行运算不改变行空间”。这可以从直接分析行运算得到,但我们在这里提供一种更正式的方式来演示这个事实。对于变换 $A$ 和集合 $X$,我们用 $A(X)$ 表示所有可以表示为 $y = A(x)$, $x \in X$ 的元素 $y$ 的集合,$A(X) := \{ y = A(x) : x \in X \}$。

如果 $A$ 是一个 $m \times n$ 矩阵, $A_e$ 是它的阶梯形,$A_e$ 是通过左乘 $A_e = EA$ 得到的,其中 $E$ 是一个 $m \times m$ 可逆矩阵(对应初等矩阵的乘积)。那么 $\text{Ran } A_e^T = \text{Ran}(A^T E^T) = A^T(\text{Ran } E^T) = A^T(\mathbb{R}^m) = \text{Ran } A^T$,所以确实 $\text{Ran } A^T = \text{Ran } A_e^T$。

\textbf{7.3. 秩定理。基本子空间的维数。}

在许多应用中,我们需要找到列空间或零空间的基。例如,正如上面所显示的,求解齐次方程 $A \mathbf{x} = \mathbf{0}$ 等价于找到零空间 $\text{Ker } A$ 的基。找到列空间的基意味着从生成集中提取基,通过移除不必要的向量(列)。然而,基本子空间计算方法最重要的应用是它们维度之间的关系。

\textbf{定理 7.1(秩定理)。} 对于矩阵 $A$,$\text{rank } A = \text{rank } A^T$。

这个定理通常表述为:矩阵的\textbf{列秩}等于它的\textbf{行秩}。这个定理的证明是微不足道的,因为 $\text{Ran } A$ 和 $\text{Ran } A^T$ 的维数都等于 $A$ 的阶梯形中的主元数量。

以下定理为我们提供了基本子空间维数之间重要的关系。它通常也称为秩定理。

\textbf{定理 7.2。} 设 $A$ 是一个 $m \times n$ 矩阵,即从 $F^n$ 到 $F^m$ 的线性变换。那么
1. $\dim \text{Ker } A + \dim \text{Ran } A = \dim \text{Ker } A + \text{rank } A = n$($A$ 的定义域的维数);
2. $\dim \text{Ker } A^T + \dim \text{Ran } A^T = \dim \text{Ker } A^T + \text{rank } A^T = \dim \text{Ker } A^T + \text{rank } A = m$($A$ 的目标空间的维数);

\textbf{证明}~ 证明,在上述计算基本子空间基的方法的意义下,几乎是微不足道的。第一个陈述仅仅是自由变量的数量($\dim \text{Ker } A$)加上基本变量的数量(即主元数量,即 $\text{rank } A$)等于列的数量(即等于 $n$)。

第二个陈述,考虑到 $\text{rank } A = \text{rank } A^T$,仅仅是将第一个陈述应用于 $A^T$。

上述定理的一个应用是,让我们回顾一下第 1 章第 6 节的例子。在那里,我们考虑了系统
$$
\begin{pmatrix} 2 & 3 & 1 & 4 & -9 \\ 1 & 1 & 1 & 1 & -3 \\ 1 & 1 & 1 & 2 & -5 \\ 2 & 2 & 2 & 3 & -8 \end{pmatrix} \mathbf{x} = \begin{pmatrix} 17 \\ 6 \\ 8 \\ 14 \end{pmatrix}
$$
并且我们声称它的通解由
$$
\mathbf{x} = \begin{pmatrix} 3 \\ 1 \\ 0 \\ 2 \\ 0 \end{pmatrix} + x_3 \begin{pmatrix} -2 \\ 1 \\ 1 \\ 0 \\ 0 \end{pmatrix} + x_5 \begin{pmatrix} 2 \\ -1 \\ 0 \\ 2 \\ 1 \end{pmatrix}, \quad x_3, x_5 \in F
$$
或者由
$$
\mathbf{x} = \begin{pmatrix} 3 \\ 1 \\ 0 \\ 2 \\ 0 \end{pmatrix} + s \begin{pmatrix} -2 \\ 1 \\ 1 \\ 0 \\ 0 \end{pmatrix} + t \begin{pmatrix} 0 \\ 0 \\ 1 \\ 2 \\ 1 \end{pmatrix}, \quad s, t \in F
$$
给出。我们在第 6 节中检查了由任一公式给出的向量 $\mathbf{x}$ 确实是方程的解。但是,我们如何保证 $(6.1)$(或 $(6.2)$)中的任何一个公式都描述了\textbf{所有}解?首先,我们知道在任一公式中,最后两个向量(被参数乘的向量)都属于 $\text{Ker } A$。很容易看出,在任一情况下,这两个向量都是线性无关的(两个向量线性相关当且仅当其中一个是一个标量倍的另一个)。现在,让我们计算维数:将第一行和第二行交换,并进行第一轮行运算
$$
\begin{pmatrix} 1 & 1 & 1 & 1 & -3 \\ 2 & 3 & 1 & 4 & -9 \\ 1 & 1 & 1 & 2 & -5 \\ 2 & 2 & 2 & 3 & -8 \end{pmatrix} \xrightarrow[{-R_1}]{-2R_1} \begin{pmatrix} 1 & 1 & 1 & 1 & -3 \\ 0 & 1 & -1 & 2 & -3 \\ 0 & 0 & 0 & 1 & -2 \\ 0 & 0 & 0 & 1 & -2 \end{pmatrix}
$$
我们看到已经有三个主元了,所以 $\text{rank } A \ge 3$。(实际上,我们可以已经看出秩是 3,但这里只需要估计量)。根据定理 7.2, $\text{rank } A + \dim \text{Ker } A = 5$,因此 $\dim \text{Ker } A \le 2$,所以 $\text{Ker } A$ 中不能有超过 2 个线性无关向量。因此,任一公式中的最后 2 个向量构成了 $\text{Ker } A$ 的基,所以任一公式都给出了方程的所有解。

秩定理的一个重要推论是以下将存在性和唯一性联系起来的关于线性方程的定理。

\textbf{定理 7.3。} 设 $A$ 是一个 $m \times n$ 矩阵。那么方程 $A \mathbf{x} = \mathbf{b}$ 对于每一个 $b \in \mathbb{R}^m$ 都有解,当且仅当对偶方程 $A^T \mathbf{x} = \mathbf{0}$ 只有唯一的(仅平凡)解。(注意,在第二个方程中我们有 $A^T$,而不是 $A$)。

\textbf{证明}~ 证明直接从定理 7.2 得出,只需计算维数即可。我们将细节留给读者作为练习。

上述定理有一个很好的几何解释。也就是说,陈述 1 表明,如果一个变换 $A: F^n \to F^m$ 具有平凡核(Ker $A = \{0\}$),那么定义域 $F^n$ 和像空间 $\text{Ran } A$ 的维数相等。如果核不是平凡的,那么变换“杀死”了 $\dim \text{Ker } A$ 个维数,所以 $\dim \text{Ran } A = n - \dim \text{Ker } A$。

\textbf{7.4. 线性无关系统的补全为基。}

正如上面第 5 节的命题 5.4 所断言的,任何线性无关系统都可以补全为基,即,给定有限维向量空间 $V$ 中的线性无关向量 $\mathbf{v}_1, \mathbf{v}_2, \dots, \mathbf{v}_r$,可以找到向量 $\mathbf{v}_{r+1}, \mathbf{v}_{r+2}, \dots, \mathbf{v}_n$ 使得向量系统 $\mathbf{v}_1, \mathbf{v}_2, \dots, \mathbf{v}_n$ 是 $V$ 中的一个基。理论上,这个命题的证明为我们提供了寻找向量 $\mathbf{v}_{r+1}, \mathbf{v}_{r+2}, \dots, \mathbf{v}_n$ 的算法,但这个算法看起来不太实用。本节的思想为我们提供了一种更实用的补全为基的方法。首先,注意到如果一个 $m \times n$ 矩阵处于阶梯形,那么它的非零行(它们显然是线性无关的)可以很容易地补全为 $F^n$ 中的基;我们只需要在适当的位置添加一些行,使得结果矩阵仍然是阶梯形并且在每一列都有主元。然后,新矩阵的非零行构成一个基,我们可以按任何我们想要的顺序排列它们,因为作为基的性质不依赖于顺序。

假设现在我们有线性无关向量 $\mathbf{v}_1, \mathbf{v}_2, \dots, \mathbf{v}_r$, $\mathbf{v}_k \in F^n$。考虑以 $\mathbf{v}_1^T, \mathbf{v}_2^T, \dots, \mathbf{v}_r^T$ 为行的矩阵 $A$,并执行行运算得到阶梯形 $A_e$。正如我们上面所讨论的,$A_e$ 的行可以很容易地补全为 $F^n$ 中的基;我们只需要在适当的位置添加行,使得结果矩阵仍然是阶梯形并且在每一列都有主元。然后,新矩阵的行构成一个基,我们可以按任何我们想要的顺序排列它们,因为作为基的性质不依赖于顺序。

\textbf{注释}~ 上面描述的补全为基的方法可能不是最简单的方法,但其主要优点之一是它适用于任意域上的向量空间。


\textbf{练习}~

7.1. 对错题:
a) 矩阵的秩等于其非零列的数量;
b) $m \times n$ 零矩阵是唯一秩为 0 的 $m \times n$ 矩阵;
c) 初等行运算保持秩;
d) 初等列运算不一定保持秩;
e) 矩阵的秩等于矩阵中线性无关列的最大数量;
f) 矩阵的秩等于矩阵中线性无关行的最大数量;
g) $n \times n$ 矩阵的秩最多为 $n$;
h) 秩为 $n$ 的 $n \times n$ 矩阵是可逆的。

7.2. 一个 $54 \times 37$ 的矩阵秩为 31。所有 4 个基本子空间的维数是多少?

7.3. 计算矩阵 $\begin{pmatrix} 1 & 1 & 0 \\ 0 & 1 & 1 \\ 1 & 1 & 0 \end{pmatrix}$, $\begin{pmatrix} 1 & 2 & 3 & 1 & 1 \\ 1 & 4 & 0 & 1 & 2 \\ 0 & 2 & -3 & 0 & 1 \\ 1 & 0 & 0 & 0 & 0 \end{pmatrix}$ 的秩和所有四个基本子空间的基。

7.4. 证明如果 $A: X \to Y$ 并且 $V$ 是 $X$ 的子空间,那么 $\dim AV \le \text{rank } A$。(这里 $AV$ 表示变换 $A$ 变换后的子空间 $V$,即 $AV$ 中的任何向量都可以表示为 $A \mathbf{v}$, $\mathbf{v} \in V$)。由此推导出 $\text{rank}(AB) \le \text{rank } A$。

\textbf{注释:} 这里你可以利用 $V \subset W$ 则 $\dim V \le \dim W$ 的事实。你知道这是为什么吗?

7.5. 证明如果 $A: X \to Y$ 并且 $V$ 是 $X$ 的子空间,那么 $\dim AV \le \dim V$。由此推导出 $\text{rank}(AB) \le \text{rank } B$。

7.6. 证明如果两个 $n \times n$ 矩阵 $A$ 和 $B$ 的乘积 $AB$ 是可逆的,那么 $A$ 和 $B$ 都是可逆的。即使你知道行列式,也不要使用它,我们还没有涉及它。提示:使用前两个问题。

7.7. 证明如果 $A \mathbf{x} = \mathbf{0}$ 只有唯一解,那么方程 $A^T \mathbf{x} = \mathbf{b}$ 对于每个右侧 $b$ 都有解。提示:计算主元。

7.8. 构造一个具有所需性质的矩阵,或解释为什么不存在这样的矩阵:
a) 列空间包含 $(1, 0, 0)^T$, $(0, 0, 1)^T$,行空间包含 $(1, 1)^T$, $(1, 2)^T$;
b) 列空间由 $(1, 1, 1)^T$ 张成,零空间由 $(1, 2, 3)^T$ 张成;
c) 列空间是 $\mathbb{R}^4$,行空间是 $\mathbb{R}^3$。
提示:首先检查维数是否匹配。

7.9. 如果 $A$ 和 $B$ 具有相同的四个基本子空间,那么 $A = B$ 吗?

7.10. 将以下向量的行补全为 $\mathbb{R}^7$ 的基:$\begin{pmatrix} e^3 \\ 3 \\ 4 \\ 0 \\ -\pi \\ 6 \\ -2 \end{pmatrix}$, $\begin{pmatrix} 0 \\ 0 \\ 2 \\ -1 \\ \pi \\ e \\ 1 \end{pmatrix}$, $\begin{pmatrix} 0 \\ 0 \\ 0 \\ 0 \\ 3 \\ -3 \\ 2 \end{pmatrix}$, $\begin{pmatrix} 0 \\ 0 \\ 0 \\ 0 \\ 0 \\ 0 \\ 1 \end{pmatrix}$。

7.11. 对于矩阵 $A = \begin{pmatrix} 1 & 2 & -1 & 2 & 3 \\ 2 & 2 & 1 & 5 & 5 \\ 3 & 6 & -3 & 0 & 24 \\ -1 & -4 & 4 & -7 & 11 \end{pmatrix}$,找到其列空间和行空间的基。

7.12. 对于上一问题的矩阵,将行空间的基补全为 $\mathbb{R}^5$ 的基。

7.13. 对于矩阵 $A = \begin{pmatrix} 1 & i \\ i & -1 \end{pmatrix}$,计算 $\text{Ran } A$ 和 $\text{Ker } A$。你能说出这些子空间之间的关系吗?

7.14. 对于实数矩阵 $A$,$\text{Ran } A = \text{Ker } A^T$ 是否可能?对于复数 $A$ 是否可能?

7.15. 将向量 $(1, 2, -1, 2, 3)^T$, $(2, 2, 1, 5, 5)^T$, $(-1, -4, 4, 7, -11)^T$ 补全为 $\mathbb{R}^5$ 的基。


\section{任意基下线性变换的表示~坐标变换公式}

我们在第 1 章中学到的关于线性变换及其矩阵的材料可以很容易地推广到有限基下的抽象向量空间中的变换。在本节中,我们将区分线性变换 $T$ 和它的矩阵,原因是我们将考虑不同的基,所以线性变换可以有不同的矩阵表示。

\textbf{8.1. 坐标向量。}

设 $V$ 是一个向量空间,具有基 $B := \{\mathbf{b}_1, \mathbf{v}_2, \dots, \mathbf{v}_n\}$。任何向量 $\mathbf{v} \in V$ 都可以唯一地表示为线性组合
$$
\mathbf{v} = x_1 \mathbf{b}_1 + x_2 \mathbf{b}_2 + \dots + x_n \mathbf{b}_n = \sum_{k=1}^n x_k \mathbf{b}_k
$$
系数 $x_1, x_2, \dots, x_n$ 称为向量 $\mathbf{v}$ 在基 $B$ 下的\textbf{坐标}。方便地将这些坐标组合成向量 $\mathbf{v}$ 相对于基 $B$ 的\textbf{坐标向量},这是一个列向量 $[\mathbf{v}]_B := \begin{pmatrix} x_1 \\ x_2 \\ \vdots \\ x_n \end{pmatrix} \in F^n$。

注意,映射 $\mathbf{v} \mapsto [\mathbf{v}]_B$ 是 $V$ 和 $F^n$ 之间的同构。它将基 $\mathbf{b}_1, \mathbf{v}_2, \dots, \mathbf{v}_n$ 映射到 $F^n$ 中的标准基 $e_1, e_2, \dots, e_n$。

\textbf{8.2. 线性变换的矩阵。}

设 $T: V \to W$ 是一个线性变换,设 $A := \{\mathbf{a}_1, \mathbf{a}_2, \dots, \mathbf{a}_n\}$, $B := \{\mathbf{b}_1, \mathbf{b}_2, \dots, \mathbf{b}_m\}$ 分别是 $V$ 和 $W$ 中的基。变换 $T$ 在基 $A$ 和 $B$ 下(或相对于基 $A$ 和 $B$)的矩阵是 $m \times n$ 矩阵,记作 $[T]_{BA}$,它关联了坐标向量 $[T \mathbf{v}]_B$ 和 $[\mathbf{v}]_A$:
$$
[T \mathbf{v}]_B = [T]_{BA} [\mathbf{v}]_A
$$
注意这里 $A$ 和 $B$ 符号的平衡:这是我们将第一个基 $A$ 放入第二个位置的原因。矩阵 $[T]_{BA}$ 很容易找到:它的第 $k$ 列就是坐标向量 $[T a_k]_B$(与从 $F^n$ 到 $F^m$ 的线性变换矩阵的查找方法进行比较!)。

正如在 $F^n$ 空间和标准基情况一样,线性变换的复合等价于它们的矩阵乘法:人们只需要在基方面稍微更小心。也就是说,设 $T_1: X \to Y$ 和 $T_2: Y \to Z$ 是线性变换,设 $A, B, C$ 分别是 $X, Y, Z$ 中的基。那么对于复合 $T = T_2 T_1$, $T: X \to Z$, $T \mathbf{x} := T_2(T_1(\mathbf{x}))$, 我们有
$$
[T]_{CA} = [T_2 T_1]_{CA} = [T_2]_{CB} [T_1]_{BA} \quad (8.1)
$$
(再次注意这里的索引平衡)。这个证明与 $F^n$ 空间和标准基情况下的证明完全相同,所以我们在此不再重复。另一种可能性是,通过坐标同构 $v \mapsto [v]_B$ 将所有内容传递到 $F^n$ 空间。然后人们就不需要任何证明了,一切都遵循矩阵乘法的相关结果。

\textbf{8.3. 坐标变换矩阵。}

设我们在向量空间 $V$ 中有两个基 $A = \{\mathbf{a}_1, \mathbf{a}_2, \dots, \mathbf{a}_n\}$ 和 $B = \{\mathbf{b}_1, \mathbf{v}_2, \dots, \mathbf{v}_n\}$。考虑恒等变换 $I = I_V$ 以及它在这些基下的矩阵 $[I]_{BA}$。根据定义 $[ \mathbf{v} ]_B = [I]_{BA} [\mathbf{v}]_A$, $\forall \mathbf{v} \in V$,也就是说,对于任何向量 $\mathbf{v} \in V$,矩阵 $[I]_{BA}$ 将其在基 $A$ 下的坐标变换为在基 $B$ 下的坐标。矩阵 $[I]_{BA}$ 通常称为\textbf{坐标变换}(从基 $A$ 到基 $B$)矩阵。矩阵 $[I]_{BA}$ 很容易计算:根据线性变换矩阵的一般查找规则,它的第 $k$ 列是第 $k$ 个基元素 $a_k$ 的坐标表示 $[a_k]_B$。

注意 $[I]_{AB} = ([I]_{BA})^{-1}$(这从矩阵乘法规则 (8.1) 中立即得出),所以任何坐标变换矩阵总是可逆的。

\textbf{8.3.1. 从标准基进行坐标变换的例子。}

设我们的空间 $V$ 是 $F^n$,并且我们有一个基 $B = \{\mathbf{b}_1, \mathbf{v}_2, \dots, \mathbf{v}_n\}$。我们还有标准基 $S = \{e_1, e_2, \dots, e_n\}$。从 $B$ 到 $S$ 的坐标变换矩阵 $[I]_{SB}$ 很容易计算:$[I]_{SB} = [b_1, b_2, \dots, b_n] =: B$,即它只是矩阵 $B$,其第 $k$ 列是向量(列)$v_k$。反向 $[I]_{BS} = [I]_{SB}^{-1} = B^{-1}$。

例如,考虑 $F^2$ 中的一个基 $B = \{ \begin{pmatrix} 1 \\ 2 \end{pmatrix}, \begin{pmatrix} 2 \\ 1 \end{pmatrix} \}$,设 $S$ 表示那里的标准基。那么 $[I]_{SB} = \begin{pmatrix} 1 & 2 \\ 2 & 1 \end{pmatrix} =: B$,并且 $[I]_{BS} = [I]_{SB}^{-1} = B^{-1} = \frac{1}{3} \begin{pmatrix} -1 & 2 \\ 2 & -1 \end{pmatrix}$(我们知道如何计算逆,并且也很容易验证上述矩阵确实是 $B$ 的逆)。

\textbf{8.3.2. 通过标准基进行变换的例子。}

在最多为 1 次的多项式空间中,我们还有基 $A = \{1, 1+t\}$, 和 $B = \{1+2t, 1-2t\}$, 并且我们想找到坐标变换矩阵 $[I]_{BA}$。当然,我们总是可以从基 $A$ 中取向量并尝试在基 $B$ 中分解它们;这涉及到求解线性系统,而我们知道如何做到这一点。然而,我认为以下方法更简单。

在 $P_1$ 中,我们还有标准基 $S = \{1, t\}$, 对于这个基 $[I]_{SA} = \begin{pmatrix} 1 & 1 \\ 0 & 1 \end{pmatrix} =: A$, $[I]_{SB} = \begin{pmatrix} 1 & 1 \\ 2 & -2 \end{pmatrix} =: B$, 并且取逆 $[I]_{AS} = A^{-1} = \begin{pmatrix} 1 & -1 \\ 0 & 1 \end{pmatrix}$, $[I]_{BS} = B^{-1} = \frac{1}{4} \begin{pmatrix} 2 & 1 \\ 2 & -1 \end{pmatrix}$。

那么 $[I]_{BA} = [I]_{BS} [I]_{SA} = B^{-1} A = \frac{1}{4} \begin{pmatrix} 2 & 1 \\ 2 & -1 \end{pmatrix} \begin{pmatrix} 1 & 1 \\ 0 & 1 \end{pmatrix}$,并且
$[I]_{AB} = [I]_{AS} [I]_{SB} = A^{-1} B = \begin{pmatrix} 1 & -1 \\ 0 & 1 \end{pmatrix} \begin{pmatrix} 1 & 1 \\ 2 & -2 \end{pmatrix}$。

\textbf{8.4. 变换的矩阵与坐标变换。}

设 $T: V \to W$ 是一个线性变换,设 $A, \tilde{A}$ 是 $V$ 中的两个基,设 $B, \tilde{B}$ 是 $W$ 中的两个基。假设我们知道矩阵 $[T]_{BA}$,并且我们想找到关于新基 $\tilde{A}, \tilde{B}$ 的矩阵表示,即矩阵 $[T]_{\tilde{B}\tilde{A}}$。规则非常简单:要得到“新”基下的矩阵,需要用坐标变换矩阵将“旧”基下的矩阵包围起来。我在这里没有提及应该将哪个坐标变换矩阵放在哪里,因为如果我们遵循索引平衡规则,我们别无选择。也就是说,线性变换的矩阵表示遵循以下公式:
$$
[T]_{\tilde{B}\tilde{A}} = [I]_{\tilde{B}B} [T]_{BA} [I]_{A\tilde{A}}
$$
请注意这里的索引平衡。
证明可以通过分析每个矩阵的作用来完成。

\textbf{8.5. 单个基的情况:相似矩阵。}

设 $V$ 是一个向量空间,设 $A = \{\mathbf{a}_1, \mathbf{a}_2, \dots, \mathbf{a}_n\}$ 是 $V$ 中的一个基。考虑一个线性变换 $T: V \to V$,并设 $[T]_{AA}$ 是它在该基下的矩阵(我们对“输入”和“输出”使用相同的基)。在这种情况下,使用更短的表示 $[T]_A$ 而不是 $[T]_{AA}$ 是非常普遍的。然而,两个索引表示 $[T]_{AA}$ 更适合索引平衡规则,所以我推荐使用它(或者至少在进行坐标变换时牢记它)。

设 $B = \{\mathbf{b}_1, \mathbf{b}_2, \dots, \mathbf{b}_n\}$ 是 $V$ 中的另一个基。根据上面的坐标变换规则 $[T]_{BB} = [I]_{BA} [T]_{AA} [I]_{AB}$。回忆 $[I]_{BA} = [I]_{AB}^{-1}$ 并记 $Q := [I]_{AB}$,我们可以将上述公式改写为 $[T]_{BB} = Q^{-1} [T]_{AA} Q$。这为以下定义提供了动机:

\begin{definition} 我们说矩阵 $A$ 与矩阵 $B$ \textbf{相似},如果存在一个可逆矩阵 $Q$ 使得 $A = Q^{-1} BQ$。
\end{definition}

从上面的推理可以立即看出,相似矩阵 $A$ 和 $B$ 必须是方阵且大小相同。如果 $A$ 与 $B$ 相似,即如果 $A = Q^{-1} BQ$,那么 $B = Q A Q^{-1} = (Q^{-1})^{-1} A (Q^{-1})$(因为 $Q^{-1}$ 是可逆的),因此 $B$ 与 $A$ 相似。所以,我们可以简单地说 $A$ 和 $B$ 是\textbf{相似}的。上述讨论表明,将 $Q$ 和 $Q^{-1}$ 放在哪里并不重要:人们可以使用公式 $A = QBQ^{-1}$ 来定义相似性。

\textbf{练习}~

8.1. 对错题:
a) 任何坐标变换矩阵都是方阵;
b) 任何坐标变换矩阵都是可逆的;
c) 如果矩阵 $A$ 和 $B$ 相似,那么 $B = Q^T A Q$ 对于某个矩阵 $Q$ 成立;
d) 如果矩阵 $A$ 和 $B$ 相似,那么 $B = Q^{-1} A Q$ 对于某个矩阵 $Q$ 成立;
e) 相似矩阵不一定是方阵。

8.2. 考虑向量系统 $(1, 2, 1, 1)^T$, $(0, 1, 3, 1)^T$, $(0, 3, 2, 0)^T$, $(0, 1, 0, 0)^T$。
a) 证明它们是 $F^4$ 中的一个基。尽量少做计算。
b) 找到将此基下的坐标变为 $F^4$ 中标准坐标(即标准基 $e_1, \dots, e_4$ 下的坐标)的坐标变换矩阵。

8.3. 找到将 $P_1$ 中的基 $1, 1+t$ 下的坐标变为基 $1-t, 2t$ 下的坐标的坐标变换矩阵。

8.4. 设 $T$ 是 $F^2$ 中的线性算子,定义为(在标准坐标下)$T(\begin{pmatrix} x \\ y \end{pmatrix}) = \begin{pmatrix} 3x + y \\ x - 2y \end{pmatrix}$。找到 $T$ 在标准基下的矩阵,以及在基 $(\begin{pmatrix} 1 \\ 1 \end{pmatrix}, \begin{pmatrix} 1 \\ 2 \end{pmatrix})$ 下的矩阵。

8.5. 证明,如果 $A$ 和 $B$ 相似,那么 $\text{trace } A = \text{trace } B$。提示:回忆 $\text{trace}(XY)$ 和 $\text{trace}(YX)$ 是如何关联的。

8.6. 矩阵 $\begin{pmatrix} 1 & 3 \\ 2 & 2 \end{pmatrix}$ 和 $\begin{pmatrix} 0 & 2 \\ 4 & 2 \end{pmatrix}$ 是否相似?请给出理由。




% 
\chapter{行列式}

\section{引言}

读者可能已经遇到过行列式,至少是在微积分或代数中遇到的 $2 \times 2$ 和 $3 \times 3$ 矩阵的行列式。对于 $2 \times 2$ 矩阵 $\begin{pmatrix} a & b \\ c & d \end{pmatrix}$,行列式就是 $ad - bc$;$3 \times 3$ 矩阵的行列式可以通过“大卫之星”法则找到。在本章中,我们想为 $n \times n$ 矩阵介绍行列式。我不想仅仅给出一个形式定义。首先我想给出一些动机,然后推导出行列式应具备的一些性质。然后,如果我们想要这些性质,我们就别无选择,只能得到行列式的几个等价定义。

从矩阵行列式开始,而不是从向量组的行列式开始,更为方便:这里没有真正的区别,因为我们可以始终将向量(列)连接在一起(比如作为列)来形成一个矩阵。设我们有 $\mathbb{R}^n$ 中的 $n$ 个向量 $\mathbf{v}_1, \mathbf{v}_2, \dots, \mathbf{v}_n$(注意向量的数量与维度一致),我们想找到由这些向量确定的平行六面体的\textbf{n维体积}。由向量 $\mathbf{v}_1, \mathbf{v}_2, \dots, \mathbf{v}_n$ 确定的平行六面体可以定义为所有向量 $\mathbf{v} \in \mathbb{R}^n$ 的集合,这些向量可以表示为 $\mathbf{v} = t_1 \mathbf{v}_1 + t_2 \mathbf{v}_2 + \dots + t_n \mathbf{v}_n$, $0 \le t_k \le 1 \ \forall k = 1, 2, \dots, n$。当 $n=2$(平行四边形)和 $n=3$(平行六面体)时,这很容易可视化。那么 $n$ 维体积是什么?在维度 1 中,它是长度。最后,让我们引入一些符号。对于向量组(列)$\mathbf{v}_1, \mathbf{v}_2, \dots, \mathbf{v}_n$,我们将把它的行列式(我们将要构造的)表示为 $D(\mathbf{v}_1, \mathbf{v}_2, \dots, \mathbf{v}_n)$。如果我们把这些向量连接成矩阵 $A$($A$ 的第 $k$ 列是 $\mathbf{v}_k$),那么我们将使用符号 $\det A$, $\det A = D(\mathbf{v}_1, \mathbf{v}_2, \dots, \mathbf{v}_n)$。

对于矩阵 $A = \begin{pmatrix} a_{1,1} & a_{1,2} & \dots & a_{1,n} \\ a_{2,1} & a_{2,2} & \dots & a_{2,n} \\ \vdots & \vdots & \ddots & \vdots \\ a_{n,1} & a_{n,2} & \dots & a_{n,n} \end{pmatrix}$,它的行列式也常常表示为
$$
\begin{vmatrix}
a_{1,1} & a_{1,2} & \dots & a_{1,n} \\
a_{2,1} & a_{2,2} & \dots & a_{2,n} \\
\vdots & \vdots & \ddots & \vdots \\
a_{n,1} & a_{n,2} & \dots & a_{n,n}
\end{vmatrix}
$$

\section{行列式应具备的性质}

我们知道,对于维度 2 和 3,“体积”平行六面体由“底乘以高”法则确定:如果我们选择一个向量,那么高是该向量到由其余向量张成的子空间的距离,底是其余向量确定的平行六面体的($n-1$ 维)体积。现在让我们将这个想法推广到更高维度。暂时我们不关心如何精确地确定高和底。我们将表明,如果我们假设高和底满足某些自然性质,那么我们就别无选择,行列式就被唯一确定了。

\textbf{2.1. 每个参数的线性。}

首先,如果我们把向量 $\mathbf{v}_1$ 乘以一个正数 $a$,那么高(即到线性张成 $L(\mathbf{v}_2, \dots, \mathbf{v}_n)$ 的距离)就会乘以 $a$。如果我们允许负高度(和负体积),那么这个性质对所有标量 $a$ 都成立,因此向量组 $\mathbf{v}_1, \mathbf{v}_2, \dots, \mathbf{v}_n$ 的行列式 $D(\mathbf{v}_1, \mathbf{v}_2, \dots, \mathbf{v}_n)$ 应该满足
$$
D(\alpha \mathbf{v}_1, \mathbf{v}_2, \dots, \mathbf{v}_n) = \alpha D(\mathbf{v}_1, \mathbf{v}_2, \dots, \mathbf{v}_n)
$$
当然,向量 $\mathbf{v}_1$ 没有什么特别之处,所以对于任何索引 $k$
$$
D(\mathbf{v}_1, \dots, \alpha \mathbf{v}_k, \dots, \mathbf{v}_n) = \alpha D(\mathbf{v}_1, \dots, \mathbf{v}_k, \dots, \mathbf{v}_n) \quad (2.1)
$$
为了得到下一个性质,让我们注意到如果我们相加两个向量,那么结果的“高度”应该等于被加数“高度”的总和,即
$$
D(\mathbf{v}_1, \dots, \mathbf{u}_k + \mathbf{v}_k, \dots, \mathbf{v}_n) = D(\mathbf{v}_1, \dots, \mathbf{u}_k, \dots, \mathbf{v}_n) + D(\mathbf{v}_1, \dots, \mathbf{v}_k, \dots, \mathbf{v}_n) \quad (2.2)
$$
换句话说,上述两个性质表明,行列式是\textbf{每个参数(向量)的线性},这意味着如果我们固定 $n-1$ 个向量并将剩余向量解释为一个变量(参数),我们就会得到一个线性函数。

\textbf{注释}~ 我们已经知道\textbf{线性}是一个非常有用的性质,在许多情况下都有帮助。因此,允许负高度(以及因此的负体积)是获得线性的一个非常小的代价,因为我们之后总是可以取绝对值。实际上,通过允许负高度,我们并没有牺牲任何东西!相反,我们甚至获得了一些东西,因为行列式的符号包含有关向量系统(方向)的一些信息。

\textbf{2.2. 在“列替换”下的保持不变性。}

下一个性质也看起来很自然。也就是说,如果我们取一个向量,比如 $\mathbf{v}_j$,并向其添加另一个向量 $\mathbf{v}_k$ 的倍数,“高度”不会改变,所以
$$
D(\mathbf{v}_1, \dots, \mathbf{v}_j + \alpha \mathbf{v}_k, \dots, \mathbf{v}_k, \dots, \mathbf{v}_n) = D(\mathbf{v}_1, \dots, \mathbf{v}_j, \dots, \mathbf{v}_k, \dots, \mathbf{v}_n) \quad (2.3)
$$
换句话说,如果我们应用第三种类型的\textbf{列运算},行列式不会改变。

\textbf{注释}~ 虽然在此并非必需,但让我们注意到第二个线性部分(性质 (2.2))不是独立的:它可以从性质 (2.1) 和 (2.3) 推导出来。我们将证明留作读者的练习。

\textbf{2.3. 反对称性。}

下一个性质是行列式应该具备的,即\textbf{关于任意两个参数的函数在交换任意两个参数时会改变符号,这类函数称为反对称函数}。

\textbf{函数具有多个变量的性质,在交换任意两个参数时会改变符号,这类函数称为反对称函数。}

也就是说,如果我们交换两个向量,行列式会改变符号:
$$
D(\mathbf{v}_1, \dots, \mathbf{v}_k, \dots, \mathbf{v}_j, \dots, \mathbf{v}_n) = - D(\mathbf{v}_1, \dots, \mathbf{v}_j, \dots, \mathbf{v}_k, \dots, \mathbf{v}_n) \quad (2.4)
$$
在第一眼看来,这个性质看起来不自然,但它可以从前面的性质推导出来。也就是说,三次应用性质 (2.3),然后使用 (2.1),我们得到
\begin{align*} D(\mathbf{v}_1, \dots, \mathbf{v}_j, \dots, \mathbf{v}_k, \dots, \mathbf{v}_n) &= D(\mathbf{v}_1, \dots, \mathbf{v}_j, \dots, \mathbf{v}_k - \mathbf{v}_j, \dots, \mathbf{v}_n) \\ &= D(\mathbf{v}_1, \dots, \mathbf{v}_j + (\mathbf{v}_k - \mathbf{v}_j), \dots, \mathbf{v}_k - \mathbf{v}_j, \dots, \mathbf{v}_n) \\ &= D(\mathbf{v}_1, \dots, \mathbf{v}_k, \dots, \mathbf{v}_k - \mathbf{v}_j, \dots, \mathbf{v}_n) \\ &= D(\mathbf{v}_1, \dots, \mathbf{v}_k, \dots, (\mathbf{v}_k - \mathbf{v}_j) - \mathbf{v}_k, \dots, \mathbf{v}_n) \\ &= D(\mathbf{v}_1, \dots, \mathbf{v}_k, \dots, -\mathbf{v}_j, \dots, \mathbf{v}_n) \\ &= - D(\mathbf{v}_1, \dots, \mathbf{v}_k, \dots, \mathbf{v}_j, \dots, \mathbf{v}_n) \end{align*}

\textbf{2.4. 归一化。}

最后一个性质是最简单的。对于 $\mathbb{R}^n$ 中的标准基 $\mathbf{e}_1, \mathbf{e}_2, \dots, \mathbf{e}_n$,对应的平行六面体是 $n$ 维单位立方体,所以
$$
D(\mathbf{e}_1, \mathbf{e}_2, \dots, \mathbf{e}_n) = 1
$$
在矩阵表示中,这可以写成 $\det I = 1$。


\section{行列式的构造}

现在的计划是:利用我们从第 2 节决定的行列式应具有的性质,我们推导出行列式的其他性质,其中一些性质非常不平凡。我们将展示如何使用这些性质通过我们熟悉的朋友——行约简来计算行列式。稍后,在第 4 节,我们将展示行列式,即具有所需性质的函数,是存在且唯一的。毕竟,我们必须确信我们正在计算和研究的对象是存在的。

虽然我们对行列式的动机及其性质的初始几何动机来自于考虑 $\mathbb{R}^n$ 中的向量,因此它们只与实数项的矩阵相关,但以下所有构造只使用代数运算(加法、乘法、除法)并且适用于具有复数项的矩阵,甚至适用于任意域上的项。

因此,在以下内容中,我们不仅为实数矩阵,也为复数矩阵(以及具有任意域项的矩阵)构造行列式。虽然我们最初的几何动机仅适用于实数情况,但在我们确定了行列式的性质(见本节的性质 1-3)之后,所有内容都适用于一般情况。

\textbf{3.1. 基本性质。}

在这一节中,我们将使用以下行列式性质:
1. 行列式在每个列中是线性的,即,在向量表示中,对于每个索引 $k$,
$$
D(\mathbf{v}_1, \dots, \alpha \mathbf{u}_k + \beta \mathbf{v}_k, \dots, \mathbf{v}_n) = \alpha D(\mathbf{v}_1, \dots, \mathbf{u}_k, \dots, \mathbf{v}_n) + \beta D(\mathbf{v}_1, \dots, \mathbf{v}_k, \dots, \mathbf{v}_n)
$$
对所有标量 $\alpha, \beta$ 成立。
2. 行列式是\textbf{反对称}的,即,如果我们交换两列,行列式改变符号。
3. 归一化性质:$\det I = 1$。

所有这些性质在第 2 节中都已讨论过。第一个性质只是 (2.1) 和 (2.2) 的组合。第二个是 (2.4),最后一个是归一化性质 (2.5)。注意,我们没有使用性质 (2.3):它可以从上述三个性质中推导出来。这三个性质完全定义了行列式!

\textbf{命题 3.1。} 对于方阵 $A$,以下陈述成立:
1. 如果 $A$ 有一个零列,那么 $\det A = 0$。
2. 如果 $A$ 有两列相等,那么 $\det A = 0$;
3. 如果 $A$ 的一列是另一列的倍数,那么 $\det A = 0$;
4. 如果 $A$ 的列是线性相关的,即如果矩阵不可逆,那么 $\det A = 0$。

\textbf{证明}~ 陈述 1 由线性性直接得出。如果我们用零乘以零列,我们不会改变矩阵及其行列式。但根据上面的性质 1,我们应该得到 0。行列式的反对称性蕴含了陈述 2。实际上,如果我们交换两列相等的列,我们什么也没改变,所以行列式保持不变。另一方面,交换两列改变了行列式的符号,所以 $\det A = -\det A$,这只有在 $\det A = 0$ 时才可能。陈述 3 是陈述 2 和线性性的直接推论。要证明最后一个陈述,让我们首先假设第一个向量 $\mathbf{v}_1$ 是其他向量的线性组合,$\mathbf{v}_1 = \alpha_2 \mathbf{v}_2 + \alpha_3 \mathbf{v}_3 + \dots + \alpha_n \mathbf{v}_n = \sum_{k=2}^n \alpha_k \mathbf{v}_k$。那么根据线性性,我们有(在向量表示中)
$$
D(\mathbf{v}_1, \mathbf{v}_2, \dots, \mathbf{v}_n) = D(\sum_{k=2}^n \alpha_k \mathbf{v}_k, \mathbf{v}_2, \dots, \mathbf{v}_n) = \sum_{k=2}^n \alpha_k D(\mathbf{v}_k, \mathbf{v}_2, \dots, \mathbf{v}_n)
$$
并且和中的每个行列式都为零,因为存在两个相等的列。现在考虑一般情况,即假设系统 $\mathbf{v}_1, \mathbf{v}_2, \dots, \mathbf{v}_n$ 是线性相关的。那么其中一个向量,比如 $\mathbf{v}_k$,可以表示为其他向量的线性组合。将此向量与 $\mathbf{v}_1$ 交换,我们得到我们刚刚处理过的情况,所以 $D(\mathbf{v}_1, \dots, \mathbf{v}_k, \dots, \mathbf{v}_n) = -D(\mathbf{v}_k, \dots, \mathbf{v}_1, \dots, \mathbf{v}_n) = -0 = 0$,所以这种情况下的行列式也为零。

下一个命题推广了性质 (2.3)。正如我们上面已经说过的,这个性质可以从我们本节中使用的三个“基本”性质中推导出来。

\textbf{命题 3.2。} 当我们向一列添加其他列的线性组合时,行列式不会改变(保持其他列不变)。特别是,行列式在“列替换”(第三类列运算)下保持不变。

\textbf{证明}~ 固定一个向量 $\mathbf{v}_k$,令 $\mathbf{u}$ 为其他向量的线性组合,$\mathbf{u} = \sum_{j \neq k} \alpha_j \mathbf{v}_j$。那么根据线性性
$$
D(\mathbf{v}_1, \dots, \mathbf{v}_k + \mathbf{u}, \dots, \mathbf{v}_n) = D(\mathbf{v}_1, \dots, \mathbf{v}_k, \dots, \mathbf{v}_n) + D(\mathbf{v}_1, \dots, \mathbf{u}, \dots, \mathbf{v}_n)
$$
并且根据命题 3.1,最后一项为零。

\textbf{3.3. 对角和三角矩阵的行列式。}

现在我们准备为一些重要的特殊矩阵类别计算行列式。第一类是所谓的\textbf{对角}矩阵。让我们回顾一下,一个方阵 $A = \{a_{j,k}\}_{n \times n}$ 称为\textbf{对角}矩阵,如果主对角线下的所有项都为零,即如果 $a_{j,k} = 0$ $\forall j \neq k$。我们将经常使用 $\text{diag}\{a_1, a_2, \dots, a_n\}$ 来表示对角矩阵:
$$
\begin{pmatrix}
a_1 & 0 & \dots & 0 \\
0 & a_2 & \dots & 0 \\
\vdots & \vdots & \ddots & \vdots \\
0 & 0 & \dots & a_n
\end{pmatrix}
$$
由于对角矩阵 $\text{diag}\{a_1, a_2, \dots, a_n\}$ 可以通过将第 $k$ 列乘以 $a_k$ 从单位矩阵 $I$ 得到,
\textbf{对角矩阵的行列式等于对角项的乘积,$\det(\text{diag}\{a_1, a_2, \dots, a_n\}) = a_1 a_2 \dots a_n$。}

下一个重要类别是所谓的\textbf{三角}矩阵。一个方阵 $A = \{a_{j,k}\}_{n \times n}$ 称为\textbf{上三角}矩阵,如果主对角线下的所有项都为零,即如果 $a_{j,k} = 0$ $\forall k < j$。一个方阵称为\textbf{下三角}矩阵,如果主对角线上的所有项都为零,即如果 $a_{j,k} = 0$ $\forall j < k$。我们称矩阵为\textbf{三角}矩阵,如果它是下三角或上三角矩阵。

很容易看出
\textbf{三角矩阵的行列式等于对角项的乘积,$\det A = a_{1,1} a_{2,2} \dots a_{n,n}$。}
实际上,如果一个三角矩阵的主对角线上有零,那么它就是不可逆的(这可以通过列运算很容易地检查出来),因此两边都等于零。如果所有对角项都非零,那么使用列替换(第三类列运算)可以将矩阵转化为具有相同对角项的对角矩阵:
对于上三角矩阵,首先应该从第 2、3、...、n 列减去第一列的适当倍数,“消去”第一行中的所有项,然后从第 3、...、n 列减去第二列的适当倍数,依此类推。

为了处理下三角矩阵的情况,需要从左到右进行“列约简”,即首先从最后一列开始,将适当倍数的最后一列从第 $n-1$, $\dots$, 2, 1 列减去,依此类推。

\textbf{3.4. 计算行列式。}

现在我们知道如何计算行列式,使用它们的性质:只需进行列约简(即对 $A^T$ 进行行约简),并跟踪改变行列式的列运算。幸运的是,最常使用的运算——行替换,即第三类运算,不会改变行列式。所以我们只需要跟踪列的交换和用标量乘以列。如果 $A^T$ 的阶梯形在每一列(和每一行)都没有主元,那么 $A$ 是不可逆的,因此 $\det A = 0$。如果 $A$ 是可逆的,我们得到一个三角矩阵,而 $\det A$ 是对角项的乘积,乘以来自列交换和乘法的校正因子。上述算法暗示 $\det A$ 仅在矩阵 $A$ 不可逆时才可能为零。结合命题 3.1 的最后一个陈述,我们得到:

\textbf{命题 3.3。} $\det A = 0$ 当且仅当 $A$ 不可逆,或者等价地说:$\det A \neq 0$ 当且仅当 $A$ 可逆。

注意,虽然我们现在知道如何计算行列式,但行列式仍然没有被定义。可以问:为什么我们不将其定义为通过上述算法得到的结果?问题在于,从形式上看,这个结果并非良好定义:我们没有证明不同的列运算序列会得到相同的结果。

\textbf{3.5. 转置行列式和乘积行列式。初等矩阵的行列式。}

在本节中,我们将证明两个重要定理。

\textbf{定理 3.4(转置行列式)。} 对于方阵 $A$,$\det A = \det(A^T)$。

这个定理意味着,我们之前讨论过的关于列的所有陈述,关于行的相应陈述也都是正确的。特别是,行列式在\textbf{行运算}下的行为与在\textbf{列运算}下的行为相同。因此,我们可以使用行运算来计算行列式。

\textbf{定理 3.5(乘积行列式)。} 对于 $n \times n$ 矩阵 $A$ 和 $B$:
$$
\det(AB) = (\det A)(\det B)
$$
换句话说,\textbf{乘积的行列式等于行列式的乘积}。

为了证明这两个定理,我们需要以下引理。

\textbf{引理 3.6。} 对于方阵 $A$ 和初等矩阵 $E$(相同大小):
$$
\det(AE) = (\det A)(\det E)
$$
\textbf{证明}~ 证明可以通过直接检查来完成:特殊矩阵的行列式很容易计算;从左边乘以初等矩阵是列运算,而列运算对行列式的影响是众所周知的。这可能看起来像一个幸运的巧合,即初等矩阵的行列式与其相应的列运算一致,但这并非巧合。也就是说,对于列运算,相应的初等矩阵可以从单位矩阵 $I$ 中通过该列运算得到。所以,它的行列式是 $1$($I$ 的行列式)乘以列运算的影响。而这一切就是这样!这可能一开始很难意识到,但上述段落是对引理的\textbf{完整且严谨}的证明!

应用引理 3.6 $N$ 次,我们得到以下推论。

\textbf{推论 3.7。} 对于任何矩阵 $A$ 和任何初等矩阵序列 $E_1, E_2, \dots, E_N$(所有矩阵均为 $n \times n$):
$$
\det(A E_1 E_2 \dots E_N) = (\det A)(\det E_1)(\det E_2) \dots (\det E_N)
$$

\textbf{引理 3.8。} 任何可逆矩阵都可以表示为初等矩阵的乘积。

\textbf{证明}~ 我们知道任何可逆矩阵都可以通过行运算(其简化阶梯形)化为单位矩阵。所以 $I = E_N E_{N-1} \dots E_2 E_1 A$,因此任何可逆矩阵都可以表示为初等矩阵的乘积,$A = (E_N \dots E_2 E_1)^{-1} I = E_1^{-1} E_2^{-1} \dots E_N^{-1}$(初等矩阵的逆也是初等矩阵)。

\textbf{定理 3.4 的证明。} 首先,可以很容易地检查出,对于初等矩阵 $E$, $\det E = \det(E^T)$。请注意,只需为可逆矩阵 $A$ 证明该定理即可,因为如果 $A$ 不可逆,那么 $A^T$ 也不可逆,并且两个行列式都为零。根据引理 3.8,矩阵 $A$ 可以表示为初等矩阵的乘积,$A = E_1 E_2 \dots E_N$,并且根据推论 3.7, $A$ 的行列式是初等矩阵行列式的乘积。由于取转置只是转置每个初等矩阵并反转它们的顺序,推论 3.7 蕴含了 $\det A = \det A^T$。

\textbf{定理 3.5 的证明。} 首先让我们假设矩阵 $B$ 是可逆的。那么引理 3.8 蕴含了 $B$ 可以表示为初等矩阵的乘积 $B = E_1 E_2 \dots E_N$,因此根据推论 3.7 $\det(AB) = (\det A)(\det E_1)(\det E_2) \dots (\det E_N) = (\det A)(\det B)$。

如果 $B$ 不可逆,那么乘积 $AB$ 也不可逆,而定理仅仅说明 $0 = 0$。要检查乘积 $AB = C$ 是否不可逆,让我们假设它是可逆的。那么将恒等式 $AB = C$ 从左边乘以 $C^{-1}$,我们得到 $C^{-1} AB = I$,所以 $C^{-1} A$ 是 $B$ 的左逆。因此 $B$ 是左可逆的,并且由于它是方的,所以它是可逆的。我们得到了一个矛盾。

\textbf{3.6. 行列式的性质总结。}

首先,让我们再说一遍,\textbf{行列式仅为方阵定义!} 由于我们现在知道 $\det A = \det(A^T)$,我们之前关于列的所有陈述也对行成立。
1. 行列式在每个行(列)中是线性的,当其他行(列)固定时。
2. 如果我们交换矩阵 $A$ 的两行(列),行列式改变符号。
3. 对于三角矩阵(特别是对角矩阵),其行列式是对角项的乘积。特别是,$\det I = 1$。
4. 如果矩阵 $A$ 有一个零行(或零列),则 $\det A = 0$。
5. 如果矩阵 $A$ 有两行(列)相等,则 $\det A = 0$。
6. 如果 $A$ 的某一行(列)是其他行(列)的线性组合,即如果矩阵不可逆,则 $\det A = 0$;更一般地,
7. $\det A = 0$ 当且仅当 $A$ 不可逆,或者等价地说
8. $\det A \neq 0$ 当且仅当 $A$ 可逆。
9. 如果我们将行(列)的线性组合加到某个行(列)上,行列式不改变。特别是,行列式在行(列)替换,即第三类行(列)运算下保持不变。
10. $\det A^T = \det A$。
11. $\det(AB) = (\det A)(\det B)$。

最后,
12. 如果 $A$ 是一个 $n \times n$ 矩阵,那么 $\det( \alpha A ) = \alpha^n \det A$。最后一个性质是从行列式的线性性质得出的,如果我们回忆起要将矩阵 $A$ 乘以 $\alpha$,我们必须将每一行乘以 $\alpha$,并且每次乘法都会将行列式乘以 $\alpha$。

\textbf{练习}~

3.1. 如果 $A$ 是一个 $n \times n$ 矩阵,$\det(3A)$ 与 $\det A$ 有何关系?
注释:$\det(3A) = 3 \det A$ 仅在 $1 \times 1$ 矩阵的平凡情况下成立。

3.2. $A = \begin{pmatrix} a_1 & a_2 & a_3 \\ b_1 & b_2 & b_3 \\ c_1 & c_2 & c_3 \end{pmatrix}$, $B = \begin{pmatrix} 2a_1 & 3a_2 & 5a_3 \\ 2b_1 & 3b_2 & 5b_3 \\ 2c_1 & 3c_2 & 5c_3 \end{pmatrix}$;
$A = \begin{pmatrix} a_1 & a_2 & a_3 \\ b_1 & b_2 & b_3 \\ c_1 & c_2 & c_3 \end{pmatrix}$, $B = \begin{pmatrix} 3a_1 & 4a_2 + 5a_1 & 5a_3 \\ 3b_1 & 4b_2 + 5b_1 & 5b_3 \\ 3c_1 & 4c_2 + 5c_1 & 5c_3 \end{pmatrix}$。$A$ 和 $B$ 的行列式之间有什么关系?

3.3. 使用列或行运算计算行列式:
$\begin{vmatrix} 0 & 1 & 2 \\ -1 & 0 & -3 \\ 2 & 3 & 0 \end{vmatrix}$, $\begin{vmatrix} 1 & 2 & 3 \\ 4 & 5 & 6 \\ 7 & 8 & 9 \end{vmatrix}$, $\begin{vmatrix} 1 & 0 & -2 & 3 \\ -3 & 1 & 1 & 2 \\ 0 & 4 & -1 & 1 \\ 2 & 3 & 0 & 1 \end{vmatrix}$, $\begin{vmatrix} 1 & x \\ 1 & y \end{vmatrix}$。

3.4. 一个方阵($n \times n$)称为\textbf{反对称}(或\textbf{反交换})矩阵,如果 $A^T = -A$。证明如果 $A$ 是反对称的且 $n$ 是奇数,则 $\det A = 0$。这对偶数 $n$ 是否成立?

3.5. 一个方阵称为\textbf{幂零}(nilpotent)矩阵,如果 $A^k = 0$ 对某个正整数 $k$ 成立。证明如果 $A$ 是幂零的,则 $\det A = 0$。

3.6. 证明如果矩阵 $A$ 和 $B$ 相似,则 $\det A = \det B$。

3.7. 一个实方阵 $Q$ 称为\textbf{正交}的,如果 $Q^T Q = I$。证明如果 $Q$ 是正交矩阵,那么 $\det Q = \pm 1$。

3.8. 证明 $\begin{vmatrix} 1 & x & x^2 \\ 1 & y & y^2 \\ 1 & z & z^2 \end{vmatrix} = (z-x)(z-y)(y-x)$。这是所谓的 Vandermonde 行列式的特例。

3.9. 设平面 $\mathbb{R}^2$ 中的点 $A, B, C$ 的坐标分别为 $(x_1, y_1), (x_2, y_2), (x_3, y_3)$。证明三角形 $ABC$ 的面积是 $\frac{1}{2} \left| \begin{vmatrix} 1 & x_1 & y_1 \\ 1 & x_2 & y_2 \\ 1 & x_3 & y_3 \end{vmatrix} \right|$ 的绝对值。提示:使用行运算和 $2 \times 2$ 行列式的几何解释(面积)。

3.10. 设 $A$ 和 $C$ 是方阵,证明块三角矩阵 $\begin{pmatrix} I & * \\ 0 & A \end{pmatrix}$, $\begin{pmatrix} A & * \\ 0 & I \end{pmatrix}$, $\begin{pmatrix} I & 0 \\ * & A \end{pmatrix}$, $\begin{pmatrix} A & 0 \\ * & I \end{pmatrix}$ 的行列式都等于 $\det A$。这里 $*$ 可以是任何东西。以下问题说明了块矩阵表示的力量。

3.11. 使用上一个问题证明,如果 $A$ 和 $C$ 是方阵,那么 $\det \begin{pmatrix} A & B \\ 0 & C \end{pmatrix} = (\det A)(\det C)$。提示:$\begin{pmatrix} A & B \\ 0 & C \end{pmatrix} = \begin{pmatrix} I & B \\ 0 & C \end{pmatrix} \begin{pmatrix} A & 0 \\ 0 & I \end{pmatrix}$。

3.12. 设 $A$ 是 $m \times n$ 矩阵,$B$ 是 $n \times m$ 矩阵。证明 $\det \begin{pmatrix} 0 & A \\ -B & I \end{pmatrix} = \det(AB)$。提示:虽然可以通过对矩阵进行行运算得到行列式易于计算的形式,但最简单的方法是右乘矩阵 $\begin{pmatrix} I & 0 \\ B & I \end{pmatrix}$。


\section{行列式的形式定义~存在性与唯一性}

在本节中,我们得到了行列式的形式定义。我们表明,一个函数,满足第 3 节中的基本性质 1, 2, 3 的存在性,而且,这样的函数是唯一的,也就是说,在构造行列式时我们别无选择。

考虑一个 $n \times n$ 矩阵 $A = \{a_{j,k}\}_{n \times n}$,并设 $\mathbf{v}_1, \mathbf{v}_2, \dots, \mathbf{v}_n$ 是它的列,即
$$
\mathbf{v}_k = \begin{pmatrix} a_{1,k} \\ a_{2,k} \\ \vdots \\ a_{n,k} \end{pmatrix} = a_{1,k} \mathbf{e}_1 + a_{2,k} \mathbf{e}_2 + \dots + a_{n,k} \mathbf{e}_n = \sum_{j=1}^n a_{j,k} \mathbf{e}_j
$$
使用行列式的线性性,我们在第一列展开:
$$
D(\mathbf{v}_1, \mathbf{v}_2, \dots, \mathbf{v}_n) = D(\sum_{j=1}^n a_{j,1} \mathbf{e}_j, \mathbf{v}_2, \dots, \mathbf{v}_n) = \sum_{j=1}^n a_{j,1} D(\mathbf{e}_j, \mathbf{v}_2, \dots, \mathbf{v}_n) \quad (4.1)
$$
然后我们在第二列展开,然后是第三列,依此类推。我们得到
$$
D(\mathbf{v}_1, \mathbf{v}_2, \dots, \mathbf{v}_n) = \sum_{j_1=1}^n \sum_{j_2=1}^n \dots \sum_{j_n=1}^n a_{j_1,1} a_{j_2,2} \dots a_{j_n,n} D(\mathbf{e}_{j_1}, \mathbf{e}_{j_2}, \dots, \mathbf{e}_{j_n})
$$
注意,我们必须为每一列使用不同的求和索引:我们称它们为 $j_1, j_2, \dots, j_n$;这里 $j_1$ 的索引与 (4.1) 中的索引 $j$ 相同。这是一个巨大的求和,包含 $n^n$ 项。幸运的是,其中一些项为零。也就是说,如果 $j_1, j_2, \dots, j_n$ 中有任何两个索引相同,则行列式 $D(\mathbf{e}_{j_1}, \mathbf{e}_{j_2}, \dots, \mathbf{e}_{j_n})$ 为零,因为这里有两个相等的列。因此,让我们重写求和,省略所有零项。最方便的方式是使用\textbf{排列}(permutation)的概念。

非正式地说,一个有序集 $\{1, 2, \dots, n\}$ 的\textbf{排列}是其元素的重新排列。一种方便的形式表示这种重新排列是通过使用一个函数 $\sigma: \{1, 2, \dots, n\} \to \{1, 2, \dots, n\}$,其中 $\sigma(1), \sigma(2), \dots, \sigma(n)$ 给出了集合 $1, 2, \dots, n$ 的新顺序。换句话说,排列 $\sigma$ 将有序集 $1, 2, \dots, n$ 重排为 $\sigma(1), \sigma(2), \dots, \sigma(n)$。这样的函数 $\sigma$ 必须是\textbf{一对一}(对不同的参数取不同的值)和\textbf{ onto}(取自目标空间的所有可能值)。一对一且 onto 的函数称为\textbf{双射}(bijection),它们在定义域和目标空间之间建立了一对一的对应关系。$^1$

$^1$ 尽管这在此处不直接相关,但让我们注意到,在组合学中是众所周知的,集合 $\{1, 2, \dots, n\}$ 的不同排列的数量恰好是 $n!$。所有 $n$ 的排列的集合将被记为 $\text{Perm}(n)$。

虽然这在此处不直接相关,但让我们注意到,在组合学中是众所周知的,集合 $\{1, 2, \dots, n\}$ 的不同排列的数量恰好是 $n!$。所有 $n$ 的排列的集合将被记为 $\text{Perm}(n)$。

使用排列的概念,我们可以将行列式重写为:
$$
D(\mathbf{v}_1, \mathbf{v}_2, \dots, \mathbf{v}_n) = \sum_{\sigma \in \text{Perm}(n)} a_{\sigma(1),1} a_{\sigma(2),2} \dots a_{\sigma(n),n} D(\mathbf{e}_{\sigma(1)}, \mathbf{e}_{\sigma(2)}, \dots, \mathbf{e}_{\sigma(n)})
$$
其中求和是遍历 $\{1, 2, \dots, n\}$ 的所有排列。

矩阵 $\mathbf{e}_{\sigma(1)}, \mathbf{e}_{\sigma(2)}, \dots, \mathbf{e}_{\sigma(n)}$ 的列可以从单位矩阵通过有限次数的列交换得到,所以行列式 $D(\mathbf{e}_{\sigma(1)}, \mathbf{e}_{\sigma(2)}, \dots, \mathbf{e}_{\sigma(n)})$ 是 $1$ 或 $-1$,取决于列交换的次数。为了形式化这一点,我们(非正式地)定义排列 $\sigma$ 的\textbf{符号}(记作 $\text{sign } \sigma$)为,如果将 $n$ 元组 $1, 2, \dots, n$ 重排为 $\sigma(1), \sigma(2), \dots, \sigma(n)$ 所需的交换次数是偶数,则符号为 1,如果交换次数是奇数,则 $\text{sign}(\sigma) = -1$。这是组合学中的一个事实,符号是明确定义的,即虽然有无数种方法可以从 $1, 2, \dots, n$ 得到 $n$ 元组 $\sigma(1), \sigma(2), \dots, \sigma(n)$,但交换次数要么总是奇数,要么总是偶数。

一种证明这一点的方法是引入另一种定义。设 $K(\sigma)$ 为 $\sigma$ 的\textbf{逆序对}(disorder)的数量,即满足 $\sigma(j) > \sigma(k)$ 的整数对 $(j, k)$ 的数量,其中 $j, k \in \{1, 2, \dots, n\}$, $j < k$,然后检查该数量是偶数还是奇数。我们将排列 $\sigma$ 称为\textbf{奇排列}如果 $K$ 是奇数,称为\textbf{偶排列}如果 $K$ 是偶数。然后定义 $\text{sign } \sigma := (-1)^{K(\sigma)}$;注意这样定义的 $\text{sign } \sigma$ 是明确定义的。

我们要证明 $\text{sign } \sigma = (-1)^{K(\sigma)}$ 可以通过将 $n$ 元组 $1, 2, \dots, n$ 重排为 $\sigma(1), \sigma(2), \dots, \sigma(n)$ 并计算交换次数来得到,如上所述。

如果 $\sigma(k) = k \ \forall k$,那么\textbf{逆序对}的数量 $K(\sigma)$ 为 0,所以这种\textbf{恒等}排列的符号是 1。还请注意,任何两个相邻元素的转置(仅交换两个相邻元素)会改变排列的符号,因为它会改变逆序对的数量(增加或减少 1)。因此,要从一个排列得到另一个排列,当排列具有相同的符号时,总是需要偶数次初等转置,而当符号不同时,则需要奇数次。最后,任何两个元素的交换都可以通过奇数次初等转置来实现。这意味着当两个元素被交换时,符号会改变。因此,要从 $1, 2, \dots, n$ 得到偶排列(正符号)总是需要偶数次交换,而得到奇排列(负符号)需要奇数次交换。因此,为了从 $1, 2, \dots, n$ 得到偶排列(正符号)总是需要偶数次交换,而得到奇排列(负符号)需要奇数次交换。因此,符号在交换两个元素时会改变。所以,要从 $1, 2, \dots, n$ 得到偶排列(正符号)总是需要偶数次交换,而得到奇排列(负符号)需要奇数次交换。因此,符号在交换两个元素时会改变。

因此,如果我们希望行列式满足第 3 节中的基本性质 1-3,我们必须将其定义为:
$$
\det A = \sum_{\sigma \in \text{Perm}(n)} a_{\sigma(1),1} a_{\sigma(2),2} \dots a_{\sigma(n),n} \text{sign}(\sigma) \quad (4.2)
$$
其中求和遍历 $\{1, 2, \dots, n\}$ 的所有排列。如果我们这样定义行列式,可以很容易地验证它满足第 3 节中的基本性质 1-3。实际上,因为每个乘积项在每一列中恰好有一个因子,并且对于任何两个相邻的列交换,我们得到的符号会改变,所以满足线性性和反对称性。而且,对于单位矩阵 $I$,右侧只有一项(对应于恒等排列 $\sigma(k)=k \ \forall k$),它的符号是 1,所以 $D(I)=1$。

\textbf{练习}~

4.1. 假设排列 $\sigma$ 将 $(1, 2, 3, 4, 5)$ 映射到 $(5, 4, 1, 2, 3)$。
a) 找到 $\sigma$ 的符号;
b) $\sigma^2 := \sigma \circ \sigma$ 对 $(1, 2, 3, 4, 5)$ 做什么?
c) 逆排列 $\sigma^{-1}$ 对 $(1, 2, 3, 4, 5)$ 做什么?
d) $\sigma^{-1}$ 的符号是什么?

4.2. 设 $P$ 是一个\textbf{排列矩阵},即一个由零和一组成的 $n \times n$ 矩阵,并且每行每列恰好有一个 1。
a) 你能描述相应的线性变换吗?这会解释它的名称。
b) 证明 $P$ 是可逆的。你能描述 $P^{-1}$ 吗?
c) 证明对于某些 $N > 0$, $P^N := P P \dots P$($N$ 次)$= I$。利用只有有限个排列的事实。

4.3. 为什么 $(1, 2, \dots, 9)$ 的排列有偶数个,并且其中恰好一半是奇排列?提示:这个问题用排列来解决可能很难,但有一个非常简单的行列式解。

4.4. 如果 $\sigma$ 是一个奇排列,解释为什么 $\sigma^2$ 是偶数但 $\sigma^{-1}$ 是奇数。

4.5. 使用 (4.2) 的行列式形式计算一个 $n \times n$ 矩阵的行列式需要多少次乘法和加法?不要计算计算 $\text{sign } \sigma$ 所需的操作。


\section{代数余子式展开}

对于 $n \times n$ 矩阵 $A = \{a_{j,k}\}_{n \times n}$,设 $A_{j,k}$ 表示通过划掉第 $j$ 行和第 $k$ 列得到的 $(n-1) \times (n-1)$ 矩阵。

\textbf{定理 5.1(行列式的代数余子式展开)。} 设 $A$ 是一个 $n \times n$ 矩阵。对于每个 $j$, $1 \le j \le n$,行列式 $A$ 可以按第 $j$ 行展开为:
$$
\det A = a_{j,1} (-1)^{j+1} \det A_{j,1} + a_{j,2} (-1)^{j+2} \det A_{j,2} + \dots + a_{j,n} (-1)^{j+n} \det A_{j,n} = \sum_{k=1}^n a_{j,k} (-1)^{j+k} \det A_{j,k}
$$
类似地,对于每个 $k$, $1 \le k \le n$,行列式可以按第 $k$ 列展开为:
$$
\det A = \sum_{j=1}^n a_{j,k} (-1)^{j+k} \det A_{j,k}
$$

\textbf{证明}~ 我们首先证明第 1 行展开的公式。第 2 行的展开公式可以通过交换第 1 行和第 2 行从它得到。然后交换第 2 行和第 3 行,我们得到第 3 行的展开公式,依此类推。由于 $\det A = \det A^T$,列展开自动跟上。

让我们首先考虑一个特殊情况,即第一行只有一个非零项 $a_{1,1}$。通过对第 2、3、...、n 列进行列运算,我们可以将 $A$ 转化为下三角形式。那么 $A$ 的行列式可以计算为三角矩阵的对角项的乘积 $\times$ 来自列运算的修正因子。但是,除了 $a_{1,1}$ 之外的所有对角项的乘积(即不包括 $a_{1,1}$)加上修正因子恰好是 $\det A_{1,1}$,所以在这种特定情况下 $\det A = a_{1,1} \det A_{1,1}$。

现在考虑所有项除了 $a_{1,2}$ 在第一行都为零的情况。这种情况可以通过交换第 1 列和第 2 列来减少到前面的情况,因此在这种情况下 $\det A = (-1)^{1+2} a_{1,2} \det A_{1,2}$。当 $a_{1,3}$ 是第一行唯一非零项的情况,可以通过交换第 2 行和第 3 行来减少到前面情况,所以在这种情况下 $\det A = a_{1,3} \det A_{1,3}$。

重复这个过程,我们得到,当 $a_{1,k}$ 是第一行唯一非零项时,$\det A = (-1)^{1+k} a_{1,k} \det A_{1,k}$。

在一般情况下,行列式在每一行上的线性意味着 $\det A = \det A^{(1)} + \det A^{(2)} + \dots + \det A^{(n)} = \sum_{k=1}^n \det A^{(k)}$,其中矩阵 $A^{(k)}$ 是通过将 $A$ 的第一行中除 $a_{1,k}$ 之外的所有项替换为 0 而得到的。正如我们上面所讨论的,$\det A^{(k)} = (-1)^{1+k} a_{1,k} \det A_{1,k}$,所以 $\det A = \sum_{k=1}^n (-1)^{1+k} a_{1,k} \det A_{1,k}$。

为了得到第二行的展开式,我们可以交换第 1 行和第 2 行,然后应用上面的公式。行交换改变了符号,所以我们得到 $\det A = -\sum_{k=1}^n (-1)^{1+k} a_{2,k} \det A_{2,k} = \sum_{k=1}^n (-1)^{2+k} a_{2,k} \det A_{2,k}$。通过交换第 3 行和第 2 行并按第二行展开,我们得到公式 $\det A = \sum_{k=1}^n (-1)^{3+k} a_{3,k} \det A_{3,k}$,依此类推。

要将行列式 $\det A$ 展开到第 $k$ 列,只需对 $A^T$ 应用行展开公式即可。

\textbf{定义}~ $C_{j,k} = (-1)^{j+k} \det A_{j,k}$ 这些数称为 $A$ 的\textbf{代数余子式}(cofactor)。

使用这个符号,在第 $j$ 行展开行列式的公式可以重写为 $\det A = a_{j,1} C_{j,1} + a_{j,2} C_{j,2} + \dots + a_{j,n} C_{j,n} = \sum_{k=1}^n a_{j,k} C_{j,k}$。类似地,在第 $k$ 列展开可以写成 $\det A = a_{1,k} C_{1,k} + a_{2,k} C_{2,k} + \dots + a_{n,k} C_{n,k} = \sum_{j=1}^n a_{j,k} C_{j,k}$。

\textbf{注释}~ 代数余子式展开公式经常被用作行列式的定义。不难证明由该公式给出的量满足行列式的基本性质:归一化性质是微不足道的,反对称性的证明很容易。然而,线性性的证明虽然不难,但有点繁琐。

\textbf{注释}~ 虽然它看起来非常不错,但代数余子式展开公式不适用于计算大于 $3 \times 3$ 的矩阵的行列式。正如可以计算的那样,它需要超过 $n!$ 次乘法(精确地说,需要 $\sum_{k=2}^n n!/k!$ 次乘法),而 $n!$ 的增长非常快。例如,计算一个 $20 \times 20$ 矩阵的代数余子式展开需要超过 $20! \approx 2.4 \times 10^{18}$ 次乘法。一台每秒执行十亿次乘法的计算机需要 77 年才能执行 $20!$ 次乘法;计算一个 $20 \times 20$ 矩阵的代数余子式展开所需的乘法将需要 132 多年。另一方面,使用行约简计算 $n \times n$ 矩阵的行列式需要 $(n^3 + 2n - 3) / 3$ 次乘法(以及大约相同数量的加法)。对于一台每秒执行一百万次运算(按当前标准非常慢)的计算机来说,计算 $100 \times 100$ 矩阵的行列式只需要一秒钟的一小部分。只有当某一行(或列)包含很多零项时,才可能实用地应用代数余子式展开公式。然而,代数余子式展开公式具有重要的理论价值,正如下一节所示。

\textbf{5.1. 逆矩阵的代数余子式公式。}

由代数余子式 $C_{j,k} = (-1)^{j+k} \det A_{j,k}$ 组成的矩阵 $C = \{C_{j,k}\}_{n \times n}$ 称为 $A$ 的\textbf{代数余子式矩阵}。

\textbf{定理 5.2。} 设 $A$ 是一个可逆矩阵,设 $C$ 是它的代数余子式矩阵。那么
$$
A^{-1} = \frac{1}{\det A} C^T
$$

\textbf{证明}~ 让我们计算乘积 $AC^T$。第 $j$ 个对角项是通过将 $A$ 的第 $j$ 行与 $C$ 的第 $j$ 列(即 $C^T$ 的第 $j$ 行)相乘得到的,所以 $(AC^T)_{j,j} = a_{j,1} C_{j,1} + a_{j,2} C_{j,2} + \dots + a_{j,n} C_{j,n} = \det A$,根据代数余子式展开公式。为了得到非对角项,我们需要将 $A$ 的第 $k$ 行与 $C^T$ 的第 $j$ 列相乘,$j \neq k$,得到 $a_{k,1} C_{j,1} + a_{k,2} C_{j,2} + \dots + a_{k,n} C_{j,n}$。根据代数余子式展开公式(在第 $j$ 行展开),这是将 $A$ 中第 $j$ 行替换为第 $k$ 行(而保持所有其他行不变)得到的矩阵的行列式。但是,这个矩阵的第 $j$ 行和第 $k$ 行是相同的,所以行列式为 0。因此,$AC^T$ 的所有非对角项都为零(而所有对角项都等于 $\det A$),所以 $AC^T = (\det A) I$。这意味着矩阵 $\frac{1}{\det A} C^T$ 是 $A$ 的右逆,由于 $A$ 是方阵,所以它是逆。

回忆一下,对于可逆矩阵 $A$,方程 $A \mathbf{x} = \mathbf{b}$ 的解是 $x = A^{-1} b = \frac{1}{\det A} C^T b$,我们得到以下定理的推论。

\textbf{推论 5.3(Cramer 法则)。} 对于可逆矩阵 $A$,方程 $A \mathbf{x} = \mathbf{b}$ 的解的第 $k$ 个项由公式给出:
$$
x_k = \frac{\det B_k}{\det A}
$$
其中矩阵 $B_k$ 是通过将 $A$ 的第 $k$ 列替换为向量 $b$ 而得到的。

\textbf{5.2. 逆矩阵的代数余子式公式的应用。}

\textbf{示例(求 $2 \times 2$ 矩阵的逆)。} 代数余子式公式在求 $2 \times 2$ 矩阵 $A = \begin{pmatrix} a & b \\ c & d \end{pmatrix}$ 的逆时,确实非常有用。代数余子式仅仅是($1 \times 1$ 矩阵)的项,代数余子式矩阵是 $\begin{pmatrix} d & -c \\ -b & a \end{pmatrix}$,所以逆矩阵 $A^{-1}$ 由公式给出:
$$
A^{-1} = \frac{1}{\det A} \begin{pmatrix} d & -b \\ -c & a \end{pmatrix}
$$
虽然对于维度大于 3 的情况,逆矩阵的代数余子式公式看起来不实用,但它具有巨大的理论价值,正如下面的例子所示。

\textbf{示例(整数逆矩阵)。} 假设我们想构造一个具有整数项的矩阵 $A$,使得其逆也具有整数项(求逆这样的矩阵会是一个很好的家庭作业:无需处理分数)。如果 $\det A = 1$ 且其项是整数,那么逆矩阵的代数余子式公式蕴含了 $A^{-1}$ 也具有整数项。注意,构造一个 $\det A = 1$ 的整数矩阵很容易:应该从主对角线上为 1 的三角矩阵开始,然后应用几次行或列替换(第三类运算)来使矩阵看起来是通用的。

\textbf{示例(多项式矩阵的逆)。} 另一个例子是考虑一个\textbf{多项式矩阵} $A(x)$,即其项不是数字而是变量 $x$ 的多项式 $a_{j,k}(x)$。如果 $\det A(x) \equiv 1$,那么逆矩阵 $A^{-1}(x)$ 也是一个多项式矩阵。如果 $\det A(x) = p(x) \neq 0$,则从代数余子式展开可知,$p(x)$ 是一个多项式,因此 $A^{-1}(x)$ 具有有理数项:更重要的是,$p(x)$ 是每个分母的倍数。

\textbf{练习}~

5.1. 使用任何方法计算行列式:
$\begin{vmatrix} 0 & 1 & 2 \\ -1 & 0 & -3 \\ 2 & 3 & 0 \end{vmatrix}$, $\begin{vmatrix} 1 & 2 & 3 \\ 4 & 5 & 6 \\ 7 & 8 & 9 \end{vmatrix}$, $\begin{vmatrix} 1 & 0 & -2 & 3 \\ -3 & 1 & 1 & 2 \\ 0 & 4 & -1 & 1 \\ 2 & 3 & 0 & 1 \end{vmatrix}$。

5.2. 使用行(列)展开计算行列式。注意,您不必使用第一行(列):选择具有许多零的行(列)将简化您的计算。
$\begin{vmatrix} 1 & 2 & 0 \\ 1 & 1 & 5 \\ 1 & -3 & 0 \end{vmatrix}$, $\begin{vmatrix} 4 & -6 & -4 & 4 \\ 2 & 1 & 0 & 0 \\ 0 & -3 & 1 & 3 \\ -2 & 2 & -3 & -5 \end{vmatrix}$。

5.3. 对于矩阵 $A = \begin{pmatrix} 0 & 0 & 0 & \dots & 0 & a_0 \\ -1 & 0 & 0 & \dots & 0 & a_1 \\ 0 & -1 & 0 & \dots & 0 & a_2 \\ \vdots & \vdots & \vdots & \ddots & \vdots & \vdots \\ 0 & 0 & 0 & \dots & 0 & a_{n-2} \\ 0 & 0 & 0 & \dots & -1 & a_{n-1} \end{pmatrix}$,计算 $\det(A + tI)$,其中 $I$ 是 $n \times n$ 单位矩阵。你应该得到一个涉及 $a_0, a_1, \dots, a_{n-1}$ 和 $t$ 的漂亮表达式。行展开和归纳可能是最好的方法。

5.4. 使用代数余子式公式计算矩阵 $\begin{pmatrix} 1 & 2 \\ 3 & 4 \end{pmatrix}$, $\begin{pmatrix} 19 & -17 \\ 3 & -2 \end{pmatrix}$, $\begin{pmatrix} 1 & 0 \\ 3 & 5 \end{pmatrix}$, $\begin{pmatrix} 1 & 1 & 0 \\ 2 & 1 & 2 \\ 0 & 1 & 1 \end{pmatrix}$ 的逆。

5.5. 设 $D_n$ 是 $n \times n$ 三对角矩阵的行列式:
$$
\begin{pmatrix}
1 & -1 & 0 & \dots & 0 \\
1 & 1 & -1 & \dots & 0 \\
0 & 1 & 1 & \dots & 0 \\
\vdots & \vdots & \ddots & \ddots & \vdots \\
0 & 0 & 0 & \dots & 1 & -1 \\
0 & 0 & 0 & \dots & 1 & 1
\end{pmatrix}
$$
使用代数余子式展开证明 $D_n = D_{n-1} + D_{n-2}$。这表明序列 $D_n$ 是斐波那契数列 $1, 2, 3, 5, 8, 13, 21, \dots$。

5.6. 重访 Vandermonde 行列式。我们的目标是证明 $(n+1) \times (n+1)$ Vandermonde 行列式的公式:
$$
\begin{vmatrix}
1 & c_0 & c_0^2 & \dots & c_0^n \\
1 & c_1 & c_1^2 & \dots & c_1^n \\
\vdots & \vdots & \vdots & \ddots & \vdots \\
1 & c_n & c_n^2 & \dots & c_n^n
\end{vmatrix} = \prod_{0 \le j < k \le n} (c_k - c_j)
$$
我们将应用归纳法。为此:
a) 验证公式对 $n=1, n=2$ 成立。
b) 将最后一行中的变量 $c_n$ 称为 $x$,并证明行列式是一个 $n+1$ 次多项式,$A_0 + A_1 x + A_2 x^2 + \dots + A_n x^n$,其中系数 $A_k$ 取决于 $c_0, c_1, \dots, c_{n-1}$。
c) 证明该多项式在 $x = c_0, c_1, \dots, c_{n-1}$ 处有零点,因此可以表示为 $A_n \cdot (x - c_0)(x - c_1) \dots (x - c_{n-1})$,其中 $A_n$ 如上所述。
d) 假设 Vandermonde 行列式的公式对 $n-1$ 成立,计算 $A_n$ 并证明对 $n$ 的公式。

5.7. 使用代数余子式展开计算 $n \times n$ 矩阵的行列式需要多少次乘法?证明这个公式。


\section{子式与秩}

对于矩阵 $A$,让我们考虑它的 $k \times k$ \textbf{子矩阵},它通过选取 $k$ 行和 $k$ 列得到。该矩阵的行列式称为 $k$ 阶\textbf{子式}(minor)。注意,一个 $m \times n$ 矩阵有 $\binom{m}{k} \cdot \binom{n}{k}$ 个不同的 $k \times k$ 子矩阵,因此它有 $\binom{m}{k} \cdot \binom{n}{k}$ 个 $k$ 阶子式。

\textbf{定理 6.1。} 对于一个非零矩阵 $A$,它的秩等于存在非零 $k$ 阶子式的最大整数 $k$。

\textbf{证明}~ 首先,让我们证明,如果 $k > \text{rank } A$,则所有 $k$ 阶子式都为零。实际上,由于 $A$ 的列空间的维数 $\text{Ran } A$ 是 $\text{rank } A < k$,因此 $A$ 的任何 $k$ 列都是线性相关的。因此,对于 $A$ 的任何 $k \times k$ 子矩阵,它的列都是线性相关的,所以所有 $k$ 阶子式都为零。

为了完成证明,我们需要证明存在一个非零的 $k$ 阶子式,其中 $k = \text{rank } A$。可能存在许多这样的子式,但也许最简单的方法是取主元行和主元列(即原始矩阵中包含主元的行和列)。这个 $k \times k$ 子矩阵具有与原始矩阵相同的主元,因此它是可逆的(每一列和每一行都有主元),并且其行列式非零。

这个定理看起来不是很有用,因为进行行约简比计算所有子式要容易得多。然而,它具有重要的理论价值,正如以下推论所示。

\textbf{推论 6.2。} 设 $A(x)$ 是一个 $m \times n$ 多项式矩阵(即其项是变量 $x$ 的多项式)。那么 $\text{rank } A(x)$ 在除了可能有限个点之外的地方是恒定的,在这些点上秩会变小。

% ---



% 6. 子式与秩。对于矩阵 $A$,我们考虑它的 $k \times k$ 子矩阵,通过选取 $k$ 行和 $k$ 列得到。这个矩阵的行列式称为 $k$ 阶子式。注意,一个 $m \times n$ 的矩阵有 $\binom{m}{k} \cdot \binom{n}{k}$ 个不同的 $k \times k$ 子矩阵,因此它有 $\binom{m}{k} \cdot \binom{n}{k}$ 个 $k$ 阶子式。

% \textbf{定理 6.1。} 对于一个非零矩阵 $A$,它的秩等于存在一个非零的 $k$ 阶子式时 $k$ 的最大整数值。

% \textbf{证明。} 我们首先证明,如果 $k > \text{rank } A$,则所有 $k$ 阶子式都为 $0$。确实,因为列空间 $\text{Ran } A$ 的维数是 $\text{rank } A < k$,所以 $A$ 的任意 $k$ 列都是线性相关的。因此,对于 $A$ 的任意 $k \times k$ 子矩阵,它的列都是线性相关的,所以所有 $k$ 阶子式都为 $0$。为了完成证明,我们需要证明存在一个秩等于 $\text{rank } A$ 的非零子式。这样的子式可能有许多,但最容易得到一个非零子式的方法是选取主元行和主元列(即包含主元的原始矩阵的行和列)。这个 $k \times k$ 子矩阵与原始矩阵有相同的主元,所以它是可逆的(每列和每行都有一个主元),其行列式非零。

% 这个定理看起来不是很有用,因为进行行变换比计算所有子式要容易得多。然而,它具有重要的理论意义,正如以下推论所示。

% \textbf{推论 6.2。} 设 $A = A(x)$ 是一个 $m \times n$ 的多项式矩阵(即其元素是关于 $x$ 的多项式)。那么 $\text{rank } A(x)$ 在除了可能有限个点之外的地方是恒定的,在这些点上秩会变小。

\textbf{证明}~ 设 $r$ 是 $\text{rank } A(x) = r$

\textbf{证明}~ 设 $r$ 是存在一个 $x$ 使得 $\text{rank } A(x) = r$ 的最大整数。为了证明这样的 $r$ 存在,我们首先尝试 $r = \min\{m, n\}$。如果存在一个 $x$ 使得 $\text{rank } A(x) = r$,我们就找到了 $r$。如果不是,我们则将 $r$ 替换为 $r-1$ 并重试。经过有限步操作,我们或者停止,或者得到 $0$。因此,$r$ 是存在的。设 $x_0$ 是一个点使得 $\text{rank } A(x_0) = r$,并且设 $M$ 是一个 $k$ 阶子式,使得 $M(x_0) \neq 0$。由于 $M(x)$ 是一个 $k \times k$ 多项式矩阵的行列式,所以 $M(x)$ 是一个多项式。由于 $M(x_0) \neq 0$,它不是恒零的,因此它只能在有限个点处为零。所以,除了可能有限个点之外,$\text{rank } A(x) \geq r$。但是根据 $r$ 的定义,对于所有的 $x$,$\text{rank } A(x) \leq r$。


\section{第3章的复习题}

7.1. 真或假:
a) 行列式只为方阵定义。
b) 如果 $A$ 的两行或两列相同,则 $\det A = 0$。
c) 如果 $B$ 是通过交换 $A$ 的两行(或两列)得到的矩阵,则 $\det B = \det A$。
d) 如果 $B$ 是通过将 $A$ 的某一行(列)乘以一个标量 $\alpha$ 得到的矩阵,则 $\det B = \det A$。
e) 如果 $B$ 是通过将 $A$ 的某一行乘以一个数加到另一行得到的矩阵,则 $\det B = \det A$。
f) 三角矩阵的行列式是其对角线元素的乘积。
g) $\det(A^T) = -\det(A)$。
h) $\det(AB) = \det(A)\det(B)$。
i) 矩阵 $A$ 可逆当且仅当 $\det A \neq 0$。
j) 如果 $A$ 是可逆矩阵,则 $\det(A^{-1}) = 1/\det(A)$。

7.2. 设 $A$ 是一个 $n \times n$ 矩阵。$\det(3A)$, $\det(-A)$ 和 $\det(A^2)$ 与 $\det A$ 的关系是什么?

7.3. 如果 $A$ 和 $A^{-1}$ 的所有元素都是整数,那么 $\det A = 3$ 是否可能?
\textbf{提示:} $\det(A)\det(A^{-1})$ 是什么?

7.4. 设 $v_1, v_2$ 是 $\mathbb{R}^2$ 中的向量,设 $A$ 是以 $v_1, v_2$ 为列的 $2 \times 2$ 矩阵。证明 $|\det A|$ 是由向量 $v_1, v_2$ 作为两边给出的平行四边形的面积。首先考虑 $v_1 = (x_1, 0)^T$ 的情况。对于一般情况 $v_1 = (x_1, y_1)^T$,左乘一个旋转矩阵,将向量 $v_1$ 变换为 $(\tilde{x}_1, 0)^T$ 来处理。
\textbf{提示:} 旋转矩阵的行列式是什么?
以下问题说明了行列式的符号与向量组的“方向”之间的关系。

7.5. 设 $v_1, v_2$ 是 $\mathbb{R}^2$ 中的向量。证明 $D(v_1, v_2) > 0$ 当且仅当存在一个旋转 $T_\alpha$ 使得向量 $T_\alpha v_1$ 与 $e_1$ 平行(并且方向相同),且 $T_\alpha v_2$ 位于上半平面 $x_2 > 0$(即 $e_2$ 所在的半平面)。
\textbf{提示:} 旋转矩阵的行列式是什么?

---



% \chapter{谱理论简介(特征值与特征向量)}

谱理论是帮助我们理解线性算子结构的主要工具。在本章中,我们只考虑从一个向量空间映到其自身的算子(或等价地说,$n \times n$ 矩阵)。如果我们有一个这样的线性变换 $A: V \to V$,我们可以将其自身相乘,取其任意幂,或任意多项式。谱理论的主要思想是将算子分解为简单的块,并分别分析每个块。为了解释主要思想,让我们考虑\textbf{差分方程}。

许多过程可以用以下类型的方程描述:
$x_{n+1} = Ax_n$, $n = 0, 1, 2, \dots$,
其中 $A: V \to V$ 是一个线性变换,而 $x_n$ 是系统在时刻 $n$ 的状态。给定初始状态 $x_0$,我们希望知道时刻 $n$ 的状态 $x_n$,分析 $x_n$ 的长期行为等。$^1$

$^1$ 差分方程是微分方程 $x'(t) = Ax(t)$ 的离散时间类似物。为了求解微分方程,需要计算 $e^{tA} := \sum_{k=0}^\infty \frac{t^k A^k}{k!}$,而谱理论也有助于完成此操作。


乍一看,这个问题似乎很简单:解由公式 $x_n = A^n x_0$ 给出。但如果 $n$ 非常大呢:成千上万,数百万?或者如果我们想分析 $x_n$ 作为 $n \to \infty$ 的行为呢?这时\textbf{特征值}和\textbf{特征向量}的概念就出现了。

假设 $Ax_0 = \lambda x_0$,其中 $\lambda$ 是某个标量。那么 $A^2 x_0 = \lambda^2 x_0$, $A^3 x_0 = \lambda^3 x_0$, $\dots$, $A^n x_0 = \lambda^n x_0$,所以解的行为得到了很好的理解。在本节中,我们只考虑有限维空间中的算子。无穷维空间中的谱理论要复杂得多,这里提出的结果大多在无穷维情况下不成立。

\section{基本定义}

    1.1. \textbf{特征值、特征向量、谱。} 标量 $\lambda$ 被称为算子 $A: V \to V$ 的\textbf{特征值},如果存在一个\textbf{非零}向量 $v \in V$ 使得 $Av = \lambda v$。向量 $v$ 被称为 $A$ 的\textbf{特征向量}(对应于特征值 $\lambda$)。如果我们知道 $\lambda$ 是一个特征值,那么寻找特征向量很容易:只需解方程 $Ax = \lambda x$,或者等价地 $(A - \lambda I)x = 0$。所以,找到对应于特征值 $\lambda$ 的\textbf{所有}特征向量,就是找到 $A - \lambda I$ 的零空间。零空间 $\text{Ker}(A - \lambda I)$,即所有特征向量和零向量的集合,被称为\textbf{特征子空间}。所有算子 $A$ 的特征值集合被称为 $A$ 的\textbf{谱},通常记作 $\sigma(A)$。

    1.2. \textbf{寻找特征值:特征多项式。} 标量 $\lambda$ 是特征值当且仅当零空间 $\text{Ker}(A - \lambda I)$ 非平凡(因此方程 $(A - \lambda I)x = 0$ 有非平凡解)。设 $A$ 作用于 $F^n$(即 $A: F^n \to F^n$)。由于 $A - \lambda I$ 的矩阵是方阵,$A - \lambda I$ 有非平凡零空间当且仅当它不可逆。我们知道一个方阵不可逆当且仅当它的行列式为 $0$。因此
    $\lambda \in \sigma(A)$,即 $\lambda$ 是 $A$ 的特征值 $\Leftrightarrow \det(A - \lambda I) = 0$。
    如果 $A$ 是一个 $n \times n$ 矩阵,那么 $\det(A - \lambda I)$ 是关于变量 $\lambda$ 的 $n$ 次多项式。这个多项式被称为 $A$ 的\textbf{特征多项式}。所以,要找到 $A$ 的所有特征值,只需计算特征多项式并找到它所有的根。





用这种方法寻找算子的谱在更高维度下并不实用。求解高次多项式的根可能是一个非常困难的问题,并且对于次数大于 4 的方程,无法用根式求解。所以,在更高维度下,通常使用不同的数值方法来寻找特征值和特征向量。

1.3. \textbf{寻找抽象算子的特征多项式和特征值。} 因此,我们知道如何找到矩阵的谱。但如何找到作用在抽象向量空间中的算子的特征值呢?方法很简单:选取任意一个基,然后计算该基下算子的矩阵的特征值。但我们如何知道这个结果不依赖于基的选取呢?有几种可能的解释。一种是基于\textbf{相似矩阵}的概念。让我们回忆一下,方阵 $A$ 和 $B$ 被称为相似的,如果存在一个可逆矩阵 $S$ 使得 $A = SBS^{-1}$。注意,相似矩阵的行列式是相等的。确实 $\det A = \det(SBS^{-1}) = \det S \det B \det S^{-1} = \det B$,因为 $\det S^{-1} = 1/\det S$。注意,如果 $A = SBS^{-1}$,那么 $A - \lambda I = SBS^{-1} - \lambda SIS^{-1} = S(B - \lambda I)S^{-1}$,所以矩阵 $A - \lambda I$ 和 $B - \lambda I$ 是相似的。因此 $\det(A - \lambda I) = \det(B - \lambda I)$,即相似矩阵的特征多项式是相同的。如果 $T: V \to V$ 是一个线性变换,并且 $A$ 和 $B$ 是 $V$ 中的两个基,那么 $[T]_{AA} = [I]_{AB}[T]_{BB}[I]_{BA}$,并且由于 $[I]_{BA} = ([I]_{AB})^{-1}$,所以线性变换的矩阵在不同基下是相似的。换句话说,线性变换的矩阵在不同基下是相似的。因此,我们可以将算子的特征多项式定义为它在某个基下的矩阵的特征多项式。正如我们上面讨论的,结果不依赖于基的选择,所以算子的特征多项式是良好定义的。

1.4. \textbf{复数与实数空间。} 代数基本定理断言任何(至少一次)多项式都有一个复根。这意味着有限维\textbf{复数}向量空间中的算子至少有一个特征值,因此它的谱是非空的。另一方面,很容易在实向量空间中构造一个没有\textbf{实数}特征数的线性变换,例如在 $\mathbb{R}^2$ 中旋转 $R_\alpha$, $\alpha \neq k\pi$ ($k \in \mathbb{Z}$) 就是一个例子。由于通常假设特征值应该属于标量域(如果一个算子作用在域 $F$ 上的向量空间中,则特征值应该在 $F$ 中),这样的算子具有空谱。因此,复数情况(即作用在复数向量空间中的算子)似乎是谱理论最自然的环境。由于 $\mathbb{R} \subset \mathbb{C}$,我们可以始终将实数 $n \times n$ 矩阵视为 $\mathbb{C}^n$ 中的算子,以允许复数特征值。将实数矩阵视为 $\mathbb{C}^n$ 中的算子在谱理论中是很典型的,我们将遵循这个约定。寻找矩阵的特征值(除非另有说明)将始终意味着寻找所有\textbf{复数}特征值,而不是仅限于实数特征值。注意,抽象实数向量空间中的算子也可以解释为复数空间中的算子。一种朴素的方法是固定一个基(本书本章所有空间都是有限维的),然后在该基下使用坐标,允许使用复数坐标:这本质上是从实数矩阵到具有复数特征值的算子的过程。这种构造描述了所谓的\textbf{复化},结果不依赖于基的选择。下面第 5 章第 8.2 节将描述复化的“高明”抽象构造,解释为什么结果不依赖于基的选择。

1.5. \textbf{特征值的重数。} 提醒读者,如果 $p$ 是一个多项式,而 $\lambda$ 是它的一个根(即 $p(\lambda) = 0$),那么 $z - \lambda$ 整除 $p(z)$,即 $p$ 可以表示为 $p(z) = (z - \lambda)q(z)$,其中 $q$ 是某个多项式。如果 $q(\lambda) = 0$,那么 $q$ 也可以被 $z - \lambda$ 整除,所以 $(z - \lambda)^2$ 整除 $p$ 等等。能够整除 $p(z)$ 的 $(z - \lambda)$ 的最大正整数 $k$ 被称为根 $\lambda$ 的\textbf{重数}。如果 $\lambda$ 是算子(矩阵)$A$ 的一个特征值,那么它就是特征多项式 $p(z) = \det(A - zI)$ 的一个根。这个根的重数被称为特征值 $\lambda$ 的\textbf{(代数)重数}。次数为 $n$ 的任何多项式 $p(z) = \sum_{k=0}^n a_k z^k$ 恰好有 $n$ 个复数根,\textbf{计入重数}。\textbf{计入重数}的意思是,如果一个根的重数是 $d$,我们就必须列出(计数)它 $d$ 次。换句话说,$p$ 可以表示为 $p(z) = a_n (z - \lambda_1)(z - \lambda_2)\dots(z - \lambda_n)$,其中 $\lambda_1, \lambda_2, \dots, \lambda_n$ 是它的复数根,计入重数。还有另一种关于特征值重数的概念:特征子空间 $\text{Ker}(A - \lambda I)$ 的维数被称为特征值 $\lambda$ 的\textbf{几何重数}。几何重数不像代数重数那样被广泛使用。所以,当人们简单地说“重数”时,他们通常指的是\textbf{代数重数}。我们在此顺便提一下,特征值的代数重数和几何重数可能不同。

\textbf{命题 1.1。} 特征值的几何重数不能超过其代数重数。

\textbf{证明}~ 见下面的练习 1.9。

1.6. \textbf{迹与行列式。}
\textbf{定理 1.2。} 设 $A$ 是一个 $n \times n$ 矩阵,设 $\lambda_1, \lambda_2, \dots, \lambda_n$ 是它的(复数)特征值(计入重数)。那么
1. $\text{trace } A = \lambda_1 + \lambda_2 + \dots + \lambda_n$。
2. $\det A = \lambda_1 \lambda_2 \dots \lambda_n$。

\textbf{证明}~ 见下面的练习 1.10, 1.11。

1.7. \textbf{三角矩阵的特征值。} 计算特征值等价于寻找矩阵的特征多项式的根(或使用某种数值方法),这可能非常耗时。然而,有一个特殊情况,我们可以直接从矩阵中读出特征值。即,三角矩阵的特征值(计入重数)正是对角线上的元素 $a_{1,1}, a_{2,2}, \dots, a_{n,n}$。这里三角矩阵是指上三角或下三角矩阵。由于对角矩阵是三角矩阵的一个特例(它既是上三角也是下三角,所以对角矩阵的特征值是其对角线元素),其证明是微不足道的:我们需要从 $A$ 的对角元素中减去 $\lambda$,并利用三角矩阵的行列式是其对角线元素乘积这一事实。我们得到


特征多项式 $\det(A - \lambda I) = (a_{1,1} - \lambda)(a_{2,2} - \lambda)\dots(a_{n,n} - \lambda)$,其根正是 $a_{1,1}, a_{2,2}, \dots, a_{n,n}$。

\textbf{练习}~

1.1. 真或假:
a) 每个 $n$ 维向量空间中的线性算子都有 $n$ 个不同的特征值;
b) 如果一个矩阵只有一个特征向量,那么它有无限多个特征向量;
c) 存在一个方实数矩阵没有实数特征值;
d) 存在一个方矩阵没有(复数)特征向量;
e) 相似矩阵总是具有相同的特征值;
f) 相似矩阵总是具有相同的特征向量;
g) 矩阵 $A$ 的两个特征向量之和总是算子 $A$ 的特征向量;
h) 对应于同一特征值 $\lambda$ 的矩阵 $A$ 的两个特征向量之和总是算子 $A$ 的特征向量。

1.2. 找出以下矩阵的特征多项式、特征值和特征向量:
$\begin{pmatrix} 4 & -5 \\ 2 & -3 \end{pmatrix}$, $\begin{pmatrix} 2 & 1 \\ -1 & 4 \end{pmatrix}$, $\begin{pmatrix} 1 & 3 & 3 \\ -3 & -5 & -3 \\ 3 & 3 & 1 \end{pmatrix}$。

1.3. 计算旋转矩阵 $\begin{pmatrix} \cos \alpha & -\sin \alpha \\ \sin \alpha & \cos \alpha \end{pmatrix}$ 的特征值和特征向量。
注意,特征值(和特征向量)不一定必须是实数。

1.4. 计算以下矩阵的特征多项式和特征值:
$\begin{pmatrix} 1 & 2 & 5 & 67 \\ 0 & 2 & 3 & 6 \\ 0 & 0 & -2 & 5 \\ 0 & 0 & 0 & 3 \end{pmatrix}$, $\begin{pmatrix} 2 & 1 & 0 & 2 \\ 0 & \pi & 43 & 2 \\ 0 & 0 & 16 & 1 \\ 0 & 0 & 0 & 54 \end{pmatrix}$, $\begin{pmatrix} 4 & 0 & 0 & 0 \\ 1 & 3 & 0 & 0 \\ 2 & 4 & e & 0 \\ 3 & 3 & 1 & 1 \end{pmatrix}$, $\begin{pmatrix} 4 & 0 & 0 & 0 \\ 1 & 0 & 0 & 0 \\ 2 & 4 & 0 & 0 \\ 3 & 3 & 1 & 1 \end{pmatrix}$。
不要展开特征多项式,将其保留为乘积形式。

1.5. 证明三角矩阵的特征值(计入重数)与其对角线元素相等。

1.6. 称算子 $A$ 为\textbf{幂零}的,如果 $A^k = 0$ 对某个 $k$ 成立。证明如果 $A$ 是幂零的,那么 $\sigma(A) = \{0\}$(即 $0$ 是 $A$ 的唯一特征值)。

\section{对角化}
一个算子的谱理论的一个应用是对算子的\textbf{对角化},即给定一个算子,找到一个基,使得该算子在该基下的矩阵是对角矩阵。这样的基并非总能找到,也就是说,并非所有算子都能对角化(是可对角化的)。可对角化算子的重要性在于,对角矩阵的幂以及更一般的函数很容易计算,因此如果我们对一个算子进行对角化,我们就可以轻松地计算它的函数。我们将在本节中解释如何计算可对角化算子的函数。我们还将给出一个算子可对角化的充要条件,以及一些简单的充分条件。

此外,对于 $F^n$ 中的算子(矩阵),$A$ 的可对角化意味着它可以表示为 $A = SDS^{-1}$,其中 $D$ 是一个对角矩阵,$S$ 是一个可逆矩阵(两者都取值于 $F$);我们将在稍后解释这一点。除非另有说明,本节中的所有结果对于复数和实数向量空间都成立。

2.1. \textbf{预备知识。} 假设一个向量空间 $V$ 中的算子 $A$ 具有一个由 $A$ 的特征向量组成的基 $B = \{b_1, b_2, \dots, b_n\}$,其中 $\lambda_1, \lambda_2, \dots, \lambda_n$ 是相应的特征值。那么 $A$ 在此基下的矩阵是对角矩阵,对角线上是 $\lambda_1, \lambda_2, \dots, \lambda_n$:
$[A]_{BB} = \text{diag}\{\lambda_1, \lambda_2, \dots, \lambda_n\} = \begin{pmatrix} \lambda_1 & & & \\ & \lambda_2 & & \\ & & \ddots & \\ & & & \lambda_n \end{pmatrix}$ (2.1)
另一方面,如果一个算子 $A$ 在基 $B = \{b_1, b_2, \dots, b_n\}$ 下的矩阵由 (2.1) 给出,那么显然 $Ab_k = \lambda_k b_k$,即 $\lambda_k$ 是特征值,$b_k$ 是相应的特征向量。注意,上述推理对于复数和实数向量空间(甚至对于任意域 $F$ 上的向量空间)都成立。

将上述推理应用于 $F^n$ 中的算子(矩阵),我们立即得到以下定理。注意,虽然本书中 $F$ 是 $\mathbb{C}$ 或 $\mathbb{R}$,但本定理对任意域 $F$ 都成立。

\textbf{定理 2.1。} 一个矩阵 $A$(取值于 $F$)允许表示为 $A = SDS^{-1}$,其中 $D$ 是一个对角矩阵,$S$ 是一个可逆矩阵(两者都取值于 $F$),当且仅当存在 $F^n$ 的一个由 $A$ 的特征向量组成的基。而且,在这种情况下,$D$ 的对角线元素是特征值,而 $S$ 的列是相应的特征向量(第 $k$ 列对应于 $D$ 的第 $k$ 个对角线元素)。

\textbf{证明}~ 设 $D = \text{diag}\{\lambda_1, \lambda_2, \dots, \lambda_n\}$,让 $b_1, b_2, \dots, b_n$ 是 $S$ 的列(注意,由于 $S$ 可逆,它的列构成了 $F^n$ 的一个基)。那么恒等式 $A = SDS^{-1}$ 意味着 $D = S^{-1}AS = [I]_{SB}A[I]_{BS}$,其中 $S = [I]_{S,B}$ 是从 $B$ 到标准基 $S$ 的坐标变换矩阵。这正好意味着 $D = [A]_{BB}$。正如我们上面讨论的,当且仅当 $\lambda_k$ 是 $A$ 的特征值,$b_k$ 是 $A$ 的相应特征向量时,$[A]_{BB} = D = \text{diag}\{\lambda_1, \lambda_2, \dots, \lambda_n\}$。

\textbf{注释} ~注意,如果一个矩阵允许表示为 $A = SDS^{-1}$ 并且 $D$ 是一个对角矩阵,那么通过简单的直接计算可以表明,$S$ 的列是 $A$ 的特征向量,而 $D$ 的对角线元素是相应的特征值。这为定理 2.1 中相应陈述提供了另一种证明。



正如我们上面讨论的,一个可对角化的算子 $A: V \to V$ 恰好有 $n = \dim V$ 个特征值(计入重数);一个复数向量空间中的算子恰好有 $n$ 个特征值(计入重数);另一方面,一个实数空间中的算子可能没有实数特征值。我们将遵循谱理论中的惯例,将实数矩阵视为 $\mathbb{C}^n$ 中的算子,从而允许复数特征值和特征向量。除非另有说明,我们将通过对矩阵的对角化来理解其\textbf{复数}对角化,即表示 $A = SDS^{-1}$,其中矩阵 $S$ 和 $D$ 可以有复数项。一个实数矩阵何时允许实数对角化($A = SDS^{-1}$,其中 $S$ 和 $D$ 都是实数矩阵)的问题,实际上是一个非常简单的问题,见下面的定理 2.9。

2.2. \textbf{一些动机:算子函数。} 设一个算子 $A$ 在基 $B = \{b_1, b_2, \dots, b_n\}$ 下的矩阵是 (2.1) 中给出的对角矩阵。那么很容易找到算子 $A$ 的 $N$ 次幂。即,在基 $B$ 下 $A^N$ 的矩阵是 $[A^N]_{BB} = \text{diag}\{\lambda_1^N, \lambda_2^N, \dots, \lambda_n^N\} = \begin{pmatrix} \lambda_1^N & & & \\ & \lambda_2^N & & \\ & & \ddots & \\ & & & \lambda_n^N \end{pmatrix}$。
而且,算子的函数也相对容易计算:例如,算子(矩阵)指数 $e^{tA}$ 定义为 $e^{tA} = I + tA + \frac{t^2 A^2}{2!} + \frac{t^3 A^3}{3!} + \dots = \sum_{k=0}^\infty \frac{t^k A^k}{k!}$,并且它在基 $B$ 下的矩阵是 $[e^{tA}]_{BB} = \text{diag}\{e^{\lambda_1 t}, e^{\lambda_2 t}, \dots, e^{\lambda_n t}\} = \begin{pmatrix} e^{\lambda_1 t} & & & \\ & e^{\lambda_2 t} & & \\ & & \ddots & \\ & & & e^{\lambda_n t} \end{pmatrix}$。

设 $A$ 是 $F^n$ 中的一个算子。为了在标准基 $S$ 下找到算子 $A^N$ 和 $e^{tA}$ 的矩阵,我们需要回忆一下坐标变换矩阵 $[I]_{SB}$ 是一个以 $b_1, b_2, \dots, b_n$ 为列的矩阵。设这个矩阵为 $S$,那么根据坐标变换公式我们得到
$A = [A]_{SS} = S \begin{pmatrix} \lambda_1 & & & \\ & \lambda_2 & & \\ & & \ddots & \\ & & & \lambda_n \end{pmatrix} S^{-1} = SDS^{-1}$,
其中我们使用 $D$ 来表示中间的对角矩阵。类似地,$A^N = SD^N S^{-1} = S \begin{pmatrix} \lambda_1^N & & & \\ & \lambda_2^N & & \\ & & \ddots & \\ & & & \lambda_n^N \end{pmatrix} S^{-1}$,
并且对于 $e^{tA}$ 类似。

另一种思考可对角化算子的幂(或其他函数)的方式是看到,如果算子 $A$ 可以表示为 $A = SDS^{-1}$,那么
$A^N = (SDS^{-1})(SDS^{-1})\dots(SDS^{-1}) \underbrace{N \text{ times}} = SD^N S^{-1}$
并且计算对角矩阵的 $N$ 次幂很容易。

2.3. \textbf{$n$ 个不同特征值的情况。} 我们现在给出一个算子可对角化的非常简单的\textbf{充分}条件,见下面的推论 2.3。

\textbf{定理 2.2。} 设 $\lambda_1, \lambda_2, \dots, \lambda_r$ 是 $A$ 的不同特征值,设 $v_1, v_2, \dots, v_r$ 是相应的特征向量。那么向量 $v_1, v_2, \dots, v_r$ 是线性无关的。

\textbf{证明}~ 我们将使用数学归纳法处理 $r$。$r=1$ 的情况是平凡的,因为根据定义,特征向量是非零的,并且由一个非零向量组成的系统是线性无关的。假设定理的陈述对 $r-1$ 是正确的。假设存在一个非平凡线性组合
$c_1 v_1 + c_2 v_2 + \dots + c_r v_r = \sum_{k=1}^r c_k v_k = 0$。
将 $(A - \lambda_r I)$ 应用于 (2.2) 并利用 $(A - \lambda_r I)v_r = 0$ 的事实,我们得到
$\sum_{k=1}^{r-1} c_k (\lambda_k - \lambda_r) v_k = 0$。
根据归纳假设,向量 $v_1, v_2, \dots, v_{r-1}$ 是线性无关的,所以 $c_k(\lambda_k - \lambda_r) = 0$ 对于 $k=1, 2, \dots, r-1$ 成立。由于 $\lambda_k \neq \lambda_r$,我们可以得出 $c_k = 0$ 对于 $k < r$。然后从 (2.2) 可知 $c_r = 0$,也就是说我们得到了平凡线性组合。

\textbf{推论 2.3。} 如果一个算子 $A: V \to V$ 恰好有 $n = \dim V$ 个不同的特征值,那么它是可对角化的。

\textbf{证明}~ 对于每个特征值 $\lambda_k$,设 $v_k$ 是一个相应的特征向量(为每个特征值只选取一个特征向量)。根据定理 2.2,系统 $\{v_1, v_2, \dots, v_n\}$ 是线性无关的,并且由于它恰好包含 $n = \dim V$ 个向量,所以它是一个基。



2.4. \textbf{子空间的基(又名子空间的直和)。} 为了描述可对角化算子,我们需要引入一些新定义。设 $V_1, V_2, \dots, V_p$ 是向量空间 $V$ 的子空间。我们说子空间的系统是 $V$ 的一个基,如果任何向量 $v \in V$ 都存在唯一的表示为和
$v = v_1 + v_2 + \dots + v_p = \sum_{k=1}^p v_k$, $v_k \in V_k$。(2.3)
我们也说,子空间的系统 $\{V_1, V_2, \dots, V_p\}$ 是线性无关的,如果方程 $v_1 + v_2 + \dots + v_p = 0$, $v_k \in V_k$ 只有平凡解 ($v_k = 0, \forall k = 1, 2, \dots, p$)。另一种表述方式是,子空间的系统 $\{V_1, V_2, \dots, V_p\}$ 是线性无关的,当且仅当任何由非零向量 $v_k$ ($v_k \in V_k$) 组成的系统是线性无关的。我们说子空间的系统 $\{V_1, V_2, \dots, V_p\}$ 是生成(或完备,或张成)的,如果任何向量 $v \in V$ 可以表示为 (2.3)(不一定是唯一的)。

\textbf{注 2.4。} 从上述定义可以立即看出,定理 2.2 实际上表明,算子 $A$ 的特征子空间 $E_k := \text{Ker}(A - \lambda_k I)$, $\lambda_k \in \sigma(A)$ 的系统是线性无关的。

\textbf{注 2.5。} 很容易看出,与向量基类似,子空间的系统 $\{V_1, V_2, \dots, V_p\}$ 是一个基当且仅当它既是生成集又是线性无关的。我们将此事实的证明留给读者作为练习。

有一个子空间基的简单例子。设 $V$ 是一个向量空间,有一个基 $\{v_1, v_2, \dots, v_n\}$。将索引集 $\{1, 2, \dots, n\}$ 分为 $p$ 个子集 $\Lambda_1, \Lambda_2, \dots, \Lambda_p$,并定义子空间 $V_k := \text{span}\{v_j : j \in \Lambda_k\}$。显然,子空间 $V_k$ 构成 $V$ 的一个基。下面的定理表明,在有限维情况下,这基本上是子空间基唯一可能的例子。

\textbf{定理 2.6。} 设 $\{V_1, V_2, \dots, V_p\}$ 是一个子空间基,并且在每个子空间 $V_k$ 中都有一个基(向量基)$B_k$。$^2$ 那么这些基的并集 $\cup_k B_k$ 是 $V$ 的一个基。

为了证明定理,我们需要以下引理:

\textbf{引理 2.7。} 设 $\{V_1, V_2, \dots, V_p\}$ 是一个线性无关的子空间族,并且在每个子空间 $V_k$ 中都有一个线性无关的向量系统 $B_k$。$^3$ 那么这些基的并集 $B := \cup_k B_k$ 是一个线性无关的系统。

\textbf{证明}~ 如果稍微思考一下,引理的证明几乎是微不足道的。书写证明的主要困难在于选择合适的记号。为了避免使用两个索引(一个表示 $k$,另一个表示 $B_k$ 中向量的编号),让我们使用“扁平化”的记号。即,设 $n$ 是 $B := \cup_k B_k$ 中向量的数量。让我们对 $B$ 中的向量集进行排序,例如如下:首先列出 $B_1$ 中的所有向量,然后是 $B_2$ 中的所有向量,依此类推,最后列出 $B_p$ 中的所有向量。这样,我们将 $B$ 中的所有向量用整数 $1, 2, \dots, n$ 索引,并且索引集 $\{1, 2, \dots, n\}$ 被分成集合 $\Lambda_1, \Lambda_2, \dots, \Lambda_p$,使得集合 $B_k$ 由向量 $\{b_j : j \in \Lambda_k\}$ 组成。假设我们有一个非平凡的线性组合
$c_1 b_1 + c_2 b_2 + \dots + c_n b_n = \sum_{j=1}^n c_j b_j = 0$。(2.4)
设 $v_k := \sum_{j \in \Lambda_k} c_j b_j$。那么 (2.4) 可以重写为
$v_1 + v_2 + \dots + v_p = 0$。
$^2$ 我们不具体列出 $B_k$ 中的向量,只需记住每个 $B_k$ 都包含有限数量的向量。
$^3$ 同样,这里我们不单独命名 $B_k$ 中的每个向量,我们只是记住每个集合 $B_k$ 都包含有限数量的向量。



由于 $v_k \in V_k$ 并且子空间 $\{V_1, V_2, \dots, V_p\}$ 是线性无关的,所以 $v_k = 0$ $\forall k$。这意味着对于每个 $k$,$\sum_{j \in \Lambda_k} c_j b_j = 0$,并且由于向量系统 $\{b_j : j \in \Lambda_k\}$(即系统 $B_k$)是线性无关的,我们得到 $c_j = 0$ 对于所有 $j \in \Lambda_k$。由于这对所有 $\Lambda_k$ 都成立,我们可以得出 $c_j = 0$ 对于所有 $j$。

\textbf{定理 2.6 的证明。} 为了证明定理,我们将使用与引理 2.7 证明相同的记号,即系统 $B_k$ 由向量 $\{b_j : j \in \Lambda_k\}$ 组成。引理 2.7 断言向量系统 $\{b_j : j = 1, 2, \dots, n\}$ 是线性无关的,所以剩下的就是证明这个系统是完备的。由于子空间系统 $\{V_1, V_2, \dots, V_p\}$ 是一个基,任何向量 $v \in V$ 可以表示为
$v = v_1 + v_2 + \dots + v_p = \sum_{k=1}^p v_k$, $v_k \in V_k$。
由于向量 $\{b_j : j \in \Lambda_k\}$ 构成了 $V_k$ 的基,向量 $v_k$ 可以表示为 $v_k = \sum_{j \in \Lambda_k} c_j b_j$。因此,$v = \sum_{j=1}^n c_j b_j$。

2.5. \textbf{可对角化判据。} 首先,让我们回顾一个必要的条件。由于对角矩阵 $D = \text{diag}\{\lambda_1, \lambda_2, \dots, \lambda_n\}$ 的特征值(计入重数)恰好是 $\lambda_1, \lambda_2, \dots, \lambda_n$,我们发现如果一个算子 $A: V \to V$ 是可对角化的,它恰好有 $n = \dim V$ 个特征值(计入重数)。下面的定理对实数和复数向量空间都成立(甚至对任意域上的空间也成立)。

\textbf{定理 2.8。} 设一个算子 $A: V \to V$ 恰好有 $n = \dim V$ 个特征值(计入重数)$^4$。那么 $A$ 是可对角化的当且仅当对于每个特征值 $\lambda$,特征子空间 $\text{Ker}(A - \lambda I)$ 的维数(即几何重数)等于 $\lambda$ 的代数重数。

\textbf{证明}~ 首先,我们注意到,对于一个对角矩阵,特征值的代数重数和几何重数是相等的,因此对于可对角化算子也是如此。
$^4$ 由于任何复数向量空间中的算子都恰好有 $n$ 个特征值(计入重数),因此在复数情况下,此假设是多余的。


现在我们来证明另一个蕴含关系。设 $\lambda_1, \lambda_2, \dots, \lambda_p$ 是 $A$ 的特征值,设 $E_k := \text{Ker}(A - \lambda_k I)$ 是相应的特征子空间。根据注 2.4,子空间 $\{E_k\}_{k=1}^p$ 是线性无关的。设 $B_k$ 是 $E_k$ 的一个基。根据引理 2.7,向量系统 $B := \cup_k B_k$ 是一个线性无关系统。我们知道每个 $B_k$ 由 $\dim E_k$(即 $\lambda_k$ 的重数)个向量组成。所以 $B$ 中的向量数量等于特征值 $\lambda_k$ 的重数之和。但是特征值重数之和就是计入重数的特征值数量,这恰好是 $n = \dim V$。因此,我们得到了一个 $n = \dim V$ 个线性无关的特征向量组成的系统,这意味着它是一个基。

2.6. \textbf{实数分解。} 下面的定理实际上已经证明过了(它本质上是定理 2.8 在实数空间上的情况)。我们在此陈述是为了总结实数矩阵实数对角化的情形。

\textbf{定理 2.9。} 一个实数 $n \times n$ 矩阵 $A$ 允许实数分解(即表示为 $A = SDS^{-1}$,其中 $S$ 和 $D$ 是实数矩阵,$D$ 是对角矩阵且 $S$ 可逆)当且仅当它允许复数分解并且 $A$ 的所有特征值都是实数。

2.7. \textbf{一些例子。}
2.7.1. \textbf{实数特征值。} 考虑矩阵 $A = \begin{pmatrix} 1 & 2 \\ 8 & 1 \end{pmatrix}$。它的特征多项式等于 $|\begin{vmatrix} 1-\lambda & 2 \\ 8 & 1-\lambda \end{vmatrix}| = (1-\lambda)^2 - 16$,其根(特征值)是 $\lambda = 5$ 和 $\lambda = -3$。对于特征值 $\lambda = 5$, $A - 5I = \begin{pmatrix} 1-5 & 2 \\ 8 & 1-5 \end{pmatrix} = \begin{pmatrix} -4 & 2 \\ 8 & -4 \end{pmatrix}$。其零空间的基由一个向量 $(1, 2)^T$ 构成,所以这是对应的特征向量。类似地,对于 $\lambda = -3$, $A - \lambda I = A + 3I = \begin{pmatrix} 1+3 & 2 \\ 8 & 1+3 \end{pmatrix} = \begin{pmatrix} 4 & 2 \\ 8 & 4 \end{pmatrix}$。
Ker$(A + 3I)$ 的零空间由向量 $(1, -2)^T$ 张成,所以这是对应的特征向量。矩阵 $A$ 可以被对角化为
$A = \begin{pmatrix} 1 & 1 \\ 2 & -2 \end{pmatrix} \begin{pmatrix} 5 & 0 \\ 0 & -3 \end{pmatrix} \begin{pmatrix} 1 & 1 \\ 2 & -2 \end{pmatrix}^{-1}$。

2.7.2. \textbf{复数特征值。} 考虑矩阵 $A = \begin{pmatrix} 1 & 2 \\ -2 & 1 \end{pmatrix}$。其特征多项式是 $|\begin{vmatrix} 1-\lambda & 2 \\ -2 & 1-\lambda \end{vmatrix}| = (1-\lambda)^2 + 4$,特征值(特征多项式的根)是 $\lambda = 1 \pm 2i$。对于 $\lambda = 1 + 2i$, $A - \lambda I = \begin{pmatrix} 1-(1+2i) & 2 \\ -2 & 1-(1+2i) \end{pmatrix} = \begin{pmatrix} -2i & 2 \\ -2 & -2i \end{pmatrix}$。
这个矩阵的秩是 1,所以特征子空间 $\text{Ker}(A - \lambda I)$ 由一个向量,例如 $(1, i)^T$ 张成。由于矩阵 $A$ 是实数的,我们不需要计算 $\lambda = 1 - 2i$ 的特征向量:通过取上述特征向量的复共轭,我们可以免费获得它,见下面的练习 2.2。所以,对于 $\lambda = 1 - 2i$,一个相应的特征向量是 $(1, -i)^T$,因此矩阵 $A$ 可以被对角化为
$A = \begin{pmatrix} 1 & 1 \\ i & -i \end{pmatrix} \begin{pmatrix} 1+2i & 0 \\ 0 & 1-2i \end{pmatrix} \begin{pmatrix} 1 & 1 \\ i & -i \end{pmatrix}^{-1}$。

2.7.3. \textbf{一个不可对角化的矩阵。} 考虑矩阵 $A = \begin{pmatrix} 1 & 1 \\ 0 & 1 \end{pmatrix}$。其特征多项式是 $|\begin{vmatrix} 1-\lambda & 1 \\ 0 & 1-\lambda \end{vmatrix}| = (1-\lambda)^2$,所以 $A$ 有一个重数为 2 的特征值 1。然而,很容易看出 $\dim \text{Ker}(A - I) = 1$(1个主元,所以 $2-1=1$ 个自由变量)。因此,特征值 1 的几何重数与其代数重数不同,所以 $A$ 是不可对角化的。还有一个不使用定理 2.8 的解释。即,我们得到特征子空间 $\text{Ker}(A - I)$ 是一维的(由向量 $(1, 0)^T$ 张成)。如果 $A$ 是可对角化的,那么它将在某个基下具有对角形式 $\begin{pmatrix} 1 & 0 \\ 0 & 1 \end{pmatrix}$$^5$,因此特征子空间的维数将是 2。所以 $A$ 不能被对角化。

$^5$ 注意,唯一具有某种基下矩阵为 $\begin{pmatrix} 1 & 0 \\ 0 & 1 \end{pmatrix}$ 的线性变换是恒等变换 $I$。由于 $A$ 肯定不是恒等变换,我们可以立即得出 $A$ 不能被对角化,所以计算特征子空间的维数是不必要的。


练习。

2.1. 设 $A$ 是 $n \times n$ 矩阵。真或假:
a) $A^T$ 与 $A$ 具有相同的特征值。
b) $A^T$ 与 $A$ 具有相同的特征向量。
c) 如果 $A$ 是可对角化的,那么 $A^T$ 也是可对角化的。
证明你的结论。

2.2. 设 $A$ 是一个实数方阵,$\lambda$ 是它的一个复数特征值。假设 $v = (v_1, v_2, \dots, v_n)^T$ 是一个相应的特征向量,$Av = \lambda v$。证明 $\bar{\lambda}$ 是 $A$ 的一个特征值,并且 $\bar{v}$ 是 $A$ 的相应特征向量。这里 $\bar{v}$ 是向量 $v$ 的复共轭,$\bar{v} := (\bar{v}_1, \bar{v}_2, \dots, \bar{v}_n)^T$。

2.3. 设 $A = \begin{pmatrix} 4 & 3 \\ 1 & 2 \end{pmatrix}$。通过对 $A$ 进行对角化,求出 $A^{2004}$。

2.4. 构建一个特征值为 1 和 3,相应特征向量为 $(1, 2)^T$ 和 $(1, 1)^T$ 的矩阵 $A$。这样的矩阵是唯一的吗?

2.5. 对以下矩阵进行对角化,如果可能:
a) $\begin{pmatrix} 4 & -2 \\ 1 & 1 \end{pmatrix}$。
b) $\begin{pmatrix} -1 & -1 \\ 6 & 4 \end{pmatrix}$。
c) $\begin{pmatrix} -2 & 2 & 6 \\ 5 & 1 & -6 \\ -5 & 2 & 9 \end{pmatrix}$ ($\lambda = 2$ 是其中一个特征值)

2.6. 考虑矩阵 $A = \begin{pmatrix} 2 & 6 & -6 \\ 0 & 5 & -2 \\ 0 & 0 & 4 \end{pmatrix}$。
a) 求它的特征值。在不计算的情况下能否求出特征值?
b) 这个矩阵可对角化吗?在不进行计算的情况下找出答案。
c) 如果矩阵可对角化,请对其进行对角化。



2.7. 对矩阵 $\begin{pmatrix} 2 & 0 & 6 \\ 0 & 2 & 4 \\ 0 & 0 & 4 \end{pmatrix}$ 进行对角化。

2.8. 求矩阵 $A = \begin{pmatrix} 5 & 2 \\ -3 & 0 \end{pmatrix}$ 的所有平方根,即求所有满足 $B^2 = A$ 的矩阵 $B$。
\textbf{提示:} 求对角矩阵的平方根很容易。你可以将答案留作乘积形式。

2.9. 回顾一下著名的斐波那契数列:0, 1, 1, 2, 3, 5, 8, 13, 21, ...,它由以下方式定义:令 $\phi_0 = 0$, $\phi_1 = 1$,并定义 $\phi_{n+2} = \phi_{n+1} + \phi_n$。我们想找到 $\phi_n$ 的一个公式。
a) 找到一个 $2 \times 2$ 矩阵 $A$,使得 $\begin{pmatrix} \phi_{n+2} \\ \phi_{n+1} \end{pmatrix} = A \begin{pmatrix} \phi_{n+1} \\ \phi_n \end{pmatrix}$。
\textbf{提示:} 结合平凡方程 $\phi_{n+1} = \phi_{n+1}$ 和斐波那契关系 $\phi_{n+2} = \phi_{n+1} + \phi_n$。
b) 对 $A$ 进行对角化,并找到 $A^n$ 的一个公式。
c) 注意到 $\begin{pmatrix} \phi_{n+1} \\ \phi_n \end{pmatrix} = A^n \begin{pmatrix} \phi_1 \\ \phi_0 \end{pmatrix} = A^n \begin{pmatrix} 1 \\ 0 \end{pmatrix}$,找到 $\phi_n$ 的一个公式。(你需要计算一个逆矩阵并进行乘法。)
d) 证明向量 $(\phi_{n+1}/\phi_n, 1)^T$ 收敛到一个 $A$ 的特征向量。你认为这是一个巧合吗?

2.10. 设 $A$ 是一个 $5 \times 5$ 矩阵,有 3 个特征值(不计重数)。假设我们知道其中一个特征子空间是三维的。你能说 $A$ 是否可对角化吗?

2.11. 给出一个 $3 \times 3$ 矩阵的例子,它不能被对角化。在构造了矩阵之后,你能使它“通用”一些,使得矩阵的特殊结构不明显吗?

2.12. 设一个非零矩阵 $A$ 满足 $A^5 = 0$。证明 $A$ 不能被对角化。更一般地说,任何非零幂零矩阵,即满足 $A^N = 0$ 对某个 $N$ 的矩阵,都不能被对角化。

2.13. \textbf{转置的特征值:}
a) 考虑 $2 \times 2$ 矩阵空间 $M_{2 \times 2}$ 上的变换 $T(A) = A^T$。找出它所有的特征值和特征向量。这个变换可能被对角化吗?
\textbf{提示:} 虽然可以写出这个线性变换在某个基下的矩阵,计算特征多项式等等,但直接从定义中找出特征值和特征向量会更容易。
b) 在 $n \times n$ 矩阵空间中,能否做同样的问题?

2.14. 证明两个子空间 $V_1$ 和 $V_2$ 是线性无关的当且仅当 $V_1 \cap V_2 = \{0\}$。





% \chapter{内积空间}

\section{$\mathbb{R}^n$ 与 $\mathbb{C}^n$ 中的内积~内积空间}

\section{正交性~正交与标准正交基}

\section{正交投影和格拉姆-施密特正交化}
回想一下二维平面几何中正交投影的定义,人们可以引入以下定义。设 $E$ 是内积空间 $V$ 的一个子空间。

\textbf{定义 3.1。} 对于向量 $v$,它到子空间 $E$ 的\textbf{正交投影} $P_E v$ 是一个向量 $w$,满足:
1. $w \in E$;
2. $v - w \perp E$。
我们将使用记号 $w = P_E v$ 表示正交投影。在引入一个对象之后,自然要问:1. 该对象是否存在?2. 该对象是否唯一?3. 如何找到它?我们将首先证明投影是唯一的。然后我们将给出一个找到投影的方法,证明它的存在。下面的定理表明了为什么正交投影很重要,同时也证明了它的唯一性。

\textbf{定理 3.2。} 正交投影 $w = P_E v$ 使 $v$ 到 $E$ 的距离最小化,即对于所有 $x \in E$,$\|v - w\| \leq \|v - x\|$。而且,如果对于某个 $x \in E$,$\|v - w\| = \|v - x\|$,则 $x = w$。

\textbf{证明}~ 设 $y = w - x$。那么 $v - x = v - w + w - x = v - w + y$。由于 $v - w \perp E$,所以 $y \perp v - w$,因此根据勾股定理 $\|v - x\|^2 = \|v - w\|^2 + \|y\|^2 \geq \|v - w\|^2$。注意,当且仅当 $y = 0$,即 $x = w$ 时,等号成立。

下面的命题表明,如果我们知道 $E$ 中的一个正交基,我们就能找到 $E$ 上的正交投影。

\textbf{命题 3.3。} 设 $\{v_1, v_2, \dots, v_r\}$ 是 $E$ 的一个正交基。那么向量 $v$ 到 $E$ 的正交投影 $P_E v$ 由以下公式给出:
$P_E v = \sum_{k=1}^r \alpha_k v_k$, 其中 $\alpha_k = \frac{(v, v_k)}{\|v_k\|^2}$。
换句话说,
(3.1) $P_E v = \sum_{k=1}^r \frac{(v, v_k)}{\|v_k\|^2} v_k$。
注意,$\alpha_k$ 的公式与 (2.1) 重合,即这个公式应用于一个正交系统(而不是基)可以得到其张成的子空间上的投影。

\textbf{3.3 的证明。} 设 $w := \sum_{k=1}^r \alpha_k v_k$, 其中 $\alpha_k = \frac{(v, v_k)}{\|v_k\|^2}$。我们想证明 $v - w \perp E$。根据引理 2.3,只要证明 $v - w \perp v_k$, $\forall k = 1, 2, \dots, n$ 即可。计算内积,我们得到对于 $k = 1, 2, \dots, r$:
$(v - w, v_k) = (v, v_k) - (w, v_k) = (v, v_k) - (\sum_{j=1}^r \alpha_j v_j, v_k) = (v, v_k) - \alpha_k (v_k, v_k) = (v, v_k) - \frac{(v, v_k)}{\|v_k\|^2} \|v_k\|^2 = 0$。

因此,如果我们知道 $E$ 中的一个正交基,我们就可以找到到 $E$ 的正交投影。特别地,由于任何只包含一个向量的系统都是正交系统,我们知道如何进行到一维空间的上的正交投影。但是如果我们只知道 $E$ 的一个基,我们该如何找到正交投影呢?幸运的是,存在一个简单的算法,可以从一个基得到一个正交基。

3.1. \textbf{格拉姆-施密特正交化算法。} 假设我们有一个线性无关系统 $\{x_1, x_2, \dots, x_n\}$。格拉姆-施密特方法从这个系统构造一个正交系统 $\{v_1, v_2, \dots, v_n\}$,使得 $\text{span}\{x_1, x_2, \dots, x_n\} = \text{span}\{v_1, v_2, \dots, v_n\}$。而且,对于所有 $r \leq n$,我们得到 $\text{span}\{x_1, x_2, \dots, x_r\} = \text{span}\{v_1, v_2, \dots, v_r\}$。
现在我们来描述这个算法。

\textbf{步骤 1。} 令 $v_1 := x_1$。记 $E_1 := \text{span}\{x_1\} = \text{span}\{v_1\}$。
\textbf{步骤 2。} 定义 $v_2$ 为
$v_2 = x_2 - P_{E_1} x_2 = x_2 - \frac{(x_2, v_1)}{\|v_1\|^2} v_1$。
定义 $E_2 = \text{span}\{v_1, v_2\}$。注意 $\text{span}\{x_1, x_2\} = E_2$。
\textbf{步骤 3。} 定义 $v_3$ 为
$v_3 := x_3 - P_{E_2} x_3 = x_3 - \frac{(x_3, v_1)}{\|v_1\|^2} v_1 - \frac{(x_3, v_2)}{\|v_2\|^2} v_2$。
令 $E_3 := \text{span}\{v_1, v_2, v_3\}$。注意 $\text{span}\{x_1, x_2, x_3\} = E_3$。还注意 $x_3 \notin E_2$ 所以 $v_3 \neq 0$。
...
\textbf{步骤 $r+1$。} 假设我们已经完成了过程的 $r$ 步,构造了一个正交系统(包含非零向量)$\{v_1, v_2, \dots, v_r\}$,使得 $E_r := \text{span}\{v_1, v_2, \dots, v_r\} = \text{span}\{x_1, x_2, \dots, x_r\}$。定义
$v_{r+1} := x_{r+1} - P_{E_r} x_{r+1} = x_{r+1} - \sum_{k=1}^r \frac{(x_{r+1}, v_k)}{\|v_k\|^2} v_k$。
注意 $x_{r+1} \notin E_r$ 所以 $v_{r+1} \neq 0$。
...
通过继续这个算法,我们将得到一个正交系统 $\{v_1, v_2, \dots, v_n\}$。

3.2. \textbf{一个例子。} 假设我们有向量 $x_1 = (1, 1, 1)^T$, $x_2 = (0, 1, 2)^T$, $x_3 = (1, 0, 2)^T$,我们想通过格拉姆-施密特来正交化它们。第一步定义 $v_1 = x_1 = (1, 1, 1)^T$。
第二步我们得到 $v_2 = x_2 - P_{E_1} x_2 = x_2 - \frac{(x_2, v_1)}{\|v_1\|^2} v_1$。
计算 $(x_2, v_1) = (\begin{pmatrix} 0 \\ 1 \\ 2 \end{pmatrix}, \begin{pmatrix} 1 \\ 1 \\ 1 \end{pmatrix}) = 3$, $\|v_1\|^2 = 3$,我们得到
$v_2 = \begin{pmatrix} 0 \\ 1 \\ 2 \end{pmatrix} - \frac{3}{3} \begin{pmatrix} 1 \\ 1 \\ 1 \end{pmatrix} = \begin{pmatrix} -1 \\ 0 \\ 1 \end{pmatrix}$。
最后,定义 $v_3 = x_3 - P_{E_2} x_3 = x_3 - \frac{(x_3, v_1)}{\|v_1\|^2} v_1 - \frac{(x_3, v_2)}{\|v_2\|^2} v_2$。
计算 $(\begin{pmatrix} 1 \\ 0 \\ 2 \end{pmatrix}, \begin{pmatrix} 1 \\ 1 \\ 1 \end{pmatrix}) = 3$, $(\begin{pmatrix} 1 \\ 0 \\ 2 \end{pmatrix}, \begin{pmatrix} -1 \\ 0 \\ 1 \end{pmatrix}) = 1$, $\|v_1\|^2 = 3$, $\|v_2\|^2 = 2$ ( $\|v_1\|^2$ 已经计算过了) 我们得到
$v_3 = \begin{pmatrix} 1 \\ 0 \\ 2 \end{pmatrix} - \frac{3}{3} \begin{pmatrix} 1 \\ 1 \\ 1 \end{pmatrix} - \frac{1}{2} \begin{pmatrix} -1 \\ 0 \\ 1 \end{pmatrix} = \begin{pmatrix} 1 \\ 0 \\ 2 \end{pmatrix} - \begin{pmatrix} 1 \\ 1 \\ 1 \end{pmatrix} - \begin{pmatrix} -1/2 \\ 0 \\ 1/2 \end{pmatrix} = \begin{pmatrix} 1/2 \\ -1 \\ 1/2 \end{pmatrix}$。

\textbf{注释} ~由于乘以标量不改变正交性,因此可以乘以任意非零数来得到由格拉姆-施密特得到的向量 $v_k$。特别地,在许多理论构造中,人们通过将向量 $v_k$ 除以它们各自的范数 $\|v_k\|$ 来\textbf{归一化}它们。然后得到的结果系统将是标准正交的,并且公式会更简单。另一方面,在进行计算时,人们可能希望避免分数项,方法是将向量乘以其元素最小公分母的倒数。因此,人们可能希望将上面例子中的向量 $v_3$ 替换为 $(1, -2, 1)^T$。

3.3. \textbf{正交补。分解 $V = E \oplus E^\perp$。}
\textbf{定义}~ 对于子空间 $E$,其\textbf{正交补} $E^\perp$ 是所有与 $E$ 正交的向量的集合,$E^\perp := \{x : x \perp E\}$。
如果 $x, y \perp E$,则对于任意线性组合 $\alpha x + \beta y \perp E$(你能看出为什么吗?)。因此 $E^\perp$ 是一个子空间。根据正交投影的定义,任何内积空间 $V$ 中的向量都可以唯一地表示为 $v = v_1 + v_2$, $v_1 \in E$, $v_2 \perp E$(等价地,$v_2 \in E^\perp$)(其中显然 $v_1 = P_E v$)。这个陈述通常被象征性地写成 $V = E \oplus E^\perp$,这意味着任何向量都可以进行上述唯一的分解。以下命题给出了正交补的一个重要性质。

\textbf{命题 3.6。} 对于子空间 $E$,$(E^\perp)^\perp = E$。
证明留给读者作为练习,见下面的练习 3.12。

\textbf{练习}~

3.1. 将向量 $(1, 2, -2)^T$, $(1, -1, 4)^T$, $(2, 1, 1)^T$ 应用于格拉姆-施密特正交化。

3.2. 将向量 $(1, 2, 3)^T$, $(1, 3, 1)^T$ 应用于格拉姆-施密特正交化。写出到由这两个向量张成的二维子空间的\textbf{正交投影}矩阵。



3.3. 将上一个问题中得到的正交系统补全为 $\mathbb{R}^3$ 中的一个正交基,即向系统中添加一些向量(多少个?)以得到一个正交基。你能描述如何将一个正交系统补全为一般情况 $\mathbb{R}^n$ 或 $\mathbb{C}^n$ 中的一个正交基吗?

3.4. 求向量 $(2, 3, 1)^T$ 到由向量 $(1, 2, 3)^T$, $(1, 3, 1)^T$ 张成的子空间的距离。注意,我只要求计算到子空间的距离,而不是正交投影。

3.5. 找到向量 $(1, 1, 1, 1)^T$ 到由向量 $v_1 = (1, 3, 1, 1)^T$ 和 $v_2 = (2, -1, 1, 0)^T$ 张成的子空间的\textbf{正交投影}(注意 $v_1 \perp v_2$)。

3.6. 求向量 $(1, 2, 3, 4)^T$ 到由向量 $v_1 = (1, -1, 1, 0)^T$ 和 $v_2 = (1, 2, 1, 1)^T$ 张成的子空间的距离(注意 $v_1 \perp v_2$)。能否在不实际计算投影的情况下找到距离?这将简化计算。

3.7. 真或假:如果 $E$ 是 $V$ 的子空间,则 $\dim E + \dim(E^\perp) = \dim V$?证明你的结论。

3.8. 设 $P$ 是到子空间 $E$ 的正交投影,$\dim V = n$, $\dim E = r$。找出它的特征值和特征向量(特征子空间)。找出每个特征值的代数重数和几何重数。

3.9. (使用特征值计算行列式)。a) 求到由向量 $(1, 1, \dots, 1)^T$ 张成的一维子空间的\textbf{正交投影}矩阵;b) 设 $A$ 是一个主对角线全为 1,其他所有元素都为 1 的 $n \times n$ 矩阵。计算它的特征值和重数(使用上一个问题);c) 计算矩阵 $A-I$(即主对角线全为零,其他所有元素都为 1 的矩阵)的特征值(和重数);d) 计算 $\det(A-I)$。

3.10. (勒让德多项式):设内积在多项式空间上由 $(f, g) = \int_{-1}^1 f(t)g(t)dt$ 定义。将格拉姆-施密特正交化应用于系统 $\{1, t, t^2, t^3\}$。勒让德多项式是所谓的正交多项式的特例,它们在数学的许多分支中起着重要作用。

3.11. 设 $P$ 是到子空间 $E$ 的正交投影。证明:
a) 矩阵 $P$ 是\textbf{自伴随}的,即 $P^* = P$。
b) $P^2 = P$。
\textbf{注:} 以上 2 个性质完全刻画了正交投影,即满足这些性质的任何矩阵都是某个正交投影的矩阵。我们稍后将讨论这一点。



3.12. 证明对于子空间 $E$,有 $(E^\perp)^\perp = E$。
\textbf{提示:} 很容易看出 $E$ 正交于 $E^\perp$(为什么?)。为了证明任何正交于 $E^\perp$ 的向量 $x$ 属于 $E$,使用上面第 3.3 节中的分解 $V = E \oplus E^\perp$。

3.13. 假设 $P$ 是到子空间 $E$ 的正交投影,而 $Q$ 是到其正交补 $E^\perp$ 的正交投影。
a) $P+Q$ 和 $PQ$ 是什么?
b) 证明 $P-Q$ 是它自己的逆。




正如第 2 章第 2 节所讨论的,方程 $Ax = b$ 有解当且仅当 $b \in \text{Ran } A$。但对于没有解的方程该怎么办?这似乎是一个愚蠢的问题,因为如果没有解,那么就没有解。但是,当我们要解一个没有解的方程时,情况可能会自然地出现,例如,如果我们从实验中得到了方程。如果我们没有错误,那么右侧 $b$ 属于列空间 $\text{Ran } A$,方程是相容的。但是在现实生活中,无法避免测量误差,所以一个理论上应该相容的方程可能没有解。那么,在这种情况下我们能做什么?

\section{最小二乘解} 

最简单的想法是写出误差 $\|Ax - b\|$ 并尝试找到最小化它的 $x$。如果我们能找到一个 $x$ 使得误差为 $0$,那么系统就是相容的,我们就得到了精确解。否则,我们就得到所谓的\textbf{最小二乘解}。\textbf{最小二乘}这个术语源于最小化 $\|Ax - b\|$ 等价于最小化 $\|Ax - b\|^2 = \sum_{k=1}^m |(Ax)_k - b_k|^2 = \sum_{k=1}^m |\sum_{j=1}^n A_{k,j} x_j - b_k|^2$,即最小化线性函数平方和。有几种方法可以找到最小二乘解。如果我们处于 $\mathbb{R}^n$ 中,并且所有内容都是实数,我们可以忽略绝对值。然后我们可以对每个变量 $x_j$ 取偏导数,并找到所有偏导数都为 $0$ 的地方,这将给我们最小值。

4.1.1. \textbf{几何方法。} 然而,有一个更简单的寻找最小值的方法。即,如果我们取所有可能的向量 $x$,那么 $Ax$ 会给出 $\text{Ran } A$ 中的所有可能向量,所以最小化 $\|Ax - b\|$ 就是从 $b$ 到 $\text{Ran } A$ 的距离。因此, $\|Ax - b\|$ 的值最小当且仅当 $Ax = P_{\text{Ran } A} b$,其中 $P_{\text{Ran } A}$ 表示到列空间 $\text{Ran } A$ 的正交投影。所以,为了找到最小二乘解,我们只需要解方程 $Ax = P_{\text{Ran } A} b$。

如果我们知道 $\text{Ran } A$ 中的一个正交基 $\{v_1, v_2, \dots, v_n\}$,我们可以通过公式 $P_{\text{Ran } A} b = \sum_{k=1}^n \frac{(b, v_k)}{\|v_k\|^2} v_k$ 来找到向量 $P_{\text{Ran } A} b$。如果我们只知道 $\text{Ran } A$ 中的一个基,我们需要使用格拉姆-施密特正交化来从它得到一个正交基。因此,理论上,问题已经解决了,但解决方案并不非常简单:它涉及格拉姆-施密特正交化,这在计算上可能很密集。幸运的是,存在一个更简单的解决方案。

4.1.2. \textbf{正规方程。} 即,$Ax$ 是正交投影 $P_{\text{Ran } A} b$ 当且仅当 $b - Ax \perp \text{Ran } A$(对所有 $x$,$Ax \in \text{Ran } A$)。如果 $a_1, a_2, \dots, a_n$ 是 $A$ 的列,那么条件 $A x \perp \text{Ran } A$ 可以重写为 $b - Ax \perp a_k$, $\forall k = 1, 2, \dots, n$。这意味着 $0 = (b - Ax, a_k) = a_k^*(b - Ax)$, $\forall k = 1, 2, \dots, n$。将行 $a_k^*$ 连接起来,我们得到这些方程等价于
$A^*(b - Ax) = 0$,
这反过来等价于所谓的\textbf{正规方程} $A^*Ax = A^*b$。该方程的解给出了 $Ax = b$ 的最小二乘解。注意,当且仅当 $A^*A$ 可逆时,最小二乘解是唯一的。

4.2. \textbf{正交投影公式。} 如上所述,如果 $x$ 是\textbf{正规方程} $A^*Ax = A^*b$ 的解(即 $Ax = b$ 的最小二乘解),那么 $Ax = P_{\text{Ran } A} b$。所以,为了找到 $b$ 到列空间 $\text{Ran } A$ 的正交投影,我们需要解正规方程 $A^*Ax = A^*b$,然后将解乘以 $A$。如果算子 $A^*A$ 可逆,则正规方程 $A^*Ax = A^*b$ 的解由 $x = (A^*A)^{-1}A^*b$ 给出,因此正交投影 $P_{\text{Ran } A} b$ 可以计算为 $P_{\text{Ran } A} b = A(A^*A)^{-1}A^*b$。由于这对所有 $b$ 都成立,
$P_{\text{Ran } A} = A(A^*A)^{-1}A^*$
是到 $\text{Ran } A$ 的正交投影矩阵的公式。




下面的定理意味着,对于一个 $m \times n$ 矩阵 $A$,矩阵 $A^*A$ 是可逆的当且仅当 $\text{rank } A = n$。

\textbf{定理 4.1。} 对于一个 $m \times n$ 矩阵 $A$,$\text{Ker } A = \text{Ker}(A^*A)$。
确实,根据秩定理,当且仅当 $\text{rank } A = n$ 时,$\text{Ker } A = \{0\}$。因此,当且仅当 $\text{rank } A = n$ 时,矩阵 $A^*A$ 是可逆的。我们把定理的证明留给读者。要证明 $\text{Ker } A = \text{Ker}(A^*A)$,需要证明两个包含关系 $\text{Ker}(A^*A) \subseteq \text{Ker } A$ 和 $\text{Ker } A \subseteq \text{Ker}(A^*A)$。其中一个包含关系是平凡的,对于另一个,使用 $\|Ax\|^2 = (Ax, Ax) = (A^*Ax, x)$ 的事实。

4.3. \textbf{一个例子:直线拟合。} 让我们引入几个最小二乘解自然出现的例子。假设我们知道两个量 $x$ 和 $y$ 之间的关系由线性规律 $y = a + bx$ 给出。系数 $a$ 和 $b$ 是未知的,我们希望通过实验数据找到它们。假设我们进行了 $n$ 次实验,得到了 $n$ 对 $(x_k, y_k)$,$k=1, 2, \dots, n$。理想情况下,所有点 $(x_k, y_k)$ 都应该在一条直线上,但由于测量误差,通常不会这样:点通常接近某条直线,但并不完全在上面。这时最小二乘解就有用了!理想情况下,系数 $a$ 和 $b$ 应该满足方程
$a + bx_k = y_k$, $k = 1, 2, \dots, n$
(注意这里,$x_k$ 和 $y_k$ 是一些固定的数字,而未知数是 $a$ 和 $b$)。如果可能找到这样的 $a$ 和 $b$,我们就有幸了。如果不行,标准做法是最小化总的二次误差 $\sum_{k=1}^n |a + bx_k - y_k|^2$。但是,最小化这个误差恰好是求解系统
$\begin{pmatrix} 1 & x_1 \\ 1 & x_2 \\ \vdots & \vdots \\ 1 & x_n \end{pmatrix} \begin{pmatrix} a \\ b \end{pmatrix} = \begin{pmatrix} y_1 \\ y_2 \\ \vdots \\ y_n \end{pmatrix}$
的最小二乘解(未知数是 $a$ 和 $b$)。

4.3.1. \textbf{一个例子。} 假设我们的数据 $(x_k, y_k)$ 由对 $(−2, 4)$, $(−1, 2)$, $(0, 1)$, $(2, 1)$, $(3, 1)$ 组成。那么我们需要求解方程
$\begin{pmatrix} 1 & -2 \\ 1 & -1 \\ 1 & 0 \\ 1 & 2 \\ 1 & 3 \end{pmatrix} \begin{pmatrix} a \\ b \end{pmatrix} = \begin{pmatrix} 4 \\ 2 \\ 1 \\ 1 \\ 1 \end{pmatrix}$
的最小二乘解。
那么 $A^*A = \begin{pmatrix} 1 & 1 & 1 & 1 & 1 \\ -2 & -1 & 0 & 2 & 3 \end{pmatrix} \begin{pmatrix} 1 & -2 \\ 1 & -1 \\ 1 & 0 \\ 1 & 2 \\ 1 & 3 \end{pmatrix} = \begin{pmatrix} 5 & 2 \\ 2 & 18 \end{pmatrix}$,
并且 $A^*b = \begin{pmatrix} 1 & 1 & 1 & 1 & 1 \\ -2 & -1 & 0 & 2 & 3 \end{pmatrix} \begin{pmatrix} 4 \\ 2 \\ 1 \\ 1 \\ 1 \end{pmatrix} = \begin{pmatrix} 9 \\ -5 \end{pmatrix}$。
所以正规方程 $A^*Ax = A^*b$ 被重写为
$\begin{pmatrix} 5 & 2 \\ 2 & 18 \end{pmatrix} \begin{pmatrix} a \\ b \end{pmatrix} = \begin{pmatrix} 9 \\ -5 \end{pmatrix}$。
该方程的解是 $a = 2$, $b = -1/2$,因此最佳拟合直线是 $y = 2 - \frac{1}{2}x$。

4.4. \textbf{其他例子:曲线和平面拟合。} 最小二乘法不限于直线拟合。它也可以应用于更一般的曲线,以及更高维度中的曲面。这里唯一的限制是我们要寻找的参数必须以线性方式参与。一般算法如下:
1. 找到如果数据是精确拟合应该满足的方程;
2. 将这些方程写成一个线性系统,其中未知数是我们想要寻找的参数。注意,系统不一定是一致的(通常不是);
3. 找到该系统的最小二乘解。

4.4.1. \textbf{曲线拟合示例。} 例如,假设我们知道 $x$ 和 $y$ 之间的关系由二次定律 $y = a + bx + cx^2$ 给出,所以我们想拟合一个抛物线 $y = a + bx + cx^2$ 到数据上。那么我们的未知数 $a, b, c$ 应该满足方程
$a + bx_k + cx_k^2 = y_k$, $k = 1, 2, \dots, n$
或者,以矩阵形式
$\begin{pmatrix} 1 & x_1 & x_1^2 \\ 1 & x_2 & x_2^2 \\ \vdots & \vdots & \vdots \\ 1 & x_n & x_n^2 \end{pmatrix} \begin{pmatrix} a \\ b \\ c \end{pmatrix} = \begin{pmatrix} y_1 \\ y_2 \\ \vdots \\ y_n \end{pmatrix}$。
例如,对于上例中的数据,我们需要求解方程
$\begin{pmatrix} 1 & -2 & 4 \\ 1 & -1 & 1 \\ 1 & 0 & 0 \\ 1 & 2 & 4 \\ 1 & 3 & 9 \end{pmatrix} \begin{pmatrix} a \\ b \\ c \end{pmatrix} = \begin{pmatrix} 4 \\ 2 \\ 1 \\ 1 \\ 1 \end{pmatrix}$
的最小二乘解。
那么 $A^*A = \begin{pmatrix} 1 & 1 & 1 & 1 & 1 \\ -2 & -1 & 0 & 2 & 3 \\ 4 & 1 & 0 & 4 & 9 \end{pmatrix} \begin{pmatrix} 1 & -2 & 4 \\ 1 & -1 & 1 \\ 1 & 0 & 0 \\ 1 & 2 & 4 \\ 1 & 3 & 9 \end{pmatrix} = \begin{pmatrix} 5 & 2 & 18 \\ 2 & 18 & 26 \\ 18 & 26 & 114 \end{pmatrix}$,
并且 $A^*b = \begin{pmatrix} 1 & 1 & 1 & 1 & 1 \\ -2 & -1 & 0 & 2 & 3 \\ 4 & 1 & 0 & 4 & 9 \end{pmatrix} \begin{pmatrix} 4 \\ 2 \\ 1 \\ 1 \\ 1 \end{pmatrix} = \begin{pmatrix} 9 \\ -5 \\ 31 \end{pmatrix}$。
因此,正规方程 $A^*Ax = A^*b$ 是
$\begin{pmatrix} 5 & 2 & 18 \\ 2 & 18 & 26 \\ 18 & 26 & 114 \end{pmatrix} \begin{pmatrix} a \\ b \\ c \end{pmatrix} = \begin{pmatrix} 9 \\ -5 \\ 31 \end{pmatrix}$
它有一个唯一解 $a = 86/77$, $b = -62/77$, $c = 43/154$。因此,$y = \frac{86}{77} - \frac{62}{77}x + \frac{43}{154}x^2$ 是最佳拟合抛物线。

4.4.2. \textbf{平面拟合。} 再举一个例子,我们拟合一个平面 $z = a + bx + cy$ 到数据 $(x_k, y_k, z_k) \in \mathbb{R}^3$, $k=1, 2, \dots, n$。在精确拟合的情况下,我们应该有的方程是
$a + bx_k + cy_k = z_k$, $k=1, 2, \dots, n$,
或者,以矩阵形式
$\begin{pmatrix} 1 & x_1 & y_1 \\ 1 & x_2 & y_2 \\ \vdots & \vdots & \vdots \\ 1 & x_n & y_n \end{pmatrix} \begin{pmatrix} a \\ b \\ c \end{pmatrix} = \begin{pmatrix} z_1 \\ z_2 \\ \vdots \\ z_n \end{pmatrix}$。
所以,为了找到最佳拟合平面,我们需要找到这个系统(未知数是 $a, b, c$)的最小二乘解。

\textbf{练习}~

4.1. 求解方程组 $\begin{pmatrix} 1 & 0 \\ 0 & 1 \\ 1 & 1 \end{pmatrix} x = \begin{pmatrix} 1 \\ 1 \\ 0 \end{pmatrix}$ 的最小二乘解。

4.2. 找出矩阵 $\begin{pmatrix} 1 & 1 \\ 2 & -1 \\ -2 & 4 \end{pmatrix}$ 的列空间的\textbf{正交投影}矩阵 $P$。使用两种方法:格拉姆-施密特正交化和投影公式。比较结果。

4.3. 找到点 $(−2, 4), (−1, 3), (0, 1), (2, 0)$ 的最佳直线拟合(最小二乘解)。

4.4. 将平面 $z = a + bx + cy$ 拟合到四个点 $(1, 1, 3), (0, 3, 6), (2, 1, 5), (0, 0, 0)$。为此:
a) 找出 4 个关于 3 个未知数 $a, b, c$ 的方程,使得平面通过所有 4 个点(这个系统不一定有解);
b) 找到该系统的最小二乘解。

4.5. \textbf{最小范数解。} 设方程 $Ax = b$ 有解,并且设 $A$ 有非平凡的核(因此解不唯一)。证明:
a) 存在唯一一个 $Ax = b$ 的解 $x_0$,它最小化范数 $\|x\|$,即存在唯一的 $x_0$ 使得 $Ax_0 = b$ 且 $\|x_0\| \leq \|x\|$ 对于任何满足 $Ax = b$ 的 $x$。
b) $x_0 = P_{(\text{Ker } A)^\perp} x$ 对于任何满足 $Ax = b$ 的 $x$。

4.6. \textbf{最小范数最小二乘解。} 将上一问题应用于方程 $Ax = P_{\text{Ran } A} b$,证明 $A x = b$ 的一个最小范数最小二乘解 $x_0$ 存在且唯一。
a) 存在唯一的最小二乘解 $x_0$ 最小化范数 $\|x\|$。
b) $x_0 = P_{(\text{Ker } A)^\perp} x$ 对于任何 $Ax = b$ 的最小二乘解 $x$。




\section{线性变换的伴随,基本子空间再次回顾}

5.1. \textbf{伴随矩阵与伴随算子。} 让我们回忆一下,对于一个 $m \times n$ 矩阵 $A$,其\textbf{共轭转置}(或简单地说\textbf{伴随})$A^*$ 定义为 $A^* := \overline{A^T}$。换句话说,矩阵 $A^*$ 是通过转置矩阵 $A^T$ 然后取每个元素的复共轭得到的。以下恒等式是伴随矩阵的主要性质:$(Ax, y) = (x, A^*y)$ $\forall x \in \mathbb{C}^n, \forall y \in \mathbb{C}^m$。
在证明这个恒等式之前,让我们引入一些有用的公式。让我们回忆一下,对于转置矩阵我们有恒等式 $(AB)^T = B^T A^T$。由于对于复数 $z$ 和 $w$ 我们有 $\overline{zw} = \bar{z}\bar{w}$,所以对于伴随有恒等式 $(AB)^* = B^*A^*$。同样,由于 $(A^T)^T = A$ 且 $\overline{\bar{z}} = z$,$(A^*)^* = A$。
现在,我们准备证明主要恒等式:$(Ax, y) = y^*Ax = (A^*y)^*x = (x, A^*y)$;这里第一个和最后一个等式遵循内积在 $F^n$ 中的定义,而中间的等式遵循 $(A^*x)^* = x^*(A^*)^* = x^*A$ 的事实。

5.1.1. \textbf{伴随的唯一性。} 上述主要恒等式 $(Ax, y) = (x, A^*y)$ 通常用作伴随算子的定义。让我们首先注意到伴随算子是唯一的:如果一个矩阵 $B$ 满足 $(Ax, y) = (x, By)$ $\forall x, y$,则 $B = A^*$。确实,根据 $A^*$ 的定义,对于给定的 $y$,我们有 $(x, A^*y) = (x, By)$ $\forall x$,因此根据推论 1.5, $A^*y = By$。由于这对所有 $y$ 都成立,线性变换,因此矩阵 $A^*$ 和 $B$ 是相等的。




5.1.2. \textbf{抽象环境下的伴随变换。} 上述主要恒等式 $(Ax, y) = (x, A^*y)$ 可用于在抽象环境中定义伴随算子,其中 $A: V \to W$ 是作用在一个内积空间到另一个内积空间上的算子。即,我们定义 $A^*: W \to V$ 为满足 $(Ax, y) = (x, A^*y)$ $\forall x \in V, \forall y \in W$ 的算子。为什么这样的算子存在?我们可以简单地构造它:考虑 $V$ 中的一组标准正交基 $\{v_1, v_2, \dots, v_n\}$ 和 $W$ 中的一组标准正交基 $\{w_1, w_2, \dots, w_m\}$。如果 $[A]_{BA}$ 是这两个基下 $A$ 的矩阵,我们定义算子 $A^*$ 为定义其矩阵 $[A^*]_{AB} = ([A]_{BA})^*$。我们将该算子是伴随算子的证明留给读者作为练习。注意,上述第 5.1.1 节中的推理意味着伴随算子是唯一的。

5.1.3. \textbf{有用的公式。} 下面我们给出将广泛使用的伴随算子(矩阵)的性质。我们将证明留给读者作为练习。
1. $(A + B)^* = A^* + B^*$;
2. $(\alpha A)^* = \bar{\alpha} A^*$;
3. $(AB)^* = B^*A^*$;
4. $(A^*)^* = A$;
5. $(y, Ax) = (A^*y, x)$。

5.2. \textbf{基本子空间之间的关系。}
\textbf{定理 5.1。} 设 $A: V \to W$ 是作用在一个内积空间到另一个内积空间上的算子。那么
1. $\text{Ker } A^* = (\text{Ran } A)^\perp$;
2. $\text{Ker } A = (\text{Ran } A^*)^\perp$;
3. $\text{Ran } A = (\text{Ker } A^*)^\perp$;
4. $\text{Ran } A^* = (\text{Ker } A)^\perp$。

\textbf{注释} ~在第 2 章第 7 节,基本子空间被定义(如文献中常见的那样)使用 $A^T$ 而不是 $A^*$。当然,对于实数矩阵没有区别,所以在实数情况下,本定理给出了那里定义的基本子空间的几何描述。第 8 章下面第 3 节(定理 3.7)给出了使用 $A^T$ 定义的基本子空间的几何解释。本定理中的公式与定理 5.1 中的公式基本相同,只是解释略有不同。

\textbf{定理 5.1 的证明。} 首先,我们注意到,对于子空间 $E$,我们有 $(E^\perp)^\perp = E$,所以陈述 1 和 3 是等价的。类似地,出于同样的原因,陈述 2 和 4 也是等价的。最后,陈述 2 恰好是应用于算子 $A^*$ 的陈述 1(这里我们使用了 $(A^*)^* = A$ 的事实)。因此,我们只需要证明陈述 1。我们将为此陈述提供两种证明:“矩阵”证明和“不变”或“坐标无关”证明。在“矩阵”证明中,我们假设 $A$ 是一个 $m \times n$ 矩阵,即 $A: F^n \to F^m$。一般情况总可以通过选取 $V$ 和 $W$ 中的标准正交基来简化为这种情况。设 $a_1, a_2, \dots, a_n$ 是 $A$ 的列。注意,$x \in (\text{Ran } A)^\perp$ 当且仅当 $x \perp a_k$(即 $(x, a_k) = 0$)$\forall k = 1, 2, \dots, n$。根据 $F^n$ 中内积的定义,这意味着 $0 = (x, a_k) = a_k^* x$ $\forall k = 1, 2, \dots, n$。由于 $a_k^*$ 是 $A^*$ 的第 $k$ 行,上述 $n$ 个等式等价于方程 $A^*x = 0$。所以,我们证明了 $x \in (\text{Ran } A)^\perp$ 当且仅当 $A^*x = 0$,而这恰好是定理 1 的陈述。现在,让我们给出“坐标无关”的证明。$x \in (\text{Ran } A)^\perp$ 的含义是 $x$ 正交于所有形式为 $Ay$ 的向量,即 $(x, Ay) = 0$ $\forall y$。由于 $(x, Ay) = (A^*x, y)$,这个恒等式等价于 $(A^*x, y) = 0$ $\forall y$,并且根据引理 1.4,这当且仅当 $A^*x = 0$。所以我们证明了 $x \in (\text{Ran } A)^\perp$ 当且仅当 $A^*x = 0$,而这恰好是定理 1 的陈述。

5.3. \textbf{线性变换的“本质”部分。} 上述定理使得算子 $A$ 的结构以及基本子空间的几何学更加清晰。从该定理可以得出,算子 $A$ 可以表示为到 $\text{Ran } A^*$ 的正交投影与从 $\text{Ran } A^*$ 到 $\text{Ran } A$ 的同构的组合。
确实,设 $\tilde{A}: \text{Ran } A^* \to \text{Ran } A$ 是 $A$ 对定义域 $\text{Ran } A^*$ 和目标空间 $\text{Ran } A$ 的限制,$\tilde{A}x = Ax$, $\forall x \in \text{Ran } A^*$。由于 $\text{Ker } A = (\text{Ran } A^*)^\perp$,我们有 $Ax = AP_{\text{Ran } A^*}x = \tilde{A}P_{\text{Ran } A^*}x$ $\forall x \in X$;这里使用了 $x - P_{\text{Ran } A^*}x \in (\text{Ran } A^*)^\perp = \text{Ker } A$ 的事实。因此我们可以写成 $A = \tilde{A}P_{\text{Ran } A^*}$ $\forall x \in X$,(5.1)或者等价地说,$A = \tilde{A}P_{\text{Ran } A^*}$。还需注意,$\tilde{A}: \text{Ran } A^* \to \text{Ran } A$ 是一个可逆变换。首先我们注意到 $\text{Ker } \tilde{A} = \{0\}$:如果 $x \in \text{Ran } A^*$ 且 $\tilde{A}x = Ax = 0$,那么 $x \in \text{Ker } A = (\text{Ran } A^*)^\perp$,所以 $x \in \text{Ran } A^* \cap (\text{Ran } A^*)^\perp$,因此 $x = 0$。然后为了证明 $\tilde{A}$ 是满射的(surjective),必须确保 $\tilde{A}$ 是满射的。但这直接从 (5.1) 得出:$\text{Ran } \tilde{A} = \tilde{A}(\text{Ran } A^*) = A P_{\text{Ran } A^*} X = AX = \text{Ran } A$。同构 $\tilde{A}$ 有时被称为算子 $A$ 的“本质部分”(非标准术语)。“本质部分” $\tilde{A}: \text{Ran } A^* \to \text{Ran } A$ 是一个同构,这隐含了以下“复数”秩定理:$\text{rank } A = \text{rank } A^*$。但是,当然,这个定理也来自一个基本观察:复共轭不改变矩阵的秩,$\text{rank } A = \text{rank } \bar{A}$。

\textbf{练习}~

5.1. 证明对于方阵 $A$,$\det(A^*) = \det(A)$ 成立。

5.2. 找出矩阵 $A = \begin{pmatrix} 1 & 1 & 1 \\ 1 & 3 & 2 \\ 2 & 4 & 3 \end{pmatrix}$ 的所有四个基本子空间的\textbf{正交投影}矩阵。注意,实际上只需要计算其中两个投影。如果你选择合适的两个,其他的 2 个可以很容易地从它们得到(回想一下,投影到 $E$ 和 $E^\perp$ 的关系)。

5.3. 设 $A$ 是一个 $m \times n$ 矩阵。证明 $\text{Ker } A = \text{Ker}(A^*A)$。为此你需要证明两个包含关系 $\text{Ker}(A^*A) \subseteq \text{Ker } A$ 和 $\text{Ker } A \subseteq \text{Ker}(A^*A)$。其中一个包含关系是平凡的,对于另一个,使用 $\|Ax\|^2 = (Ax, Ax) = (A^*Ax, x)$ 的事实。

5.4. 使用 $\text{Ker } A = \text{Ker}(A^*A)$ 的等式来证明:
a) $\text{rank } A = \text{rank}(A^*A)$;
b) 如果 $Ax = 0$ 只有平凡解,则 $A$ 是左可逆的。(你只需要写出一个左逆的公式)。

5.5. 假设矩阵 $A$ 的 $A^*A$ 是可逆的,因此到 $\text{Ran } A$ 的正交投影由公式 $A(A^*A)^{-1}A^*$ 给出。你能写出到其他 3 个基本子空间($\text{Ker } A$, $\text{Ker } A^*$, $\text{Ran } A^*$)的正交投影的公式吗?

5.6. 设矩阵 $P$ 是自伴随的 ($P^* = P$) 并且 $P^2 = P$。证明 $P$ 是一个正交投影的矩阵。
\textbf{提示:} 考虑分解 $x = x_1 + x_2$, $x_1 \in \text{Ran } P$, $x_2 \perp \text{Ran } P$,并证明 $Px_1 = x_1$, $Px_2 = 0$。对于其中一个等式,你将需要自伴随性,对于另一个等式,你需要 $P^2 = P$ 的性质。





6. 
\section{等距同构和酉算子~酉矩阵和正交矩阵}

6.1. \textbf{基本定义。}
\textbf{定义}~ 算子 $U: X \to Y$ 被称为\textbf{等距同构},如果它保持范数,即 $\|Ux\| = \|x\|$ $\forall x \in X$。
以下定理表明等距同构保持内积:

\textbf{定理 6.1。} 算子 $U: X \to Y$ 是等距同构当且仅当它保持内积,即当且仅当 $(x, y) = (Ux, Uy)$ $\forall x, y \in X$。

\textbf{证明}~ 证明使用了极化恒等式(第 5 章引理 1.9)。例如,如果 $X$ 是复数空间($U$ 是复数算子),$(Ux, Uy) = \frac{1}{4} \sum_{\alpha = \pm 1, \pm i} \alpha \|Ux + \alpha Uy\|^2 = \frac{1}{4} \sum_{\alpha = \pm 1, \pm i} \alpha \|U(x + \alpha y)\|^2 = \frac{1}{4} \sum_{\alpha = \pm 1, \pm i} \alpha \|x + \alpha y\|^2 = (x, y)$。
类似地,对于实数空间 $X$($U$ 是实数算子),$(Ux, Uy) = \frac{1}{4} (\|Ux + Uy\|^2 - \|Ux - Uy\|^2) = \frac{1}{4} (\|U(x+y)\|^2 - \|U(x-y)\|^2) = \frac{1}{4} (\|x+y\|^2 - \|x-y\|^2) = (x, y)$。

\textbf{引理 6.2。} 算子 $U: X \to Y$ 是等距同构当且仅当 $U^*U = I$。

\textbf{证明}~ 如果 $U^*U = I$,那么根据伴随算子的定义,$(x, x) = (U^*Ux, x) = (Ux, Ux)$ $\forall x \in X$。因此 $\|x\| = \|Ux\|$,所以 $U$ 是等距同构。另一方面,如果 $U$ 是等距同构,那么根据伴随算子的定义和定理 6.1,对于所有 $x \in X$,$(U^*Ux, y) = (Ux, Uy) = (x, y)$ $\forall y \in X$,因此根据推论 1.5, $U^*Ux = x$。由于这对所有 $x \in X$ 都成立,所以 $U^*U = I$。

上面的引理意味着等距同构总是左可逆的($U^*$ 是左逆)。

\textbf{定义}~ 等距同构 $U: X \to Y$ 被称为\textbf{酉}算子,如果它是可逆的。

\textbf{命题 6.3。} 等距同构 $U: X \to Y$ 是酉算子当且仅当 $\dim X = \dim Y$。

\textbf{证明}~ 由于 $U$ 是等距同构,它是左可逆的,并且由于 $\dim X = \dim Y$,它是可逆的(左可逆的方阵才是可逆的)。另一方面,如果 $U: X \to Y$ 是可逆的,那么 $\dim X = \dim Y$(只有方阵才是可逆的,同构的空间具有相等的维度)。

一个方阵 $U$ 被称为\textbf{酉}矩阵,如果 $U^*U = I$,即酉矩阵是作用在 $F^n$ 上的酉算子的矩阵。实数项的酉矩阵被称为\textbf{正交}矩阵。一个正交矩阵可以解释为作用在实数空间 $\mathbb{R}^n$ 上的酉算子的矩阵。

酉算子的几个性质:
1. 对于酉变换 $U$, $U^{-1} = U^*$;
2. 如果 $U$ 是酉的,那么 $U^* = U^{-1}$ 也必须是酉的;
3. 如果 $U$ 是等距同构,并且 $\{v_1, v_2, \dots, v_n\}$ 是一个标准正交基,那么 $\{Uv_1, Uv_2, \dots, Uv_n\}$ 是一个标准正交系统。而且,如果 $U$ 是酉的,$\{Uv_1, Uv_2, \dots, Uv_n\}$ 是一个标准正交基。
4. 酉算子的乘积也是酉算子。

6.2. \textbf{例子。} 首先,让我们注意到,一个矩阵 $U$ 是等距同构当且仅当它的列构成一个标准正交系统。通过计算乘积 $U^*U$ 可以很容易地检验这一点。可以很容易地检验出旋转矩阵 $\begin{pmatrix} \cos \alpha & -\sin \alpha \\ \sin \alpha & \cos \alpha \end{pmatrix}$ 的列彼此正交,并且每列的范数都为 1。因此,旋转矩阵是等距同构,并且由于它是方阵,所以它是酉的。由于旋转矩阵的所有元素都是实数,它是一个正交矩阵。下一个例子更抽象。
设 $X$ 和 $Y$ 是内积空间,$\dim X = \dim Y = n$,并且设 $\{x_1, x_2, \dots, x_n\}$ 和 $\{y_1, y_2, \dots, y_n\}$ 分别是 $X$ 和 $Y$ 中的标准正交基。定义一个算子 $U: X \to Y$ 为 $Ux_k = y_k$, $k = 1, 2, \dots, n$。由于对于向量 $x = \sum_{k=1}^n c_k x_k$, $\|x\|^2 = \sum_{k=1}^n |c_k|^2$ 和 $\|Ux\|^2 = \|\sum_{k=1}^n c_k y_k\|^2 = \sum_{k=1}^n |c_k|^2$,我们可以得出 $\|Ux\| = \|x\|$ $\forall x \in X$,所以 $U$ 是一个酉算子。

6.3. \textbf{酉算子的性质。}
\textbf{命题 6.4。} 设 $U$ 是一个酉矩阵。那么
1. $|\det U| = 1$。特别是,对于正交矩阵 $\det U = \pm 1$;
2. 如果 $\lambda$ 是 $U$ 的一个特征值,那么 $|\lambda| = 1$。

\textbf{注释} ~注意,对于正交矩阵,特征值(不像行列式)不一定必须是实数。我们老朋友,旋转矩阵就是一个例子。




\textbf{命题 6.4 的证明。} 设 $\det U = z$。由于 $\det(U^*) = \det(U)$,见问题 5.1,我们有 $|z|^2 = z\bar{z} = \det(U^*U) = \det I = 1$,所以 $|\det U| = |z| = 1$。陈述 1 证毕。为了证明陈述 2,我们注意到如果 $Ux = \lambda x$,那么 $\|Ux\| = \|\lambda x\| = |\lambda|\|x\|$,所以 $|\lambda| = 1$。

6.4. \textbf{酉等价算子。}
\textbf{定义}~ 算子(矩阵)$A$ 和 $B$ 被称为\textbf{酉等价}的,如果存在一个酉算子 $U$ 使得 $A = UBU^*$。由于对于酉 $U$ 我们有 $U^{-1} = U^*$,任何两个酉等价的矩阵也是相似的。反之则不然,很容易构造一对酉等价但不是酉等价的相似矩阵。下面的命题提供了一种构造反例的方法。

\textbf{命题 6.5。} 矩阵 $A$ 是酉等价于一个对角矩阵当且仅当它具有一个\textbf{正交}(或标准正交)的特征向量基。

\textbf{证明}~ 设 $A$ 与对角矩阵 $D$ 酉等价,即 $A = UDU^*$。设 $Bx = \lambda x$。那么 $AUx = UBU^*Ux = U Bx = U(\lambda x) = \lambda Ux$,即 $Ux$ 是 $A$ 的特征向量。所以,设 $A$ 具有由特征向量 $u_1, u_2, \dots, u_n$ 组成的\textbf{正交}基。通过将每个向量 $u_k$ 除以其范数(如果需要),我们可以假设系统 $\{u_1, u_2, \dots, u_n\}$ 是一个\textbf{标准正交}基。设 $D$ 是 $A$ 在基 $B = \{u_1, u_2, \dots, u_n\}$ 下的矩阵。显然,$D$ 是一个对角矩阵。设 $U$ 为以 $u_1, u_2, \dots, u_n$ 为列的矩阵。由于列向量构成了标准正交基, $U$ 是酉的。标准坐标变换公式意味着 $A = [A]_{SS} = [I]_{SB}[A]_{BB}[I]_{BS} = UDU^{-1}$,并且由于 $U$ 是酉的,$A = UDU^*$。



练习。

6.1. 对以下矩阵进行\textbf{正交对角化},即对每个矩阵 $A$,找出酉矩阵 $U$ 和对角矩阵 $D$,使得 $A = UDU^*$:
$\begin{pmatrix} 1 & 2 \\ 2 & 1 \end{pmatrix}$, $\begin{pmatrix} 0 & -1 \\ 1 & 0 \end{pmatrix}$, $\begin{pmatrix} 0 & 2 & 2 \\ 2 & 0 & 2 \\ 2 & 2 & 0 \end{pmatrix}$。

6.2. 真或假:一个矩阵是酉等价于一个对角矩阵当且仅当它具有一个\textbf{正交}的特征向量基。

6.3. 证明极化恒等式 $(Ax, y) = \frac{1}{4} [ (A(x+y), x+y) - (A(x-y), x-y) ]$(实数情况,$A=A^*$),以及 $(Ax, y) = \frac{1}{4} \sum_{\alpha = \pm 1, \pm i} \alpha (A(x+\alpha y), x+\alpha y)$(复数情况,$A$ 任意)。

6.4. 证明酉(正交)矩阵的乘积也是酉(正交)的。

6.5. 设 $U: X \to X$ 是一个有限维内积空间上的线性变换。真或假:
a) 如果 $\|Ux\| = \|x\|$ $\forall x \in X$,那么 $U$ 是酉的。
b) 如果 $\|Ue_k\| = \|e_k\|$, $k=1, 2, \dots, n$ 对于某个标准正交基 $\{e_1, e_2, \dots, e_n\}$,那么 $U$ 是酉的。
用证明或反例证明你的答案。

6.6. 设 $A$ 和 $B$ 是酉等价的 $n \times n$ 矩阵。
a) 证明 $\text{trace}(A^*A) = \text{trace}(B^*B)$。
b) 使用 a) 证明 $\sum_{j,k=1}^n |A_{j,k}|^2 = \sum_{j,k=1}^n |B_{j,k}|^2$。
c) 使用 b) 证明矩阵 $\begin{pmatrix} 1 & 2 \\ 2 & i \end{pmatrix}$ 和 $\begin{pmatrix} i & 4 \\ 1 & 1 \end{pmatrix}$ 不是酉等价的。

6.7. 以下哪些矩阵对是酉等价的:
a) $\begin{pmatrix} 1 & 0 \\ 0 & 1 \end{pmatrix}$ 和 $\begin{pmatrix} 0 & 1 \\ 1 & 0 \end{pmatrix}$。
b) $\begin{pmatrix} 0 & 1 \\ 1 & 0 \end{pmatrix}$ 和 $\begin{pmatrix} 0 & 1/2 \\ 1/2 & 0 \end{pmatrix}$。
c) $\begin{pmatrix} 0 & 1 & 0 \\ -1 & 0 & 0 \\ 0 & 0 & 1 \end{pmatrix}$ 和 $\begin{pmatrix} 2 & 0 & 0 \\ 0 & -1 & 0 \\ 0 & 0 & 0 \end{pmatrix}$。
d) $\begin{pmatrix} 0 & 1 & 0 \\ -1 & 0 & 0 \\ 0 & 0 & 1 \end{pmatrix}$ 和 $\begin{pmatrix} 1 & 0 & 0 \\ 0 & -i & 0 \\ 0 & 0 & i \end{pmatrix}$。
e) $\begin{pmatrix} 1 & 1 & 0 \\ 0 & 2 & 2 \\ 0 & 0 & 3 \end{pmatrix}$ 和 $\begin{pmatrix} 1 & 0 & 0 \\ 0 & 2 & 0 \\ 0 & 0 & 3 \end{pmatrix}$。
\textbf{提示:} 很容易排除不酉等价的矩阵:记住酉等价矩阵是相似的,而相似矩阵的迹、行列式和特征值是相同的。此外,一个矩阵是酉等价于一个对角矩阵当且仅当它具有一个特征向量的正交基。

6.8. 设 $U$ 是一个行列式为 1 的 $2 \times 2$ 正交矩阵。证明 $U$ 是一个旋转矩阵。

6.9. 设 $U$ 是一个行列式为 1 的 $3 \times 3$ 正交矩阵。证明:
a) $1$ 是 $U$ 的一个特征值。
b) 如果 $\{v_1, v_2, v_3\}$ 是一个标准正交基,使得 $Uv_1 = v_1$(记住 $1$ 是一个特征值),那么在基 $\{v_1, v_2, v_3\}$ 下 $U$ 的矩阵是 $\begin{pmatrix} 1 & 0 & 0 \\ 0 & \cos \alpha & -\sin \alpha \\ 0 & \sin \alpha & \cos \alpha \end{pmatrix}$,其中 $\alpha$ 是某个角度。
\textbf{提示:} 证明,由于 $v_1$ 是 $U$ 的特征向量,1 下方的所有元素必须为零,并且由于 $v_1$ 也是 $U^*$(为什么?)的特征向量,1 右侧的所有元素也必须为零。然后证明下方的 $2 \times 2$ 矩阵是一个行列式为 1 的正交矩阵,并使用上一问题。

7. 
\section{$\mathbb{R}^n$ 中的刚性运动}

$\mathbb{R}^n$ 中的一个\textbf{刚性运动}是一个变换 $f: V \to V$,它保持点之间的距离,即 $\|f(x) - f(y)\| = \|x - y\|$ $\forall x, y \in V$。
注意,定义中我们没有假设变换 $f$ 是线性的。显然,任何酉变换都是刚性运动。另一个刚性运动的例子是平移(移位)$a \in V$,$f(x) = x + a$。




下面的定理是主要结果,它表明任何实内积空间中的刚性运动都是正交变换和翻译的组合。

\textbf{定理 7.1。} 设 $f$ 是实内积空间 $X$ 中的一个刚性运动,设 $T(x) := f(x) - f(0)$。那么 $T$ 是一个正交变换。

为了证明这个定理,我们需要一个简单的引理。

\textbf{引理 7.2。} 设 $T$ 如定理 7.1 中所定义。那么对于所有 $x, y \in X$:
1. $\|Tx\| = \|x\|$;
2. $\|T(x) - T(y)\| = \|x - y\|$;
3. $(Tx, Ty) = (x, y)$。

\textbf{证明}~ 为了证明陈述 1,注意到 $\|T(x)\| = \|f(x) - f(0)\| = \|x - 0\| = \|x\|$。
陈述 2 源于以下恒等式链:$\|T(x) - T(y)\| = \|(f(x) - f(0)) - (f(y) - f(0))\| = \|f(x) - f(y)\| = \|x - y\|$。
另一种解释是 $T$ 是两个刚性运动的组合(先是 $f$,然后是平移 $-f(0)$),并且可以很容易地看出刚性运动的组合是刚性运动。由于 $T(0) = 0$,并且 $\|T(x)\| = \|T(x) - T(0)\|$,所以陈述 1 可以视为陈述 2 的一个特例。为了证明陈述 3,让我们注意到在实内积空间中
$\|T(x) - T(y)\|^2 = \|T(x)\|^2 + \|T(y)\|^2 - 2(T(x), T(y))$,
并且 $\|x - y\|^2 = \|x\|^2 + \|y\|^2 - 2(x, y)$。
回想一下 $\|T(x) - T(y)\| = \|x - y\|$ 并且 $\|T(x)\| = \|x\|$, $\|T(y)\| = \|y\|$,我们立即得到期望的结论。

\textbf{定理 7.1 的证明。} 首先,注意到对于所有 $x \in X$, $\|Tx\| = \|f(x) - f(0)\| = \|x - 0\| = \|x\|$,所以 $T$ 保持范数,$\|Tx\| = \|x\|$。我们想说 $T$ 是一个等距同构,但要能够说出这一点,我们需要证明 $T$ 是一个线性变换。为此,让我们在 $X$ 中固定一个标准正交基 $\{e_1, e_2, \dots, e_n\}$,并设 $b_k := T(e_k)$, $k=1, 2, \dots, n$。由于 $T$ 保持内积(引理 7.2 的陈述 3),我们可以得出 $\{b_1, b_2, \dots, b_n\}$ 是一个标准正交系统。事实上,由于 $\dim X = n$(因为基 $\{e_1, e_2, \dots, e_n\}$ 包含 $n$ 个向量),我们可以得出 $\{b_1, b_2, \dots, b_n\}$ 是一个标准正交\textbf{基}。设 $x = \sum_{k=1}^n \alpha_k e_k$。回忆一下根据抽象正交傅里叶分解 (2.2),我们有 $\alpha_k = (x, e_k)$。将抽象正交傅里叶分解 (2.2) 应用于 $T(x)$ 和标准正交基 $\{b_1, b_2, \dots, b_n\}$,我们得到 $T(x) = \sum_{k=1}^n (T(x), b_k) b_k$。由于 $(T(x), b_k) = (T(x), T(e_k)) = (x, e_k) = \alpha_k$,我们得到 $T(\sum_{k=1}^n \alpha_k e_k) = \sum_{k=1}^n \alpha_k b_k$。这意味着 $T$ 是一个线性变换,其在基 $\{e_1, e_2, \dots, e_n\}$ 和 $\{b_1, b_2, \dots, b_n\}$ 下的矩阵是单位矩阵,$[T]_{B,S} = I$。另一种证明 $T$ 是线性变换的方法是进行以下直接计算:$\|T(x + \alpha y) - (T(x) + \alpha T(y))\|^2 = \|(T(x + \alpha y) - T(x)) - \alpha T(y)\|^2 = \|T(x + \alpha y) - T(x)\|^2 + \alpha^2 \|T(y)\|^2 - 2\alpha (T(x + \alpha y) - T(x), T(y)) = \|x + \alpha y - x\|^2 + \alpha^2\|y\|^2 - 2\alpha(T(x+\alpha y), T(y)) + 2\alpha(T(x), T(y)) = \|\alpha y\|^2 + \alpha^2\|y\|^2 - 2\alpha(x+\alpha y, y) + 2\alpha(x, y) = \alpha^2\|y\|^2 + \alpha^2\|y\|^2 - 2\alpha(x, y) - 2\alpha^2(y, y) + 2\alpha(x, y) = 2\alpha^2\|y\|^2 - 2\alpha^2\|y\|^2 = 0$。因此 $T(x+\alpha y) = T(x) + \alpha T(y)$,这意味着 $T$ 是线性的(取 $x=0$ 或 $\alpha=1$ 可得到线性变换定义的两个性质)。所以,$T$ 是一个满足 $\|Tx\| = \|x\|$ 的线性变换,即 $T$ 是一个等距同构。由于 $T: X \to X$, $T$ 是一个酉变换(见命题 6.3)。这完成了证明,因为一个正交变换仅仅是一个实内积空间中的酉变换。

\textbf{练习}~

7.1. 在 $\mathbb{C}^n$ 中给出一个刚性运动 $T$, $T(0)=0$,但 $T$ 不是线性变换。




8. 
\section{复化与反复化}
本节可能比本章的其余部分更抽象一些,可以首次阅读时跳过。

8.1. \textbf{反复化。}
8.1.1. \textbf{向量空间的复化。} 任何复数向量空间都可以解释为一个实向量空间:我们只需要忘记我们可以乘以复数,并且只允许乘以实数。例如,空间 $\mathbb{C}^n$ \textbf{典型地}被识别为实向量空间 $\mathbb{R}^{2n}$:每个复数坐标 $z_k = x_k + iy_k$ 给出两个实坐标 $x_k$ 和 $y_k$。“典型地”在这里意味着这是一种标准、最自然地识别 $\mathbb{C}^n$ 和 $\mathbb{R}^{2n}$ 的方式。注意,虽然上述定义给了我们一种从复数坐标得到实数坐标的典型方法,但它并没有说明坐标的排序。事实上,有两种标准的方法来排序坐标 $x_k, y_k$。一种方法是先取实部,然后取虚部,所以排序是 $x_1, x_2, \dots, x_n, y_1, y_2, \dots, y_n$。另一种标准选择是排序 $x_1, y_1, x_2, y_2, \dots, x_n, y_n$。本节的内容不依赖于坐标的排序选择,所以读者不必担心选择排序。

8.1.2. \textbf{内积的复化。} 结果表明,如果我们有一个复内积(在一个复数空间中),我们可以以典型的方式从中得到一个实内积:实际上,你可能已经在不知不觉中做过了。即,考虑 $\mathbb{C}^n$ 的上述例子,它典型地被识别为 $\mathbb{R}^{2n}$。设 $(x, y)_{\mathbb{C}}$ 表示 $\mathbb{C}^n$ 中的标准内积,$(x, y)_{\mathbb{R}}$ 表示 $\mathbb{R}^{2n}$ 中的标准内积(注意 $\mathbb{R}^n$ 中的标准内积不依赖于坐标的排序)。那么(见下面的练习 8.1)
$(x, y)_{\mathbb{R}} = \text{Re}((x, y)_{\mathbb{C}})$ (8.1)
这个公式可以用于典型地从复内积中定义一个实内积,在一般情况下也是如此。即,很容易检查出,如果 $(x, y)_{\mathbb{C}}$ 是一个复内积空间中的内积,那么 $(x, y)_{\mathbb{R}}$ 定义为 (8.1) 是一个实内积(在其相应的实空间上)。总结一下,我们可以说,要对一个复内积空间进行反复化,我们只需“忘记”我们可以乘以复数,即我们只允许乘以实数。被反复化空间中的典型实内积由公式 (8.1) 给出。

\textbf{注释} ~任何(复数)线性变换作用在 $\mathbb{C}^n$ 上(或更广泛地说,在复向量空间上)都会产生一个实线性变换:这仅仅是因为如果 $T(\alpha x + \beta y) = \alpha T x + \beta T y$ 对 $\alpha, \beta \in \mathbb{C}$ 成立,那么它当然对 $\alpha, \beta \in \mathbb{R}$ 也成立。反之则不成立,即作用在 $\mathbb{C}^n$ 的反复化 $\mathbb{R}^{2n}$ 上的(实数)线性变换并不总是产生 $\mathbb{C}^n$ 的(复数)线性变换(在抽象情况下也是一样)。例如,如果考虑 $n=1$ 的情况,那么乘以一个复数 $z$(复数空间 $\mathbb{C}^1$ 中的线性变换的一般形式)被视为 $\mathbb{R}^2$ 中的线性变换时,具有一个非常特殊的结构(你能描述它吗?)。

8.2. \textbf{复化。} 我们也可以做相反的事情,即从一个实空间得到一个复空间:实际上,你可能已经做过了,而没有太在意。即,给定一个实内积空间 $\mathbb{R}^n$,我们可以从它得到一个复空间 $\mathbb{C}^n$,方法是允许复数坐标(在两种情况下都使用标准内积)。在这种情况下,空间 $\mathbb{R}^n$ 将是 $\mathbb{C}^n$ 的一个\textbf{实子空间},由具有实数坐标的向量组成。抽象地说,这个构造可以描述如下:给定一个实向量空间 $X$,我们可以将其复化 $X_{\mathbb{C}}$ 定义为所有对 $[x_1, x_2]$, $x_1, x_2 \in X$ 的集合,其中加法和实数 $\alpha$ 的乘法是逐坐标定义的:$[x_1, x_2] + [y_1, y_2] = [x_1 + y_1, x_2 + y_2]$, $\alpha [x_1, x_2] = [\alpha x_1, \alpha x_2]$。如果 $X = \mathbb{R}^n$,那么向量 $x_1$ 由 $\mathbb{C}^n$ 中复数坐标的实部组成,向量 $x_2$ 由虚部组成。因此,非正式地说,我们可以将对 $[x_1, x_2]$ 写成 $x_1 + ix_2$。为了定义复数乘法,我们定义 $i$ 的乘法为 $i[x_1, x_2] = [-x_2, x_1]$(将 $[x_1, x_2]$ 写成 $x_1 + ix_2$ 时,我们可以看到它必须这样定义),并使用第二个分配律 $\alpha(v+w) = \alpha v + \alpha w$ 来定义任意复数的乘法。如果 $X$ 是一个内积空间,我们还可以将内积扩展到 $X_{\mathbb{C}}$:
$([x_1, x_2], [y_1, y_2])_{X_{\mathbb{C}}} = (x_1, y_1)_X + (x_2, y_2)_X$。
要看到一切都定义良好,最简单的方法是固定一个基(在实内积空间的情况下是标准正交基),然后查看坐标表示下会发生什么。然后我们可以看到,如果我们把向量 $x_1$ 看作由复数坐标的实部组成的向量,把向量 $x_2$ 看作由坐标的虚部组成的向量,那么这个构造恰好是 $\mathbb{R}^n$ 的标准复化(通过允许复数坐标)正如上面描述的那样。我们可以用坐标无关的方式来解释这个构造,而不必选取基并处理坐标,这意味着结果不依赖于基的选择。所以,思考复化的最简单方法可能是这样的:要构造一个实向量空间 $X$ 的复化,我们可以选取一个基(如果 $X$ 是实内积空间,则选取一个标准正交基),然后处理坐标,允许复数坐标。结果空间不依赖于基的选择;我们可以通过标准的坐标变换公式从一个坐标集得到另一个。注意,任何实空间 $X$ 中的线性变换 $T$ 都会产生其复化 $X_{\mathbb{C}}$ 中的一个线性变换 $T_{\mathbb{C}}$。看到这一点最简单的方法是固定 $X$ 中的一个基(如果 $X$ 是实内积空间,则选取一个标准正交基),并以坐标表示进行处理:在这种情况下,$T_{\mathbb{C}}$ 与 $T$ 具有相同的矩阵。在抽象表示中,我们可以写成 $T_{\mathbb{C}}[x_1, x_2] = [Tx_1, Tx_2]$。另一方面,并非所有 $X_{\mathbb{C}}$ 中的线性变换都可以从 $X$ 中的变换得到;如果我们进行坐标复化,只有具有实数矩阵的变换才有效。请注意,这与第 8.1 节中描述的反复化的情景完全相反。一个细心的读者可能已经注意到,复化和反复化的操作不是彼此的逆。首先,空间及其复化具有相同的维度,而一个 $n$ 维空间的复化其维度为 $2n$。此外,正如我们刚才讨论的,实数和复数线性变换之间的关系在这些情况下是完全相反的。在下一节中,我们将讨论一个操作,在某种意义上是反复化的逆。

8.3. \textbf{向实数空间引入复数结构。} 本节介绍的构造仅适用于偶数维的实数空间。
8.3.1. \textbf{引入复数结构的初等方法。} 设 $X$ 是一个 $2n$ 维的实内积空间。我们想通过引入 $X$ 上的复数结构来逆转反复化过程,即识别这个空间与一个复数空间,使其反复化(见第 8.1 节)得到原始空间 $X$。最简单的想法是固定 $X$ 中的一个标准正交基,然后将该基下的坐标分成两半。我们然后将一半坐标(例如,坐标 $x_1, x_2, \dots, x_n$)视为复数坐标的实部,并将其余部分视为虚部。然后我们需要将实部和虚部组合起来:例如,如果我们处理 $x_1, x_2, \dots, x_n$ 作为实部,$x_{n+1}, x_{n+2}, \dots, x_{2n}$ 作为虚部,我们可以定义复数坐标 $z_k = x_k + ix_{n+k}$。当然,结果通常取决于标准正交基的选择,以及我们如何分割实数坐标为实部和虚部,以及如何将它们组合起来。从第 8.1 节描述的反复化构造也可以看出,所有实内积空间 $X$ 上的复数结构都可以通过这种方式获得。
8.3.2. \textbf{从初等到抽象构造复数结构。} 上述构造可以用抽象的、坐标无关的方式来描述。即,设我们将空间 $X$ 分解为 $X = E \oplus E^\perp$,其中 $E$ 是一个子空间,$\dim E = n$(因此 $\dim E^\perp = n$),并且设 $U_0: E \to E^\perp$ 是一个酉(更确切地说,是正交的,因为我们的空间是实数的)变换。注意,如果 $\{v_1, v_2, \dots, v_n\}$ 是 $E$ 中的一个标准正交基,那么系统 $\{U_0 v_1, U_0 v_2, \dots, U_0 v_n\}$ 是 $E^\perp$ 中的一个标准正交基,因此 $\{v_1, v_2, \dots, v_n, U_0 v_1, U_0 v_2, \dots, U_0 v_n\}$ 是整个空间 $X$ 中的一个标准正交基。如果 $x_1, x_2, \dots, x_{2n}$ 是该基下向量 $x$ 的坐标,并且我们将 $x_k + ix_{n+k}$, $k=1, 2, \dots, n$ 作为 $x$ 的复数坐标,那么 $i$ 的乘法由正交变换 $U$ 表示,该变换在子空间 $E, E^\perp$ 的正交基下由块对角矩阵 $U = \begin{pmatrix} 0 & -U_0^* \\ U_0 & 0 \end{pmatrix}$ 给出。这意味着 $i \begin{pmatrix} x_1 \\ x_2 \end{pmatrix} = U \begin{pmatrix} x_1 \\ x_2 \end{pmatrix} = \begin{pmatrix} 0 & -U_0^* \\ U_0 & 0 \end{pmatrix} \begin{pmatrix} x_1 \\ x_2 \end{pmatrix}$,$x_1 \in E$, $x_2 \in E^\perp$。显然,$U$ 是一个正交变换,且 $U^2 = -I$。因此,任何实内积空间 $X$ 上的复数结构都由满足 $U^2 = -I$ 的正交变换 $U$ 给出;变换 $U$ 赋予了我们虚数单位 $i$ 的乘法。




$i x = U x$, 那么复数乘法必须定义为 $( \alpha + \beta i ) x := \alpha x + \beta U x = (\alpha I + \beta U) x$, $\alpha, \beta \in \mathbb{R}$, $x \in X$。(8.2)
我们将使用这个公式来定义复数乘法。不难检查出,对于由 (8.2) 定义的复数乘法,所有复向量空间的公理都得到了满足。例如,可以通过利用实空间 $X$ 中的线性并注意到,关于代数运算(加法和乘法),形式为 $\alpha I + \beta U$ 的线性变换的行为方式与复数 $\alpha + \beta i$ 完全相同,即这样的变换给了我们复数域的一个\textbf{表示}。这意味着首先,形式为 $\alpha I + \beta U$ 的变换的和与积是相同形式的变换,并且为了得到结果的系数 $\alpha, \beta$,我们可以对相应的复数执行运算并取结果的实部和虚部。注意,这里我们需要 $U^2 = -I$ 的恒等式,但我们不需要 $U$ 是正交变换的事实。因此,我们得到了一个复向量空间的结构。为了得到一个复\textbf{内积}空间,我们需要引入一个复内积,使得原始实内积是它的实部。我们在这里确实没有其他选择:注意到对于复内积 $\text{Im}(x, y) = \text{Re}[-i(x, y)_{\mathbb{R}}] = \text{Re}((x, i y)_{\mathbb{R}})$,我们发现定义复内积的唯一方法是
$(x, y)_{\mathbb{C}} := (x, y)_{\mathbb{R}} + i(x, Uy)_{\mathbb{R}}$。(8.3)

我们来证明这是一个内积。我们需要 $U^* = -U$ 的事实,见下面的练习 8.4(这里的 $U^*$ 是指相对于原始实内积的伴随)。为了证明 $(y, x)_{\mathbb{C}} = (x, y)_{\mathbb{C}}$,我们使用恒等式 $U^* = -U$ 和实内积的对称性:$(y, x)_{\mathbb{C}} = (y, x)_{\mathbb{R}} + i(y, Ux)_{\mathbb{C}} = (x, y)_{\mathbb{R}} + i(Ux, y)_{\mathbb{R}} = (x, y)_{\mathbb{R}} - i(x, Uy)_{\mathbb{R}} = (x, y)_{\mathbb{R}} + i(x, Uy)_{\mathbb{R}} = (x, y)_{\mathbb{C}}$。为了证明复内积的线性性,让我们首先注意到 $(x, y)_{\mathbb{C}}$ 在第一个(实际上是每个)参数上是\textbf{实线性}的,即对于 $\alpha, \beta \in \mathbb{R}$,$( \alpha x + \beta y, z )_{\mathbb{C}} = \alpha (x, z)_{\mathbb{C}} + \beta (y, z)_{\mathbb{C}}$;这是正确的,因为右侧的每个加数在参数上都是实线性的。使用 $(x, y)_{\mathbb{C}}$ 的实线性以及 $U^* = -U$(这意味着 $(Ux, y)_{\mathbb{R}} = -(x, Uy)_{\mathbb{R}}$)以及 $U$ 的正交性,我们得到以下等式链:$(\alpha I + \beta U)x, y)_{\mathbb{C}} = (\alpha x, y)_{\mathbb{C}} + \beta (Ux, y)_{\mathbb{C}} = \alpha (x, y)_{\mathbb{C}} + \beta [(Ux, y)_{\mathbb{R}} + i(Ux, Uy)_{\mathbb{R}}] = \alpha (x, y)_{\mathbb{C}} + \beta [-(x, Uy)_{\mathbb{R}} + i(x, y)_{\mathbb{R}}] = \alpha (x, y)_{\mathbb{C}} + \beta i [(x, y)_{\mathbb{R}} + i(x, Uy)_{\mathbb{R}}] = \alpha (x, y)_{\mathbb{C}} + \beta i (x, y)_{\mathbb{C}} = (\alpha + \beta i)(x, y)_{\mathbb{C}}$,这证明了\textbf{复}线性。最后,为了证明 $(x, x)_{\mathbb{C}}$ 的非负性,让我们注意到(见练习 8.3)$(x, Ux)_{\mathbb{R}} = 0$,所以 $(x, x)_{\mathbb{C}} = (x, x)_{\mathbb{R}} = \|x\|^2 \geq 0$。

8.3. \textbf{从初等到抽象构造复数结构。} 对于不习惯如此“高明”和抽象的证明的读者,还有另一种更实际的解释。即,可以证明(见练习 8.5),存在一个子空间 $E$,$\dim E = n$(回想一下 $\dim X = 2n$),使得在分解 $X = E \oplus E^\perp$ 下 $U$ 的矩阵由块对角矩阵 $U = \begin{pmatrix} 0 & -U_0^* \\ U_0 & 0 \end{pmatrix}$ 给出,其中 $U_0: E \to E^\perp$ 是某个正交变换。



设 $\{v_1, v_2, \dots, v_n\}$ 是 $E$ 中的一个标准正交基。那么 $\{U_0 v_1, U_0 v_2, \dots, U_0 v_n\}$ 是 $E^\perp$ 中的一个标准正交基,所以 $\{v_1, v_2, \dots, v_n, U_0 v_1, U_0 v_2, \dots, U_0 v_n\}$ 是整个空间 $X$ 中的一个标准正交基。考虑该基下的坐标 $x_1, x_2, \dots, x_{2n}$,并将 $x_k + ix_{n+k}$, $k=1, 2, \dots, n$ 作为 $x$ 的复数坐标,那么 $i$ 的乘法由变换 $U$ 来表示:它对于 $x \in E$ 是平凡的,对于 $y \in E^\perp$ 也是平凡的,因此它对于所有实数线性组合 $\alpha x + \beta y$ 都成立,即对于 $X$ 中的所有向量都成立。但这意味着抽象复数结构的引入以及相应的初等方法给出了相同的结果!而且,由于初等方法清楚地给出复数结构,抽象方法也给出了相同的复数结构。

\textbf{练习}~

8.1. 证明公式 (8.1)。即,证明如果 $x = (z_1, z_2, \dots, z_n)^T$, $y = (w_1, w_2, \dots, w_n)^T$, $z_k = x_k + iy_k$, $w_k = u_k + iv_k$, $x_k, y_k, u_k, v_k \in \mathbb{R}$,那么 $\text{Re}(\sum_{k=1}^n z_k \bar{w}_k) = \sum_{k=1}^n x_k u_k + \sum_{k=1}^n y_k v_k$。

8.2. 证明如果 $(x, y)_{\mathbb{C}}$ 是复内积空间中的内积,那么 $(x, y)_{\mathbb{R}}$ 由 (8.1) 定义的是一个实内积空间。

8.3. 设 $U$ 是一个满足 $U^2 = -I$ 的正交变换(在实内积空间 $X$ 中)。证明对于所有 $x \in X$, $Ux \perp x$。

8.4. 证明,如果 $U$ 是一个满足 $U^2 = -I$ 的正交变换,那么 $U^* = -U$。

8.5. 设 $U$ 是一个满足 $U^2 = -I$ 的实内积空间中的正交变换。证明在这种情况下 $\dim X = 2n$,并且存在一个子空间 $E \subset X$,$\dim E = n$,以及一个正交变换 $U_0: E \to E^\perp$,使得在 $X = E \oplus E^\perp$ 的分解下,$U$ 由块对角矩阵 $U = \begin{pmatrix} 0 & -U_0^* \\ U_0 & 0 \end{pmatrix}$ 给出。这个陈述可以很容易地从第 6 章定理 5.1 得到,如果我们注意到 $\mathbb{R}^2$ 中的唯一满足 $R_\alpha^2 = -I$ 的旋转是角度为 $\pm \pi/2$ 的旋转。但是,可以找到一个初等的证明,而无需使用该定理。例如,该陈述在 $\dim X = 2$ 时是平凡的:在这种情况下,我们可以选择任何一维子空间作为 $E$,见练习 8.3。




然后,不难证明,这样的变换 $U$ 不存在于 $\mathbb{R}^2$ 中,并且我们可以通过归纳 $\dim X$ 来完成证明。






% \chapter{内积空间中算子的结构}

在本章中,我们再次假设所有空间都是有限维的。同样,我们只处理复数或实数空间,内积空间的理论不适用于任意域上的空间。当没有提及我们所处的空间时,所有结果都适用于复数和实数空间。为了避免重复书写基本相同的公式,我们将使用复数情况的记号:在实数情况下,它给出正确但有时稍显复杂的公式。

\section{算子的上三角(舒尔)表示}

\textbf{定理 1.1。} 设 $A: X \to X$ 是作用在复内积空间中的算子。存在一个标准正交基 $\{u_1, u_2, \dots, u_n\}$ 在 $X$ 中,使得 $A$ 在该基下的矩阵是上三角矩阵。换句话说,任何 $n \times n$ 矩阵 $A$ 都可以表示为 $A = UTU^*$,其中 $U$ 是酉矩阵,而 $T$ 是上三角矩阵。

\textbf{证明}~ 我们使用 $\dim X$ 的数学归纳法来证明定理。如果 $\dim X = 1$,则定理是平凡的,因为任何 $1 \times 1$ 矩阵都是上三角矩阵。假设我们已经证明了当 $\dim X = n-1$ 时定理成立,并且我们想证明对于 $\dim X = n$ 时定理成立。
设 $\lambda_1$ 是 $A$ 的一个特征值,设 $u_1$, $\|u_1\| = 1$ 是相应的特征向量,$Au_1 = \lambda_1 u_1$。记 $E = u_1^\perp$,并设 $\{v_2, \dots, v_n\}$ 是 $E$ 中的某个标准正交基(显然 $\dim E = \dim X - 1 = n-1$),那么 $\{u_1, v_2, \dots, v_n\}$ 是 $X$ 中的一个标准正交基。在这一基下,$A$ 的矩阵具有形式 (1.1)
$\begin{pmatrix} \lambda_1 & * & \dots & * \\ 0 & & & \\ \vdots & & A_1 & \\ 0 & & & \end{pmatrix}$;
这里 $\lambda_1$ 下方的所有元素都是零,而 $*$ 表示我们不关心 $\lambda_1$ 右侧的元素。我们足够关心右下角的 $(n-1) \times (n-1)$ 块,以便给它命名:我们将其记为 $A_1$。注意,$A_1$ 定义了 $E$ 中的一个线性变换,并且由于 $\dim E = n-1$,归纳假设表明存在一个标准正交基(我们将其记为 $\{u_2, \dots, u_n\}$),使得 $A_1$ 在该基下的矩阵是上三角矩阵。因此,$A$ 在该标准正交基 $\{u_1, u_2, \dots, u_n\}$ 下的矩阵也是上三角矩阵。

\textbf{注释} ~注意,在证明中引入的子空间 $E = u_1^\perp$ 对于 $A$ 不是不变的,即 $AE \subseteq E$ 不一定成立。这意味着 $A_1$ 不是 $A$ 的一部分,它是从 $A$ 构建的某个算子。还需注意,$AE \subseteq E$ 当且仅当所有标记为 $*$ 的元素(即除了 $\lambda_1$ 之外的第一行的所有元素)都为零。

\textbf{注释} ~注意,即使我们从一个实数矩阵 $A$ 开始,矩阵 $U$ 和 $T$ 也可以是复数的。旋转矩阵 $\begin{pmatrix} \cos \alpha & -\sin \alpha \\ \sin \alpha & \cos \alpha \end{pmatrix}$, $\alpha \neq k\pi, k \in \mathbb{Z}$ 与实数上三角矩阵不是酉等价的(甚至不是相似的)。因为该矩阵的特征值是复数,而上三角矩阵的特征值是对角线元素。

\textbf{注释} ~1.1 定理的一个类似版本可以陈述并证明用于任意向量空间,而不要求它具有内积。在这种情况下,定理声称在某个基下任何算子都有上三角形式。可以通过模仿定理 1.1 的证明来完成。另一种方法是为 $V$ 装备内积,方法是固定一个基并声明它是标准正交基,见第 5 章第 2.4 节。


注意,内积空间版本(定理 1.1)比向量空间版本更强大,因为它说明我们总能找到一个标准正交基,而不仅仅是一个基。下面的定理是定理 1.1 的实数版本:

\textbf{定理 1.2。} 设 $A: X \to X$ 是作用在\textbf{实数}内积空间中的算子。假设 $A$ 的所有特征值都是实数(意味着 $A$ 恰好有 $n = \dim X$ 个实数特征值,计入重数)。那么存在 $X$ 中的一个标准正交基 $\{u_1, u_2, \dots, u_n\}$,使得 $A$ 在该基下的矩阵是上三角矩阵。换句话说,任何具有所有实数特征值的实数 $n \times n$ 矩阵 $A$ 都可以表示为 $A = UTU^* = U T U^T$,其中 $U$ 是正交矩阵,而 $T$ 是实数上三角矩阵。

\textbf{证明}~ 为了证明定理,我们只需要分析定理 1.1 的证明。让我们假设(我们可以无损于一般性地这样做)算子(矩阵)$A$ 作用在 $\mathbb{R}^n$ 上。假设定理对 $(n-1) \times (n-1)$ 矩阵成立。在定理 1.1 的证明中,设 $\lambda_1$ 是 $A$ 的一个实特征值,$u_1 \in \mathbb{R}^n$, $\|u_1\| = 1$ 是相应的特征向量,设 $\{v_2, \dots, v_n\}$ 是 $\mathbb{R}^n$ 中的一个标准正交系统,使得 $\{u_1, v_2, \dots, v_n\}$ 是 $\mathbb{R}^n$ 中的一个标准正交基。在该基下 $A$ 的矩阵具有形式 (1.1),其中 $A_1$ 是某个实数矩阵。如果我们能证明矩阵 $A_1$ 只有实数特征值,那么我们就完成了。确实,根据归纳假设,存在 $E = u_1^\perp$ 中的一个标准正交基 $\{u_2, \dots, u_n\}$,使得 $A_1$ 在该基下的矩阵是上三角矩阵,因此 $A$ 在 $\{u_1, u_2, \dots, u_n\}$ 基下的矩阵也是上三角矩阵。为了证明 $A_1$ 只有实数特征值,让我们注意到 $\det(A - \lambda I) = ( \lambda_1 - \lambda ) \det( A_1 - \lambda )$(例如,通过第一行的代数余子式展开),所以 $A_1$ 的任何特征值也是 $A$ 的特征值。但是 $A$ 只有实数特征值!

\section{自伴随和正规算子的谱定理}
在本章中,我们处理的是酉等价于对角矩阵的矩阵(算子)。

让我们回忆一下,如果一个算子满足 $A = A^*$,则称其为\textbf{自伴随}的。在某个标准正交基下的自伴随算子(即满足 $A^* = A$ 的矩阵)称为\textbf{Hermitian}矩阵。术语“自伴随”和“Hermitian”基本上是同义的。通常人们在谈论算子(变换)时说自伴随,在谈论矩阵时说 Hermitian。我们将尝试遵循这个约定,但由于我们经常不区分算子和它们的矩阵,所以有时会混合使用这两个术语。

\textbf{定理 2.1。} 设 $A = A^*$ 是内积空间 $X$(空间可以是复数或实数)中的一个自伴随算子。那么 $A$ 的所有特征值都是实数,并且 $X$ 中存在 $A$ 的特征向量的标准正交基。

这个定理可以用矩阵形式重述如下:

\textbf{定理 2.2。} 设 $A = A^*$ 是一个自伴随(因此是方阵)矩阵。那么 $A$ 可以表示为 $A = UDU^*$,其中 $U$ 是酉矩阵,$D$ 是具有实数项的对角矩阵。而且,如果矩阵 $A$ 是实数的,矩阵 $U$ 可以选择为实数的(即正交的)。

\textbf{证明}~ 为了证明定理 2.1 和定理 2.2,我们首先对内积空间 $X$(或实数空间 $X$)应用定理 1.1(或定理 1.2)来找到一个标准正交基,使得 $A$ 在该基下的矩阵是上三角矩阵。现在让我们问自己一个问题:什么样的上三角矩阵是自伴随的?答案是显而易见的:上三角矩阵是自伴随的当且仅当它是具有实数项的对角矩阵。定理 2.1(以及因此定理 2.2)得证。

\textbf{注释} ~注意,在许多教科书中只考虑实数矩阵,并且定理 2.2 通常被称为“\textbf{对称矩阵的谱定理}”。然而,我们应该强调,定理 2.2 的结论对于\textbf{复数}对称矩阵是不成立的:该定理适用于 Hermitian 矩阵,特别是\textbf{实数}对称矩阵。

让我们给出一个 $A=A^*$ 的算子的特征值是实数的独立证明。设 $A=A^*$ 且 $Ax=\lambda x$, $x \neq 0$。那么 $(Ax, x) = (\lambda x, x) = \lambda(x, x) = \lambda\|x\|^2$。另一方面,$(Ax, x) = (x, A^*x) = (x, Ax) = (x, \lambda x) = \bar{\lambda}(x, x) = \bar{\lambda}\|x\|^2$(这里我们用了 $(x, \lambda y) = \bar{\lambda}(x, y)$)。所以 $\lambda\|x\|^2 = \bar{\lambda}\|x\|^2$。由于 $x \neq 0$,$\|x\|^2 \neq 0$,我们可以得出 $\lambda = \bar{\lambda}$,所以 $\lambda$ 是实数。

从定理 2.1 也可以得出,自伴随算子的特征子空间是相互正交的。让我们给出一个该结果的独立证明。

\textbf{命题 2.3。} 设 $A = A^*$ 是一个自伴随算子,设 $u, v$ 是它的特征向量,$Au = \lambda u$, $Av = \mu v$。那么,如果 $\lambda \neq \mu$,则特征向量 $u$ 和 $v$ 是正交的。

\textbf{证明}~ 这个命题虽然可以从谱定理(定理 1.1)得出,但我们在这里给出一个直接的证明。即,$(Au, v) = (\lambda u, v) = \lambda(u, v)$。另一方面,$(Au, v) = (u, A^*v) = (u, Av) = (u, \mu v) = \mu(u, v)$(最后一个等式成立是因为自伴随算子的特征值是实数),所以 $\lambda(u, v) = \mu(u, v)$。如果 $\lambda \neq \mu$,则这只有在 $(u, v) = 0$ 时才可能。

现在让我们尝试找到哪些矩阵是酉等价于一个对角矩阵。可以很容易地检验出,对于对角矩阵 $D$, $D^*D = DD^*$。因此,如果 $A$ 在某个标准正交基下的矩阵是对角矩阵,那么 $A^*A = AA^*$。

\textbf{定义}~ 称算子(矩阵)$N$ 是\textbf{正规}的,如果 $N^*N = NN^*$。显然,任何自伴随算子($A^*A = AA^*$)都是正规的。同样,任何酉算子 $U: X \to X$ 也是正规的,因为 $U^*U = UU^* = I$。注意,正规算子是作用在同一个空间上的算子,而不是从一个空间到另一个空间。所以,如果 $U$ 是作用在一个空间到另一个空间上的酉算子,我们就不能说 $U$ 是正规的。

\textbf{定理 2.4。} 任何复数向量空间中的正规算子 $N$ 都有一个标准正交的特征向量基。换句话说,任何满足 $N^*N = NN^*$ 的矩阵 $N$ 都可以表示为 $N = UDU^*$,其中 $U$ 是酉矩阵,$D$ 是对角矩阵。

\textbf{注释} ~注意,在上述定理中,即使 $N$ 是实数矩阵,我们也没有声称矩阵 $U$ 和 $D$ 是实数。而且,可以很容易地证明,如果 $D$ 是实数,那么 $N$ 必须是自伴随的。



\textbf{定理 2.4 的证明。} 为了证明定理 2.4,我们应用定理 1.1 来得到一个标准正交基,使得 $N$ 在该基下的矩阵是上三角矩阵。为了完成定理的证明,我们只需要证明一个上三角正规矩阵必须是对角矩阵。我们将使用矩阵维数的数学归纳法来证明这一点。$1 \times 1$ 矩阵的情况是平凡的,因为任何 $1 \times 1$ 矩阵都是对角矩阵。假设我们已经证明了任何 $(n-1) \times (n-1)$ 的上三角正规矩阵都是对角矩阵,并且我们想证明对于 $n \times n$ 矩阵也成立。设 $N$ 是一个 $n \times n$ 上三角正规矩阵。我们可以将其写成
$N = \begin{pmatrix} a_{1,1} & a_{1,2} & \dots & a_{1,n} \\ 0 & & & \\ \vdots & & N_1 & \\ 0 & & & \end{pmatrix}$
其中 $N_1$ 是一个 $(n-1) \times (n-1)$ 的上三角矩阵。让我们比较 $N^*N$ 和 $NN^*$ 的左上角元素(第一行第一列)。直接计算表明 $(N^*N)_{1,1} = \bar{a}_{1,1}a_{1,1} = |a_{1,1}|^2$,而 $(NN^*)_{1,1} = |a_{1,1}|^2 + |a_{1,2}|^2 + \dots + |a_{1,n}|^2$。所以,$(N^*N)_{1,1} = (NN^*)_{1,1}$ 当且仅当 $a_{1,2} = \dots = a_{1,n} = 0$。因此,矩阵 $N$ 具有形式
$N = \begin{pmatrix} a_{1,1} & 0 & \dots & 0 \\ 0 & & & \\ \vdots & & N_1 & \\ 0 & & & \end{pmatrix}$。
从上述表示可以得出 $N^*N = \begin{pmatrix} |a_{1,1}|^2 & & \\ & N_1^*N_1 & \\ & & \end{pmatrix}$,$NN^* = \begin{pmatrix} |a_{1,1}|^2 & & \\ & N_1N_1^* & \\ & & \end{pmatrix}$。所以 $N_1^*N_1 = N_1N_1^*$。这意味着矩阵 $N_1$ 也是正规的,并且根据归纳假设它是对角矩阵。所以矩阵 $N$ 也是对角矩阵。

以下命题给出了正规算子的一个非常有用的刻画。

\textbf{命题 2.5。} 算子 $N: X \to X$ 是正规的当且仅当 $\|Nx\| = \|N^*x\|$ $\forall x \in X$。

\textbf{证明}~ 设 $N$ 是正规的,$N^*N = NN^*$。那么 $\|Nx\|^2 = (Nx, Nx) = (N^*Nx, x) = (NN^*x, x) = (N^*x, N^*x) = \|N^*x\|^2$,所以 $\|Nx\| = \|N^*x\|$。现在设 $\|Nx\| = \|N^*x\|$ $\forall x \in X$。极化恒等式(第 5 章引理 1.9)暗示对于所有 $x, y \in X$,$(N^*Nx, y) = (Nx, Ny) = \frac{1}{4} \sum_{\alpha = \pm 1, \pm i} \alpha \|Nx + \alpha Ny\|^2 = \frac{1}{4} \sum_{\alpha = \pm 1, \pm i} \alpha \|N(x + \alpha y)\|^2 = \frac{1}{4} \sum_{\alpha = \pm 1, \pm i} \alpha \|N^*(x + \alpha y)\|^2 = (N^*x, N^*y) = (NN^*x, y)$。因此(见推论 1.6),$N^*N = NN^*$。

\textbf{练习}~

2.1. 真或假:
a) 任何酉算子 $U: X \to X$ 都是正规的。
b) 矩阵是酉的当且仅当它是可逆的。
c) 如果两个矩阵酉等价,那么它们也相似。
d) 两个自伴随算子之和是自伴随的。
e) 酉算子的伴随是酉的。
f) 正规算子的伴随是正规的。
g) 如果一个线性算子的所有特征值都是 1,那么该算子必须是酉的或正交的。
h) 如果一个正规算子的所有特征值都是 1,那么该算子是恒等算子。
i) 线性算子可能保持范数但不保持内积。

2.2. 真或假:两个正规算子之和是正规的?证明你的结论。

2.3. 证明一个酉等价于对角矩阵的矩阵是正规的。

2.4. \textbf{正交对角化}矩阵 $\begin{pmatrix} 3 & 2 \\ 2 & 3 \end{pmatrix}$。找出 $A$ 的所有平方根,即找出所有满足 $B^2 = A$ 的矩阵 $B$。
\textbf{注:} $A$ 的所有平方根都是自伴随的。




2.5. 真或假:任何自伴随矩阵都有一个自伴随的平方根。证明你的结论。

2.6. \textbf{正交对角化}矩阵 $A = \begin{pmatrix} 7 & 2 \\ 2 & 4 \end{pmatrix}$,即将其表示为 $A = UDU^*$,其中 $D$ 是对角矩阵,$U$ 是酉矩阵。在 $A$ 的所有平方根中,找出具有正特征值的平方根。你可以将 $B$ 表示为乘积形式。

2.7. 真或假:a) 两个自伴随矩阵的乘积是自伴随的。b) 如果 $A$ 是自伴随的,那么 $A^k$ 是自伴随的。证明你的结论。

2.8. 设 $A$ 是 $m \times n$ 矩阵。证明:
a) $A^*A$ 是自伴随的。
b) $A^*A$ 的所有特征值都是非负的。
c) $A^*A + I$ 是可逆的。

2.9. 如果陈述为真,则证明;如果陈述为假,则给出反例:
a) 如果 $A$ 是自伴随的,那么 $A + iI$ 是可逆的。
b) 如果 $U$ 是酉的,$U + \frac{3}{4}I$ 是可逆的。
c) 如果矩阵 $A$ 是实数的,那么 $A - iI$ 是可逆的。

2.10. \textbf{正交对角化}旋转矩阵 $R_\alpha = \begin{pmatrix} \cos \alpha & -\sin \alpha \\ \sin \alpha & \cos \alpha \end{pmatrix}$,其中 $\alpha$ 不是 $\pi$ 的整数倍。注意,在这种情况下你会得到复数特征值。

2.11. \textbf{正交对角化}矩阵 $A = \begin{pmatrix} \cos \alpha & \sin \alpha \\ \sin \alpha & -\cos \alpha \end{pmatrix}$。
\textbf{提示:} 你会得到实数特征值。此外,三角恒等式 $\sin^2 x = 2 \sin x \cos x$, $\sin^2 x = (1 - \cos 2x)/2$, $\cos^2 x = (1 + \cos 2x)/2$(应用于 $x = \alpha/2$)将有助于简化特征向量的表达式。

2.12. 你能从几何上描述上一问题中矩阵 $A$ 所代表的线性变换吗?它有一个非常简单的几何解释。

2.13. 证明一个具有模为 1 的特征值(即所有特征值满足 $|\lambda_k| = 1$)的正规算子是酉的。
\textbf{提示:} 考虑对角化。

2.14. 证明一个具有实数特征值的正规算子是自伴随的。




\section{正定算子~平方根}
\textbf{定义}~ 称自伴随算子 $A: X \to X$ 为\textbf{正定}的,如果 $(Ax, x) > 0$ $\forall x \neq 0$,称其为\textbf{半正定}的,如果 $(Ax, x) \geq 0$ $\forall x \in X$。我们将使用记号 $A > 0$ 表示正定算子,$A \geq 0$ 表示半正定算子。下面的定理描述了正定和半正定算子。

\textbf{定理 3.1。} 设 $A = A^*$。那么
1. $A > 0$ 当且仅当 $A$ 的所有特征值都是正的。
2. $A \geq 0$ 当且仅当 $A$ 的所有特征值都是非负的。
3. $A < 0$ 当且仅当 $A$ 的所有特征值都是负的。
4. $A \leq 0$ 当且仅当 $A$ 的所有特征值都是非正的。
5. $A$ 是不定的,当且仅当它既有正特征值也有负特征值。

\textbf{证明}~ 通过选取一个标准正交基,使得 $A$ 在该基下的矩阵是对角矩阵(见定理 2.1),我们可以无损于一般性地证明。要完成证明,只需注意到,对于对角矩阵,当且仅当其对角线元素都为正(非负)时,该矩阵才是正定(半正定)的。

\textbf{推论 3.2。} 设 $A = A^* \geq 0$ 是一个半正定算子。存在一个唯一的半正定算子 $B$,使得 $B^2 = A$。这样的 $B$ 被称为 $A$ 的(正)\textbf{平方根},并记作 $\sqrt{A}$ 或 $A^{1/2}$。

\textbf{证明}~ 我们来证明 $\sqrt{A}$ 的存在性。设 $\{v_1, v_2, \dots, v_n\}$ 是 $A$ 的特征向量的标准正交基,并设 $\lambda_1, \lambda_2, \dots, \lambda_n$ 是相应的特征值。注意,由于 $A \geq 0$,所有 $\lambda_k \geq 0$。在基 $\{v_1, v_2, \dots, v_n\}$ 下,$A$ 的矩阵是对角矩阵 $\text{diag}\{\lambda_1, \lambda_2, \dots, \lambda_n\}$,对角线上是 $\lambda_1, \lambda_2, \dots, \lambda_n$。定义 $B$ 在同一基下的矩阵为 $\text{diag}\{\sqrt{\lambda_1}, \sqrt{\lambda_2}, \dots, \sqrt{\lambda_n}\}$。显然,$B = B^* \geq 0$ 且 $B^2 = A$。

为了证明 $B$ 的唯一性,让我们假设存在一个算子 $C = C^* \geq 0$ 使得 $C^2 = A$。设 $\{u_1, u_2, \dots, u_n\}$ 是 $C$ 的特征向量的标准正交基,并设 $\mu_1, \mu_2, \dots, \mu_n$ 是相应的特征值(注意 $\mu_k \geq 0$ $\forall k$)。$C$ 在该基下的矩阵是对角矩阵 $\text{diag}\{\mu_1, \mu_2, \dots, \mu_n\}$,因此 $A = C^2$ 在同一基下的矩阵是 $\text{diag}\{\mu_1^2, \mu_2^2, \dots, \mu_n^2\}$。这暗示 $A$ 的任何特征值 $\lambda$ 都必须是 $\mu_k^2$ 的形式,并且,更重要的是,如果 $Ax = \lambda x$,那么 $Cx = \sqrt{\lambda} x$。因此,在上面的基 $\{v_1, v_2, \dots, v_n\}$ 下,$C$ 的矩阵是对角矩阵 $\text{diag}\{\sqrt{\lambda_1}, \sqrt{\lambda_2}, \dots, \sqrt{\lambda_n}\}$,即 $B = C$。

3.2. \textbf{算子的模。奇异值。} 考虑算子 $A: X \to Y$。它的\textbf{Hermitian平方} $A^*A$ 是作用在 $X$ 上的半正定算子。确实,$(A^*A)^* = A^*(A^*)^* = A^*A$ 并且 $(A^*Ax, x) = (Ax, Ax) = \|Ax\|^2 \geq 0$ $\forall x \in X$。因此,存在一个(唯一的)半正定算子 $R = \sqrt{A^*A}$。这个算子 $R$ 被称为算子 $A$ 的\textbf{模},通常记为 $|A|$。$A$ 的模显示了算子 $A$ 的“大小”:

\textbf{命题 3.3。} 对于线性算子 $A: X \to Y$,$\| |A| x \| = \|Ax\|$ $\forall x \in X$。

\textbf{证明}~ 对于任何 $x \in X$,$\| |A| x \|^2 = (|A|x, |A|x) = (|A|^*|A|x, x) = ( |A|^2 x, x ) = (A^*Ax, x) = (Ax, Ax) = \|Ax\|^2$。

\textbf{推论 3.4。} $\text{Ker } A = \text{Ker } |A| = (\text{Ran } |A|)^\perp$。

\textbf{证明}~ 第一个等式直接来自命题 3.3,第二个等式来自恒等式 $\text{Ker } T = (\text{Ran } T^*)^\perp$($|A|$ 是自伴随的)。

\textbf{定理 3.5(算子的极分解)。} 设 $A: X \to X$ 是一个算子(方阵)。那么 $A$ 可以表示为 $A = U|A|$,其中 $U$ 是酉算子。

\textbf{注释} ~酉算子 $U$ 通常不是唯一的。正如从定理的证明中可以看出,$U$ 仅在 $A$ 可逆时才唯一。

\textbf{注释} ~极分解 $A = U|A|$ 也适用于作用在一个空间到另一个空间上的算子 $A: X \to Y$。但在这种情况下,我们只能保证 $U$ 是从 $\text{Ran } |A| = (\text{Ker } A)^\perp$ 到 $Y$ 的一个等距同构。如果 $\dim X \leq \dim Y$,则此等距同构可以扩展为从整个 $X$ 到 $Y$ 的等距同构(如果 $\dim X = \dim Y$,则它将是一个酉算子)。

\textbf{定理 3.5 的证明。} 考虑向量 $x \in \text{Ran } |A|$。那么向量 $x$ 可以表示为 $x = |A|v$ 对于某个向量 $v \in X$。定义 $U_0 x := Av$。根据命题 3.3 $\|U_0 x\| = \|Ax\| = \||A|v\| = \|x\|$,所以看起来 $U$ 是从 $\text{Ran } |A|$ 到 $X$ 的一个等距同构。但首先我们需要证明 $U_0$ 是良好定义的。设 $v_1$ 是另一个使得 $x = |A|v_1$ 的向量。但是 $x = |A|v = |A|v_1$ 意味着 $v - v_1 \in \text{Ker } |A| = \text{Ker } A$(参见推论 3.4),所以 $Av = Av_1$,这意味着 $U_0 x$ 是良好定义的。根据构造,$A = U_0|A|$。我们将 $U_0$ 扩展为一个酉算子 $U$ 的证明留给读者,读者需要检查 $U_0$ 是一个线性变换。设 $U = U_0 + U_1$,其中 $U_1$ 是从 $(\text{Ran } |A|)^\perp = \text{Ker } A$ 到 $(\text{Ran } A)^\perp$ 的一个酉算子(注意,从秩定理可知 $\dim \text{Ker } A = \dim \text{Ker } A^* = \dim (\text{Ran } A)^\perp$)。容易看出 $U$ 是一个酉算子,并且 $A = U|A|$。

3.3. \textbf{奇异值。施密特分解。}
\textbf{定义}~ $|A|$ 的特征值被称为 $A$ 的\textbf{奇异值}。换句话说,如果 $\lambda_1, \lambda_2, \dots, \lambda_n$ 是 $A^*A$ 的特征值,那么 $\sqrt{\lambda_1}, \sqrt{\lambda_2}, \dots, \sqrt{\lambda_n}$ 就是 $A$ 的奇异值。
\textbf{注释} ~在许多文献中,奇异值被定义为 $A^*A$ 的特征值的非负平方根,而不提及算子 $|A|$。我认为算子 $|A|$ 的概念很重要,所以上面已经介绍了。然而,算子 $|A|$ 的概念对于后续内容(定义舒尔和奇异值分解)不是必需的。此外,正如下面将要显示的,算子 $|A|$ 可以很容易地从奇异值分解构造出来。

设 $A: X \to Y$ 是一个算子,并设 $\sigma_1, \sigma_2, \dots, \sigma_n$ 是 $A$ 的奇异值(计入重数)。假设 $\sigma_1, \sigma_2, \dots, \sigma_r$ 是 $A$ 的\textbf{非零}奇异值(计入重数)。这意味着,特别地,$\sigma_k = 0$ 对于 $k > r$。




根据奇异值的定义,数字 $\sigma_1^2, \sigma_2^2, \dots, \sigma_n^2$ 是 $A^*A$ 的特征值。设 $\{v_1, v_2, \dots, v_n\}$ 是 $A^*A$ 的特征向量的标准正交基,$A^*Av_k = \sigma_k^2 v_k$。

\textbf{命题 3.6。} 系统 $\{w_k := \frac{1}{\sigma_k} Av_k, k = 1, 2, \dots, r\}$ 是一个标准正交系统。

\textbf{证明}~ $(Av_j, Av_k) = (A^*Av_j, v_k) = (\sigma_j^2 v_j, v_k) = \sigma_j^2 (v_j, v_k) = \begin{cases} 0, & j \neq k \\ \sigma_j^2, & j = k \end{cases}$,因为 $\{v_1, v_2, \dots, v_r\}$ 是一个标准正交系统。

在上述命题的记号中,算子 $A$ 可以表示为
$A = \sum_{k=1}^r \sigma_k w_k v_k^*$, (3.1)
或者等价地
$Ax = \sum_{k=1}^r \sigma_k (x, v_k) w_k$。(3.2)
确实,我们知道 $\{v_1, v_2, \dots, v_n\}$ 是 $X$ 的一个标准正交基。那么将 $x = v_j$ 代入 (3.2) 的右侧,我们得到 $\sum_{k=1}^r \sigma_k (v_j, v_k) w_k = \sigma_j (v_j, v_j) w_j = \sigma_j w_j = Av_j$ 如果 $j=1, 2, \dots, r$,并且 $\sum_{k=1}^r \sigma_k (v_k^* v_j) w_k = 0 = Av_j$ 对于 $j > r$。所以,(3.1) 中左右两侧的算子在基 $\{v_1, v_2, \dots, v_n\}$ 上是相同的,因此它们是相等的。

\textbf{定义}~ 上述分解 (3.1)(或 (3.2))被称为算子 $A$ 的\textbf{施密特分解}。

\textbf{注释} ~施密特分解不是唯一的。为什么?



\textbf{引理 3.7。} 设 $A = \sum_{k=1}^r \sigma_k w_k v_k^*$,其中 $\sigma_k > 0$ 并且 $\{v_1, v_2, \dots, v_r\}$, $\{w_1, w_2, \dots, w_r\}$ 是某些标准正交系统。那么这个表示给出了 $A$ 的施密特分解。

\textbf{证明}~ 我们只需要证明 $v_1, v_2, \dots, v_r$ 是 $A^*A$ 的特征向量,$A^*Av_k = \sigma_k^2 v_k$。由于 $\{w_1, w_2, \dots, w_r\}$ 是标准正交系统,$w_k^* w_j = (w_j, w_k) = \delta_{k,j} := \begin{cases} 0, & j \neq k \\ 1, & j = k \end{cases}$,因此 $A^*A = \sum_{k=1}^r \sigma_k^2 v_k v_k^*$。由于 $\{v_1, v_2, \dots, v_r\}$ 是标准正交系统,$A^*Av_j = \sum_{k=1}^r \sigma_k^2 v_k v_k^* v_j = \sigma_j^2 v_j$,因此 $v_k$ 是 $A^*A$ 的特征向量。

\textbf{推论 3.8。} 设 $A = \sum_{k=1}^r \sigma_k w_k v_k^*$ 是 $A$ 的施密特分解。那么 $A^* = \sum_{k=1}^r \sigma_k v_k w_k^*$ 是 $A^*$ 的施密特分解。

3.4. \textbf{施密特分解的矩阵表示。奇异值分解。} 施密特分解可以写成一个很好的矩阵形式。即,假设 $A: F^n \to F^m$(这里 $F$ 总是 $\mathbb{C}$ 或 $\mathbb{R}$;我们可以通过选取 $X$ 和 $Y$ 中的标准正交基并处理这些基下的坐标来完成)。设 $\sigma_1, \sigma_2, \dots, \sigma_r$ 是 $A$ 的非零奇异值(计入重数),并设 $A = \sum_{k=1}^r \sigma_k w_k v_k^*$ 是 $A$ 的施密特分解。
如你所见,这个等式可以重写为
$A = \tilde{W} \tilde{\Sigma} \tilde{V}^*$, (3.3)
其中 $\tilde{\Sigma} = \text{diag}\{\sigma_1, \sigma_2, \dots, \sigma_r\}$ 并且 $\tilde{V}$ 和 $\tilde{W}$ 是分别以 $v_1, v_2, \dots, v_r$ 和 $w_1, w_2, \dots, w_r$ 为列的矩阵。(你能说出每个矩阵的大小吗?)注意,由于 $\{v_1, v_2, \dots, v_r\}$ 和 $\{w_1, w_2, \dots, w_r\}$ 是标准正交系统,矩阵 $\tilde{V}$ 和 $\tilde{W}$ 是等距同构。还需注意 $r = \text{rank } A$,见下面的练习 3.1。如果矩阵 $A$ 是可逆的,那么 $m=n=r$,矩阵 $\tilde{V}$, $\tilde{W}$ 是酉的,并且 $\tilde{\Sigma}$ 是一个可逆的对角矩阵。事实证明,总是可以写出一个类似的表示(3.3),用酉矩阵 $V$ 和 $W$ 来代替 $\tilde{V}$ 和 $\tilde{W}$,并且在许多情况下,处理这样的表示会更方便。
为了写出这个表示,我们首先需要将系统 $\{v_1, v_2, \dots, v_r\}$ 和 $\{w_1, w_2, \dots, w_r\}$ \textbf{补全}为 $F^n$ 和 $F^m$ 中的正交基。回想一下,要将 $\{v_1, v_2, \dots, v_r\}$ 补全为 $F^n$ 中的标准正交基,只需找到 $\text{Ker } V^*$ 的一个标准正交基 $\{v_{r+1}, \dots, v_n\}$;那么系统 $\{v_1, v_2, \dots, v_n\}$ 将是 $F^n$ 中的一个标准正交基。并且人们总是能通过格拉姆-施密特正交化从任意系统得到一个标准正交基。然后 $A$ 可以表示为
$A = W \Sigma V^*$, (3.4)
其中 $V \in M_F^{n \times n}$ 和 $W \in M_F^{m \times m}$ 是以 $v_1, v_2, \dots, v_n$ 和 $w_1, w_2, \dots, w_m$ 为列的酉矩阵,而 $\Sigma \in M_\mathbb{R}^{+, m \times n}$ 是一个“对角”矩阵(意思是 $\sigma_{k,k} \geq 0$ 对于所有 $k = 1, 2, \dots, \min\{m, n\}$,并且 $\sigma_{j,k} = 0$ 对于所有 $j \neq k$)。(3.5)
也就是说,为了得到矩阵 $\Sigma$,你需要取对角矩阵 $\text{diag}\{\sigma_1, \sigma_2, \dots, \sigma_r\}$ 并通过在“南方”和“东方”添加额外的零将其变成一个 $m \times n$ 矩阵。

\textbf{定义 3.9。} 对于矩阵 $A \in M_F^{m \times n}$(这里 $F$ 总是 $\mathbb{C}$ 或 $\mathbb{R}$),其\textbf{奇异值分解} (SVD) 是形如 (3.4) 的分解,即分解 $A = W \Sigma V^*$,其中 $W \in M_F^{n \times n}$ 和 $V \in M_F^{m \times m}$ 是酉矩阵,而 $\Sigma \in M_\mathbb{R}^{+, m \times n}$ 是“对角”矩阵(意思是 $\sigma_{k,k} \geq 0$ 对于所有 $k = 1, 2, \dots, \min\{m, n\}$,并且 $\sigma_{j,k} = 0$ 对于所有 $j \neq k$)。更精确地说,\textbf{约简 SVD} 是一个表示 $A = \tilde{W} \tilde{\Sigma} \tilde{V}^*$,其中 $\tilde{\Sigma} \in M_\mathbb{R}^{+, r \times r}$, $r \leq \min\{m, n\}$ 是一个对角矩阵,其对角线元素严格为正,而 $\tilde{W} \in M_F^{n \times r}$, $\tilde{V} \in M_F^{m \times r}$ 是等距同构;而且,我们要求 $\tilde{W}$ 和 $\tilde{V}$ 中至少有一个不是方阵。

\textbf{注 3.10。} 很容易看出,如果 $A = W \Sigma V^*$ 是 $A$ 的奇异值分解,那么 $\sigma_k := \sigma_{k,k}$ 是 $A$ 的奇异值,即 $\sigma_k^2$ 是 $A^*A$ 的特征值。而且,$V$ 的列 $v_k$ 是 $A^*A$ 的相应特征向量,$A^*Av_k = \sigma_k^2 v_k$。还要注意,如果 $\sigma_k \neq 0$,那么 $w_k = \frac{1}{\sigma_k} Av_k$。所有这些都意味着任何奇异值分解 $A = W \Sigma V^*$ 都可以通过本节上面描述的构造从施密特分解 (3.2) 得到。对于不可逆矩阵 $A$,约简奇异值分解可以解释为施密特分解 (3.2) 的矩阵形式。对于可逆矩阵 $A$,施密特分解的矩阵形式给出了奇异值分解。

\textbf{注 3.11。} $A = W \Sigma V^*$ 的奇异值分解的另一种解释是,$\Sigma$ 是 $A$ 在(标准正交)基 $\{v_1, v_2, \dots, v_n\}$ 和 $\{w_1, w_2, \dots, w_n\}$ 下的矩阵,即 $\Sigma = [A]_{B,A}$。我们将在后面使用这个解释。

3.4.1. \textbf{从奇异值分解到极坐标分解。} 注意,如果我们知道方阵 $A$ 的奇异值分解 $A = W \Sigma V^*$,我们可以写出 $A$ 的极坐标分解:
(3.6) $A = W \Sigma V^* = (WV^*) (V \Sigma V^*) = U|A|$
其中 $|A| = V \Sigma V^*$ 并且 $U = WV^*$。为了说明这确实是一个极坐标分解,让我们注意到 $V\Sigma V^*$ 是一个自伴随的、半正定的算子,并且 $A^*A = (W\Sigma V^*)^*(W\Sigma V^*) = V \Sigma^* W^* W \Sigma V^* = V \Sigma^*\Sigma V^* = V (\Sigma^* \Sigma) V^* = (V \Sigma V^*)(V \Sigma V^*) = (|A|)^2$。所以根据 $|A|$ 的定义(它是 $A^*A$ 的唯一半正定平方根),我们可以看出 $|A| = V \Sigma V^*$。变换 $WV^*$ 显然是一个酉变换,因为它是由两个酉变换相乘得到的,所以(3.6)确实给出了 $A$ 的一个极坐标分解。请注意,此推理仅适用于方阵,因为如果 $A$ 不是方阵,则乘积 $V\Sigma$ 是未定义的(维度不匹配,你能看出为什么吗?)。

\textbf{练习}~

3.1. 证明矩阵 $A$ 的非零奇异值的数量(计入重数)与其秩相等。





3.2. 为以下矩阵 $A$ 找出施密特分解 $A = \sum_{k=1}^r s_k w_k v_k^*$:
$\begin{pmatrix} 2 & 3 \\ 0 & 2 \end{pmatrix}$, $\begin{pmatrix} 7 & 1 & 0 \\ 0 & 0 & 5 \\ 5 & 0 & 5 \end{pmatrix}$, $\begin{pmatrix} 1 & 1 & 0 \\ 1 & 2 & 2 \\ 0 & -1 & 1 \end{pmatrix}$。

3.3. 设 $A$ 是一个可逆矩阵,设 $A = W \Sigma V^*$ 是它的奇异值分解。求 $A^*$ 和 $A^{-1}$ 的奇异值分解。

3.4. 为以下矩阵 $A$ 找出奇异值分解 $A = W \Sigma V^*$,其中 $V$ 和 $W$ 是酉矩阵:
a) $A = \begin{pmatrix} -3 & 1 \\ 6 & -2 \\ 6 & -2 \end{pmatrix}$;
b) $A = \begin{pmatrix} 3 & 2 & 2 \\ 2 & 3 & -2 \end{pmatrix}$。

3.5. 找出矩阵 $A = \begin{pmatrix} 2 & 3 \\ 0 & 2 \end{pmatrix}$ 的奇异值分解。使用它来找出:
a) $\max_{\|x\| \leq 1} \|Ax\|$ 以及最大值达到的向量;
b) $\min_{\|x\|=1} \|Ax\|$ 以及最小值达到的向量;
c) $A$ 对 $\mathbb{R}^2$ 中的闭单位球 $B = \{x \in \mathbb{R}^2 : \|x\| \leq 1\}$ 的像 $A(B)$。几何上描述 $A(B)$。

3.6. 证明对于方阵 $A$,$|\det A| = \det |A|$。

3.7. 真或假:
a) 矩阵的奇异值也是该矩阵的特征值。
b) 矩阵 $A$ 的奇异值是 $A^*A$ 的特征值。
c) 如果 $s$ 是矩阵 $A$ 的一个奇异值,而 $c$ 是一个标量,那么 $|c|s$ 是 $cA$ 的奇异值。
d) 任何线性算子的奇异值都是非负的。
e) 自伴随矩阵的奇异值与其特征值相等。

3.8. 设 $A$ 是一个 $m \times n$ 矩阵。证明 $A^*A$ 和 $AA^*$ 的\textbf{非零}特征值(计入重数)是相同的。你能说出 $A^*A$ 的零特征值和 $AA^*$ 的零特征值何时具有相同的重数吗?

3.9. 设 $s$ 是算子 $A$ 的最大奇异值,设 $\lambda$ 是 $A$ 具有最大绝对值的特征值。证明 $|\lambda| \leq s$。

3.10. 证明矩阵的秩等于其非零奇异值的数量(计入重数)。





3.11. 证明算子范数 $\|A\|$ 与 Frobenius 范数 $\|A\|_2$ 相等当且仅当该矩阵秩为 1。
\textbf{提示:} 上一个问题可能有所帮助。

3.12. 对于矩阵 $A = \begin{pmatrix} 2 & -3 \\ 0 & 2 \end{pmatrix}$,描述单位球的逆像,即所有 $x \in \mathbb{R}^2$ 使得 $\|Ax\| \leq 1$ 的集合。使用奇异值分解。

\section{奇异值分解的应用}

正如上面讨论的,奇异值分解仅仅是在两个不同标准正交基下的对角化。由于这里有两个不同的基,我们无法从其奇异值分解中得知关于算子谱性质的太多信息。例如,奇异值分解中的 $\Sigma$ 的对角线元素不是 $A$ 的特征值。注意,对于 $A = W \Sigma V^*$,其中 $A$ 不一定是方阵,并且(或)奇异值不全非零。考虑“对角”矩阵 $\Sigma$ 的形式 (3.5)。很容易看出单位球 $B$ 的像 $\Sigma(B)$ 是一个椭球体(不是在整个空间中,而是在 $\text{Ran } \Sigma$ 中)其半轴为 $\sigma_1, \sigma_2, \dots, \sigma_r$。考虑一般情况, $A = W \Sigma V^*$,其中 $W$ 和 $V$ 是酉算子。酉变换不改变单位球(因为它们保持范数),所以 $V^*(B) = B$。我们知道 $\Sigma(B)$ 是 $\text{Ran } \Sigma$ 中的一个椭球体,其半轴为 $\sigma_1, \sigma_2, \dots, \sigma_r$。酉变换不改变物体的几何形状,所以 $W(\Sigma(B))$ 也是一个具有相同半轴的椭球体。很容易从分解 $A = W \Sigma V^*$(利用 $W$ 和 $V^*$ 都是可逆的事实)看出 $W$ 将 $\text{Ran } \Sigma$ 映射到 $\text{Ran } A$,所以我们可以得出结论:闭单位球 $B$ 的像 $A(B)$ 是 $\text{Ran } A$ 中的一个椭球体,其半轴为 $\sigma_1, \sigma_2, \dots, \sigma_r$。这里 $r$ 是非零奇异值的数量,即 $A$ 的秩。

4.2. \textbf{线性变换的算子范数。} 给定一个线性变换 $A: X \to Y$,让我们考虑以下优化问题:找到在闭单位球 $B = \{x \in X : \|x\| \leq 1\}$ 上 $\|Ax\|$ 的最大值。同样,奇异值分解允许我们解决这个问题。对于具有非负项的对角矩阵 $A$,最大值恰好是最大的对角项。确实,设 $s_1, s_2, \dots, s_n$ 是 $A$ 的非零对角项,设 $s_1$ 是最大的。由于对于 $x = (x_1, x_2, \dots, x_n)^T$,
$Ax = \sum_{k=1}^n s_k x_k e_k$, (4.1)
我们可以得出 $\|Ax\|^2 = \sum_{k=1}^n s_k^2 |x_k|^2 \leq s_1^2 \sum_{k=1}^n |x_k|^2 = s_1^2 \|x\|^2$,所以 $\|Ax\| \leq s_1 \|x\|$。另一方面,$\|Ae_1\| = \|s_1 e_1\| = s_1 \|e_1\|$,所以确实 $s_1$ 是闭单位球 $B$ 上 $\|Ax\|$ 的最大值。注意,在上述推理中,我们没有假设矩阵 $A$ 是方阵,我们只假设主对角线以外的所有项都为零,所以公式 (4.1) 成立。为了处理一般情况,让我们考虑奇异值分解 (3.5), $A = W \Sigma V^*$,其中 $W$ 和 $V$ 是酉算子,而 $\Sigma$ 是具有非负项的对角矩阵。由于酉变换不改变范数,我们可以得出 $B$ 上 $\|Ax\|$ 的最大值等于 $\Sigma$ 的最大对角项,即 $A$ 的算子范数等于其最大奇异值,即 $\|A\| = s_1$。所以我们可以得出 $\|A\| \leq \|A\|_2$,即矩阵的算子范数不能大于其 Frobenius 范数。这个陈述也可以通过使用柯西-施瓦茨不等式直接证明,并且这样的证明在一些教科书中都有介绍。我们这里提出的证明的美妙之处在于,它不需要任何计算,并且阐明了不等式背后的原因。

\textbf{定义}~ $A: X \to Y$ 的线性变换的\textbf{算子范数}被定义为 $\max\{\|Ax\| : x \in X, \|x\| \leq 1\}$,记作 $\|A\|$。很容易看出 $\|A\|$ 满足范数的所有性质:
1. $\|\alpha A\| = |\alpha| \|A\|$ 对于所有 $A$ 和所有标量 $\alpha$。
2. $\|A + B\| \leq \|A\| + \|B\|$。
3. $\|A\| \geq 0$ 对于所有 $A$;
4. $\|A\| = 0$ 当且仅当 $A = 0$,所以它确实是 $X$ 到 $Y$ 的线性变换空间上的一个范数。算子范数的一个主要性质是不等式 $\|Ax\| \leq \|A\|\|x\|$,这很容易从范数 $\|x\|$ 的齐次性得出。事实上,可以证明算子范数 $\|A\|$ 是最小的数 $C \geq 0$,使得 $\|Ax\| \leq C\|x\|$ $\forall x \in X$。这通常作为算子范数的定义。在线性变换空间中,我们已经有一个范数,即 Frobenius 范数,或 Hilbert-Schmidt 范数 $\|A\|_2 = \sqrt{\text{trace}(A^*A)}$。所以,让我们研究一下这两个范数如何比较。设 $s_1, s_2, \dots, s_n$ 是 $A$ 的非零奇异值(计入重数),设 $s_1$ 是最大的奇异值。那么 $s_1^2, s_2^2, \dots, s_n^2$ 是 $A^*A$ 的非零特征值(同样计入重数)。回想起迹等于特征值之和,我们得出 $\|A\|_2^2 = \text{trace}(A^*A) = \sum_{k=1}^n s_k^2$。另一方面,我们知道算子范数 $\|A\|$ 等于其最大奇异值,即 $\|A\| = s_1$。所以我们可以得出 $\|A\| \leq \|A\|_2$,即矩阵的算子范数不能大于其 Frobenius 范数。这个陈述也接受一个直接证明,使用柯西-施瓦茨不等式,并且这样的证明在一些教科书中被提出。我们这里提出的证明的美妙之处在于,它不需要任何计算,并且阐明了不等式背后的原因。

4.3. \textbf{矩阵的条件数。} 假设我们有一个可逆矩阵 $A$ 并且我们想求解方程 $Ax = b$。当然,解由 $x = A^{-1}b$ 给出,但我们想研究如果我们只近似知道数据会发生什么。这种情况在现实生活中会发生,当数据例如通过某些实验获得时。但即使我们有精确的数据,计算机计算过程中的舍入误差也可能产生相同的影响,即扭曲数据。让我们考虑最简单的模型,假设右侧方程有一个小的误差。这意味着,我们求解的是 $Ax = b + \Delta b$,而不是 $Ax = b$,其中 $\Delta b$ 是右侧 $b$ 的微小扰动。所以,我们得到的是 $A(x + \Delta x) = b + \Delta b$ 的近似解,而不是精确解 $x$。我们假设 $A$ 是可逆的,所以 $\Delta x = A^{-1}\Delta b$。我们想知道解中的相对误差 $\| \Delta x \| / \| x \|$ 与右侧的相对误差 $\| \Delta b \| / \| b \|$ 相比有多大。很容易看出 $\| \Delta x \| / \| x \| = \| A^{-1} \Delta b \| / \| x \| = \| A^{-1} \Delta b \| / \| b \| \cdot \| b \| / \| x \| = \| A^{-1} \Delta b \| / \| b \| \cdot \| b \| / \| x \|$。由于 $\| A^{-1} \Delta b \| \leq \| A^{-1} \| \cdot \| \Delta b \|$ 且 $\| Ax \| \leq \| A \| \cdot \| x \|$,我们可以得出 $\| \Delta x \| / \| x \| \leq \| A^{-1} \| \cdot \| A \| \cdot \| \Delta b \| / \| b \|$。数量 $\|A\| \cdot \|A^{-1}\|$ 被称为矩阵的\textbf{条件数}。它估计了解 $x$ 中的相对误差如何依赖于右侧 $b$ 中的相对误差。让我们看看这个数量与奇异值有何关系。设 $\sigma_1, \sigma_2, \dots, \sigma_n$ 是 $A$ 的奇异值,并假设 $\sigma_1$ 是最大奇异值,$\sigma_n$ 是最小奇异值。我们知道算子范数 $\|A\|$ 等于其最大奇异值,所以 $\|A\| = \sigma_1$, $\|A^{-1}\| = 1/\sigma_n$,所以 $\|A\| \cdot \|A^{-1}\| = \sigma_1/\sigma_n$。换句话说,矩阵的条件数等于最大和最小奇异值之比。





我们上面推导出 $\| \Delta x \| / \| x \| \leq \| A^{-1} \| \cdot \| A \| \cdot \| \Delta b \| / \| b \|$。不难看出这个估计是尖锐的,即可以选择右侧 $b$ 和误差 $\Delta b$ 使得我们得到等式 $\| \Delta x \| / \| x \| = \| A^{-1} \| \cdot \| A \| \cdot \| \Delta b \| / \| b \|$。我们只需选择 $b = w_1$ 和 $\Delta b = \alpha w_n$,其中 $w_1$ 和 $w_n$ 分别是奇异值分解 $A = W \Sigma V^*$ 中 $W$ 的第一个和最后一个列,$\alpha \neq 0$ 是任意标量。这里,正如通常所做的那样,奇异值假定为非递增顺序 $\sigma_1 \geq \sigma_2 \geq \dots \geq \sigma_n$,所以 $\sigma_1$ 是最大的,$\sigma_n$ 是最小的。我们将细节留给读者。如果一个矩阵的条件数不是太大的话,它被称为\textbf{良适定}的。如果条件数很大,则该矩阵被称为\textbf{病态}的。这里“大”取决于具体问题:你能在什么精度下找到你的右侧,$x$ 的解需要什么精度等等。

4.4. \textbf{矩阵的有效秩。} 理论上,计算矩阵的秩很容易:只需对矩阵进行行变换并计数主元。然而,在实际应用中,并非一切都那么容易。主要原因是,我们通常不知道精确的矩阵,只知道其近似值,精确到某个精度。此外,即使我们知道精确矩阵,大多数计算机程序在计算过程中也会引入舍入误差,所以我们实际上无法区分零主元和非常小的非零主元。一个简单的朴素想法是处理舍入误差:计算秩(以及与它相关的其他对象,如列空间、核等),只需设置一个容差(某个小的数),如果主元小于容差,则将其视为零。这种方法的优点是其简单性,因为它非常容易编程。然而,主要缺点是无法看出容差的作用。例如,如果我们设置容差为 $10^{-6}$,我们会丢失什么?如果设置为 $10^{-8}$ 会好多少?虽然上述方法对于良适定的矩阵效果很好,但在一般情况下并不可靠。更好的方法是使用奇异值。这需要更多的计算,但结果要好得多,也更容易解释。在这种方法中,我们也设置一个小的数字作为容差,然后执行奇异值分解。然后,我们只需将小于容差的奇异值视为零。这种方法的优点是我们能够看到我们正在做什么。奇异值是椭球体 $A(B)$($B$ 是闭单位球)的半轴,所以通过设置容差,我们只是决定椭球体应该“多薄”才能被认为是“扁平”的。

4.5. \textbf{摩尔-彭罗斯(伪)逆。} 正如我们在第 5 章第 4 节中讨论的,最小二乘解在方程 $Ax = b$ 没有解的情况下,为我们提供了“最好的替代方案”(并且在方程存在解时,为我们提供了 $Ax = b$ 的解)。注意,最小二乘解并没有解决唯一性的问题:正规方程 $A^*Ax = A^*b$ 的解不一定唯一。一个自然的、特定的解是具有最小范数的解;这样的解确实是唯一的,并且可以通过取任意一个解,然后将其投影到 $(\text{Ker } A^*)^\perp = (\text{Ker } A)^\perp$ 来得到(见第 5 章问题 4.5 和 4.6)。不难看出,如果 $A = \tilde{W} \tilde{\Sigma} \tilde{V}^*$ 是 $A$ 的\textbf{约简}奇异值分解,那么最小范数最小二乘解 $x_0$ 由下式给出:
$x_0 = \tilde{V} \tilde{\Sigma}^{-1} \tilde{W}^* b$。(4.2)
确实,$x_0$ 是 $Ax = b$ 的最小二乘解(即 $Ax = P_{\text{Ran } A} b$ 的解):$Ax_0 = \tilde{W} \tilde{\Sigma} \tilde{V}^* \tilde{V} \tilde{\Sigma}^{-1} \tilde{W}^* b = \tilde{W} \tilde{\Sigma} \tilde{\Sigma}^{-1} \tilde{W}^* b = \tilde{W} \tilde{W}^* b = P_{\text{Ran } A} b$(在最后一个等式链中,我们使用了 $\tilde{W} \tilde{W}^* = P_{\text{Ran } \tilde{W}}$($P_{\text{Ran } \tilde{W}} = \tilde{W}(\tilde{W}^*\tilde{W})^{-1}\tilde{W}^* = \tilde{W}\tilde{W}^*$)并且 $\text{Ran } \tilde{W} = \text{Ran } A$(见下面的问题 4.4)。$Ax = P_{\text{Ran } A} b$ 的一般解由 $x = x_0 + y$, $y \in \text{Ker } A$ 给出,所以 $x_0$ 确实是 $Ax = b$ 的最小范数最小二乘解。

\textbf{定义 4.1。} 算子 $A^+ := \tilde{V} \tilde{\Sigma}^{-1} \tilde{W}^*$,其中 $A = \tilde{W} \tilde{\Sigma} \tilde{V}^*$ 是 $A$ 的\textbf{约简}奇异值分解,被称为 $A$ 的\textbf{摩尔-彭罗斯逆}(或\textbf{摩尔-彭罗斯伪逆})。换句话说,\textbf{摩尔-彭罗斯逆}是给出 $Ax = b$ 的唯一最小范数最小二乘解的算子。

\textbf{注 4.2。} 在文献中,摩尔-彭罗斯逆通常定义为一个矩阵 $A^+$,满足:
1. $AA^+A = A$;
2. $A^+AA^+ = A^+$;
3. $(AA^+)^* = AA^+$;
4. $(A^+A)^* = A^+A$。
很容易验证算子 $A^+ := \tilde{V} \tilde{\Sigma}^{-1} \tilde{W}^*$ 满足上述性质 1-4。也可以(虽然稍微困难一些)证明满足性质 1-4 的算子 $A^+$ 是唯一的。确实,通过将恒等式 1 分别用 $A^+$ 从左边和右边乘以,我们得到 $(A^+A)^2 = A^+A$ 和 $(AA^+)^2 = AA^+$;结合性质 3 和 4,这意味着 $A^+A$ 和 $AA^+$ 是正交投影(见第 5 章问题 5.6)。显然,$\text{Ker } A \subseteq \text{Ker } A^+A$。另一方面,恒等式 1 暗示 $\text{Ker } A^+A \subseteq \text{Ker } A$(为什么?),所以 $\text{Ker } A^+A = \text{Ker } A$。但这表示 $A^+A$ 是到 $(\text{Ker } A)^\perp = \text{Ran } A^*$ 的正交投影,$A^+A = P_{\text{Ran } A^*}$。性质 1 也暗示 $AA^+y = y$ 对于所有 $y \in \text{Ran } A$。由于 $AA^+$ 是正交投影,我们得出 $\text{Ran } A \subseteq \text{Ran } AA^+$。相反的包含关系 $\text{Ran } AA^+ \subseteq \text{Ran } A$ 是平凡的,所以 $AA^+$ 是到 $\text{Ran } A$ 的正交投影,$AA^+ = P_{\text{Ran } A}$。知道 $A^+A$ 和 $AA^+$,我们可以重写性质 2 为 $P_{\text{Ran } A^*} A^+ = A^+$ 或 $A^+ P_{\text{Ran } A} = A^+$。结合上述恒等式,我们得到 $P_{\text{Ran } A^*} A^+ P_{\text{Ran } A} = A^+$。最后,对于目标空间 $A$ 的任何 $b$, $x_0 := A^+b = P_{\text{Ran } A^*} A^+ b \in \text{Ran } A^*$ 并且 $Ax_0 = AA^+ b = P_{\text{Ran } A} b$,这意味着 $x_0$ 是 $Ax = b$ 的最小二乘解。由于 $x_0 \in \text{Ran } A^* = (\text{Ker } A)^\perp$, $x_0$ 是最小范数最小二乘解,如我们上面所讨论的。但是,正如我们前面所显示的,这样的最小范数解由(4.2)给出,所以 $A^+ = \tilde{V} \tilde{\Sigma}^{-1} \tilde{W}^*$。

\textbf{练习}~

4.1. 找出以下矩阵的范数和条件数:
a) $A = \begin{pmatrix} 4 & 0 \\ 1 & 3 \end{pmatrix}$。
b) $A = \begin{pmatrix} 5 & 3 \\ -3 & 3 \end{pmatrix}$。
对于 a) 部分的矩阵 $A$,给出一个右侧 $b$ 和误差 $\Delta b$ 的例子,使得 $\|\Delta x\|/\|x\| = \|A\| \cdot \|A^{-1}\| \cdot \|\Delta b\|/\|b\|$;这里 $Ax = b$ 并且 $A \Delta x = \Delta b$。

4.2. 设 $A$ 是一个正规算子,$\lambda_1, \lambda_2, \dots, \lambda_n$ 是它的特征值(计入重数)。证明 $A$ 的奇异值是 $|\lambda_1|, |\lambda_2|, \dots, |\lambda_n|$。

4.3. 找出矩阵 $A = \begin{pmatrix} 2 & 1 & 1 \\ 1 & 2 & 1 \\ 1 & 1 & 2 \end{pmatrix}$ 的奇异值、范数和条件数。你可以在几乎不进行任何计算的情况下完成这个问题,如果你能回答以下问题:
a) 某个子空间 $E$ 上的\textbf{正交投影} $P_E$ 的奇异值是什么?
b) 由向量 $(1, 1, 1)^T$ 张成子空间的\textbf{正交投影}矩阵是什么?
c) $T$ 和 $aT + bI$ 的特征值(其中 $a$ 和 $b$ 是标量)之间有什么关系?当然,你也可以老老实实地进行计算。

4.4. 设 $A = \tilde{W} \tilde{\Sigma} \tilde{V}^*$ 是 $A$ 的\textbf{约简}奇异值分解。证明 $\text{Ran } A = \text{Ran } \tilde{W}$,然后通过取伴随得到 $\text{Ran } A^* = \text{Ran } \tilde{V}$。

4.5. 用奇异值分解 $A = W \Sigma V^*$ 写出摩尔-彭罗斯逆 $A^+$ 的公式。

4.6. \textbf{泰洪诺夫正则化:} 证明摩尔-彭罗斯逆 $A^+$ 可以计算为极限 $A^+ = \lim_{\epsilon \to 0^+} (A^*A + \epsilon I)^{-1} A^* = \lim_{\epsilon \to 0^+} A^*(AA^* + \epsilon I)^{-1}$。





\section{正交矩阵的结构}
行列式为 1 的正交矩阵 $U$ 通常被称为\textbf{旋转}。下面的定理解释了这个名称。

\textbf{定理 5.1。} 设 $U$ 是 $\mathbb{R}^n$ 中的一个正交算子,且 $\det U = 1$。那么存在一个标准正交基 $\{v_1, v_2, \dots, v_n\}$,使得 $U$ 在该基下的矩阵具有块对角形式
$\begin{pmatrix} R_{\phi_1} & & & & \\ & R_{\phi_2} & & & \\ & & \ddots & & \\ & & & R_{\phi_k} & \\ & & & & I_{n-2k} \end{pmatrix}$,
其中 $R_{\phi_k}$ 是 $2$ 维旋转矩阵,$R_{\phi_k} = \begin{pmatrix} \cos \phi_k & -\sin \phi_k \\ \sin \phi_k & \cos \phi_k \end{pmatrix}$,而 $I_{n-2k}$ 表示 $(n-2k) \times (n-2k)$ 的单位矩阵。

\textbf{证明}~ 我们利用数学归纳法来证明定理。如果 $p$ 是一个具有实数系数的多项式,$\lambda$ 是它的复数根,$p(\lambda) = 0$,那么 $\bar{\lambda}$ 也是 $p$ 的一个根,$p(\bar{\lambda}) = 0$(这可以通过将 $\lambda$ 代入 $p(z) = \sum_{k=0}^n a_k z^k$ 来轻易地验证)。因此,$A$ 的所有复数特征值可以配对为 $\lambda_k$ 和 $\bar{\lambda}_k$。我们知道酉矩阵的特征值的模为 1,所以 $A$ 的所有复数特征值都可以写成 $\lambda_k = \cos \alpha_k + i \sin \alpha_k$, $\bar{\lambda}_k = \cos \alpha_k - i \sin \alpha_k$。固定一对复数特征值 $\lambda$ 和 $\bar{\lambda}$,设 $u \in \mathbb{C}^n$ 是 $U$ 的特征向量,$Uu = \lambda u$。那么 $U\bar{u} = \bar{\lambda}\bar{u}$。现在,将 $u$ 分为实部和虚部,即定义 $x := \text{Re } u = (u + \bar{u})/2$, $y = \text{Im } u = (u - \bar{u})/(2i)$,所以 $u = x + iy$(注意,$x, y$ 是实向量,即具有实数项的向量)。那么 $Ux = U \frac{1}{2}(u + \bar{u}) = \frac{1}{2}(\lambda u + \bar{\lambda} \bar{u}) = \text{Re}(\lambda u)$。类似地,$Uy = \frac{1}{2i} U(u - \bar{u}) = \frac{1}{2i}(\lambda u - \bar{\lambda} \bar{u}) = \text{Im}(\lambda u)$。由于 $\lambda = \cos \alpha + i \sin \alpha$,我们有 $\lambda u = (\cos \alpha + i \sin \alpha)(x + iy) = ((\cos \alpha)x - (\sin \alpha)y) + i((\cos \alpha)y + (\sin \alpha)x)$,所以 $Ux = \text{Re}(\lambda u) = (\cos \alpha)x - (\sin \alpha)y$, $Uy = \text{Im}(\lambda u) = (\cos \alpha)y + (\sin \alpha)x$。换句话说,$U$ 不动向量 $x, y$ 张成的二维子空间 $E_\lambda$(注意 $x \perp y$),并且 $U$ 在该子空间上的限制矩阵是旋转矩阵 $R_{-\alpha} = \begin{pmatrix} \cos \alpha & \sin \alpha \\ -\sin \alpha & \cos \alpha \end{pmatrix}$。注意,由于向量 $u$ 和 $\bar{u}$(不同特征值的特征向量)是正交的,根据勾股定理 $\|x\| = \|y\| = \frac{1}{\sqrt{2}}\|u\|$。很容易看出 $x \perp y$,所以 $\{x, y\}$ 是 $E_\lambda$ 的一个正交基。如果我们乘以每个基向量相同的非零数,我们不会改变线性变换的矩阵,所以我们可以无损于一般性地假设 $\|x\| = \|y\| = 1$,即 $\{x, y\}$ 是 $E_\lambda$ 的一个正交基。将标准正交系统 $\{v_1 = x, v_2 = y\}$ 补全为 $\mathbb{R}^n$ 中的一个标准正交基。由于 $UE_\lambda \subset E_\lambda$,即 $E_\lambda$ 是 $U$ 的不变子空间,所以 $U$ 在该基下的矩阵具有块三角形式 $\begin{pmatrix} R_{-\alpha} & * \\ 0 & U_1 \end{pmatrix}$(其中 $0$ 是 $(n-2) \times 2$ 的零块)。由于旋转矩阵 $R_{-\alpha}$ 是可逆的,我们有 $U E_\lambda = E_\lambda$。因此 $U^*E_\lambda = U^{-1}E_\lambda = E_\lambda$,所以 $U$ 在我们构造的基下的矩阵实际上是块对角矩阵:$\begin{pmatrix} R_{-\alpha} & 0 \\ 0 & U_1 \end{pmatrix}$。由于 $U$ 是酉的, $I = U^*U = \begin{pmatrix} I_2 & 0 \\ 0 & U_1^*U_1 \end{pmatrix}$,所以(由于 $U_1$ 是方阵)$U_1^*U_1 = I$,即 $U_1$ 也是酉的。如果 $U_1$ 有复数特征值,我们可以应用相同的过程来减小其尺寸(通过 2),直到我们只剩下只有实数特征值的块。实数特征值只能是 $+1$ 或 $-1$,所以在一个标准正交基下,$U$ 的矩阵具有形式 $\begin{pmatrix} R_{-\alpha_1} & & & & \\ & R_{-\alpha_2} & & & \\ & & \ddots & & \\ & & & R_{-\alpha_d} & \\ & & & & -I_r \end{pmatrix}$;这里 $I_r$ 和 $I_l$ 是 $r \times r$ 和 $l \times l$ 的单位矩阵。由于 $\det U = 1$,$-1$ 的特征值的重数(即 $r$)必须是偶数。注意,$2 \times 2$ 矩阵 $-I_2$ 可以解释为通过角度 $\pi$ 的旋转。因此,上述矩阵具有定理结论中给出的形式,其中 $\phi_k = -\alpha_k$ 或 $\phi_k = \pi$。

我们把证明留给读者。将定理 5.1 的证明修改一下,以适应实数矩阵的情况,这并不难。注意,对于实数矩阵,$U$ 也可以是实数(正交矩阵)。




\textbf{定理 5.2。} 设 $U$ 是 $\mathbb{R}^n$ 中的一个正交算子,且 $\det U = -1$。那么存在一个标准正交基 $\{v_1, v_2, \dots, v_n\}$,使得 $U$ 在该基下的矩阵具有块对角形式
$\begin{pmatrix} R_{\phi_1} & & & & & \\ & R_{\phi_2} & & & & \\ & & \ddots & & & \\ & & & R_{\phi_k} & & \\ & & & & I_r & \\ & & & & & -1 \end{pmatrix}$,
其中 $r = n - 2k - 1$ 并且 $R_{\phi_k}$ 是 $2$ 维旋转矩阵,$R_{\phi_k} = \begin{pmatrix} \cos \phi_k & -\sin \phi_k \\ \sin \phi_k & \cos \phi_k \end{pmatrix}$,而 $I_r$ 是 $(n-2k-1) \times (n-2k-1)$ 的单位矩阵。
我们将证明留给读者。对定理 5.1 的证明所需要做的修改是显而易见的。注意,对于正交矩阵 $U$,$\det U = -1$,它总是\textbf{反射}。设我们现在固定一个标准正交基,例如 $\mathbb{R}^n$ 中的标准基。我们将\textbf{基本旋转}$^3$ 定义为在 $x_j - x_k$ 平面上的旋转,即一个只改变坐标 $x_j$ 和 $x_k$ 的线性变换,并且它在这两个坐标上表现为一个平面旋转。

\textbf{定理 5.3。} 任何旋转 $U$(即 $\det U = 1$ 的正交变换)都可以表示为至多 $n(n-1)/2$ 个基本旋转的乘积。

为了证明定理,我们将需要以下简单引理。

\textbf{引理 5.4。} 设 $x = (x_1, x_2)^T \in \mathbb{R}^2$。存在一个 $\mathbb{R}^2$ 的旋转 $R_\alpha$,它将向量 $x$ 移动到向量 $(a, 0)^T$,其中 $a = \sqrt{x_1^2 + x_2^2}$。
证明是基础的,我们留给读者。人们可以画一张图或者写出 $R_\alpha$ 的公式。

\textbf{引理 5.5。} 设 $x = (x_1, x_2, \dots, x_n)^T \in \mathbb{R}^n$。存在 $n-1$ 个基本旋转 $R_1, R_2, \dots, R_{n-1}$ 使得 $R_{n-1} \dots R_2 R_1 x = (a, 0, \dots, 0)^T \in \mathbb{R}^n$,其中 $a = \sqrt{x_1^2 + x_2^2 + \dots + x_n^2}$。

\textbf{证明}~ 引理证明的思路非常简单。我们使用一个基本旋转 $R_1$ 在 $x_{n-1} - x_n$ 平面上“消去” $x$ 的最后一个坐标(引理 5.4 保证了这种旋转的存在)。然后使用一个基本旋转 $R_2$ 在 $x_{n-2} - x_{n-1}$ 平面上“消去” $R_1 x$ 的第 $n-1$ 个坐标(旋转 $R_2$ 不改变最后一个坐标,所以 $R_2 R_1 x$ 的最后一个坐标保持为零),依此类推... 对于形式证明,我们将使用数学归纳法处理 $n$。$n=1$ 的情况是平凡的,因为 $\mathbb{R}^1$ 中的任何向量都具有所需的格式。$n=2$ 的情况由引理 5.4 处理。现在假设引理对 $n-1$ 成立,我们想为 $n$ 证明它。根据引理 5.4,存在一个 $2 \times 2$ 旋转矩阵 $R_\alpha$,使得 $R_\alpha \begin{pmatrix} x_{n-1} \\ x_n \end{pmatrix} = \begin{pmatrix} a_{n-1} \\ 0 \end{pmatrix}$,其中 $a_{n-1} = \sqrt{x_{n-1}^2 + x_n^2}$。所以如果我们定义 $n \times n$ 的基本旋转 $R_1$ 为 $R_1 = \begin{pmatrix} I_{n-2} & 0 \\ 0 & R_\alpha \end{pmatrix}$ ($I_{n-2}$ 是 $(n-2) \times (n-2)$ 的单位矩阵),那么 $R_1 x = (x_1, x_2, \dots, x_{n-2}, a_{n-1}, 0)^T$。我们假设引理的结论对 $n-1$ 成立,所以存在 $n-2$ 个基本旋转(我们称它们为 $R_2, R_3, \dots, R_{n-1}$)在 $\mathbb{R}^{n-1}$ 中将向量 $(x_1, x_2, \dots, x_{n-1}, a_{n-1})^T$ 变换为向量 $(a, 0, \dots, 0)^T \in \mathbb{R}^{n-1}$。换句话说,$R_{n-1} \dots R_3 R_2 (x_1, x_2, \dots, x_{n-1}, a_{n-1})^T = (a, 0, \dots, 0)^T$。我们可以总是假设基本旋转 $R_2, R_3, \dots, R_{n-1}$ 作用在整个 $\mathbb{R}^n$ 上,只需假设它们不改变最后一个坐标。那么 $R_{n-1} \dots R_3 R_2 R_1 x = (a, 0, \dots, 0)^T \in \mathbb{R}^n$。现在我们来证明 $a = \sqrt{x_1^2 + x_2^2 + \dots + x_n^2}$。可以直接计算证明这一点,但我们采用间接推理。我们知道正交变换保持范数,并且我们知道 $a \geq 0$。但是,那么我们就没有选择,对于 $a$ 唯一的可能性就是 $a = \sqrt{x_1^2 + x_2^2 + \dots + x_n^2}$。

\textbf{定理 5.3 的证明。} 根据引理 5.5,存在基本旋转 $R_1, R_2, \dots, R_N$, $N \leq n(n-1)/2$,使得矩阵 $U_1 = R_N \dots R_2 R_1 U$ 是上三角的,并且除了最后一个对角元素 $B_{n,n}$ 之外,所有对角元素的非负值。注意,矩阵 $U_1$ 是正交的。任何正交矩阵都是正规的,并且我们知道只有对角矩阵才是上三角正规矩阵。因此,$U_1$ 是一个对角矩阵。我们知道正交矩阵的特征值只能是 $1$ 或 $-1$,所以 $U_1$ 的对角线上的值只能是 $1$ 或 $-1$。但是,我们知道 $U_1$ 的所有对角元素,除了最后的可能值之外,都是非负的,所以 $U_1$ 的所有对角元素,除了最后的可能值之外,都是 $1$。最后一个对角元素可以是 $\pm 1$。由于基本旋转的行列式是 1,我们可以得出 $\det U_1 = \det U = 1$,所以最后一个对角元素也必须是 1。所以 $U_1 = I$,因此 $U$ 可以表示为基本旋转的乘积 $U = R_1^{-1} R_2^{-1} \dots R_N^{-1}$。这里我们使用了基本旋转的逆也是基本旋转的事实。

\section{方向}

6.1. \textbf{动机。} 在图 1 和图 2 中,我们看到 $\mathbb{R}^2$ 和 $\mathbb{R}^3$ 中的三个标准正交基。在每个图中,基 b) 可以通过旋转从标准基 a) 得到,而无法通过旋转将标准基旋转得到基 c)(即 $e_k$ 映射到 $v_k$ $\forall k$)。你可能以前听过“方向”这个词,你可能知道基 a) 和 b) 具有正方向,而基 c) 的方向是负的。你也可能知道一些确定方向的规则,比如物理学中的右手定则。所以,如果你能\textbf{看到}一个基,比如在 $\mathbb{R}^3$ 中,你可能就能说出它的方向是什么。但是如果只给你向量 $v_1, v_2, v_3$ 的坐标呢?当然,你可以尝试画一张图来可视化向量,然后确定方向。但这并不总是容易的。此外,你该如何“向计算机解释”这一点呢?事实证明,有一个更简单的方法。让我们来解释一下。我们需要检查是否可能通过旋转得到 $\mathbb{R}^3$ 中的基 $\{v_1, v_2, v_3\}$。存在唯一的线性变换 $U$,使得 $Ue_k = v_k$, $k=1, 2, 3$;它的矩阵(在标准基下)是 $v_1, v_2, v_3$ 作为列的矩阵。它是一个正交矩阵(因为它将标准正交基变换为标准正交基),所以我们需要看看何时它是旋转。定理 5.1 和 5.2 给了我们答案:矩阵 $U$ 是旋转当且仅当 $\det U = 1$。注意,(对于 $3 \times 3$ 矩阵)如果 $\det U = -1$,那么 $U$ 是关于某个轴的旋转与在旋转平面(即与该轴正交的平面)上的反射的组合。这为下面的形式化定义提供了动机。

e1 e2 v1 v2 v1 v2 v3 v1 v2 v3 a) b) c)
图 1. $\mathbb{R}^2$ 中的方向。

e1 e3 e2 v1 v2 v3 v1 v2 v3 a) b) c)
图 2. $\mathbb{R}^3$ 中的方向。

6.2. \textbf{形式定义。} 设 $A = \{a_1, a_2, \dots, a_n\}$ 和 $B = \{b_1, b_2, \dots, b_n\}$ 是 $X$ 中的两个基。我们说基 $A$ 和 $B$ 具有\textbf{相同方向},如果坐标变换矩阵 $[I]_{B,A}$ 的行列式为正,并且说它们具有\textbf{不同方向},如果 $[I]_{B,A}$ 的行列式为负。注意,由于 $[I]_{A,B} = [I]_{B,A}^{-1}$,可以使用矩阵 $[I]_{A,B}$ 来定义。我们通常假设 $\mathbb{R}^n$ 中的标准基 $\{e_1, e_2, \dots, e_n\}$ 具有正方向。在一个抽象空间中,只需固定一个基并声明其方向为正。

如果 $\mathbb{R}^n$ 中的一个标准正交基 $\{v_1, v_2, \dots, v_n\}$ 具有正方向(即与标准基具有相同方向),那么定理 5.1 和 5.2 说明基 $\{v_1, v_2, \dots, v_n\}$ 是通过旋转从标准基得到的。

6.3. \textbf{基的连续变换和方向。}
\textbf{定义}~ 我们说基 $A = \{a_1, a_2, \dots, a_n\}$ 可以\textbf{连续地}变换为基 $B = \{b_1, b_2, \dots, b_n\}$,如果存在一个连续的基族 $\{v_1(t), v_2(t), \dots, v_n(t)\}$, $t \in [a, b]$ 使得 $v_k(a) = a_k$, $v_k(b) = b_k$, $k=1, 2, \dots, n$。“连续的基族”意味着向量函数 $v_k(t)$ 是连续的(它们在某个基下的坐标是连续函数),并且,至关重要的是,系统 $\{v_1(t), v_2(t), \dots, v_n(t)\}$ 是所有 $t \in [a, b]$ 的基。注意,通过进行变量替换,如果需要,我们可以总是假设 $[a, b] = [0, 1]$。

\textbf{定理 6.1。} 两个基 $A = \{a_1, a_2, \dots, a_n\}$ 和 $B = \{b_1, b_2, \dots, b_n\}$ 具有相同方向,当且仅当一个基可以连续地变换为另一个基。

\textbf{证明}~ 假设基 $A$ 可以连续地变换为基 $B$,并设 $\{v_1(t), v_2(t), \dots, v_n(t)\}$, $t \in [a, b]$ 是执行此变换的连续基族。考虑矩阵函数 $V(t)$,其列是 $v_k(t)$ 在基 $A$ 下的坐标向量 $[v_k(t)]_A$。显然,$V(t)$ 的元素是连续函数,并且 $V(a) = I$, $V(b) = [I]_{A,B}$。注意,由于 $V(t)$ 始终是一个基,$\det V(t)$ 永远不为零。那么,根据介值定理,$\det V(a)$ 和 $\det V(b)$ 具有相同的符号。由于 $\det V(a) = \det I = 1$,我们可以得出 $[I]_{A,B}$ 的行列式 $\det[I]_{A,B} = \det V(b) > 0$,所以基 $A$ 和 $B$ 具有相同方向。为了证明反向蕴含,即定理的“仅当”部分,需要证明单位矩阵 $I$ 可以通过可逆矩阵连续地变换为任何满足 $\det B > 0$ 的矩阵 $B$。换句话说,存在一个在 $[a, b]$ 上的连续矩阵函数 $V(t)$,使得对于所有 $t \in [a, b]$,$V(t)$ 是可逆的,并且 $V(a) = I$, $V(b) = B$。我们将此事实的证明留给读者。有几种方法可以证明这一点,其中一种在下面的问题 6.2-6.5 中进行了概述。

\textbf{练习}~

6.1. 设 $R_\alpha$ 是通过 $\alpha$ 的旋转,所以它在标准基下的矩阵是 $\begin{pmatrix} \cos \alpha & -\sin \alpha \\ \sin \alpha & \cos \alpha \end{pmatrix}$。找出 $R_\alpha$ 在基 $\{v_1, v_2\}$, 其中 $v_1 = e_2$, $v_2 = e_1$ 下的矩阵。

6.2. 设 $R_\alpha$ 是旋转矩阵 $R_\alpha = \begin{pmatrix} \cos \alpha & -\sin \alpha \\ \sin \alpha & \cos \alpha \end{pmatrix}$。证明 $2 \times 2$ 单位矩阵 $I_2$ 可以通过可逆矩阵连续地变换为 $R_\alpha$。

6.3. 设 $U$ 是一个 $n \times n$ 正交矩阵,且 $\det U > 0$。证明单位矩阵 $I_n$ 可以通过可逆矩阵连续地变换为 $U$。
\textbf{提示:} 使用上一问题和 $n$ 维旋转表示为平面旋转的乘积。

6.4. 设 $A$ 是一个 $n \times n$ 正定 Hermite 矩阵,$A = A^* > 0$。证明单位矩阵 $I_n$ 可以通过可逆矩阵连续地变换为 $A$。
\textbf{提示:} 对角矩阵呢?

6.5. 使用极坐标分解和上面的问题 6.3, 6.4 完成定理 6.3 的“仅当”部分的证明。



% \chapter{双线性与二次型}



% \chapter{对偶空间与张量}



% \chapter{高级谱理论}



%%%%%%%%%%%%%%%%%%%%%% 参考文献 %%%%%%%%%%%%%%%%%%%%%

% 生成参考文献, 两种方式任选一种

% 第一种方式, 使用 bib 文件
%\nocite{*}  % 可以显示全部参考文献
% \bibliography{reference}

%--------------------------------------------------%

% 第二种方式, 手动添加文献信息
%
%%%%%%%%%%%%%%% 手动添加参考文献  %%%%%%%%%%%%%%%

\clearpage
\phantomsection
\addcontentsline{toc}{chapter}{参考文献} % 添加  "参考文献 " 到目录

\begin{thebibliography}{99}
\bibitem{Tadmor2012} Tadmor~E. A review of numerical methods for nonlinear partial differential equations\allowbreak[J]. Bull. Amer. Math. Soc., 2012, 49(4): 507-554.

\bibitem{LiLiu1997} 李荣华, 刘播. 微分方程数值解法\allowbreak[M]. 第四版. 北京: 高等教育出版社, 2009.

\bibitem{Adams2003} Adams~R~A, Fournier~J~J~F. Sobolev spaces\allowbreak[M]. 2nd ed. Amsterdam: Elsevier, 2003.

\bibitem{TreWei2014}Trefethen~L~N, Weideman~J~A~C. The exponentially convergent trapezoidal rule\allowbreak[J]. SIAM Rev., 2014, 56(3): 385-458.

\bibitem{Shen1994} Shen~J. Efficient spectral-Galerkin method I. Direct solvers of second- and fourth-order equations using Legendre polynomials\allowbreak[J]. SIAM J. Sci. Comput., 1994, 15(6): 1489-1505.

\end{thebibliography}




%%%%%%%%%%%%%%%%%%%%%% 附录 %%%%%%%%%%%%%%%%%%%%%%%%

% 添加附录, 如不需要可以注释
% 
%%%%%%%%%%%%%%%%%%%% 附录 %%%%%%%%%%%%%%%%%%%%%

% 添加附录, 如不需要可以把附录部分注释
\appendix

% 附录正文
\chapter{这是第一个附录}

\section{附录A的小节}

这里是附录环境\index{附录环境}.

附录公式及编号
\begin{equation}\label{eq:abc}
  a^2+b^2=c^2.
\end{equation}

附录的图片: 如图~\ref{fig:sinx2}.
\begin{figure}[htp!]
  \centering
  \includegraphics[width=0.45\linewidth]{image1.eps}
  \caption{函数 $y=\sin(x)$ 的图像}\label{fig:sinx2}
\end{figure}


附录的表格: 表~\ref{tab:heightweight2}.

\begin{table}[htp!]
\centering
% PLCR已经定义
\caption{某校学生身高体重样本}
\label{tab:heightweight2}
\begin{tabularx}{0.9\textwidth}{P{1.5cm}CCC}
\toprule
序号 & 年龄 & 身高 & 体重 \\
\midrule
001 & 15 & 156 & 42 \\
002 & 16 & 158 & 45 \\
003 & 14 & 162 & 48 \\
004 & 15 & 163 & 50 \\
\cmidrule{2-4}
平均 & 15 & 159.75 & 46.25 \\
\bottomrule
\end{tabularx}
\end{table}


\chapter{这是第二个附录}

\section{附录B的小节}

这里是附录环境.

附录内容附录内容附录内容内容附录内容附录内容附录内容附录内容附录内容附录内容附录内容附录内容附录内容附录内容附录内容附录内容附录内容附录内容附录内容附录内容附录内容正文内容正文.内容附录内容附录内容附录内容附录内容附录内容附录内容附录内容附录内容附录内容附录内容附录内容附录内容附录内容附录内容附录内容附录内容附录内容文内容正文内容.



%%%%%%%%%%%%%%%%%%%%%%%%%%%%%%%%%%%%%%%%%%%%%%%%%%%%

\backmatter  % 结束章节自动编号

%%%%%%%%%%%%%%%%%%%%%% 索引  %%%%%%%%%%%%%%%%%%%%%%%%

\clearpage
\printindex


%%%%%%%%%%%%%%%%%%%%%%% 后记 %%%%%%%%%%%%%%%%%%%%%%%%


%%%%%%%%%%%%%%%%%%%% 后记 %%%%%%%%%%%%%%%%%%%%%

\chapter{译~~后~~记}

从大一开学一个月到今天,这本书的翻译工作终于完成了。

这一切都始于李耀文老师的线性代数课和我们使用的核心参考书——《Linear Algebra Done Wrong》。它是一本非常出色的教材,但全英文的内容对不少同学来说,确实是一个不小的挑战。

为了帮助大家,也为了提升自己,我当时定下了几个目标:
力求译文精准易懂、采用专业的 \LaTeX{} 排版以媲美原书的阅读体验、成果永久免费分享,并于GitHub开源以汇集众智。

现在回看,我可以欣慰地说,当初的目标和承诺都已实现。还原英文原稿的排版,亲手敲下书中每一个复杂的 \LaTeX{} 公式——这份工作量确实超出了最初的预估,但也正是在这个过程中,我收获了无与伦比的成长与满足。


在此,衷心感谢李耀文老师的关注与指导,感谢匡亚明学院各位同学的支持与鼓励,正是有了你们的帮助,我才能够顺利完成这本书的翻译工作。

希望这本译稿能真正帮助到同级的同学以及未来的学弟学妹们。如果它能为你的学习带来一点点便利,那么我所有的付出就都是值得的。




\vspace{5ex}
\begin{flushright}
董耀择~~~~~~~~~

2025年10月~~~~~
\end{flushright}




\end{document}


