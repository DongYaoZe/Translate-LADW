
\chapter{行列式}

\section{引言}

读者可能已经遇到过行列式,至少是在微积分或代数中遇到的 $2 \times 2$ 和 $3 \times 3$ 矩阵的行列式。对于 $2 \times 2$ 矩阵 $\begin{pmatrix} a & b \\ c & d \end{pmatrix}$,行列式就是 $ad - bc$;$3 \times 3$ 矩阵的行列式可以通过“大卫之星”法则找到。在本章中,我们想为 $n \times n$ 矩阵介绍行列式。我不想仅仅给出一个形式定义。首先我想给出一些动机,然后推导出行列式应具备的一些性质。然后,如果我们想要这些性质,我们就别无选择,只能得到行列式的几个等价定义。

从矩阵行列式开始,而不是从向量组的行列式开始,更为方便:这里没有真正的区别,因为我们可以始终将向量(列)连接在一起(比如作为列)来形成一个矩阵。设我们有 $\mathbb{R}^n$ 中的 $n$ 个向量 $\mathbf{v}_1, \mathbf{v}_2, \dots, \mathbf{v}_n$(注意向量的数量与维度一致),我们想找到由这些向量确定的平行六面体的\textbf{n维体积}。由向量 $\mathbf{v}_1, \mathbf{v}_2, \dots, \mathbf{v}_n$ 确定的平行六面体可以定义为所有向量 $\mathbf{v} \in \mathbb{R}^n$ 的集合,这些向量可以表示为 $\mathbf{v} = t_1 \mathbf{v}_1 + t_2 \mathbf{v}_2 + \dots + t_n \mathbf{v}_n$, $0 \le t_k \le 1 \ \forall k = 1, 2, \dots, n$。当 $n=2$(平行四边形)和 $n=3$(平行六面体)时,这很容易可视化。那么 $n$ 维体积是什么?在维度 1 中,它是长度。最后,让我们引入一些符号。对于向量组(列)$\mathbf{v}_1, \mathbf{v}_2, \dots, \mathbf{v}_n$,我们将把它的行列式(我们将要构造的)表示为 $D(\mathbf{v}_1, \mathbf{v}_2, \dots, \mathbf{v}_n)$。如果我们把这些向量连接成矩阵 $A$($A$ 的第 $k$ 列是 $\mathbf{v}_k$),那么我们将使用符号 $\det A$, $\det A = D(\mathbf{v}_1, \mathbf{v}_2, \dots, \mathbf{v}_n)$。

对于矩阵 $A = \begin{pmatrix} a_{1,1} & a_{1,2} & \dots & a_{1,n} \\ a_{2,1} & a_{2,2} & \dots & a_{2,n} \\ \vdots & \vdots & \ddots & \vdots \\ a_{n,1} & a_{n,2} & \dots & a_{n,n} \end{pmatrix}$,它的行列式也常常表示为
$$
\begin{vmatrix}
a_{1,1} & a_{1,2} & \dots & a_{1,n} \\
a_{2,1} & a_{2,2} & \dots & a_{2,n} \\
\vdots & \vdots & \ddots & \vdots \\
a_{n,1} & a_{n,2} & \dots & a_{n,n}
\end{vmatrix}
$$

\section{行列式应具备的性质}

我们知道,对于维度 2 和 3,“体积”平行六面体由“底乘以高”法则确定:如果我们选择一个向量,那么高是该向量到由其余向量张成的子空间的距离,底是其余向量确定的平行六面体的($n-1$ 维)体积。现在让我们将这个想法推广到更高维度。暂时我们不关心如何精确地确定高和底。我们将表明,如果我们假设高和底满足某些自然性质,那么我们就别无选择,行列式就被唯一确定了。

\textbf{2.1. 每个参数的线性。}

首先,如果我们把向量 $\mathbf{v}_1$ 乘以一个正数 $a$,那么高(即到线性张成 $L(\mathbf{v}_2, \dots, \mathbf{v}_n)$ 的距离)就会乘以 $a$。如果我们允许负高度(和负体积),那么这个性质对所有标量 $a$ 都成立,因此向量组 $\mathbf{v}_1, \mathbf{v}_2, \dots, \mathbf{v}_n$ 的行列式 $D(\mathbf{v}_1, \mathbf{v}_2, \dots, \mathbf{v}_n)$ 应该满足
$$
D(\alpha \mathbf{v}_1, \mathbf{v}_2, \dots, \mathbf{v}_n) = \alpha D(\mathbf{v}_1, \mathbf{v}_2, \dots, \mathbf{v}_n)
$$
当然,向量 $\mathbf{v}_1$ 没有什么特别之处,所以对于任何索引 $k$
$$
D(\mathbf{v}_1, \dots, \alpha \mathbf{v}_k, \dots, \mathbf{v}_n) = \alpha D(\mathbf{v}_1, \dots, \mathbf{v}_k, \dots, \mathbf{v}_n) \quad (2.1)
$$
为了得到下一个性质,让我们注意到如果我们相加两个向量,那么结果的“高度”应该等于被加数“高度”的总和,即
$$
D(\mathbf{v}_1, \dots, \mathbf{u}_k + \mathbf{v}_k, \dots, \mathbf{v}_n) = D(\mathbf{v}_1, \dots, \mathbf{u}_k, \dots, \mathbf{v}_n) + D(\mathbf{v}_1, \dots, \mathbf{v}_k, \dots, \mathbf{v}_n) \quad (2.2)
$$
换句话说,上述两个性质表明,行列式是\textbf{每个参数(向量)的线性},这意味着如果我们固定 $n-1$ 个向量并将剩余向量解释为一个变量(参数),我们就会得到一个线性函数。

\textbf{注释}~ 我们已经知道\textbf{线性}是一个非常有用的性质,在许多情况下都有帮助。因此,允许负高度(以及因此的负体积)是获得线性的一个非常小的代价,因为我们之后总是可以取绝对值。实际上,通过允许负高度,我们并没有牺牲任何东西!相反,我们甚至获得了一些东西,因为行列式的符号包含有关向量系统(方向)的一些信息。

\textbf{2.2. 在“列替换”下的保持不变性。}

下一个性质也看起来很自然。也就是说,如果我们取一个向量,比如 $\mathbf{v}_j$,并向其添加另一个向量 $\mathbf{v}_k$ 的倍数,“高度”不会改变,所以
$$
D(\mathbf{v}_1, \dots, \mathbf{v}_j + \alpha \mathbf{v}_k, \dots, \mathbf{v}_k, \dots, \mathbf{v}_n) = D(\mathbf{v}_1, \dots, \mathbf{v}_j, \dots, \mathbf{v}_k, \dots, \mathbf{v}_n) \quad (2.3)
$$
换句话说,如果我们应用第三种类型的\textbf{列运算},行列式不会改变。

\textbf{注释}~ 虽然在此并非必需,但让我们注意到第二个线性部分(性质 (2.2))不是独立的:它可以从性质 (2.1) 和 (2.3) 推导出来。我们将证明留作读者的练习。

\textbf{2.3. 反对称性。}

下一个性质是行列式应该具备的,即\textbf{关于任意两个参数的函数在交换任意两个参数时会改变符号,这类函数称为反对称函数}。

\textbf{函数具有多个变量的性质,在交换任意两个参数时会改变符号,这类函数称为反对称函数。}

也就是说,如果我们交换两个向量,行列式会改变符号:
$$
D(\mathbf{v}_1, \dots, \mathbf{v}_k, \dots, \mathbf{v}_j, \dots, \mathbf{v}_n) = - D(\mathbf{v}_1, \dots, \mathbf{v}_j, \dots, \mathbf{v}_k, \dots, \mathbf{v}_n) \quad (2.4)
$$
在第一眼看来,这个性质看起来不自然,但它可以从前面的性质推导出来。也就是说,三次应用性质 (2.3),然后使用 (2.1),我们得到
\begin{align*} D(\mathbf{v}_1, \dots, \mathbf{v}_j, \dots, \mathbf{v}_k, \dots, \mathbf{v}_n) &= D(\mathbf{v}_1, \dots, \mathbf{v}_j, \dots, \mathbf{v}_k - \mathbf{v}_j, \dots, \mathbf{v}_n) \\ &= D(\mathbf{v}_1, \dots, \mathbf{v}_j + (\mathbf{v}_k - \mathbf{v}_j), \dots, \mathbf{v}_k - \mathbf{v}_j, \dots, \mathbf{v}_n) \\ &= D(\mathbf{v}_1, \dots, \mathbf{v}_k, \dots, \mathbf{v}_k - \mathbf{v}_j, \dots, \mathbf{v}_n) \\ &= D(\mathbf{v}_1, \dots, \mathbf{v}_k, \dots, (\mathbf{v}_k - \mathbf{v}_j) - \mathbf{v}_k, \dots, \mathbf{v}_n) \\ &= D(\mathbf{v}_1, \dots, \mathbf{v}_k, \dots, -\mathbf{v}_j, \dots, \mathbf{v}_n) \\ &= - D(\mathbf{v}_1, \dots, \mathbf{v}_k, \dots, \mathbf{v}_j, \dots, \mathbf{v}_n) \end{align*}

\textbf{2.4. 归一化。}

最后一个性质是最简单的。对于 $\mathbb{R}^n$ 中的标准基 $\mathbf{e}_1, \mathbf{e}_2, \dots, \mathbf{e}_n$,对应的平行六面体是 $n$ 维单位立方体,所以
$$
D(\mathbf{e}_1, \mathbf{e}_2, \dots, \mathbf{e}_n) = 1
$$
在矩阵表示中,这可以写成 $\det I = 1$。


\section{行列式的构造}

现在的计划是:利用我们从第 2 节决定的行列式应具有的性质,我们推导出行列式的其他性质,其中一些性质非常不平凡。我们将展示如何使用这些性质通过我们熟悉的朋友——行约简来计算行列式。稍后,在第 4 节,我们将展示行列式,即具有所需性质的函数,是存在且唯一的。毕竟,我们必须确信我们正在计算和研究的对象是存在的。

虽然我们对行列式的动机及其性质的初始几何动机来自于考虑 $\mathbb{R}^n$ 中的向量,因此它们只与实数项的矩阵相关,但以下所有构造只使用代数运算(加法、乘法、除法)并且适用于具有复数项的矩阵,甚至适用于任意域上的项。

因此,在以下内容中,我们不仅为实数矩阵,也为复数矩阵(以及具有任意域项的矩阵)构造行列式。虽然我们最初的几何动机仅适用于实数情况,但在我们确定了行列式的性质(见本节的性质 1-3)之后,所有内容都适用于一般情况。

\textbf{3.1. 基本性质。}

在这一节中,我们将使用以下行列式性质:
1. 行列式在每个列中是线性的,即,在向量表示中,对于每个索引 $k$,
$$
D(\mathbf{v}_1, \dots, \alpha \mathbf{u}_k + \beta \mathbf{v}_k, \dots, \mathbf{v}_n) = \alpha D(\mathbf{v}_1, \dots, \mathbf{u}_k, \dots, \mathbf{v}_n) + \beta D(\mathbf{v}_1, \dots, \mathbf{v}_k, \dots, \mathbf{v}_n)
$$
对所有标量 $\alpha, \beta$ 成立。
2. 行列式是\textbf{反对称}的,即,如果我们交换两列,行列式改变符号。
3. 归一化性质:$\det I = 1$。

所有这些性质在第 2 节中都已讨论过。第一个性质只是 (2.1) 和 (2.2) 的组合。第二个是 (2.4),最后一个是归一化性质 (2.5)。注意,我们没有使用性质 (2.3):它可以从上述三个性质中推导出来。这三个性质完全定义了行列式!

\textbf{命题 3.1。} 对于方阵 $A$,以下陈述成立:
1. 如果 $A$ 有一个零列,那么 $\det A = 0$。
2. 如果 $A$ 有两列相等,那么 $\det A = 0$;
3. 如果 $A$ 的一列是另一列的倍数,那么 $\det A = 0$;
4. 如果 $A$ 的列是线性相关的,即如果矩阵不可逆,那么 $\det A = 0$。

\textbf{证明}~ 陈述 1 由线性性直接得出。如果我们用零乘以零列,我们不会改变矩阵及其行列式。但根据上面的性质 1,我们应该得到 0。行列式的反对称性蕴含了陈述 2。实际上,如果我们交换两列相等的列,我们什么也没改变,所以行列式保持不变。另一方面,交换两列改变了行列式的符号,所以 $\det A = -\det A$,这只有在 $\det A = 0$ 时才可能。陈述 3 是陈述 2 和线性性的直接推论。要证明最后一个陈述,让我们首先假设第一个向量 $\mathbf{v}_1$ 是其他向量的线性组合,$\mathbf{v}_1 = \alpha_2 \mathbf{v}_2 + \alpha_3 \mathbf{v}_3 + \dots + \alpha_n \mathbf{v}_n = \sum_{k=2}^n \alpha_k \mathbf{v}_k$。那么根据线性性,我们有(在向量表示中)
$$
D(\mathbf{v}_1, \mathbf{v}_2, \dots, \mathbf{v}_n) = D(\sum_{k=2}^n \alpha_k \mathbf{v}_k, \mathbf{v}_2, \dots, \mathbf{v}_n) = \sum_{k=2}^n \alpha_k D(\mathbf{v}_k, \mathbf{v}_2, \dots, \mathbf{v}_n)
$$
并且和中的每个行列式都为零,因为存在两个相等的列。现在考虑一般情况,即假设系统 $\mathbf{v}_1, \mathbf{v}_2, \dots, \mathbf{v}_n$ 是线性相关的。那么其中一个向量,比如 $\mathbf{v}_k$,可以表示为其他向量的线性组合。将此向量与 $\mathbf{v}_1$ 交换,我们得到我们刚刚处理过的情况,所以 $D(\mathbf{v}_1, \dots, \mathbf{v}_k, \dots, \mathbf{v}_n) = -D(\mathbf{v}_k, \dots, \mathbf{v}_1, \dots, \mathbf{v}_n) = -0 = 0$,所以这种情况下的行列式也为零。

下一个命题推广了性质 (2.3)。正如我们上面已经说过的,这个性质可以从我们本节中使用的三个“基本”性质中推导出来。

\textbf{命题 3.2。} 当我们向一列添加其他列的线性组合时,行列式不会改变(保持其他列不变)。特别是,行列式在“列替换”(第三类列运算)下保持不变。

\textbf{证明}~ 固定一个向量 $\mathbf{v}_k$,令 $\mathbf{u}$ 为其他向量的线性组合,$\mathbf{u} = \sum_{j \neq k} \alpha_j \mathbf{v}_j$。那么根据线性性
$$
D(\mathbf{v}_1, \dots, \mathbf{v}_k + \mathbf{u}, \dots, \mathbf{v}_n) = D(\mathbf{v}_1, \dots, \mathbf{v}_k, \dots, \mathbf{v}_n) + D(\mathbf{v}_1, \dots, \mathbf{u}, \dots, \mathbf{v}_n)
$$
并且根据命题 3.1,最后一项为零。

\textbf{3.3. 对角和三角矩阵的行列式。}

现在我们准备为一些重要的特殊矩阵类别计算行列式。第一类是所谓的\textbf{对角}矩阵。让我们回顾一下,一个方阵 $A = \{a_{j,k}\}_{n \times n}$ 称为\textbf{对角}矩阵,如果主对角线下的所有项都为零,即如果 $a_{j,k} = 0$ $\forall j \neq k$。我们将经常使用 $\text{diag}\{a_1, a_2, \dots, a_n\}$ 来表示对角矩阵:
$$
\begin{pmatrix}
a_1 & 0 & \dots & 0 \\
0 & a_2 & \dots & 0 \\
\vdots & \vdots & \ddots & \vdots \\
0 & 0 & \dots & a_n
\end{pmatrix}
$$
由于对角矩阵 $\text{diag}\{a_1, a_2, \dots, a_n\}$ 可以通过将第 $k$ 列乘以 $a_k$ 从单位矩阵 $I$ 得到,
\textbf{对角矩阵的行列式等于对角项的乘积,$\det(\text{diag}\{a_1, a_2, \dots, a_n\}) = a_1 a_2 \dots a_n$。}

下一个重要类别是所谓的\textbf{三角}矩阵。一个方阵 $A = \{a_{j,k}\}_{n \times n}$ 称为\textbf{上三角}矩阵,如果主对角线下的所有项都为零,即如果 $a_{j,k} = 0$ $\forall k < j$。一个方阵称为\textbf{下三角}矩阵,如果主对角线上的所有项都为零,即如果 $a_{j,k} = 0$ $\forall j < k$。我们称矩阵为\textbf{三角}矩阵,如果它是下三角或上三角矩阵。

很容易看出
\textbf{三角矩阵的行列式等于对角项的乘积,$\det A = a_{1,1} a_{2,2} \dots a_{n,n}$。}
实际上,如果一个三角矩阵的主对角线上有零,那么它就是不可逆的(这可以通过列运算很容易地检查出来),因此两边都等于零。如果所有对角项都非零,那么使用列替换(第三类列运算)可以将矩阵转化为具有相同对角项的对角矩阵:
对于上三角矩阵,首先应该从第 2、3、...、n 列减去第一列的适当倍数,“消去”第一行中的所有项,然后从第 3、...、n 列减去第二列的适当倍数,依此类推。

为了处理下三角矩阵的情况,需要从左到右进行“列约简”,即首先从最后一列开始,将适当倍数的最后一列从第 $n-1$, $\dots$, 2, 1 列减去,依此类推。

\textbf{3.4. 计算行列式。}

现在我们知道如何计算行列式,使用它们的性质:只需进行列约简(即对 $A^T$ 进行行约简),并跟踪改变行列式的列运算。幸运的是,最常使用的运算——行替换,即第三类运算,不会改变行列式。所以我们只需要跟踪列的交换和用标量乘以列。如果 $A^T$ 的阶梯形在每一列(和每一行)都没有主元,那么 $A$ 是不可逆的,因此 $\det A = 0$。如果 $A$ 是可逆的,我们得到一个三角矩阵,而 $\det A$ 是对角项的乘积,乘以来自列交换和乘法的校正因子。上述算法暗示 $\det A$ 仅在矩阵 $A$ 不可逆时才可能为零。结合命题 3.1 的最后一个陈述,我们得到:

\textbf{命题 3.3。} $\det A = 0$ 当且仅当 $A$ 不可逆,或者等价地说:$\det A \neq 0$ 当且仅当 $A$ 可逆。

注意,虽然我们现在知道如何计算行列式,但行列式仍然没有被定义。可以问:为什么我们不将其定义为通过上述算法得到的结果?问题在于,从形式上看,这个结果并非良好定义:我们没有证明不同的列运算序列会得到相同的结果。

\textbf{3.5. 转置行列式和乘积行列式。初等矩阵的行列式。}

在本节中,我们将证明两个重要定理。

\textbf{定理 3.4(转置行列式)。} 对于方阵 $A$,$\det A = \det(A^T)$。

这个定理意味着,我们之前讨论过的关于列的所有陈述,关于行的相应陈述也都是正确的。特别是,行列式在\textbf{行运算}下的行为与在\textbf{列运算}下的行为相同。因此,我们可以使用行运算来计算行列式。

\textbf{定理 3.5(乘积行列式)。} 对于 $n \times n$ 矩阵 $A$ 和 $B$:
$$
\det(AB) = (\det A)(\det B)
$$
换句话说,\textbf{乘积的行列式等于行列式的乘积}。

为了证明这两个定理,我们需要以下引理。

\textbf{引理 3.6。} 对于方阵 $A$ 和初等矩阵 $E$(相同大小):
$$
\det(AE) = (\det A)(\det E)
$$
\textbf{证明}~ 证明可以通过直接检查来完成:特殊矩阵的行列式很容易计算;从左边乘以初等矩阵是列运算,而列运算对行列式的影响是众所周知的。这可能看起来像一个幸运的巧合,即初等矩阵的行列式与其相应的列运算一致,但这并非巧合。也就是说,对于列运算,相应的初等矩阵可以从单位矩阵 $I$ 中通过该列运算得到。所以,它的行列式是 $1$($I$ 的行列式)乘以列运算的影响。而这一切就是这样!这可能一开始很难意识到,但上述段落是对引理的\textbf{完整且严谨}的证明!

应用引理 3.6 $N$ 次,我们得到以下推论。

\textbf{推论 3.7。} 对于任何矩阵 $A$ 和任何初等矩阵序列 $E_1, E_2, \dots, E_N$(所有矩阵均为 $n \times n$):
$$
\det(A E_1 E_2 \dots E_N) = (\det A)(\det E_1)(\det E_2) \dots (\det E_N)
$$

\textbf{引理 3.8。} 任何可逆矩阵都可以表示为初等矩阵的乘积。

\textbf{证明}~ 我们知道任何可逆矩阵都可以通过行运算(其简化阶梯形)化为单位矩阵。所以 $I = E_N E_{N-1} \dots E_2 E_1 A$,因此任何可逆矩阵都可以表示为初等矩阵的乘积,$A = (E_N \dots E_2 E_1)^{-1} I = E_1^{-1} E_2^{-1} \dots E_N^{-1}$(初等矩阵的逆也是初等矩阵)。

\textbf{定理 3.4 的证明。} 首先,可以很容易地检查出,对于初等矩阵 $E$, $\det E = \det(E^T)$。请注意,只需为可逆矩阵 $A$ 证明该定理即可,因为如果 $A$ 不可逆,那么 $A^T$ 也不可逆,并且两个行列式都为零。根据引理 3.8,矩阵 $A$ 可以表示为初等矩阵的乘积,$A = E_1 E_2 \dots E_N$,并且根据推论 3.7, $A$ 的行列式是初等矩阵行列式的乘积。由于取转置只是转置每个初等矩阵并反转它们的顺序,推论 3.7 蕴含了 $\det A = \det A^T$。

\textbf{定理 3.5 的证明。} 首先让我们假设矩阵 $B$ 是可逆的。那么引理 3.8 蕴含了 $B$ 可以表示为初等矩阵的乘积 $B = E_1 E_2 \dots E_N$,因此根据推论 3.7 $\det(AB) = (\det A)(\det E_1)(\det E_2) \dots (\det E_N) = (\det A)(\det B)$。

如果 $B$ 不可逆,那么乘积 $AB$ 也不可逆,而定理仅仅说明 $0 = 0$。要检查乘积 $AB = C$ 是否不可逆,让我们假设它是可逆的。那么将恒等式 $AB = C$ 从左边乘以 $C^{-1}$,我们得到 $C^{-1} AB = I$,所以 $C^{-1} A$ 是 $B$ 的左逆。因此 $B$ 是左可逆的,并且由于它是方的,所以它是可逆的。我们得到了一个矛盾。

\textbf{3.6. 行列式的性质总结。}

首先,让我们再说一遍,\textbf{行列式仅为方阵定义!} 由于我们现在知道 $\det A = \det(A^T)$,我们之前关于列的所有陈述也对行成立。
1. 行列式在每个行(列)中是线性的,当其他行(列)固定时。
2. 如果我们交换矩阵 $A$ 的两行(列),行列式改变符号。
3. 对于三角矩阵(特别是对角矩阵),其行列式是对角项的乘积。特别是,$\det I = 1$。
4. 如果矩阵 $A$ 有一个零行(或零列),则 $\det A = 0$。
5. 如果矩阵 $A$ 有两行(列)相等,则 $\det A = 0$。
6. 如果 $A$ 的某一行(列)是其他行(列)的线性组合,即如果矩阵不可逆,则 $\det A = 0$;更一般地,
7. $\det A = 0$ 当且仅当 $A$ 不可逆,或者等价地说
8. $\det A \neq 0$ 当且仅当 $A$ 可逆。
9. 如果我们将行(列)的线性组合加到某个行(列)上,行列式不改变。特别是,行列式在行(列)替换,即第三类行(列)运算下保持不变。
10. $\det A^T = \det A$。
11. $\det(AB) = (\det A)(\det B)$。

最后,
12. 如果 $A$ 是一个 $n \times n$ 矩阵,那么 $\det( \alpha A ) = \alpha^n \det A$。最后一个性质是从行列式的线性性质得出的,如果我们回忆起要将矩阵 $A$ 乘以 $\alpha$,我们必须将每一行乘以 $\alpha$,并且每次乘法都会将行列式乘以 $\alpha$。

\textbf{练习}~

3.1. 如果 $A$ 是一个 $n \times n$ 矩阵,$\det(3A)$ 与 $\det A$ 有何关系?
注释:$\det(3A) = 3 \det A$ 仅在 $1 \times 1$ 矩阵的平凡情况下成立。

3.2. $A = \begin{pmatrix} a_1 & a_2 & a_3 \\ b_1 & b_2 & b_3 \\ c_1 & c_2 & c_3 \end{pmatrix}$, $B = \begin{pmatrix} 2a_1 & 3a_2 & 5a_3 \\ 2b_1 & 3b_2 & 5b_3 \\ 2c_1 & 3c_2 & 5c_3 \end{pmatrix}$;
$A = \begin{pmatrix} a_1 & a_2 & a_3 \\ b_1 & b_2 & b_3 \\ c_1 & c_2 & c_3 \end{pmatrix}$, $B = \begin{pmatrix} 3a_1 & 4a_2 + 5a_1 & 5a_3 \\ 3b_1 & 4b_2 + 5b_1 & 5b_3 \\ 3c_1 & 4c_2 + 5c_1 & 5c_3 \end{pmatrix}$。$A$ 和 $B$ 的行列式之间有什么关系?

3.3. 使用列或行运算计算行列式:
$\begin{vmatrix} 0 & 1 & 2 \\ -1 & 0 & -3 \\ 2 & 3 & 0 \end{vmatrix}$, $\begin{vmatrix} 1 & 2 & 3 \\ 4 & 5 & 6 \\ 7 & 8 & 9 \end{vmatrix}$, $\begin{vmatrix} 1 & 0 & -2 & 3 \\ -3 & 1 & 1 & 2 \\ 0 & 4 & -1 & 1 \\ 2 & 3 & 0 & 1 \end{vmatrix}$, $\begin{vmatrix} 1 & x \\ 1 & y \end{vmatrix}$。

3.4. 一个方阵($n \times n$)称为\textbf{反对称}(或\textbf{反交换})矩阵,如果 $A^T = -A$。证明如果 $A$ 是反对称的且 $n$ 是奇数,则 $\det A = 0$。这对偶数 $n$ 是否成立?

3.5. 一个方阵称为\textbf{幂零}(nilpotent)矩阵,如果 $A^k = 0$ 对某个正整数 $k$ 成立。证明如果 $A$ 是幂零的,则 $\det A = 0$。

3.6. 证明如果矩阵 $A$ 和 $B$ 相似,则 $\det A = \det B$。

3.7. 一个实方阵 $Q$ 称为\textbf{正交}的,如果 $Q^T Q = I$。证明如果 $Q$ 是正交矩阵,那么 $\det Q = \pm 1$。

3.8. 证明 $\begin{vmatrix} 1 & x & x^2 \\ 1 & y & y^2 \\ 1 & z & z^2 \end{vmatrix} = (z-x)(z-y)(y-x)$。这是所谓的 Vandermonde 行列式的特例。

3.9. 设平面 $\mathbb{R}^2$ 中的点 $A, B, C$ 的坐标分别为 $(x_1, y_1), (x_2, y_2), (x_3, y_3)$。证明三角形 $ABC$ 的面积是 $\frac{1}{2} \left| \begin{vmatrix} 1 & x_1 & y_1 \\ 1 & x_2 & y_2 \\ 1 & x_3 & y_3 \end{vmatrix} \right|$ 的绝对值。提示:使用行运算和 $2 \times 2$ 行列式的几何解释(面积)。

3.10. 设 $A$ 和 $C$ 是方阵,证明块三角矩阵 $\begin{pmatrix} I & * \\ 0 & A \end{pmatrix}$, $\begin{pmatrix} A & * \\ 0 & I \end{pmatrix}$, $\begin{pmatrix} I & 0 \\ * & A \end{pmatrix}$, $\begin{pmatrix} A & 0 \\ * & I \end{pmatrix}$ 的行列式都等于 $\det A$。这里 $*$ 可以是任何东西。以下问题说明了块矩阵表示的力量。

3.11. 使用上一个问题证明,如果 $A$ 和 $C$ 是方阵,那么 $\det \begin{pmatrix} A & B \\ 0 & C \end{pmatrix} = (\det A)(\det C)$。提示:$\begin{pmatrix} A & B \\ 0 & C \end{pmatrix} = \begin{pmatrix} I & B \\ 0 & C \end{pmatrix} \begin{pmatrix} A & 0 \\ 0 & I \end{pmatrix}$。

3.12. 设 $A$ 是 $m \times n$ 矩阵,$B$ 是 $n \times m$ 矩阵。证明 $\det \begin{pmatrix} 0 & A \\ -B & I \end{pmatrix} = \det(AB)$。提示:虽然可以通过对矩阵进行行运算得到行列式易于计算的形式,但最简单的方法是右乘矩阵 $\begin{pmatrix} I & 0 \\ B & I \end{pmatrix}$。


\section{行列式的形式定义~存在性与唯一性}

在本节中,我们得到了行列式的形式定义。我们表明,一个函数,满足第 3 节中的基本性质 1, 2, 3 的存在性,而且,这样的函数是唯一的,也就是说,在构造行列式时我们别无选择。

考虑一个 $n \times n$ 矩阵 $A = \{a_{j,k}\}_{n \times n}$,并设 $\mathbf{v}_1, \mathbf{v}_2, \dots, \mathbf{v}_n$ 是它的列,即
$$
\mathbf{v}_k = \begin{pmatrix} a_{1,k} \\ a_{2,k} \\ \vdots \\ a_{n,k} \end{pmatrix} = a_{1,k} \mathbf{e}_1 + a_{2,k} \mathbf{e}_2 + \dots + a_{n,k} \mathbf{e}_n = \sum_{j=1}^n a_{j,k} \mathbf{e}_j
$$
使用行列式的线性性,我们在第一列展开:
$$
D(\mathbf{v}_1, \mathbf{v}_2, \dots, \mathbf{v}_n) = D(\sum_{j=1}^n a_{j,1} \mathbf{e}_j, \mathbf{v}_2, \dots, \mathbf{v}_n) = \sum_{j=1}^n a_{j,1} D(\mathbf{e}_j, \mathbf{v}_2, \dots, \mathbf{v}_n) \quad (4.1)
$$
然后我们在第二列展开,然后是第三列,依此类推。我们得到
$$
D(\mathbf{v}_1, \mathbf{v}_2, \dots, \mathbf{v}_n) = \sum_{j_1=1}^n \sum_{j_2=1}^n \dots \sum_{j_n=1}^n a_{j_1,1} a_{j_2,2} \dots a_{j_n,n} D(\mathbf{e}_{j_1}, \mathbf{e}_{j_2}, \dots, \mathbf{e}_{j_n})
$$
注意,我们必须为每一列使用不同的求和索引:我们称它们为 $j_1, j_2, \dots, j_n$;这里 $j_1$ 的索引与 (4.1) 中的索引 $j$ 相同。这是一个巨大的求和,包含 $n^n$ 项。幸运的是,其中一些项为零。也就是说,如果 $j_1, j_2, \dots, j_n$ 中有任何两个索引相同,则行列式 $D(\mathbf{e}_{j_1}, \mathbf{e}_{j_2}, \dots, \mathbf{e}_{j_n})$ 为零,因为这里有两个相等的列。因此,让我们重写求和,省略所有零项。最方便的方式是使用\textbf{排列}(permutation)的概念。

非正式地说,一个有序集 $\{1, 2, \dots, n\}$ 的\textbf{排列}是其元素的重新排列。一种方便的形式表示这种重新排列是通过使用一个函数 $\sigma: \{1, 2, \dots, n\} \to \{1, 2, \dots, n\}$,其中 $\sigma(1), \sigma(2), \dots, \sigma(n)$ 给出了集合 $1, 2, \dots, n$ 的新顺序。换句话说,排列 $\sigma$ 将有序集 $1, 2, \dots, n$ 重排为 $\sigma(1), \sigma(2), \dots, \sigma(n)$。这样的函数 $\sigma$ 必须是\textbf{一对一}(对不同的参数取不同的值)和\textbf{ onto}(取自目标空间的所有可能值)。一对一且 onto 的函数称为\textbf{双射}(bijection),它们在定义域和目标空间之间建立了一对一的对应关系。$^1$

$^1$ 尽管这在此处不直接相关,但让我们注意到,在组合学中是众所周知的,集合 $\{1, 2, \dots, n\}$ 的不同排列的数量恰好是 $n!$。所有 $n$ 的排列的集合将被记为 $\text{Perm}(n)$。

虽然这在此处不直接相关,但让我们注意到,在组合学中是众所周知的,集合 $\{1, 2, \dots, n\}$ 的不同排列的数量恰好是 $n!$。所有 $n$ 的排列的集合将被记为 $\text{Perm}(n)$。

使用排列的概念,我们可以将行列式重写为:
$$
D(\mathbf{v}_1, \mathbf{v}_2, \dots, \mathbf{v}_n) = \sum_{\sigma \in \text{Perm}(n)} a_{\sigma(1),1} a_{\sigma(2),2} \dots a_{\sigma(n),n} D(\mathbf{e}_{\sigma(1)}, \mathbf{e}_{\sigma(2)}, \dots, \mathbf{e}_{\sigma(n)})
$$
其中求和是遍历 $\{1, 2, \dots, n\}$ 的所有排列。

矩阵 $\mathbf{e}_{\sigma(1)}, \mathbf{e}_{\sigma(2)}, \dots, \mathbf{e}_{\sigma(n)}$ 的列可以从单位矩阵通过有限次数的列交换得到,所以行列式 $D(\mathbf{e}_{\sigma(1)}, \mathbf{e}_{\sigma(2)}, \dots, \mathbf{e}_{\sigma(n)})$ 是 $1$ 或 $-1$,取决于列交换的次数。为了形式化这一点,我们(非正式地)定义排列 $\sigma$ 的\textbf{符号}(记作 $\text{sign } \sigma$)为,如果将 $n$ 元组 $1, 2, \dots, n$ 重排为 $\sigma(1), \sigma(2), \dots, \sigma(n)$ 所需的交换次数是偶数,则符号为 1,如果交换次数是奇数,则 $\text{sign}(\sigma) = -1$。这是组合学中的一个事实,符号是明确定义的,即虽然有无数种方法可以从 $1, 2, \dots, n$ 得到 $n$ 元组 $\sigma(1), \sigma(2), \dots, \sigma(n)$,但交换次数要么总是奇数,要么总是偶数。

一种证明这一点的方法是引入另一种定义。设 $K(\sigma)$ 为 $\sigma$ 的\textbf{逆序对}(disorder)的数量,即满足 $\sigma(j) > \sigma(k)$ 的整数对 $(j, k)$ 的数量,其中 $j, k \in \{1, 2, \dots, n\}$, $j < k$,然后检查该数量是偶数还是奇数。我们将排列 $\sigma$ 称为\textbf{奇排列}如果 $K$ 是奇数,称为\textbf{偶排列}如果 $K$ 是偶数。然后定义 $\text{sign } \sigma := (-1)^{K(\sigma)}$;注意这样定义的 $\text{sign } \sigma$ 是明确定义的。

我们要证明 $\text{sign } \sigma = (-1)^{K(\sigma)}$ 可以通过将 $n$ 元组 $1, 2, \dots, n$ 重排为 $\sigma(1), \sigma(2), \dots, \sigma(n)$ 并计算交换次数来得到,如上所述。

如果 $\sigma(k) = k \ \forall k$,那么\textbf{逆序对}的数量 $K(\sigma)$ 为 0,所以这种\textbf{恒等}排列的符号是 1。还请注意,任何两个相邻元素的转置(仅交换两个相邻元素)会改变排列的符号,因为它会改变逆序对的数量(增加或减少 1)。因此,要从一个排列得到另一个排列,当排列具有相同的符号时,总是需要偶数次初等转置,而当符号不同时,则需要奇数次。最后,任何两个元素的交换都可以通过奇数次初等转置来实现。这意味着当两个元素被交换时,符号会改变。因此,要从 $1, 2, \dots, n$ 得到偶排列(正符号)总是需要偶数次交换,而得到奇排列(负符号)需要奇数次交换。因此,为了从 $1, 2, \dots, n$ 得到偶排列(正符号)总是需要偶数次交换,而得到奇排列(负符号)需要奇数次交换。因此,符号在交换两个元素时会改变。所以,要从 $1, 2, \dots, n$ 得到偶排列(正符号)总是需要偶数次交换,而得到奇排列(负符号)需要奇数次交换。因此,符号在交换两个元素时会改变。

因此,如果我们希望行列式满足第 3 节中的基本性质 1-3,我们必须将其定义为:
$$
\det A = \sum_{\sigma \in \text{Perm}(n)} a_{\sigma(1),1} a_{\sigma(2),2} \dots a_{\sigma(n),n} \text{sign}(\sigma) \quad (4.2)
$$
其中求和遍历 $\{1, 2, \dots, n\}$ 的所有排列。如果我们这样定义行列式,可以很容易地验证它满足第 3 节中的基本性质 1-3。实际上,因为每个乘积项在每一列中恰好有一个因子,并且对于任何两个相邻的列交换,我们得到的符号会改变,所以满足线性性和反对称性。而且,对于单位矩阵 $I$,右侧只有一项(对应于恒等排列 $\sigma(k)=k \ \forall k$),它的符号是 1,所以 $D(I)=1$。

\textbf{练习}~

4.1. 假设排列 $\sigma$ 将 $(1, 2, 3, 4, 5)$ 映射到 $(5, 4, 1, 2, 3)$。
a) 找到 $\sigma$ 的符号;
b) $\sigma^2 := \sigma \circ \sigma$ 对 $(1, 2, 3, 4, 5)$ 做什么?
c) 逆排列 $\sigma^{-1}$ 对 $(1, 2, 3, 4, 5)$ 做什么?
d) $\sigma^{-1}$ 的符号是什么?

4.2. 设 $P$ 是一个\textbf{排列矩阵},即一个由零和一组成的 $n \times n$ 矩阵,并且每行每列恰好有一个 1。
a) 你能描述相应的线性变换吗?这会解释它的名称。
b) 证明 $P$ 是可逆的。你能描述 $P^{-1}$ 吗?
c) 证明对于某些 $N > 0$, $P^N := P P \dots P$($N$ 次)$= I$。利用只有有限个排列的事实。

4.3. 为什么 $(1, 2, \dots, 9)$ 的排列有偶数个,并且其中恰好一半是奇排列?提示:这个问题用排列来解决可能很难,但有一个非常简单的行列式解。

4.4. 如果 $\sigma$ 是一个奇排列,解释为什么 $\sigma^2$ 是偶数但 $\sigma^{-1}$ 是奇数。

4.5. 使用 (4.2) 的行列式形式计算一个 $n \times n$ 矩阵的行列式需要多少次乘法和加法?不要计算计算 $\text{sign } \sigma$ 所需的操作。


\section{代数余子式展开}

对于 $n \times n$ 矩阵 $A = \{a_{j,k}\}_{n \times n}$,设 $A_{j,k}$ 表示通过划掉第 $j$ 行和第 $k$ 列得到的 $(n-1) \times (n-1)$ 矩阵。

\textbf{定理 5.1(行列式的代数余子式展开)。} 设 $A$ 是一个 $n \times n$ 矩阵。对于每个 $j$, $1 \le j \le n$,行列式 $A$ 可以按第 $j$ 行展开为:
$$
\det A = a_{j,1} (-1)^{j+1} \det A_{j,1} + a_{j,2} (-1)^{j+2} \det A_{j,2} + \dots + a_{j,n} (-1)^{j+n} \det A_{j,n} = \sum_{k=1}^n a_{j,k} (-1)^{j+k} \det A_{j,k}
$$
类似地,对于每个 $k$, $1 \le k \le n$,行列式可以按第 $k$ 列展开为:
$$
\det A = \sum_{j=1}^n a_{j,k} (-1)^{j+k} \det A_{j,k}
$$

\textbf{证明}~ 我们首先证明第 1 行展开的公式。第 2 行的展开公式可以通过交换第 1 行和第 2 行从它得到。然后交换第 2 行和第 3 行,我们得到第 3 行的展开公式,依此类推。由于 $\det A = \det A^T$,列展开自动跟上。

让我们首先考虑一个特殊情况,即第一行只有一个非零项 $a_{1,1}$。通过对第 2、3、...、n 列进行列运算,我们可以将 $A$ 转化为下三角形式。那么 $A$ 的行列式可以计算为三角矩阵的对角项的乘积 $\times$ 来自列运算的修正因子。但是,除了 $a_{1,1}$ 之外的所有对角项的乘积(即不包括 $a_{1,1}$)加上修正因子恰好是 $\det A_{1,1}$,所以在这种特定情况下 $\det A = a_{1,1} \det A_{1,1}$。

现在考虑所有项除了 $a_{1,2}$ 在第一行都为零的情况。这种情况可以通过交换第 1 列和第 2 列来减少到前面的情况,因此在这种情况下 $\det A = (-1)^{1+2} a_{1,2} \det A_{1,2}$。当 $a_{1,3}$ 是第一行唯一非零项的情况,可以通过交换第 2 行和第 3 行来减少到前面情况,所以在这种情况下 $\det A = a_{1,3} \det A_{1,3}$。

重复这个过程,我们得到,当 $a_{1,k}$ 是第一行唯一非零项时,$\det A = (-1)^{1+k} a_{1,k} \det A_{1,k}$。

在一般情况下,行列式在每一行上的线性意味着 $\det A = \det A^{(1)} + \det A^{(2)} + \dots + \det A^{(n)} = \sum_{k=1}^n \det A^{(k)}$,其中矩阵 $A^{(k)}$ 是通过将 $A$ 的第一行中除 $a_{1,k}$ 之外的所有项替换为 0 而得到的。正如我们上面所讨论的,$\det A^{(k)} = (-1)^{1+k} a_{1,k} \det A_{1,k}$,所以 $\det A = \sum_{k=1}^n (-1)^{1+k} a_{1,k} \det A_{1,k}$。

为了得到第二行的展开式,我们可以交换第 1 行和第 2 行,然后应用上面的公式。行交换改变了符号,所以我们得到 $\det A = -\sum_{k=1}^n (-1)^{1+k} a_{2,k} \det A_{2,k} = \sum_{k=1}^n (-1)^{2+k} a_{2,k} \det A_{2,k}$。通过交换第 3 行和第 2 行并按第二行展开,我们得到公式 $\det A = \sum_{k=1}^n (-1)^{3+k} a_{3,k} \det A_{3,k}$,依此类推。

要将行列式 $\det A$ 展开到第 $k$ 列,只需对 $A^T$ 应用行展开公式即可。

\textbf{定义}~ $C_{j,k} = (-1)^{j+k} \det A_{j,k}$ 这些数称为 $A$ 的\textbf{代数余子式}(cofactor)。

使用这个符号,在第 $j$ 行展开行列式的公式可以重写为 $\det A = a_{j,1} C_{j,1} + a_{j,2} C_{j,2} + \dots + a_{j,n} C_{j,n} = \sum_{k=1}^n a_{j,k} C_{j,k}$。类似地,在第 $k$ 列展开可以写成 $\det A = a_{1,k} C_{1,k} + a_{2,k} C_{2,k} + \dots + a_{n,k} C_{n,k} = \sum_{j=1}^n a_{j,k} C_{j,k}$。

\textbf{注释}~ 代数余子式展开公式经常被用作行列式的定义。不难证明由该公式给出的量满足行列式的基本性质:归一化性质是微不足道的,反对称性的证明很容易。然而,线性性的证明虽然不难,但有点繁琐。

\textbf{注释}~ 虽然它看起来非常不错,但代数余子式展开公式不适用于计算大于 $3 \times 3$ 的矩阵的行列式。正如可以计算的那样,它需要超过 $n!$ 次乘法(精确地说,需要 $\sum_{k=2}^n n!/k!$ 次乘法),而 $n!$ 的增长非常快。例如,计算一个 $20 \times 20$ 矩阵的代数余子式展开需要超过 $20! \approx 2.4 \times 10^{18}$ 次乘法。一台每秒执行十亿次乘法的计算机需要 77 年才能执行 $20!$ 次乘法;计算一个 $20 \times 20$ 矩阵的代数余子式展开所需的乘法将需要 132 多年。另一方面,使用行约简计算 $n \times n$ 矩阵的行列式需要 $(n^3 + 2n - 3) / 3$ 次乘法(以及大约相同数量的加法)。对于一台每秒执行一百万次运算(按当前标准非常慢)的计算机来说,计算 $100 \times 100$ 矩阵的行列式只需要一秒钟的一小部分。只有当某一行(或列)包含很多零项时,才可能实用地应用代数余子式展开公式。然而,代数余子式展开公式具有重要的理论价值,正如下一节所示。

\textbf{5.1. 逆矩阵的代数余子式公式。}

由代数余子式 $C_{j,k} = (-1)^{j+k} \det A_{j,k}$ 组成的矩阵 $C = \{C_{j,k}\}_{n \times n}$ 称为 $A$ 的\textbf{代数余子式矩阵}。

\textbf{定理 5.2。} 设 $A$ 是一个可逆矩阵,设 $C$ 是它的代数余子式矩阵。那么
$$
A^{-1} = \frac{1}{\det A} C^T
$$

\textbf{证明}~ 让我们计算乘积 $AC^T$。第 $j$ 个对角项是通过将 $A$ 的第 $j$ 行与 $C$ 的第 $j$ 列(即 $C^T$ 的第 $j$ 行)相乘得到的,所以 $(AC^T)_{j,j} = a_{j,1} C_{j,1} + a_{j,2} C_{j,2} + \dots + a_{j,n} C_{j,n} = \det A$,根据代数余子式展开公式。为了得到非对角项,我们需要将 $A$ 的第 $k$ 行与 $C^T$ 的第 $j$ 列相乘,$j \neq k$,得到 $a_{k,1} C_{j,1} + a_{k,2} C_{j,2} + \dots + a_{k,n} C_{j,n}$。根据代数余子式展开公式(在第 $j$ 行展开),这是将 $A$ 中第 $j$ 行替换为第 $k$ 行(而保持所有其他行不变)得到的矩阵的行列式。但是,这个矩阵的第 $j$ 行和第 $k$ 行是相同的,所以行列式为 0。因此,$AC^T$ 的所有非对角项都为零(而所有对角项都等于 $\det A$),所以 $AC^T = (\det A) I$。这意味着矩阵 $\frac{1}{\det A} C^T$ 是 $A$ 的右逆,由于 $A$ 是方阵,所以它是逆。

回忆一下,对于可逆矩阵 $A$,方程 $A \mathbf{x} = \mathbf{b}$ 的解是 $x = A^{-1} b = \frac{1}{\det A} C^T b$,我们得到以下定理的推论。

\textbf{推论 5.3(Cramer 法则)。} 对于可逆矩阵 $A$,方程 $A \mathbf{x} = \mathbf{b}$ 的解的第 $k$ 个项由公式给出:
$$
x_k = \frac{\det B_k}{\det A}
$$
其中矩阵 $B_k$ 是通过将 $A$ 的第 $k$ 列替换为向量 $b$ 而得到的。

\textbf{5.2. 逆矩阵的代数余子式公式的应用。}

\textbf{示例(求 $2 \times 2$ 矩阵的逆)。} 代数余子式公式在求 $2 \times 2$ 矩阵 $A = \begin{pmatrix} a & b \\ c & d \end{pmatrix}$ 的逆时,确实非常有用。代数余子式仅仅是($1 \times 1$ 矩阵)的项,代数余子式矩阵是 $\begin{pmatrix} d & -c \\ -b & a \end{pmatrix}$,所以逆矩阵 $A^{-1}$ 由公式给出:
$$
A^{-1} = \frac{1}{\det A} \begin{pmatrix} d & -b \\ -c & a \end{pmatrix}
$$
虽然对于维度大于 3 的情况,逆矩阵的代数余子式公式看起来不实用,但它具有巨大的理论价值,正如下面的例子所示。

\textbf{示例(整数逆矩阵)。} 假设我们想构造一个具有整数项的矩阵 $A$,使得其逆也具有整数项(求逆这样的矩阵会是一个很好的家庭作业:无需处理分数)。如果 $\det A = 1$ 且其项是整数,那么逆矩阵的代数余子式公式蕴含了 $A^{-1}$ 也具有整数项。注意,构造一个 $\det A = 1$ 的整数矩阵很容易:应该从主对角线上为 1 的三角矩阵开始,然后应用几次行或列替换(第三类运算)来使矩阵看起来是通用的。

\textbf{示例(多项式矩阵的逆)。} 另一个例子是考虑一个\textbf{多项式矩阵} $A(x)$,即其项不是数字而是变量 $x$ 的多项式 $a_{j,k}(x)$。如果 $\det A(x) \equiv 1$,那么逆矩阵 $A^{-1}(x)$ 也是一个多项式矩阵。如果 $\det A(x) = p(x) \neq 0$,则从代数余子式展开可知,$p(x)$ 是一个多项式,因此 $A^{-1}(x)$ 具有有理数项:更重要的是,$p(x)$ 是每个分母的倍数。

\textbf{练习}~

5.1. 使用任何方法计算行列式:
$\begin{vmatrix} 0 & 1 & 2 \\ -1 & 0 & -3 \\ 2 & 3 & 0 \end{vmatrix}$, $\begin{vmatrix} 1 & 2 & 3 \\ 4 & 5 & 6 \\ 7 & 8 & 9 \end{vmatrix}$, $\begin{vmatrix} 1 & 0 & -2 & 3 \\ -3 & 1 & 1 & 2 \\ 0 & 4 & -1 & 1 \\ 2 & 3 & 0 & 1 \end{vmatrix}$。

5.2. 使用行(列)展开计算行列式。注意,您不必使用第一行(列):选择具有许多零的行(列)将简化您的计算。
$\begin{vmatrix} 1 & 2 & 0 \\ 1 & 1 & 5 \\ 1 & -3 & 0 \end{vmatrix}$, $\begin{vmatrix} 4 & -6 & -4 & 4 \\ 2 & 1 & 0 & 0 \\ 0 & -3 & 1 & 3 \\ -2 & 2 & -3 & -5 \end{vmatrix}$。

5.3. 对于矩阵 $A = \begin{pmatrix} 0 & 0 & 0 & \dots & 0 & a_0 \\ -1 & 0 & 0 & \dots & 0 & a_1 \\ 0 & -1 & 0 & \dots & 0 & a_2 \\ \vdots & \vdots & \vdots & \ddots & \vdots & \vdots \\ 0 & 0 & 0 & \dots & 0 & a_{n-2} \\ 0 & 0 & 0 & \dots & -1 & a_{n-1} \end{pmatrix}$,计算 $\det(A + tI)$,其中 $I$ 是 $n \times n$ 单位矩阵。你应该得到一个涉及 $a_0, a_1, \dots, a_{n-1}$ 和 $t$ 的漂亮表达式。行展开和归纳可能是最好的方法。

5.4. 使用代数余子式公式计算矩阵 $\begin{pmatrix} 1 & 2 \\ 3 & 4 \end{pmatrix}$, $\begin{pmatrix} 19 & -17 \\ 3 & -2 \end{pmatrix}$, $\begin{pmatrix} 1 & 0 \\ 3 & 5 \end{pmatrix}$, $\begin{pmatrix} 1 & 1 & 0 \\ 2 & 1 & 2 \\ 0 & 1 & 1 \end{pmatrix}$ 的逆。

5.5. 设 $D_n$ 是 $n \times n$ 三对角矩阵的行列式:
$$
\begin{pmatrix}
1 & -1 & 0 & \dots & 0 \\
1 & 1 & -1 & \dots & 0 \\
0 & 1 & 1 & \dots & 0 \\
\vdots & \vdots & \ddots & \ddots & \vdots \\
0 & 0 & 0 & \dots & 1 & -1 \\
0 & 0 & 0 & \dots & 1 & 1
\end{pmatrix}
$$
使用代数余子式展开证明 $D_n = D_{n-1} + D_{n-2}$。这表明序列 $D_n$ 是斐波那契数列 $1, 2, 3, 5, 8, 13, 21, \dots$。

5.6. 重访 Vandermonde 行列式。我们的目标是证明 $(n+1) \times (n+1)$ Vandermonde 行列式的公式:
$$
\begin{vmatrix}
1 & c_0 & c_0^2 & \dots & c_0^n \\
1 & c_1 & c_1^2 & \dots & c_1^n \\
\vdots & \vdots & \vdots & \ddots & \vdots \\
1 & c_n & c_n^2 & \dots & c_n^n
\end{vmatrix} = \prod_{0 \le j < k \le n} (c_k - c_j)
$$
我们将应用归纳法。为此:
a) 验证公式对 $n=1, n=2$ 成立。
b) 将最后一行中的变量 $c_n$ 称为 $x$,并证明行列式是一个 $n+1$ 次多项式,$A_0 + A_1 x + A_2 x^2 + \dots + A_n x^n$,其中系数 $A_k$ 取决于 $c_0, c_1, \dots, c_{n-1}$。
c) 证明该多项式在 $x = c_0, c_1, \dots, c_{n-1}$ 处有零点,因此可以表示为 $A_n \cdot (x - c_0)(x - c_1) \dots (x - c_{n-1})$,其中 $A_n$ 如上所述。
d) 假设 Vandermonde 行列式的公式对 $n-1$ 成立,计算 $A_n$ 并证明对 $n$ 的公式。

5.7. 使用代数余子式展开计算 $n \times n$ 矩阵的行列式需要多少次乘法?证明这个公式。


\section{子式与秩}

对于矩阵 $A$,让我们考虑它的 $k \times k$ \textbf{子矩阵},它通过选取 $k$ 行和 $k$ 列得到。该矩阵的行列式称为 $k$ 阶\textbf{子式}(minor)。注意,一个 $m \times n$ 矩阵有 $\binom{m}{k} \cdot \binom{n}{k}$ 个不同的 $k \times k$ 子矩阵,因此它有 $\binom{m}{k} \cdot \binom{n}{k}$ 个 $k$ 阶子式。

\textbf{定理 6.1。} 对于一个非零矩阵 $A$,它的秩等于存在非零 $k$ 阶子式的最大整数 $k$。

\textbf{证明}~ 首先,让我们证明,如果 $k > \text{rank } A$,则所有 $k$ 阶子式都为零。实际上,由于 $A$ 的列空间的维数 $\text{Ran } A$ 是 $\text{rank } A < k$,因此 $A$ 的任何 $k$ 列都是线性相关的。因此,对于 $A$ 的任何 $k \times k$ 子矩阵,它的列都是线性相关的,所以所有 $k$ 阶子式都为零。

为了完成证明,我们需要证明存在一个非零的 $k$ 阶子式,其中 $k = \text{rank } A$。可能存在许多这样的子式,但也许最简单的方法是取主元行和主元列(即原始矩阵中包含主元的行和列)。这个 $k \times k$ 子矩阵具有与原始矩阵相同的主元,因此它是可逆的(每一列和每一行都有主元),并且其行列式非零。

这个定理看起来不是很有用,因为进行行约简比计算所有子式要容易得多。然而,它具有重要的理论价值,正如以下推论所示。

\textbf{推论 6.2。} 设 $A(x)$ 是一个 $m \times n$ 多项式矩阵(即其项是变量 $x$ 的多项式)。那么 $\text{rank } A(x)$ 在除了可能有限个点之外的地方是恒定的,在这些点上秩会变小。

% ---



% 6. 子式与秩。对于矩阵 $A$,我们考虑它的 $k \times k$ 子矩阵,通过选取 $k$ 行和 $k$ 列得到。这个矩阵的行列式称为 $k$ 阶子式。注意,一个 $m \times n$ 的矩阵有 $\binom{m}{k} \cdot \binom{n}{k}$ 个不同的 $k \times k$ 子矩阵,因此它有 $\binom{m}{k} \cdot \binom{n}{k}$ 个 $k$ 阶子式。

% \textbf{定理 6.1。} 对于一个非零矩阵 $A$,它的秩等于存在一个非零的 $k$ 阶子式时 $k$ 的最大整数值。

% \textbf{证明。} 我们首先证明,如果 $k > \text{rank } A$,则所有 $k$ 阶子式都为 $0$。确实,因为列空间 $\text{Ran } A$ 的维数是 $\text{rank } A < k$,所以 $A$ 的任意 $k$ 列都是线性相关的。因此,对于 $A$ 的任意 $k \times k$ 子矩阵,它的列都是线性相关的,所以所有 $k$ 阶子式都为 $0$。为了完成证明,我们需要证明存在一个秩等于 $\text{rank } A$ 的非零子式。这样的子式可能有许多,但最容易得到一个非零子式的方法是选取主元行和主元列(即包含主元的原始矩阵的行和列)。这个 $k \times k$ 子矩阵与原始矩阵有相同的主元,所以它是可逆的(每列和每行都有一个主元),其行列式非零。

% 这个定理看起来不是很有用,因为进行行变换比计算所有子式要容易得多。然而,它具有重要的理论意义,正如以下推论所示。

% \textbf{推论 6.2。} 设 $A = A(x)$ 是一个 $m \times n$ 的多项式矩阵(即其元素是关于 $x$ 的多项式)。那么 $\text{rank } A(x)$ 在除了可能有限个点之外的地方是恒定的,在这些点上秩会变小。

\textbf{证明}~ 设 $r$ 是 $\text{rank } A(x) = r$

\textbf{证明}~ 设 $r$ 是存在一个 $x$ 使得 $\text{rank } A(x) = r$ 的最大整数。为了证明这样的 $r$ 存在,我们首先尝试 $r = \min\{m, n\}$。如果存在一个 $x$ 使得 $\text{rank } A(x) = r$,我们就找到了 $r$。如果不是,我们则将 $r$ 替换为 $r-1$ 并重试。经过有限步操作,我们或者停止,或者得到 $0$。因此,$r$ 是存在的。设 $x_0$ 是一个点使得 $\text{rank } A(x_0) = r$,并且设 $M$ 是一个 $k$ 阶子式,使得 $M(x_0) \neq 0$。由于 $M(x)$ 是一个 $k \times k$ 多项式矩阵的行列式,所以 $M(x)$ 是一个多项式。由于 $M(x_0) \neq 0$,它不是恒零的,因此它只能在有限个点处为零。所以,除了可能有限个点之外,$\text{rank } A(x) \geq r$。但是根据 $r$ 的定义,对于所有的 $x$,$\text{rank } A(x) \leq r$。


\section{第3章的复习题}

7.1. 真或假:
a) 行列式只为方阵定义。
b) 如果 $A$ 的两行或两列相同,则 $\det A = 0$。
c) 如果 $B$ 是通过交换 $A$ 的两行(或两列)得到的矩阵,则 $\det B = \det A$。
d) 如果 $B$ 是通过将 $A$ 的某一行(列)乘以一个标量 $\alpha$ 得到的矩阵,则 $\det B = \det A$。
e) 如果 $B$ 是通过将 $A$ 的某一行乘以一个数加到另一行得到的矩阵,则 $\det B = \det A$。
f) 三角矩阵的行列式是其对角线元素的乘积。
g) $\det(A^T) = -\det(A)$。
h) $\det(AB) = \det(A)\det(B)$。
i) 矩阵 $A$ 可逆当且仅当 $\det A \neq 0$。
j) 如果 $A$ 是可逆矩阵,则 $\det(A^{-1}) = 1/\det(A)$。

7.2. 设 $A$ 是一个 $n \times n$ 矩阵。$\det(3A)$, $\det(-A)$ 和 $\det(A^2)$ 与 $\det A$ 的关系是什么?

7.3. 如果 $A$ 和 $A^{-1}$ 的所有元素都是整数,那么 $\det A = 3$ 是否可能?
\textbf{提示:} $\det(A)\det(A^{-1})$ 是什么?

7.4. 设 $v_1, v_2$ 是 $\mathbb{R}^2$ 中的向量,设 $A$ 是以 $v_1, v_2$ 为列的 $2 \times 2$ 矩阵。证明 $|\det A|$ 是由向量 $v_1, v_2$ 作为两边给出的平行四边形的面积。首先考虑 $v_1 = (x_1, 0)^T$ 的情况。对于一般情况 $v_1 = (x_1, y_1)^T$,左乘一个旋转矩阵,将向量 $v_1$ 变换为 $(\tilde{x}_1, 0)^T$ 来处理。
\textbf{提示:} 旋转矩阵的行列式是什么?
以下问题说明了行列式的符号与向量组的“方向”之间的关系。

7.5. 设 $v_1, v_2$ 是 $\mathbb{R}^2$ 中的向量。证明 $D(v_1, v_2) > 0$ 当且仅当存在一个旋转 $T_\alpha$ 使得向量 $T_\alpha v_1$ 与 $e_1$ 平行(并且方向相同),且 $T_\alpha v_2$ 位于上半平面 $x_2 > 0$(即 $e_2$ 所在的半平面)。
\textbf{提示:} 旋转矩阵的行列式是什么?

---
